\textbf{Alessandra Martins Parente} é psicanalista, fundadora e coordenadora do
Projeto \emph{Causdequê?} Doutora em Psicologia Social e do Trabalho pela \versal{USP} com
estágio"-sanduíche pelo \versal{ZfL}-Berlin (bolsa \versal{CAPES}), Mestre em Psicologia Clínica pela
\versal{PUC"-SP}, formada em Psicologia pela \versal{PUC"-SP} e em Filosofia pela \versal{USP}. Em 2018 realizou seu Pós"-doutorado pelo Departamento de Filosofia da \versal{USP}, com estágio na
Birkbeck, University of London (bolsa \versal{FAPESP/FAPESP"-BEPE}). É autora de
\emph{Sublimação e Unheimliche} (Pearson, 2017) e de artigos que estabelecem interfaces entre
Psicanálise, Filosofia e Artes Visuais. Atende e oferece supervisões clínicas em
consultório particular. Foi professora de Filosofia durante oito anos no Ensino Médio
em diferentes escolas de São Paulo e por sete anos em cursos de Psicologia do Ensino
Superior. É membra do Laboratório de Teoria Social, Filosofia e Psicanálise (Latesfip-\versal{USP}) e do \versal{GT} de Filosofia e Psicanálise da \versal{ANPOF}.

\textbf{Léa Silveira} é professora de Filosofia da Universidade Federal de Lavras, instituição na
qual participou da criação do curso de licenciatura em Filosofia, do Departamento de
Ciências Humanas e do mestrado em Filosofia. Fez a pós"-graduação em Filosofia na
Universidade Federal de São Carlos e formou"-se em Psicologia pela Universidade
Federal do Ceará. Suas publicações são voltadas para a área de Filosofia da Psicanálise.
É membro do comitê executivo da International Society of Psychoanalysis and
Philosophy (\versal{SIPP"-ISPP}) e do núcleo de sustentação do \versal{GT} de Filosofia e Psicanálise da \versal{ANPOF}. É membro do conselho editorial das revistas de Filosofia \emph{Ipseitas}, \emph{Eleutheria} e \emph{Em curso} e da revista de psicanálise \emph{Analytica}. Atua como assessora para a Fapesp.

\textbf{Freud e o patriarcado} parte da constatação de que o campo da teoria psicanalítica põe em jogo uma forma de conceber o psíquico --- ou a subjetividade --- como algo que se constrói a partir de um modelo que assume, em seu centro, uma equivalência generalizada entre cultura, civilização e masculinidade. Ao assumir esse tipo de encaminhamento, a psicanálise coloca, no centro de seus modelos teóricos, algo que deveria ser explicado, em vez de ser tomado como dado. Sobre essa trilha as autoras e os autores desse livro tecem suas considerações, seja para explorar a legitimidade e a preservação dos modelos descritivos psicanalíticos ancorados nas inspirações originárias de Freud e em seus desdobramentos, buscando sua potência própria; seja para apontar, nos próprios textos de Freud, elementos que permitiriam vislumbrar modelos distintos; ou ainda para problematizar algumas de suas teses, apostando mais diretamente na necessidade de repensá"-las. 







