\part{Interstícios do texto freudiano:\\ acerto de contas}

\chapter*{Dentro do sonho\footnote{Este capítulo é uma adaptação do
  Capítulo 3 da obra de Stephen Frosh, \emph{Diferença Sexual:
  Masculinidade e psicanálise}. Londres: Routledge, 1994.}}
\addcontentsline{toc}{chapter}{Dentro do sonho, \footnotesize\emph{por Stephen Frosh}}
\hedramarkboth{Dentro do sonho}{}


\begin{flushright}
\emph{Stephen Frosh}
\end{flushright}

\section{As origens da psicanálise}

Não importa o que se diga sobre diferenças sexuais, a maneira de o dizer
se torna uma expressão de como essa diferença está sendo sentida; ao
falar, a pessoa também se posiciona, homem ou mulher, com todas
associações derivadas de domínio ou resistência, conhecer ou ser, visão
ou tato. Isso não significa que a diferença sexual ``é'' algo
absolutamente fixo; na verdade, a organização do mundo social em torno
da diferença produz pessoas que se relacionam com o gênero, de forma que
o que são, em princípio, categorias ``vazias'' (masculino, feminino),
tornam"-se repletas de expectativas, estereótipos e projeções. Isso,
porém, não torna os seus efeitos menos reais: apesar das distinções de
gênero serem construídas e (em importantes sentidos) ``arbitrárias'',
elas possuem um poder sobre nós e são difíceis, talvez impossíveis, de
transcender. Ao se explorar a maneira com que a diferença sexual é
construída, analisando isso como um processo social, a centralidade do
gênero para a subjetividade --- por exemplo, para o eixo central do senso
de identidade das pessoas --- vem à tona. Consequentemente, escrever
sobre diferença sexual nunca resulta em uma atividade neutra: sempre nos
coloca frente a frente com nossas próprias convicções e desejos.

A psicanálise também participa desse processo. Um de seus mais poderosos
ensinamentos é que cada discurso tem seu subtexto, seu reverso ou
antagonismo, posicionado dentro dele; esse subtexto fala através das
fendas, através das escorregadas e dos momentos de incerteza da
linguagem, sempre colocando o todo em dúvida. Assim como isso é verdade
em relação ao discurso do analisante, é também verdadeiro em relação ao
discurso da própria psicanálise. A qualquer momento, o que está
``realmente'' sendo dito pode não ser o que está aparentemente sendo o
alvo da conversa, mas algo diferente, algo que requer interpretação para
retirá"-lo de seu esconderijo. Além disso, aquele que fala pode não ser o
``eu'' sereno e autoral que parece estar ali, possivelmente apresentando
uma faceta racional e equânime, porém algo inconsciente, envolto em
complicadas convicções, incluindo aquelas em torno da diferença sexual.
A superfície da linguagem psicanalítica, igual à linguagem de qualquer
paciente, assim, mascara algo significativo; e esse algo é, ele próprio,
momentâneo e escorregadio, e fala a partir de uma posição de incerteza,
de constante mudança e conflito.

A tradicional associação da masculinidade com racionalidade e da
feminilidade com irracionalidade começa a se encaixar aqui. Há um
sentido em que as origens da psicanálise podem ser descritas como
provenientes dessa tensão entre razão masculina e expressividade
feminina, sentida particularmente como histeria: em psicanálise, aquela
diz desta, mas precisa da última para poder ter algum conteúdo
específico. A psicanálise é construída sobre essa situação de tensão:
ela brinca com as fronteiras da razão, recusando"-se a abrir mão de sua
busca por significado, mas mostrando igualmente como o irracional é um
elemento intrínseco à condição humana e nunca irá desaparecer. Não
surpreende, portanto, que mesmo quando a psicanálise se apresenta em sua
forma mais dominante e de aparência claramente científica, ela também
pode ser obscura e desconfortável consigo mesma, sugerindo um estado
interno de inquietação. Essa situação torna"-se particularmente pungente
quando o paralelo implícito entre feminilidade e inconsciente vem à
tona. Traduzido em termos de gênero, a atividade aparentemente
``masculina'' de saber e organizar os fenômenos está sempre relacionada
com a atividade supostamente ``feminina'' de personificação e disrupção.
Quando ela fala, com suas muitas vozes contraditórias, de diferença
sexual, a psicanálise delineia conceitos que ajudam no desafio de
encontrar significado e obter sentido a partir da maneira como o gênero
se insinua na vida psíquica das pessoas. Porém, a fala da psicanálise
também \emph{expressa} essa presença do gênero. Nos textos de
psicanálise, assim como nos discursos dos pacientes, é possível
encontrar as vozes da diferença sexual, canalizando, influenciando e
minando as narrativas que se desdobram. De fato, por causa de sua
preocupação com sexualidade e subjetividade, a linguagem da psicanálise
é uma linguagem particularmente sexuada, marcada pela instabilidade de
suas posições identificatórias e por sua jocosidade quando defrontada
com questões de referência. Sempre que fala com uma voz aparentemente
``masculina'', ela igualmente sugere o tema da feminilidade: ao levar
adiante a palavra do mestre, a subversão dessa palavra também surge em
segundo plano.

Neste capítulo, eu exploro essa ambiguidade da própria psicanálise na
sua posição sobre gênero por meio da análise da questão da diferença
sexual em um dos mais importantes textos de fundação da psicanálise,
\emph{A interpretação dos sonhos} (1900), de Freud. Argumentarei que
esse texto trabalha com categorias convencionais de masculinidade e
feminilidade, de forma tal que ambos os lados da tensão antes mencionada
estão visíveis --- tanto o modo pelo qual a diferença sexual funciona
como princípio organizador central à identidade pessoal, como também a
maneira pela qual seu caráter construído torna possível a transgressão
da categorização tradicional de gênero. Em relação à psicanálise, os
conteúdos do texto não possuem somente importância histórica. Durante
toda sua história, a psicanálise esteve enredada na personalidade, assim
como nas ideias de Freud. Embora os desenvolvimentos na teoria e na
prática tenham sido significativos desde a sua época, o trabalho de
Freud continua definindo os princípios básicos da teoria e da prática
analíticas, e sua figura continua a ser objeto de identificação, o
superego corporativo de todos os outros que surgiram desde então. Além
do mais, a psicanálise é uma ciência incomum, no sentido de que seu
conjunto de conhecimentos iniciais foi construído, em grande parte,
baseado na personalidade e auto"-escrutínio de Freud. Sua famosa
``autoanálise'', perpetrada nos últimos anos do século \versal{XIX}, é comumente
representada (particularmente por Jones {[}1955{]} na primeira biografia
``oficial'') como o heroico ato de fundação da psicanálise; dado que se
volta explicitamente aos impulsos subjetivos e conflitos internos de
Freud, significa que seus próprios desejos, conscientes e inconscientes,
estão colocados como as linhas de partida de onde todas as análises se
posicionam. Posteriormente, a autoanálise de Freud tornou"-se o protótipo
para a análise didática realizada como parte do processo de se tornar um
psicanalista, produzindo uma sequência dinástica que se inicia com o ato
de autocriação. Aqui, já podemos encontrar algumas ambiguidades
estruturais que andam em paralelo com aquelas encontradas na diferença
sexual. Assim como no mito bíblico do Gênesis, o criador original reúne
em seu próprio ser tudo que é necessário para gerar a vida, elementos
tanto masculinos como femininos, poderíamos dizer; mais tarde, ocorre a
divisão, inicia"-se a diferenciação e ninguém dá conta da missão
analítica criativa por conta própria.

\emph{A interpretação dos sonhos} é a obra que está mais intrinsecamente
ligada a essa autoanálise e a que, consequentemente, expressa mais
vigorosamente as ambiguidades e tensões envolvidas no processo criativo
do qual a psicanálise nasceu. Nesse livro, publicado no início do século
\versal{XX}, encontramos um texto que é, ao mesmo tempo, científico e
confessional, incorporando teoria formal lado a lado com revelação
pessoal consciente (ele é construído, principalmente, em torno de
interpretações dos próprios sonhos de Freud, assim como os de alguns dos
seus pacientes) e ansiedade e desejo inconscientes. E, bem no cerne
desse texto, conforme este capítulo pretende revelar, reside uma
incerteza relacionada ao gênero: uma masculinidade que brinca, reprime e
exprime sua própria fragmentação e identificações femininas
ambivalentes. \emph{A interpretação dos sonhos} é apresentada por Freud
como uma triunfal elucidação dos recônditos obscuros do irracional --- o
desconhecido terreno da noite; mas a obra continuamente levanta a
questão da identidade sexual do sonhador e do intérprete, bem como, a do
sonho e de sua análise.

Todos os sonhos descritos por Freud na primeira edição desse grande e
abrangente livro foram sonhados no final do século \versal{XIX}, em uma época de
conflito para Freud, na qual nasce algo novo, algo terrível e, ainda
assim, excitante. Portanto, esses sonhos e suas interpretações podem ser
vistos como um tipo de comentário, não apenas da situação em que Freud
entendia estar, mas também das forças e experiências sociais que deram
origem aos grandes movimentos modernistas do início do século \versal{XX} ---
forças essas que são mais facilmente caracterizadas como
``revolucionárias'', como uma quebra com as correntes do passado que
leva a um conjunto de circunstâncias em que mudanças ocorrem em um ritmo
nunca antes visto, muitas vezes por meio da inversão radical de crença e
organização social. Anzieu (1975) assim comenta o estado de mente de
Freud no momento em que ele sonhava (em 1895) o seu mais famoso sonho,
seu sonho ``espécime'', da injeção de Irma, no qual aqui focaremos:

\begin{quote}
Freud, fragmentado em peças díspares, estava buscando sua verdadeira
unidade. O sistema de identificações que havia lhe governado até então
ruiu. Por ora, sua vida havia sido dominada pelos desejos de outros.
Durante aquela noite de 23-24 de julho de 1895, seu sonho o questionou
sobre seus próprios desejos. (p.132)\footnote{Foi curioso notar que na
  versão brasileira de A auto"-análise de Freud a passagem citada por
  Stephen Frosh simplesmente foi suprimida. Ela deveria estar localizada
  na p. 38 do livro editado pelas Artes Médicas, 1989, mas simplesmente
  não está lá. Em razão disso, a paginação acima segue a edição inglesa
  originalmente citada por Frosh com trad. Nossa (N. do T.)}
\end{quote}

Essa imagem do indivíduo em fragmentos, incerto sobre a verdade do saber
e sobre o ``sistema de identificações'' por meio do qual havia vivido
anteriormente, buscando desesperadamente por um significado alternativo
e unificador que tornará possível a fabricação de uma vida novamente
integrada --- essa é uma imagem modernista característica, do indivíduo
em conflito com a fragmentação da modernidade e tentando, por meio de
sua criatividade, atravessá"-la vitoriosamente. Ela também é
predominantemente apresentada como uma imagem \emph{masculina} da
autoconstrução heroica, em que o indivíduo tem poder suficiente para
forçar algo novo para dentro do mundo, para ``livrar"-se de suas
amarras'', como diz Anzieu (p.146), e descobrir um novo caminho para si.
No entanto, mesmo nisso existe ambiguidade sexual. A tarefa de Freud,
nesse momento, era a de abrir mão das tradições de objetividade
científica e repressão social e começar a se estudar, encontrando uma
voz segura que pudesse articular sua própria experiência --- uma missão
de autoexpressão e ``política de identidade'' que se associa de perto
com o feminismo contemporâneo. Mais especificamente, Anzieu define a
consequência para Freud do processo de questionamento que se inicia a
partir do sonho de Irma como um futuro ``segundo nascimento'' (que ``não
pode deixar de evocar nele seu primeiro nascimento'' --- p. 146),
simbolizado pelo deslocado exame ginecológico presente no próprio sonho
(olhando o fundo da garganta de Irma para encontrar ``algumas notáveis
estruturas encaracoladas que eram evidentemente modeladas a partir dos
ossos turbinados do nariz'' --- Freud, 1900, p. 107; Freud mais tarde
relata {[}p. 117{]} como seu amigo Wilhelm Fliess associa ``os ossos
turbinados e os órgãos reprodutivos femininos''). A imagem é de Freud
dando um novo parto a si mesmo, libertando uma energia que produz a
psicanálise como consequência. A sexualidade aqui é uma combinada, pois
há tanto o gênio determinado como o maternal ato de dar à luz; mais
ainda, há um aspecto hermafrodita do novo Freud deslizando para fora da
pele do antigo.

O próprio Freud é menos delicado e está mais em sintonia com os
sentimentos revolucionários daqueles tempos. Ele admite a vontade de
abrir o capítulo sobre terapia que há no livro com o título ``Flavit et
dissipati sunt'' --- ``Ele soprou e eles se espalharam''. Ainda que isso
refira"-se aos problemas inerentes ao seu trabalho clínico e teórico, aos
seus oponentes, ou a toda a ordem estabelecida das coisas, a imagem aqui
é a de uma revolução.

\section{Identificações masculinas: de José a Moisés}

É com \emph{A interpretação dos sonhos} que a psicanálise obtém sua
própria voz, a voz do inconsciente e a voz da personalidade de Freud.
Freud carimba sua autoridade em seu material; ele mostra como tudo se
encaixa, ele resolve o segredo dos sonhos, ele sabe o que apenas
profetas sabem. A partir disso, surge um movimento de crentes, um corpo
de conhecedores, aqueles que, por causa de Freud, conseguem ver,
interpretar e entender.

A imagética religiosa dessa descrição deriva diretamente de uma
estratégia consciente de identificação adotada pelo próprio Freud. Essa
estratégia produz uma posição autoral que é conscientemente masculina,
inquestionada como tal durante o texto, e que também permite a Freud
estabelecer uma conexão com sua herança cultural judaica, de forma que
ele se torne o novo profeta --- secular, talvez, mas seguindo firmemente
os passos dos heróis do Antigo Testamento. Parece claro que ambos os
impulsos são de grande importância para Freud. A questão da
masculinidade é uma que, conforme será descrito mais detalhadamente
adiante, surge para Freud dentro do contexto de sua tentativa de
solucionar seu relacionamento com seu pai. Dessa forma, uma possível
trajetória de \emph{A interpretação dos sonhos} seria a tentativa de
Freud de penetrar sua masculinidade --- de resolver as questões que lhe
impediam de completar a transição da posição de filho (de seu pai de
fato e dos tutores masculinos idealizados de seu passado) para a de pai
(de seu próprio legado intelectual, o movimento psicanalítico). Para
fazê"-la, Freud adota identificações que brincam com a dinâmica
filho/pai, particularmente em relação às figuras bíblicas José e Moisés.
Uma das ironias que aqui emergem, contudo, é que a construção
autoconsciente e um tanto estereotipada dessa persona ``masculina'' está
constantemente em tensão com um conjunto de identificações ``femininas''
implícitas, novamente expressando a instabilidade daquela naturalizada
polaridade.

O segundo dos impulsos acima mencionados, aquele de inserir"-se na
herança cultural de seu povo, foi algo que acompanhou Freud durante toda
a sua obra. Ele pode ser observado de diferentes formas como, por
exemplo, em relação ao impacto de suas próprias experiências com o
antissemitismo (atestadas diretamente em \emph{A interpretação dos
sonhos}), à congenialidade de seus laços com outros médicos e leigos
judeus (como em sua associação à organização B'nai B'rith), em suas
ansiedades decorrentes da concentração de judeus entre os primeiros
psicanalistas ou em sua resposta ao nazismo (Frosh, 2005). Tom e
conteúdo do que é encontrado em \emph{A Interpretação dos Sonhos}
alinham"-se a isso, como se Freud estivesse usando os elementos de sua
herança judaica para criar seu próprio livro das origens. Mais ainda:
como o padrão de identificações do material é masculino, o livro
torna"-se, em parte, um compromisso com o patriarcado.

Vale a pena explorar em mais detalhes como o texto trabalha. A
consciência de Freud em relação à sua posição histórica transicional ---
entre o velho e o novo, entre Egito e Israel --- leva"-o a identificações
com duas figuras bíblicas centrais, José, o profeta, e Moisés, o
libertador. Em relação ao primeiro, Freud escreve:

\begin{quote}
Já terá percebido que o nome Josef tem papel importante nos meus sonhos
{[}\ldots{}{]}. Meu Eu pode se esconder facilmente por trás das pessoas que
assim se chamam nos sonhos, pois Josef é também o nome do interpretador
dos sonhos que conhecemos da Bíblia. (Freud, 1900/2019, p. 812, n.
2)\footnote{Para as citações, adota"-se a versão da Companhia das Letras,
  trad. Paulo César de Souza.}
\end{quote}

O dom de José vem de Deus e depende da sua fé em Deus; seu exercício
leva"-o ao exílio e em seguida à redenção, um processo que é
posteriormente recapitulado (mas ao reverso) por todo seu povo. Moisés
chega ao final dessa sequência; Moisés também aparece em vários momentos
da vida de Freud, de modo mais pungente no fim, em sua meditação confusa
e evocativa sobre o nazismo, em \emph{Moisés e o Monoteísmo} (Freud,
1939). Em \emph{A interpretação dos sonhos}, Moisés aparece
explicitamente pela primeira vez como uma figura de identificação, em um
dos sonhos de ``Roma'' (p. 194):

\begin{quote}
Em outro sonho, alguém me levava para o alto de uma colina e me mostrava
a cidade de Roma meio velada pela neblina e ainda tão distante que a
nitidez da vista me espantava. O conteúdo do sonho era mais rico do que
devo expor aqui. O tema ``ver a terra prometida à distância'' pode
facilmente ser reconhecido. (Freud, 1900/2019, p. 267-8)
\end{quote}

Moisés liberta seu povo da escravidão no Egito --- a opressão física, a
perda de identidade, os deuses falsos --- e os conduz ao deserto. Porém,
sua própria rebeldia agressiva o impede de ser aquele que leva os
israelitas à terra prometida; ele apenas a vislumbra de longe. Não
obstante, no caminho, ele deu ao seu povo a Lei; agora, para atingir sua
soberania e seu verdadeiro \emph{status}, eles {[}os membros do povo{]}
devem segui"-la. Que isso se prove, a longo prazo, uma missão impossível,
nada mais é do que parte das condições de existência humana.

Há uma boa dose de reflexividade na maneira como José e Moisés aparecem
em \emph{A interpretação dos sonhos}; isto é, como figuras
significativas em um território onírico, eles dizem muito sobre o texto
em si e, claro, sobre Freud. A visão pessoal de José é uma em que os
futuros problemas de um povo tornam"-se entrelaçados; é como se eles
fossem arrastados para dentro do turbilhão de seus próprios sonhos e
ambições. Filho predileto que, assim como Freud, era o primogênito da
segunda esposa de seu pai, José sonha sonhos imensos de grandeza e
ambição. Por ser o escolhido de seu pai, ele sofre abusos e é
escravizado, suscita inveja e desejo, é preso e acorrentado. Então, no
ponto mais baixo de sua fortuna, ele descobre seu grande dom, uma
habilidade mágica de enxergar através do mistério dos sonhos. Por meio
da própria percepção de seu conflito interno, José aprende a perceber os
outros; ao interpretar os sonhos de outros, tornando claro suas
estruturas e significados, ele compreende seus próprios sonhos e
torna"-se o mestre de seus irmãos. A inveja deles transforma"-se em medo e
gratidão, os murmúrios se silenciam; as conquistas de José são evidência
de que, afinal, sua grandeza não era uma fantasia.

Isso em si já possui a estrutura de um sonho, a realização de um desejo
em que o vilipendiado triunfa sobre seus opressores. E ressoa através de
\emph{A interpretação dos sonhos} como uma resposta à memória produzida
ao sonho do ``Conde Thun''. Quando tinha ``sete ou oito'' anos, Freud
urinou no quarto de seus pais; enfurecido, seu pai descarta todos os
sinais anteriores de grandeza com a seguinte profecia: ``esse garoto não
será nada na vida'' (p. 296). Freud comenta:

\begin{quote}
Deve ter sido uma humilhação terrível para a minha ambição, pois meus
sonhos fazem alusões frequentes a essa cena e regularmente vêm
acompanhados de uma enumeração dos meus trabalhos e sucessos, como se eu
quisesse dizer: ``Está vendo? Consegui ser algo na vida''. (Freud,
1900/2019, p. 294-5).
\end{quote}

A estrutura da história de José e as constantes interposições entre os
conflitos de Freud em torno de seu próprio pai e sua ambivalência em
relação ao seu por algum tempo patrono e colaborador Josef Breuer
permitem a Freud fazer tal afirmação --- ``Eu dei em algo'' --- e legitimar
sua grandiosidade e reivindicação à fama.

Há, porém, um lado negro em José, com o qual Freud também parece haver
feito uma conexão pessoal. José suscita inveja e recriminação por causa
de sua falta de humildade. Vivendo em um mundo de sonhos, ele força a
realidade a se dobrar às suas vontades para realizar os desejos
triunfantes do início de sua vida. Ao fazê"-lo, ele salva sua família e
seu povo da fome, apenas para conduzi"-los a um longo pesadelo de
escravidão e desespero. A ambiguidade dessa história --- a maneira como
triunfo pessoal se transforma posteriormente em tragédia comunal --- é
algo profundamente enraizado na consciência judaica de Freud (ver
Diller, 1991). Em relação à obra \emph{A interpretação dos sonhos} e à
fundação da psicanálise, também trata"-se de uma pungente expressão de
questionamento: seria tudo isso apenas o sonho de Freud, e a que
pesadelo ele levará aqueles que ficarem enredados nele? É difícil, de
maneira retrospectiva, não enxergar aqui a influência de uma consciência
interna sobre as forças que mais tarde levaram ao surgimento do Nazismo,
uma certa expressão inconsciente do reconhecimento explícito de Freud e
de seu desconforto com o elo entre psicanálise e judaísmo, e um medo em
relação a onde isso poderia levar. No mínimo, isso sugere que o duplo
desafio que Freud lança ao mundo --- primeiro esboçando uma teoria
ultrajante (sonhos são realizações de desejos sexuais) e, em seguida,
explicitamente ligando"-a ao seu judaísmo --- era, em certa medida, uma
reação ao seu senso de rejeição por parte das comunidades científicas e
não judaicas.

Moisés faz a viagem reversa àquela de José, do Egito às fronteiras de
Canaã, da escravidão à liberdade inerente à Lei. Ele leva consigo os
ossos de José, para que possa enterrá"-los em seu devido lugar ao fim da
longa noite de exílio. Aqui, a relação entre tempo e espaço é bem
complexa, intimamente ligada à relação do sonho com a realidade. Ao
deixar o Egito, Moisés leva seu povo ao deserto; tendo falhado em manter
uma fé perfeita, a geração de pessoas que trabalhou como escrava precisa
se extinguir antes que uma nova geração, que nada sabe de seu
\emph{status} anterior no Egito, possa ser confiada com a missão de
tomar posse da terra prometida. O deserto, o espaço entre Egito e
Israel, é empregado para permitir que o tempo inflija seus efeitos; por
quarenta anos os israelitas são obrigados a vaguear, até que não reste
nenhum ex"-escravo. Geralmente, o tempo conquista o espaço; qualquer
distância pode ser vencida, com tempo suficiente. Mas Moisés conduz o
seu povo em círculos, esperando. O tempo torna"-se vassalo do espaço; o
espaço relativamente pequeno se expande indefinidamente, até que o tempo
disponível aos indivíduos se esgota.

O próprio Moisés é punido por sua presunção e impaciência, não lhe sendo
permitida a experiência da realização do sonho do qual ele foi parte.
Esse não é o seu sonho: ele é uma peça relutante de algo muito maior do
que ele mesmo. De alguma maneira, ainda é o sonho de José, que não está
terminado até Josué enterrar os seus ossos em Israel; Moisés é
simplesmente o instrumento principal por meio do qual a redenção do
pesadelo se concretiza. De modo mais amplo, é o sonho de Deus, o sonho
do inconsciente, do Outro. Moisés é recalcitrante, ele discute com Deus
quando ele é escolhido, no local da sarça ardente, para liberar os
hebreus, mas ele deve colaborar; algo viaja através dele, ele é uma
figura por meio da qual um desejo é expresso e alcançado. Ao
identificar"-se com ele, Freud vê sua irascibilidade, seu poder e sua
legislatura; mas ele também evoca seu \emph{pathos}, o modo como seus
próprios desejos e temores têm que sofrer certa renuncia para que sua
missão possa ser atingida.

As identificações de Freud com José e Moisés envolvem um conjunto
complexo de ambiguidades relacionadas a tempo e a espaço, sujeito e
objeto, sonhador e sonho. Essas ambiguidades se conectam de perto com a
questão da autoridade, que permeia tanto esse texto como o contexto
maior da psicanálise. É também por meio disso que surge a voz falante do
inconsciente e da diferença sexual. As identificações explícitas de
Freud são com autoridades masculinas, figuras heroicas de tempos
bíblicos. Sustentando"-as, entretanto, está uma estrutura de incerteza,
em que o herói é um desastre, ou um objeto do plano de outro. Não mais
senhor do sonho, não mais capaz de falar e comandar, essa representação
da masculinidade torna"-se problemática, na forma de algo instável e
incerto. O sonho em si toma conta, com todas as suas ambiguidades e
fluidez de posicionamentos. Se a diferença sexual é um eixo primário em
torno do qual se forja a identidade, as complexidades de identificações
presentes em \emph{A interpretação dos sonhos} revelam a precariedade da
identidade e da própria diferença sexual.

\section{O sonho da injeção de Irma}

É no sonho ``espécime'' de Freud, o sonho da ``injeção de Irma'', que
essa incerteza recorrente sobre identidade de gênero e diferença sexual
aparece de modo mais claro. Freud apresenta esse sonho e sua análise
como uma oportunidade de demonstrar seu ``método de interpretação''
(Freud, 1900, p. 105); porém, fica claro por meio dos detalhes e da
energia da análise que ele está engajado em trabalhá"-lo por motivos que
ultrapassam a intenção pedagógica. De fato, é em comemoração à
ocorrência desse sonho em particular que Freud sugere a Fliess a
colocação de uma placa de mármore na casa em que ele o sonhou --- uma
placa com os dizeres ``Nesta casa, em 24 de julho de 1895, o segredo dos
sonhos foi revelado ao Dr. Sigm. Freud'' (p. 121, n.1). Uma nota
relatando esse intuito é colocada pelos editores de \emph{A
interpretação dos sonhos} imediatamente após estas últimas palavras do
capítulo sobre Irma: ``Após completar o trabalho de interpretação,
percebemos que o sonho é a realização de um desejo'' (p. 179). Todavia,
como Lacan (1955-6) observou, a enorme significância atribuída a esse
sonho, e a natureza revolucionária da conclusão que dele se obtém, pouco
parecem se relacionar ao conteúdo manifesto da interpretação do sonho em
si, do qual foi removido todo reconhecimento explícito de sua subversão
e perturbação da sexualidade. Lacan questiona:

\begin{quote}
{[}\ldots{}{]} como é que Freud, ele que vai mais adiante desenvolver a
função do desejo inconsciente, contenta"-se aqui, para o primeiro passo
de sua demonstração, em apresentar um sonho inteiramente explicado pela
satisfação de um desejo que não se pode chamar de outro modo a não ser
de pré"-consciente, e até mesmo de inteiramente consciente? (Lacan,
1955-6/ 1985, p. 193)
\end{quote}

A resposta de Lacan a essa pergunta concerne ao significado das
revelações sobre análises de sonhos em relação às estruturas de ego e
identidade; retornaremos a ela mais adiante. Porém, em uma análise mais
simples, Freud parece estar usando o sonho como uma demonstração de sua
própria autoridade --- novamente, como uma expressão de sua identificação
com José, mas também, dado o conteúdo do sonho em si, como uma fantasia
de conquista sexual.

Anzieu comenta que a análise do sonho de Irma parece ``desarrumada'',
mas, na verdade, ``é notavelmente bem"-estruturada, e se desenrola como
uma peça, com os personagens sendo introduzidos nos primeiros atos e o
desenlace ocorrendo no último'' (1976, pp. 137-8). A peça em questão é
um tipo de drama de tribunal:

\begin{quote}
O primeiro ato ceça por uma oração de defesa de Freud {[}\ldots{}{]} ele
termina em seu medo. No segundo ato, Freud é acusado com provas
terríveis. No terceiro ato, as testemunhas, os advogados, destroem estas
provas. A questão que é núcleo da inquirição é então apresentada
claramente: quem é responsável? (Anzieu,1989, pp. 47-8)
\end{quote}

Uma possível resposta, de acordo com a leitura de Anzieu, é que a
responsável é a ``injeção de trimetilamina''; isso é, ``a causa dos
males de Irma é sua vida sexual insatisfatória. Freud tem razão contra
seus detratores, pois sustenta a etiologia sexual das neuroses'' (p.
48). Nessa parte de sua análise, Anzieu está enxergando o sonho de Irma
como parte da tendência de auto"-justificativa presente em \emph{A
intepretação dos sonhos}: Freud derrubando a maldição de seu pai para
descobrir o segredo da vida mental. Irma permite que ele demonstre essa
descoberta com autoridade: todas aquelas mulheres teimosas e homens
ignorantes que, juntos, causaram uma bagunça, e mesmo assim acusam
Freud, são despachados e o feito de Freud é mais uma vez asseverado. É a
sórdida injeção que deve ser responsabilizada, a suja seringa; qualquer
freudiano notaria isso a partir do material em questão.

O próprio Freud fornece uma analogia famosa a sua postura defensiva no
sonho, com uma referência específica à maneira com que o sonho está
estruturado, como uma contestação formal contra as acusações de
responsabilidade pela doença de Irma.

\begin{quote}
Toda essa alegação --- esse sonho não é outra coisa --- lembra vivamente a
defesa do homem que era acusado pelo vizinho de ter devolvido uma
chaleira em estado defeituoso. Em primeiro lugar, ele diz que a
devolvera em perfeito estado; em segundo lugar, que a chaleira já estava
furada quando ele a tomou emprestada; em terceiro lugar, que jamais
tomou emprestada a chaleira do vizinho. (Freud, 1900/2019, p. 178)
\end{quote}

O Freud como intérprete de seu sonho consegue olhar com ironia para o
Freud que o sonha, revelando as fragilidades em meio aos seus desejos.
Mas os dois Freuds não estão tão completamente divididos, e o Freud
analítico não tem o controle absoluto de seus procedimentos, como
gostaria que acreditássemos. Anzieu (1989, p. 49) comenta que ``Freud
advertirá, mais tarde, que os pensamentos que sobrevêm após o sonho são
ainda pensamentos do sonho''. A análise da contestação múltipla que está
presente no próprio sonho (a própria Irma é responsável por suas dores,
suas dores são orgânicas, elas têm relação com sua viuvez, com a injeção
equivocada, com a seringa suja) produz a generalização de que sonhos são
realizações de desejo; porém isso também é a realização de desejo, o
desejo de Freud de decifrar o mistério dos sonhos. A estrutura de gênero
do próprio sonho amplia isso. O objeto central do sonho é feminino:
Irma, com sua boca e torso abertos para inspeção pelo sujeito masculino
do sonho (o sonhador) e seus apoiadores e competidores. Similarmente, o
próprio sonho está prostrado para inspeção e análise pelo autor e
psicanalista (o sonhador, novamente), sonhando com sua fama com base no
corpo desse sonho. Assim, o sonho está associado ao objeto feminino,
como dois espelhos que se refletem; a análise do sonho é a análise do
corpo, examinando os tubos genitais de Irma através de sua boca.
Conquistando o mistério dos sonhos e conquistando a mulher; a autoridade
não é tão mundana, não é apenas uma contestação ao desdém professional.

Freud nos fornece esse sonho como uma ilustração de sua arte e, por meio
de uma cuidadosa ordenação do material, edita o que acontece, permitindo
que a força do sonho comunique mais pela centralidade que lhe é
outorgada --- e pelo caráter evocativo da própria figurabilidade do sonho
--- do que pela abrangência da análise. A descrição superficial que demos
até agora, porém, já revela até que ponto o domínio sexual e sua
incerteza permeiam o que está expresso no texto. Freud é explícito
quanto à supressão de material sexual que é pessoalmente revelador, mas
ele também permite que a sexualidade das imagens do sonho fale
claramente em sua interpretação, assim preservando a energia e a
perturbação do sonho. Freud recorda ``outras lembranças de exames
médicos e de pequenos segredos revelados em seu decorrer --- para o
desagrado de todos'' (Freud, 1900/2019, p. 165); ele pensa em uma
``amiga íntima'' de Irma e relembra que ``agora me lembro também de ter
cogitado várias vezes a possibilidade de essa mulher recorrer aos meus
serviços para livrá"-las de seus sintomas'', mas ``ela é de natureza
muito reservada'' (ibid.); a amiga de Irma ``seria mais inteligente'' do
que ela, ``ou seja, cederia mais rapidamente. \emph{A boca se abre
facilmente,} ela revelaria mais do que Irma'' (p. 167). Explorando o
elemento do sonho em que uma área infectada do ombro de Irma é notada
``apesar do vestido'', Freud comenta: ``Francamente, não sinto
inclinação a me aprofundar nesse ponto''\footnote{No original, lemos:
  ``{[}\ldots{}{]} ich habe, offen gesagt, keine Neigung, mich hier
  tiefer einzulassen''. (Freud, 1900/ versão digital com variação de
  paginação conforme alterações no tamanho da letra e do visor), o que
  indica que o termo ``aqui'', ao invés de ``nesse ponto'' é mais
  indicado para sugerir a ambiguidade entre além do vestido ou do
  sintoma manifesto.} (p. 113). E, finalmente, há a referência de Freud
à teoria de Fliess sobre ``algumas associações muito marcantes entre os
ossos turbinados e os órgãos reprodutivos femininos'' (p. 170).
Independentemente de quaisquer outros elementos presentes na análise do
texto, Freud permite que ele se comunique em linguagem suavemente
codificada da sexualidade e da dominação, em grande parte a partir de
uma posição abertamente sexuada, como a autoridade ``penetradora'', mas
com ansiedade suficiente para dar algum espaço a sua própria
identificação feminina.

É com a questão da identificação que lida mais claramente a notável
leitura de Lacan (1955-6) do sonho de Irma e sua interpretação, e é isso
que o permite mostrar o quão radical o capítulo ``espécime'' é, quando
comparado com o que ele seria se restrito apenas à (não obstante,
radical) constatação da ubiquidade do desejo sexual. Nesse contexto,
vale a pena recordar o comentário de Anzieu que diz que o sonho da
injeção de Irma encapsulou a maneira como ``o sistema de identificações
que havia governado {[}Freud{]} até aquele momento ruiu'' (1975, p.
132)\footnote{Foi curioso notar que na versão brasileira de A
  auto"-análise de Freud a passagem citada por Stephen Frosh simplesmente
  foi suprimida. Ela deveria estar localizada na p. 38 do livro editado
  pelas Artes Médicas, 1989, mas simplesmente não está lá. Em razão
  disso, a paginação acima segue a edição inglesa originalmente citada
  por Frosh com trad. Nossa (N. do T.)}. Para Lacan, igualmente, o sonho
revela esse desmoronamento, mas não de forma a permitir que Freud
renasça livre das identificações anteriores e capaz de seguir seu
próprio rumo. Na verdade, o sonho revela a \emph{necessária}
desconstrução das identificações que constituem o ego. Sem dúvidas, ele
lida com o desprendimento de identidades, com fragmentação e ambiguidade
discursiva; mas ele lida com essas coisas para poder demonstrar sua
centralidade à condição humana.

Lacan analisa o sonho em suas duas partes, a primeira sendo o exame que
Freud faz em Irma e a segunda a discussão com seus colegas da área
médica --- seu ``congresso''. A visão de Lacan sobre o ego está exposta
sucintamente neste Seminário:

\begin{quote}
o eu é a soma das identificações do sujeito, com tudo o que possa
comportar de radical mente contingente. Se me permitirem pô"-lo em
imagens, o eu é como a superposição dos diferentes mantos tomados
emprestado àquilo que chamarei de bricabraque de sua loja de acessórios.
(Lacan, 1955-6/1985, p. 198).
\end{quote}

Essa é a familiar noção lacaniana da especiosidade do ego como entidade
unificada: o que aceitamos na vida cotidiana como o componente central
do eu se revela, sob análise, um bricabraque, partes e pedaços
posicionados juntos, ou um sobre o outro, mais ou menos por acaso,
cobrindo o vazio que há por baixo. Na leitura de Lacan, a primeira parte
do sonho de Irma, em que Freud encara as profundezas da boca de Irma,
traz o sonhador e o analista cara a cara com o horror que existe por
trás dessa superfície egóica.

Lacan afirma que ``o objeto está sempre mais ou menos estruturado como a
imagem do corpo do sujeito'' (1955-6/1985, p. 212), sugerindo que quando
Freud olha para as entranhas de Irma ele acaba vendo sua própria psique
lá dentro, ou, nas palavras de Lacan (1955-6/1985), ``o que podemos
chamar de revelação do real naquilo que tem de menos penetrável'' (p.
209). Lacan se expressa de maneira excepcionalmente lírica nessa parte,
com uma linguagem repleta de imagens de horror e dissolução. Ele
localiza o tema da morte no sonho por meio da referência às ``três
mulheres da primeira parte: Irma, a paciente ideal, e a esposa de Freud.
A própria análise de Freud sobre o tema das três mulheres em seu ensaio
sobre os ``três escrínios'' (Freud, 1913) é, conforme Lacan observa, ao
mesmo tempo claro e místico: ``O último termo é pura e simplesmente a
morte'' (Lacan, 1955-6/1985, p. 200). Refletindo sobre isso em diversas
partes de seu Seminário, Lacan produz o seguinte confronto com o que ele
nomeia (p. 154) ``a fundação das coisas, o outro lado da cabeça'' --- um
confronto majoritariamente reunido em uma enorme sentença. A primeira
parte do sonho:

\begin{quote}
A primeira vai dar no surgimento da imagem aterradora, angustiante,
nesta verdadeira cabeça de Medusa, na revelação deste algo de inominável
propriamente falando, o fundo desta garganta, cuja forma complexa,
insituável, faz dela tanto o objeto primitivo por excelência, o abismo
do órgão feminino, de onde sai toda vida, quanto o vórtice da boca, onde
tudo é tragado, como ainda a imagem da morte onde tudo vem"-se acabar, já
que em relação com a doença de sua filha, que poderia ter sido mortal, a
morte da doente que ele perdeu numa época contigua a da doença de sua
filha, ele a considerou como sendo não sei que retaliação do destino por
sua negligencia profissional - \emph{uma Mathilde por outra}, escreve
ele. (Lacan, 1955-6/1985, p. 208).
\end{quote}

A energia extraordinária e o tumultuoso empilhamento de orações
existente nessa passagem, seu modo de rodear temas como a perda e a
morte \emph{(``essa Matilde por aquela Matilde''}), sua imagética de
horror, que permeia o Seminário --- essa é uma leitura que ecoa do
tradicional, mas de modo algum trivial, elo culturalmente estabelecido
entre sexualidade feminina e morte. A morte vem de mãos dadas com o
sexo; porém, junto a isso, há o tema paralelo do retorno ao nada,
simbolizado pelo desejo do útero, ambos elementos que estão presentes na
analogia dos três escrínios e, de modo mais formal, na (posterior)
conceptualização de Freud da pulsão de morte (Freud, 1920). Ao ler a
interpretação de Lacan para o sonho de Freud, sentimos que o sonhador se
encontra cara a cara com seu próprio desejo, ao descartar seu egóico
si"-mesmo. Ao olhar dentro da boca do outro, para o abismo da
subjetividade, o que se revela é a impossibilidade da posição do sujeito
consciente e integrado. As identificações que mantêm esse ego no lugar
não se sustentam quando confrontadas com a erupção do real. ``Este sonho
nos ensina, portanto, o seguinte - o que está em jogo na função do sonho
se acha para além do \emph{ego,} aquilo que no sujeito é do sujeito e
não é do sujeito, isto é, o inconsciente'' (Lacan, 1955-6/1985, p. 203).

No momento desse encontro com o real, um segundo elenco de personagens
entra no sonho, agora um trio de homens (Dr. M, Otto, Leopoldo). Lacan
interpreta esse acontecimento da seguinte forma (p. 164):

\begin{quote}
As relações do sujeito mudam completamente. Ele passa a ser algo
totalmente diferente, não há mais Freud, não há mais ninguém que possa
dizer {[}\emph{eu}{]}. É o momento que denominei entrada do bufão, já
que é mais ou menos este o papel que desempenham os sujeitos para os
quais Freud apela. (Lacan, 1955-6/1985, p. 209).
\end{quote}

Mas esses sujeitos não são externos a Freud: estando em seu sonho, eles
são aspectos dele mesmo, eles são identificações a partir das quais seu
ego é constituído. E essas identificações estão de fato desmoronando,
decompondo"-se; o sonho revela o real estado das ocorrências psíquicas ---
o estado normal, não o patológico. Com a ansiedade que o sonho produz
pela revelação da fragilidade do eu, nós ``assistimos a esta
decomposição imaginária, que é apenas a revelação dos componentes
normais da percepção''. (Lacan, 1955-6/1985, p. 212) --- isso é, assim os
diferentes e desconexos elementos do supostamente integrado ego são
dramatizados. Freud está presente em todos esses elementos, em cada um
dos membros desse elenco de tolos que procura uma maneira de escapar da
responsabilidade pelo horror das entranhas de Irma; afinal, é seu sonho.
Para Lacan, não é o conteúdo do discurso assim produzido que importa,
mas a presença desses elementos imaginários do ego e o modo como eles
desaparecem para revelar atrás deles algo mais poderosos e insidioso, um
nome que, apesar de completamente inescrutável, coloca todo o padrão
simbólico no lugar.

\begin{quote}
A descrição de Freud de seu sonho culmina com sua visão da fórmula da
trimetilamina (\versal{N-CH3/CH3/CH3}) e um comentário sobre a seringa não estar
limpa. Lacan toma essa fórmula abordando"-a como inscrição de uma ``outra
voz'', algo que aparece escrito apenas no ar, sem um agente responsável
--- ao que ele se refere, novamente com alusões Bíblicas (dessa vez ao
livro de Daniel), como seu ``\emph{Mene, Mene, Tekel, Upharsin}'' (p.
201). Essa fórmula, de nenhuma significância naquilo que diz, é da maior
significância por seu ato de falar; é assim que ela revela ao sonhador o
que está no cerne do sonho. Esta é a exposição completa de Lacan em
relação a esse aspecto (p. 170):

No sonho da injeção de Irma, é quando o mundo do sonhador está
mergulhado no maior caos imaginário que o discurso entra em jogo, o
discurso como tal, independentemente de seu sentido, já que é um
discurso insensato. Parece, então, que o sujeito se descompõe e
desaparece. Há neste sonho o reconhecimento do caráter fundamentalmente
acéfalo do sujeito, passado um certo limite. Este ponto está designado
pelo \emph{\versal{AZ}} da fórmula da trimetilamina. É aí que, neste momento, se
acha o {[}\emph{eu}{]} do sujeito. E não e sem humorismo nem sem
hesitação, já que isto é quase um \emph{Witz}, que eu lhes propus ver aí
a derradeira palavra do sonho. No ponto em que a hidra perdeu as
cabeças, uma voz que não é senão \emph{a voz de} \emph{ninguém} faz
surgir a fórmula da trimetilamina, como a derradeira palavra daquilo de
que se trata, a palavra de tudo. E esta palavra não quer dizer nada,
senão que é uma palavra.
(Lacan, 1955-6/1985, p. 215-6)
\end{quote}

Se essa versão sobre o sonho de Irma foi produzida para estar alinhada
com a teoria já existente de Lacan, ou se a teoria se segue à análise, é
uma questão interessante, mas uma que não é essencial agora. Certamente,
a leitura de Lacan é diferente da de Freud, e não apenas porque Freud
suprimiu conteúdo que revelava demasiado sobre sua pessoa. Não obstante,
a reflexão de Lacan sobre a reflexão de Freud sobre o sonho de Irma que
Freud teve expressa um sonho em si próprio, uma visão do que significa
ser um sonhador e um criador. ``Não sou mais nada'', é o que Lacan
entende que Freud está dizendo: ``É o meu inconsciente, é esta fala que
fala em mim, para além de mim'' (p. 217). Com a dissolução do ego, há
uma relação simétrica entre as três mulheres da primeira parte do sonho
e os três homens da segunda parte. Freud, o sonhador, está fora de todos
eles, mas pela própria natureza das coisas ele incorpora todos eles ---
eles são todos aspectos de sua própria imaginação, criaturas de seu
sonho. Tipicamente, ele olha dentro da garganta da mulher para encontrar
horror e dissolução; ele se volta para as baboseiras dos homens para
receber consolação e conivência pela renúncia da responsabilidade pela
destruição que ele ocasionou. Contudo, conforme o ego dissolve e as
identificações a partir das quais ele é construído são reveladas como
nada mais do que lealdades superficiais, uma fórmula mais verdadeira se
revela, algo que é sexual mas não é em si sexuado. O sonho de Freud
abarca morte e alteridade, tornando"-as disponíveis à consciência do
sonhador e analista. Elas também estão disponíveis aos sonhadores e
analistas que se seguem e que podem ver o inconsciente de Freud
trabalhando. Assim, o sonho demonstra a arte de Freud e um pouco da
verdade de sua teoria, porém não por meio de seu domínio, mas através
das quebras que há nele, causadas pela atividade de seu inconsciente.

Nessa leitura, a diferença sexual se torna menos fixa do que nunca. À
primeira vista, o sonho da injeção de Irma é um relato
paradigmaticamente masculino de penetração e dominação. Repelido pela
intransigência da mulher, o homem se força sobre ela, causando nada além
de sofrimento e, ao mesmo tempo, fantasiando sobre a mulher perfeita que
está um pouco além do alcance. Então, um elenco de portentosos tolos,
outros homens, é convocado para participar de um jogo de evasão da
culpa, perdoando o sonhador pela responsabilidade de causar a aflição da
mulher. Está tudo bem; já estava quebrado antes; é culpa de outra
pessoa. O que poderia ser mais sexuado do que isso? Todavia, ao ler esse
sonho e sua interpretação mais a fundo, encontramos uma série de
identificações operando que, convencionalmente, seriam associadas tanto
à feminilidade quanto à masculinidade. Freud pode assumir ambas as
posições de sujeito, ele é o ponto nodal (o ``N''), com escrínios
femininos e masculinos se ramificando a partir dele; entretanto, ele não
é expresso completamente por nenhum deles. A bissexualidade disso também
é bastante convencional, apesar de ser um avanço em relação à
estereotipagem do relato anterior. Agora, Freud, apesar de ser o homem
sonhador, revela a capacidade de possuir atributos tanto masculinos
quanto femininos; ele consegue olhar e enxergar pelo prisma dos outros
até entender o que cada um tem a dizer.

A leitura de Lacan, contudo, vai além dessa ``bissexualidade''. Ainda há
uma dinâmica representativa em torno da diferença sexual: o inconsciente
fala de horror e morte através da imersão na mulher, o outro lado da
pele. Essa perda do eu e da identidade, essa castração, é um elemento
comum à imagética lacaniana (e também a alguns trabalhos posteriores a
ele, particularmente os de Kristeva, 1980) e está sempre expressa por
meio da ``alteridade'' feminina, aquilo que está excluído do discurso do
simbólico. Porém, na análise do sonho da injeção de Irma, Lacan mostra o
quanto um rico entendimento sobre desejo e perda pode ser construído a
partir dessa aparente e quem sabe estéril --- ou até misógina ---
tradicional diferenciação sexual, e como isso pode romper qualquer
posição fixa sobre a própria diferença. Com a dissolução das
identificações que constituem o ego, o sujeito termina por referir"-se a
algo diferente do que ele experimenta como sendo ele mesmo, algo que
escreve a fórmula para ele. Esse algo não é em si sexuado; ele é muito
impessoal para tal. Falando pelo sujeito, ele desorganiza as posições do
sujeito, misturando o que é um com o que é outro, sujeito e objeto. Isso
não dispensa a diferença sexual, mas a interroga de maneira impessoal,
sem uma lealdade formal. Há esse ``masculino'' e esse ``feminino'',
porém o sujeito é outro que não essas categorias; inserido nelas, é
claro, mas, mesmo assim, fora do que elas têm a dizer. As categorias
sociais de gênero são pertinentes às identificações do ego; porém, atrás
do ego há outra coisa, visível quando se consegue fitá"-la sem piscar ou
desviar o olhar. Nessa leitura, a autoanálise de Freud é genuinamente
heroica: a mirada analítica que desafia todos nós a vermos mais do que
costumamos enxergar.

\section{Ruptura e sonho}

\emph{A interpretação dos sonhos} é um texto marcado pela presença de
Freud em diversos níveis e em diferentes posições: como autor e sujeito,
como analista e sujeito, como sonhador e sujeito, e como autor, analista
e sonhador e objeto. É um texto dedicado ao domínio, por meio do qual
Freud demonstra um controle virtuoso dos princípios de interpretação de
sonhos, assim banindo os segredos da noite, bem como as dúvidas de seus
adversários. Freud, como autoridade, está competindo com outros: o corpo
de conhecimento está disponível para ser desvelado, é um aspecto da
natureza, algo que evadiu os predecessores de Freud por causa de sua
estranheza e obscuridade --- seu relacionamento com o ``continente
negro'' --- mas algo que é, não obstante, uma entidade material,
disponível às atividades da razão e da análise. De fato, como a
psicanálise está, em geral, preocupada com a colonização do irracional e
sua subjugação pelas forças da razão e da ciência, nesse quesito
específico ela também está dedicada a reduzir a área de recalcitrância
na natureza, tornando o mundo mais previsível e seguro.

A tradicional diferenciação sexual dessa descrição é bem visível. O
obscuro é o feminino --- natureza, a noite, o continente negro, o sonho.
A masculinidade é identificada com a racionalidade, com o domínio sobre
essa obscuridade, com a luz na escuridão, com o triunfo da ciência sobre
a natureza. Ela é caracterizada pela exaltação própria, uma economia da
competição: Freud explicitamente constrói seu livro"-sonho sobre a
rivalidade com outros, que, por sua vez, é derivada de sua necessidade
de superar o menosprezo de seu pai. A natureza, portanto, está sendo
atacada para que se atinja uma posição privilegiada no mundo dos homens.
Porém, a natureza também está sendo atacada por si só, como objeto
ambivalente de desejo. O ``impulso epistemológico'' do menino é
convencionalmente atribuído pelos psicanalistas à curiosidade em relação
ao corpo da mãe: um desejo por conhecê"-lo e entender seus segredos, mas
também de agir sobre ele, de obter vingança nele pela terrível
dependência característica da primeira infância, assim como pelo
abandono experimentado conforme a mãe o afasta. As sutilezas e a
imprevisibilidade do corpo maternal, a reminiscência de seu cheiro e
toque, oferecem uma base para o desejo de conhecer e conquistar a
natureza, ligando"-se a todas as dicotomias mente/corpo, razão/natureza,
sanidade/insanidade que são tão familiares a todas as partes do mundo
social.

Nesse esquema de coisas, o sonho é o objeto da investigação
psicanalítica, portanto, feminino em seu tom. Misterioso e obscuro, uma
criatura da noite, o sonho está estirado sobre a mesa, dissecado e
inspecionado, penetrado com o zelo racional do investigador freudiano.
Compreender sonhos fornece poder no mundo dos homens (José); contudo,
também torna a noite menos assustadora --- é um modo de articular e lidar
com o horror interior, repudiado mas profundamente sentido. Interpretar
sonhos é, então, de fato um ato de domínio, mas ele confessa sua própria
incerteza dinâmica. A articulação que Freud faz da análise de sonhos em
\emph{A interpretação dos sonhos} está atravessada de ambivalência e da
voz do desejo inconsciente. Essa voz fala em seu lugar, mostrando como
suas identificações pessoais deslizam por entre aquelas que podem ser
caracterizadas como masculinas e aquelas que podem ser femininas, sempre
procurando. O ato de descrever os sonhos e apresentar as respectivas
análises ao mundo, um ato triunfante de domínio, é também um em que é
concedida a palavra ao irracional, e a personalidade do próprio Freud,
permeada de conflitos, é oferecida para análise. Sujeito e objeto
alteram"-se novamente: Freud na mesa, sendo examinado, o objeto feminino
do desejo masculino.

Tudo isso também se aplica à psicanálise em geral. Ela comumente fala
através de uma voz repleta de autoridade e certeza; todavia, ao fim e ao
cabo, ela nunca está certa de sua posição. Trabalhando com os mistérios
da noite, ela irá sempre patinar; dando voz à irracionalidade, ela nunca
será completamente racional. Se a psicanálise é um discurso expressivo,
o que ela expressa irá sempre incluir o inconsciente --- tanto sujeito
quanto objeto da empreitada psicanalítica, tanto ``masculino'' quanto
``feminino'' da divisão sexual. Longe da diferença sexual ser algo fixo
e absoluto, a fundação diante da qual a análise cessa em espanto, ela
termina por parecer"-se mais com uma névoa --- e, talvez, possamos
simplesmente atravessá"-la.

\section{Referências}

Anzieu, D. (1975) \emph{Freud's Self"-Analysis.} London: Hogarth Press,
1986.

Anzieu, D. (1975) A auto"-análise de Freud. Porto Alegre: Artes Médicas,
1989.

Diller, J. (1991) \emph{Freud's Jewish Identity: A Case Study in the
Impact of Ethnicity}. London: Associated University Presses.

Freud, S. (1900) The Interpretation of Dreams. \emph{The Standard
Edition of the Complete Psychological Works of Sigmund Freud, Volume \versal{IV}
(1900): The Interpretation of Dreams (First Part)}, ix-627. London:
Hogarth Press.

Freud, S. (1900) \emph{A interpretação dos sonhos}. São Paulo: Companhia
das Letras, 2019.

Freud, S. (1913) The Theme of the Three Caskets. \emph{The Standard
Edition of the Complete Psychological Works of Sigmund Freud, Volume \versal{XII}
(1911-1913): The Case of Schreber, Papers on Technique and Other Works},
289-302. London: Hogarth Press.

Freud, S. (1920) Beyond the Pleasure Principle. \emph{The Standard
Edition of the Complete Psychological Works of Sigmund Freud, Volume
\versal{XVIII} (1920-1922): Beyond the Pleasure Principle, Group Psychology and
Other Works}, 1-64. London: Hogarth Press.

Freud, S. (1939) Moses and Monotheism. \emph{The Standard Edition of the
Complete Psychological Works of Sigmund Freud, Volume \versal{XXIII} (1937-1939):
Moses and Monotheism, An Outline of Psycho"-Analysis and Other Works},
1-138. London: Hogarth Press.

Frosh, S. (2005) \emph{Hate and the Jewish Science: Antisemitism, Nazism
and Psychoanalysis}. London: Palgrave.

Jones, E. (1955) \emph{Sigmund Freud: Life and Work}. London: Hogarth
Press.

Kristeva, J. (1980) \emph{Powers of Horror}. New York: Columbia
University Press.

Lacan, J. (1955-6) \emph{The Seminars of Jacques Lacan, Book 2}.
Cambridge: Cambridge University Press.

Lacan, J. (1955-6) O Seminário, livro 2. Rio de Janeiro: Zahar, 1985.

\chapter*{Por uma leitura reparadora do destino das mulheres criado
pela psicanálise: de volta à Sociedade das Quartas"-Feiras}
\addcontentsline{toc}{chapter}{Por uma leitura reparadora do destino das mulheres criado
pela psicanálise: de volta à Sociedade das Quartas"-Feiras,\\ \footnotesize\emph{por Aline de Souza Martins e Lívia Santiago Moreira}}
\hedramarkboth{Por uma leitura reparadora do destino das mulheres\ldots{}}{}


\epigraph{Tudo o que os homens escreveram sobre as mulheres deve ser suspeito,
pois eles são, a um tempo, juiz e parte.}{Simone de Beauvoir citando Poulain de la Barre}

\epigraph{Tudo nesse mundo é sobre sexo exceto sexo. Sexo é sobre poder.}{Oscar Wilde}

Em seu nascimento, a psicanálise foi criada para compreender e tratar as
patologias cujas origens eram localizadas em questões psíquicas, e não
na fisiologia corporal. Nos registros das reuniões de Freud e seus
convidados na Sociedade das Quartas"-feiras (1906-1910), é possível
recolher os indícios do ambiente discursivo patriarcal que rodeava os
primórdios da psicanálise. Partindo da premissa de que não há como
separar o psíquico do social, este trabalho surgiu do interesse em
entender melhor o ambiente no qual Freud discutia a psicanálise de
maneira menos formalizada. Como na lógica do \emph{a posteriori} que
ressignifica as cenas e os sentidos do que foi anteriormente vivido sem
conflito, buscaremos problematizar o que figurava como questão nas
discussões desse grupo. A seguir, veremos como o complexo de Édipo pode
ser visto como a consequência da ideologia patriarcal na psicanálise,
tanto naquilo que fundamenta a construção de sua teoria quanto na
introjeção desta dinâmica de poder e hierarquia representados pela
família patriarcal na psique do sujeito.

A teoria de Eve K. Sedgwick (2003) mostra"-se útil para pensarmos a
releitura do patriarcado na psicanálise ao levar em conta a dimensão
performativa do conhecimento. Isto é, como as categorias criadas, a
maneira de expor e organizar o conhecimento tem efeitos materiais sobre
o que tenta descrever, assim, precisamos criar uma abertura para
deslocar a questão que gira em torno de sua veracidade para questões que
investigam o que faz o conhecimento e sua exposição, pensando nos
efeitos de receber, novamente, aquilo que já se sabe. A busca da Verdade
pode ser vista como um recalque de outras formas de saber, escamoteando
seu próprio ponto de partida localizado. O que nos leva a perguntar:
quais disputas de poder estão implicadas na construção dessa forma de
conhecimento e como lidar com a política inerente à produção de qualquer
saber sobre os corpos?

As arqueologias do presente, que buscam desmistificar e subverter
através da detecção dos padrões escondidos de violência e da sua
exposição, têm"-se tornado praticamente a norma dos estudos sobre a
cultura. O problema é que essa espécie de ``hermenêutica paranoide''
acaba por converter"-se em uma forma consensual da maneira correta de
historicizar. A ausência de questionamento sobre o que constitui uma
adequada forma narrativa e de explicação empobrece as perspectivas
críticas, diminuindo a nossa capacidade e habilidade de resposta às
mudanças políticas e ambientais.

Ainda segundo Sedgwick (2003), teorias diferentes e concorrentes
constituem a ecologia mental da maioria das pessoas. Ela nos alerta
sobre a forma de leitura ``paranoica'', que pode tomar lugar na crítica
feminista e queer, promovendo um scanneamento vigilante a partir do
pressuposto de que, \emph{a priori}, nenhuma área do pensamento
psicanalítico poderia ser declarado imune à influência da reificação do
gênero. A perspectiva de leitura paranóica acredita que é mais perigoso
que essas reificações sejam inesperadas do que muitas vezes não
contestadas. A autora nos diz que a fé na exposição desmistificadora da
paranóia é uma presunção cruel de que

\begin{quote}
a única coisa que falta para a revolução global --- como a explosão dos
papéis sociais --- é que as (outras) pessoas tenham os efeitos dolorosos
de sua opressão, pobreza e ilusão exacerbados suficientemente de modo a
tornar a dor consciente e intolerável, como se as situações intoleráveis
não pudessem ser conhecidas de outra forma, ou pudessem gerar excelentes
soluções. (Sedgwick, 2003, tradução nossa, p. 144).
\end{quote}

Assim, a crítica não deveria expor somente o que tem sido escondido, mas
a luta entre diferentes formas de visibilidade. Ela oferece uma leitura
que problematiza o que parece ser uma teoria ``forte'', nessa lógica
paranóide em que a verdade do conhecimento esclarecido repousa em seu
pessimismo preditivo e na capacidade de antecipar os ataques. Mas,
quando nossa cultura dá notícias de desastres e catástrofes naturais e
humanas a todo tempo, a conclusão a que chegamos é de que não se pode
ser nunca suficientemente paranóico. Dizer que as coisas estão ruins e
irão piorar é uma afirmação imune à refutação. A conseqüência dessa
posição, diz Sedgwick, é que as tentativas de se fazer uma outra
estratégia de resistência e oposição têm sido consideradas nulas, já que
motivos reparativos são inadmissíveis na teoria paranóide por serem
considerados meramente reformistas e estéticos.

Não é novidade que o patriarcado é inerente à estrutura de pensamento
daqueles que forjaram a psicanálise. É preciso lembrar que muitas das
denúncias, testemunhos e desmistificações têm uma força efetiva de
transformação, contudo, se muitas delas são igualmente verdadeiras e
convincentes mas não produzem efeito, temos que admitir que ``a eficácia
e a direcionalidade de tais atos (de denúncia ou revelação que conseguem
ter uma força efetiva) residem em outro lugar que aquele de sua relação
com o conhecimento \emph{per se}'' (2003, p. 141). Para evitar que os
deslocamentos experimentados não acabem gerando apenas a interpretação
de um eterno retorno, a crítica precisa levar em conta quais são os
afetos mobilizados pelo tipo de conhecimento que é trazido e a forma com
que chegam até nós.

Sedgwick propõe que talvez seja possível levar a sério a noção de que as
teorias criadas no cotidiano afetam qualitativamente as formas de
experienciar e conhecer, e que há muito a se pensar sobre essas
práticas. A ideia é que essas outras teorias não universalistas e
baseadas em afetos não negativos, em sua variedade dinâmica e suas
diferentes formas de interagir com as contingências históricas, produzam
nas suas interações uma nova ecologia dos saberes. Seria necessário
abrir mão do prestígio paranoico que se acredita mais astuto e
inteligente que os demais, em uma onipotência de pensamento que tenta
controlar tudo, inclusive a fragilidade do próprio pensamento,
extirpando a oportunidade do erro em sua lógica. Essa proteção leva
consigo a possibilidade também de boas surpresas, não só o terror.

Optamos, então, por uma leitura reparadora das atas, que não é menos
realista, menos fantasmática, delirante ou menos ligada ao projeto de
sobrevivência que a leitura paranóica. Trata"-se, para Sedgwick, de uma
leitura que busca outros tipos de afetos, riscos e ambições, ``que nos
mostra como tantas pessoas e comunidades conseguiram extrair sustentação
dos objetos da cultura, mesmo de uma cultura cujo desejo autorizado tem
sido muitas vezes de não sustentá"-los''. (Sedgwick, 2003,p. 150). De
acordo com a autora, apesar da ansiedade e do medo estarem sempre
presentes, a leitura de uma posição reparadora precisa levar em conta a
necessidade da surpresa, nenhum horror deveria ser visto ou lido sem
espanto, por mais que se tente de todas as formas antecipá"-lo na
tentativa de preveni"-lo. Essa leitura requer um espaço no qual o futuro
possa ser pensado como diferente do presente, e é preciso tornar
possível ``entreter a possibilidade profundamente dolorosa, aliviadora e
eticamente crucial de que o passado poderia ter acontecido de forma
diferente'' (2003, p. 145).

Depois de apresentarmos a ``Sociedade das quartas"-feiras'' e partes das
discussões do grupo sobre as mulheres, sua sexualidade e patologias,
iremos trazer para o debate os argumentos de Carole Pateman (1988),
Juliet Mitchell (1974) e Jéssica Benjamin (1980, 1988), que se dedicaram
ao estudo do patriarcado para comentarem a teoria freudiana,
especialmente através do conceito de complexo de Édipo. Como teórica da
política, Carole Pateman (1988) denuncia o silenciamento sobre o
patriarcado na atualidade como mecanismo próprio da repressão inerente
ao mesmo, e aponta que o contrato sexual também teria sido recalcado sob
a teoria do contrato social, encontrando ressonâncias com a teoria
freudiana da origem da sociedade na horda primeva. Voltando para Freud,
Juliet Mitchell (1974) foi uma das primeiras psicanalistas a
evidenciarem o papel desta estrutura de poder na teoria da diferenciação
sexual freudiana, tornando visível como a clínica psicanalítica trataria
também as consequências do patriarcado na constituição dos sujeitos.
Para incorporar esse debate à psicanálise, Jéssica Benjamin (1980, 1988)
inclui o desejo da mulher/mãe como parte dos elementos a serem
analisados no complexo de Édipo e aposta no reconhecimento mútuo como
ética para lidar com o outro como sujeito, e não como objeto.

Trazer mulheres para discutirem o saber que os homens construíram sobre
elas é apostar no debate de ideias, sem apontar verdades ou queimar
bruxas. Retomar as discussões do grupo de Freud não tem o objetivo único
de desvelar o patriarcado, que já sabemos fazer parte da socialização
dos homens e mulheres através dos séculos, mas também trazê"-lo para o
debate como mais um dos componentes políticos que fazem parte da
constituição dos sujeitos, no qual as diferentes formas de visibilidade
possam ser respeitadas como uma luta constante, dinâmica e sem fim.

\section{Atas da Sociedade Psicanalítica de Viena: mulheres fazendo os
biscoitos enquanto os homens discutem coisas sérias }

A Sociedade Psicanalítica de Viena, conhecida como Sociedade das
Quartas"-feiras, foi a primeira organização criada para fomentar os
estudos em psicanálise. Segundo Roudinesco (2016), neste momento Freud
desejava inscrever seu ensino na herança das grandes escolas filosóficas
da Grécia antiga, tornando"-se o Sócrates dos tempos modernos. Para dar
cabo do seu projeto a psicanálise não poderia ficar restrita ao ensino
universitário, ele ``precisava fundar um movimento político'' (p.135).
Assim, esse pequeno grupo iniciava com a participação de Wilhelm Stekel,
Max Kahane, Rudolf Reitler, Alfred Adler e depois outros psiquiatras e
amigos foram incluídos com consentimento de todos. O processo de
institucionalização contou com a contratação de Otto Rank como escrivão
oficial e deu origem a 250 registros de reuniões, compiladas em quatro
volumes chamados ``Atas da sociedade Psicanalítica de Viena'' entre
1906-1918. Cada encontro consistia em um ritual: colocavam em uma urna o
nome dos futuros oradores e escutavam a comunicação dos escolhidos, esta
poderia ser uma apresentação de pesquisas pessoais, casos ou pesquisas
importantes da comunidade científica, sendo que muitos casos eram
relatos de seus próprios traumas, experiências e sonhos, ou de suas
esposas, amantes e irmãs. Depois de uma breve pausa para tomar café
preto e comer biscoitos amanteigados se colocavam imediatamente em
fervorosas discussões enquanto fumavam charutos. O grupo era finalizado
com um breve resumo e apontamentos de Freud, que tinha sempre a última
palavra.

Com o crescimento do grupo e as discussões sobre propriedade
intelectual, Freud resolveu dissolver o círculo em 1907 e recriá"-lo com
o nome de Sociedade Psicanalítica de Viena, na qual a inclusão de novos
membros dependia de voto secreto, a regra que todos eram obrigados a
tomar a palavra foi abolida e instaurou"-se uma regulamentação que se
fundava em uma hierarquia. O primeiro presidente desta nova instituição
foi Alfred Adler e seus membros mais proeminentes foram: Sigmund Freud,
Max Eitingon, Wilhelm Reich, Otto Rank, Karl Abraham, Carl Jung, Sándor
Ferenczi, Ernest Jones, Isidor Isaak Sadger, Hanns Sachs e o pastor
Oskar Pfister (Roudinesco, 2016). Assim se constituiu entre 1907 e 1910
o primeiro núcleo de discípulos de Freud, homens que contribuíram para a
internacionalização do movimento para a América e Europa. Eles
praticavam a psicanálise em seus consultórios com conhecidos e
familiares --- após terem feito análise com Freud, Ferenczi ou Federn.
Segundo Roudinesco (2016), constituíram uma família expandida cujas
contribuições foram imprescindíveis para o desenvolvimento do pensamento
de Freud e a criação da psicanálise.

A partir de 1910 algumas mulheres entraram nesse seleto grupo, as
pioneiras foram Hermine von Hug"-Hellmuth, Tatiana Rosenthal, Eugénie
Sokolnicka, Margarethe Hilferding, Lou Andreas Salomé e, mais tarde,
Sabina Spielrein, que foi a primeira mulher do movimento psicanalítico a
seguir de fato uma carreira. Várias das mulheres que seguiram a
psicanálise começaram como pacientes e se tornaram discípulas de Freud,
uma prática recorrente também entre os homens, que costumavam analisar
os sonhos e sintomas uns dos outros em um excesso de interpretações
selvagens.

Embora sejam apenas registros não aprofundados característicos do gênero
ata, essas citações dão mostra das discussões que estavam em vigor no
grupo, as preconcepções típicas do momento histórico e a moralidade na
qual se assentavam as primeiras hipóteses do movimento psicanalítico.

Em 17 de Outubro de 1906, Otto Rank apresenta a segunda parte da sua
conferência, ``A relação incestuosa entre irmãos''. Nela aparece o
debate sobre o poder na relação pai e filho. Enquanto Paul Federn, que
se identificava como o discípulo e oficial subalterno do movimento
psicanalítico, defendia que o motivo da disputa seria o poder, Rank
defende que ``o motivo da luta pelo poder apenas encoberta o motivo
sexual mais profundo'' (2015, p.63). Adler apresenta um receio sobre a
preservação da família contra as ``tendências incestuosas'' através da
pedagogia, e supõe que o homem político teria sempre motivações
pessoais, ``os políticos que defendem em seu programa social a dissolução
da família têm uma vaga intuição da existência de tendências
incestuosas'' (2015, p.65). Poderíamos ler um certo paralelismo na
reunião seguinte, de 7 de Novembro de 1906, quando Freud emite algumas
palavras sobre o conflito entre superiores e subalternos, sempre
tendendo para explicações ligadas à sexualidade (Cf. 2015, p.149).

Em 30 de Janeiro de 1907 é feita a discussão das questões sobre
etiologia e terapia das neuroses por Max Eitingon, sionista convicto,
segundo filho de uma família de judeus ortodoxos que realizou seu
estágio de psiquiatria em Zurique, onde entrou em contato com a
psicanálise. Nesse encontro, Eitingon questiona ``quais outros fatores
devem estar presentes, além dos mecanismos conhecidos por nós, para que
se constitua uma neurose? (em que consiste à predisposição à histeria?)
Deve"-se ter em conta fatores sociais?'' (2015, p.164). Reitler afirma
que ``acontecimentos banais da vida social nunca podem desencadear um
sintoma se não estiverem intimamente associados a aspectos sexuais''
(2015, p.167). Eitingon parece se irritar e ``declara que sua questão
acerca dos fatores sociais fica enfraquecida se tudo é atribuído à
sexualidade'' (2015, p.168). Já Stekel diz que ``o mais importante
parecem ser as dificuldades impostas pelo meio (e não aquelas de
natureza orgânica). O fator orgânico não estaria relacionado com a
neurose. Os momentos sociais têm um papel importante, como demonstra a
nervosidade dos judeus russos'' (p.171).

No dia 10 de Abril de 1907, o conferencista Wittels se põe a falar sobre
Tatjana Leontiev (revolucionária russa) e são abordados temas como
gênero e ideologia. Fritz Wittels é sobrinho de Sadger, neurologista
responsável pela cunhagem dos termos ``narcisismo'' e
``sadomasoquismo'', o qual acreditava que a análise teria o objetivo e a
capacidade de alterar a imagem interna dos objetos sexuais dos
homossexuais, mudando, assim, sua orientação sexual desviante ---
segundo Roudinesco (2016), ambos eram misóginos e muito freudianos.
Nesta ata, como o que já aconteceu em outros momentos, existe um claro
preconceito de gênero e a psicanálise é usada para justificar a
patologização das histéricas e do ``desejo revolucionário das mulheres
que as impulsiona a matarem homens em prol das causas revolucionárias''
(p.265). Wittels, no final da sua apresentação sobre o caráter histérico
da revolucionária russa, ``manifesta, por fim, sua antipatia pessoal por
Leonitiev e por todas as histéricas'' (2015, p.256). Adler critica a
conferência e relata que ``também não se pode concordar com a tese de
Wittels de que a ideologia possa ser totalmente dissociada, num
acontecimento concreto, daquilo que chamamos vida emocional ou ambiente.
A ideologia não pode explicar nada, mas podemos explicar a ideologia''
(p.257). Freud afirma que

\begin{quote}
é o erotismo reprimido que põe a arma na mão dessas mulheres. Toda a
ação que envolve o ódio tem sua origem em tendências eróticas. É
sobretudo o amor desdenhado que torna possível esta transformação. Ele
se apodera da componente sádica (Adler). Na maioria das vezes gira em
torno do pai, e a afirmação de Bach de que as mulheres em questão
normalmente são filhas de generais confirma isso (2015, p.258-259)
\end{quote}

O que aparece aqui e nos dá indícios do que permanece na lógica da
psicanálise é como a reivindicação política das mulheres pode ser
deslegitimada e desautorizada pelas explicações que tentam demonstrar a
sexualidade como origem do conflito. Se num primeiro momento a teoria da
sexualidade infantil é revolucionária em seus pressupostos, num segundo
momento, ela pode ser o elemento que recalca o político do sexual.

Em 15 de Maio de 1907 o conferencista Wittels apresenta um texto sobre
``As mulheres médicas''. Federn faz um contraponto afirmando que

\begin{quote}
ele (conferencista) comete um grande erro ao afirmar que a sexualidade é
a única pulsão do ser humano. A questão do trabalho e da busca de
satisfação na vida também merecem atenção quando analisamos o estudo
realizado por mulheres. A necessidade de trabalhar não se funda somente
na condição social, mas é um dos instintos que apareceram tardiamente no
desenvolvimento do homem. Como contraponto à visão unilateral de
Wittels, Federn menciona a perversidade lasciva de muitos homens e a
exploração sexual a que muitas mulheres são submetidas por médicos
homens (2015, p. 300)
\end{quote}

Na ata de 05 e 12 de Fevereiro de 1908, Eduard Hitschmann, médico da
família de Freud e mais tarde diretor do Ambulatorium (a clínica
psicanalítica criada em 1922) discute a anestesia sexual e a relação da
mulher com o trabalho. Na discussão Adler defende que ``as mulheres
estão ainda começando a ter uma vida independente, longe da família, e a
desenvolver seu caráter; essas aspirações se opõem em certa medida a que
se entreguem ao coito. Isso talvez explique a frequência particular da
anestesia em nossos tempos'' (2015, p.445). Ao que Freud surpreende com
uma posição bastante interessante

\begin{quote}
a anestesia da mulher deve ser considerada essencialmente um produto da
cultura (caso contrário, ela aparece apenas isoladamente), uma
consequência da educação: ela se deve ao homem (objetos sexuais
inadequados) ou diretamente aos efeitos da educação. Um grande número de
mulheres frígidas foram meninas demasiadas bem"-educadas. O recalque
sexual não apenas atingiu seus fins, como também foi muito além de suas
intenções. (\ldots{}) Mulheres que tem muitos amantes são na verdade mulheres
acometidas de anestesia em busca de um homem que as satisfaça. Não se
deve subestimar as influências acidentais; por exemplo, não conhecemos
nenhum caso de anestesia sexual em mulheres que exercem o poder (2015,
pp.449-450)
\end{quote}

Nesta ata a discussão continua problematizando se ``a terapia deve ser em
primeiro lugar social'' (2015, p.451) e

\begin{quote}
destaca especialmente a importância da etiologia social, mencionada
brevemente por Adler, a anestesia é uma doença da burguesia, dos
círculos em que a escolha do cônjuge não é determinada em primeiro lugar
pela seleção natural, mas por fatores sociais e econômicos, em que as
meninas são educadas diretamente para o recalque e em que a virgindade e
a monogamia são, ao menos em teoria, exigências estritas (2015, p. 451).
\end{quote}

Aqui vemos os riscos da interpretação que entende a insatisfação
feminina somente como de ordem sexual, e não social. A tradução da
anestesia sexual como resultado do pudor que acompanha a ``boa
educação'' e a seguinte relação criada por Freud entre sexo e poder
demonstra como o que se apresenta como sintoma histérico é um índice da
reivindicação feminina de poder ocupar um lugar de sujeito e não somente
aquele de objeto de desejo do outro - única via que aparece como
disponível para o reconhecimento da mulher.

Wittels, em 11 de março de 1908 fala da ``posição natural da mulher''.
Nesta discussão Adler afirma que ``enquanto todos supõem que a repartição
atual dos papéis dos homens e das mulheres é imutável, os socialistas
propõem que o quadro da família já se encontra hoje abalado e se abalará
cada vez mais. As mulheres não tolerarão que a maternidade as impeçam de
exercer uma profissão''. (2015, p. 506). E continua, ``os estudos de Marx
descrevem como, sobre o domínio da propriedade, tudo se converte em
domínio. A mulher se torna uma propriedade e esta é a origem de seu
destino. Por essa razão, deve"-se começar por abolir a mulher como
propriedade'' (2015, p.506). Por mais inconcebível que possa nos parecer
hoje, essa posição gerou muita discordância no grupo, principalmente por
parte de Wittels.

A discussão do dia 23 de fevereiro de 1910 tem por tema o
hermafroditismo psíquico --- que será retomado na ata do dia 23 de
novembro de 1910 ---, ali a preocupação está em separar o que há na
neurose de feminino e de masculino. Por mais que a tradição
psicanalítica tenha tentado deslocar para o campo simbólico essa divisão
que diria respeito ao que é da ordem da passividade e da atividade que
se encontram misturadas em cada sujeito, esse binarismo que reaparece
descolado do corpo repete a estrutura valorativa que condiciona o que
entendemos como um: homem"-masculino"-ativo e o outro:
mulher"-feminino"-passivo. Essa separação é uma derivação direta da ideia
de inferioridade orgânica. Nos registros lemos: ``Parece que o que ele
(Adler) entende por ``feminino'' é quase unicamente o que é mau, e
certamente, tudo o que é inferior, e que se tente, assim, defender"-se de
tudo isto que é patológico'' (1910,p.415). O orador da ocasião fala
sobre como a disposição à neurose é criada pelo sentimento de
inferioridade, referindo"-se ao momento da infância onde a criança tem
medo de ser tratada como alguém negligenciável, ``problema que será mais
tarde expresso na frase: ``não sou totalmente um homem''. Em sua teoria
sobre a inferioridade orgânica, ``as realizações que são reconhecidas
como sendo inferiores são ressentidas como não viris, então, como
femininas. O esforço que um indivíduo faz para eliminar esses traços
``femininos'' é sentido como masculino''. (1910, p. 415) E ainda:

\begin{quote}
O infante feminino vê também seu ideal no homem, com uma restrição,
contudo: ele deve tornar"-se um homem pelos meios femininos. Não dissemos
provavelmente nada novo sublinhando que a tarefa da análise consiste em
trazer à luz, expor a mulher no homem neurótico e a mostrar que todos os
seus traços patológicos são atravessados pela corrente dessas tendências
femininas'' (1910, p. 415).
\end{quote}

Nada poderia ser mais explícito para expor a tese de Simone de Beauvoir
de que a mulher é o segundo sexo, o outro. Através de uma amostragem
mínima das atas é possível perceber não só como o conteúdo expressa uma
resistência à crença na superioridade masculina de origem anatômica, mas
o quanto também sua forma é problemática, uma vez que o ponto de vista é
representada por um grupo de homens discorrendo sobre a sexualidade,
inferioridade, frigidez, motivação revolucionária e trabalho femininos.
Ali temos os registros do ambiente patriarcal no qual a psicanálise
estava inserido, e, embora o termo patriarcado não seja citado, este é
um conceito operador importante para descrever a forma de organização
social que está baseada na superioridade masculina --- que justificaria a
autoridade paterna e a subordinação das mulheres e dos filhos na
sucessão patrilinear. A seguir, propomos tensionar as relações que foram
construídas por aquele grupo e reativar esse conceito através das
críticas e contribuições trazidas pelo conhecimento que foi produzido
por mulheres.

\section{Destinos do Édipo}

O processo histórico no qual a psicanálise se insere poderia ser
apresentado através de uma linha que parte da guerra entre os homens e
os deuses que decidiam o seu destino, desloca"-se para a guerra entre os
homens que decidem os rumos da política e chega na guerra entre os
sexos, onde o destino seria dado pela anatomia: Freud parafraseia
Napoleão dizendo que a ``anatomia é o destino'' (Freud, 1924, p. 211),
substituindo justamente o significante do que o ex"-imperador francês
considerava essencial, a política, na frase célebre ``\emph{Le destin,
c'est la politique}''. Quando Freud afirma que o destino é a anatomia,
podemos ler na própria teoria o conflito que ela busca solucionar, por
exemplo, através da elaboração do conceito fundamental de pulsão que diz
respeito ao que está entre o somático e o psíquico. Tal interpretação
parece ser herdeira do desejo de neutralidade cientificista que
``desconhece'' os elementos inconscientes presentes na valoração que é
feita daquilo que é observável na ``natureza''. Como veremos, essa
espécie de recalque do político que está na forma como o conhecimento é
produzido pode ser considerado um sintoma do patriarcado na teoria
psicanalítica freudiana.

Simone de Beauvoir no indispensável ``Segundo Sexo'', escrito em 1949,
dedicou"-se a desmontar o apelo biologizante dos argumentos que tentavam
explicar a diferença entre um sexo e o ``Outro'' ao longo de toda a
primeira sessão intitulada ``Destino'':

\begin{quote}
Quando um indivíduo ou um grupo de indivíduos é mantido numa situação de
inferioridade, ele é de fato inferior; mas é sobre o alcance da palavra
ser que precisamos entender"-nos; a má fé consiste em dar"-lhe um valor
substancial quando tem o sentido dinâmico hegeliano: ser é ter"-se
tornado, é ter sido feito tal qual se manifesta. Sim, as mulheres, em
seu conjunto, são hoje inferiores aos homens, isto é, sua situação
oferece"-lhes possibilidades menores. (Beauvoir, 1980, p.18).
\end{quote}

Como vimos nas atas, a tendência apresentada pelo grupo de interpretar a
sexualidade feminina a partir do ideal de corpo masculino gera
consequências para toda a teoria freudiana desenvolvida posteriormente,
o que Adler evidencia quando diz que o ``infante feminino vê também seu
ideal no homem, com uma restrição, contudo: ele deve tornar"-se um homem
pelos meios femininos''.

Ao escrever sobre a dissolução do complexo de Édipo, Freud (1924) faz
uma distinção entre o processo de escolha objetal no menino e na menina.
Mesmo deixando claro que o conhecimento sobre a sexualidade feminina é
ainda obscuro e insuficiente, defende que a percepção da menina é que
ela ``saiu perdendo'' (Freud, 1924, p. 211) --- o que na tradução
alternativa de Paulo César de Souza também poderia significar ``saiu
curto de mais'' (Freud, 1924, p. 211). Essa sensação seria vivida como
``desvantagem e razão para a inferioridade'' (Freud, 1924, p. 211), o que
faz com que a menina aceite sua castração mais facilmente através de uma
tentativa de compensação: o pênis perdido pode vir a ser substituído
pela promessa de no futuro receber do pai um filho --- o que para Freud
explicaria a identificação e a rivalidade das meninas com suas mães, num
espelhamento da lógica do Édipo masculino.

Ao recolhermos o efeito dessa lógica encontramos um grave problema ---
que explicitamente aparece no Caso Dora ---, a solução dos conflitos
histéricos perpassaria pela aceitação da própria castração e
inferioridade e não pelo questionamento dos critérios de julgamento que
a tornaram um ser inferior. Podemos dizer que há um deslocamento
conservador em Freud, o trauma dos abusos que aparecem na sua Teoria da
Sedução acaba tendo recalcada a sua dimensão de violência quando a
interpretação freudiana localiza o conflito e o sofrimento como oriundos
do pudor da consciência de entrar em contato com um desejo proibido, e
não pelo elemento da agressividade e invasão sofridos pelo outro.

Deste modo, interessa perseguir os efeitos produzidos pela retomada
daquilo que permanece disponível nos registros e dão testemunho de como
a teoria psicanalítica era construída e ao mesmo tempo se tornava uma
ferramenta para justificar os pressupostos dos lugares de poder de onde
falavam os médicos que compunham as primeiras reuniões.

\section{``Ainda essa discussão sobre patriarcado?'': qual a regra do jogo
e quem é o dono da bola}

No artigo sobre as mulheres médicas apresentado por Wittels em 15 de
março de 1907, lemos que a verdadeira vocação da mulher é atrair os
homens e o interesse das mulheres em estudar medicina viria da
necessidade de sobrepujar outras mulheres.

\begin{quote}
Quanto mais histérica for, tanto melhor tanto melhor será seu desempenho
como estudante, pois a histérica é capaz de desviar sua pulsão sexual do
alvo sexual. Ela pode ser tão imoral quanto quiser sem ter de se
envergonhar. Os homens que afirmam ser feministas, mas que na verdade
não são senão masoquistas, aprovam que as mulheres estudem medicina: mas
o estudante comum e relativamente normal considera sua colega uma
prostituta. Enquanto ainda é estudante de medicina, a mulher prejudica
apenas a si mesma; quando começa a exercer a profissão, torna"-se um
perigo para os outros. As pacientes mulheres não confiam nela, as
enfermeiras não gostam dela, e um homem doente jamais se submeteria a um
exame realizado por uma mulher sem alimentar pensamentos de cunho
sexual. (1915, p. 298)
\end{quote}

Na discussão que segue, Graf, percebe no afeto impróprio de Wittels, uma
espécie de ressentimento de que as mulheres estudem ao invés de praticar
coito. Para Graf, ``a mulher nunca poderá realizar algo importante como
o homem, pois carece de grande influência pessoal, do poder sugestivo
que, além do conhecimento, é indispensável à boa formação do médico.
(\ldots{}) As médicas, que não dispõem desta autoridade (paterna), são mais
qualificadas para atuarem como substitutas da mãe, isto é, para serem
enfermeiras.'' (1915,p. 301) Já Reitler entende que ``a maioria das
mulheres que estuda medicina voluntariamente renunciou aos homens (por
conhecerem seus próprios defeitos físicos) (1915, p.301).

Hitschmann diz que a formação rica dessas mulheres é uma profilaxia
contra a histeria, e que seria necessário admitir que ``a maioria das
estudantes é feia e que as verdadeiras amazonas não têm seios''. (1915,
p. 302) Para Hitschmann:

\begin{quote}
O perigo de ser capturado por um tal ``monstro'', que segundo Wittels,
ameaça os estudantes do século masculino, não é tão grande assim, e o
destino deste jovem estudante é preferível ao de tantos outros que
dissipam sua juventude com prostitutas. O comportamento em certo sentido
livre das estudantes (segundo ele, elas são prostitutas) é ainda
preferível à hipocrisia mentirosa de algumas \emph{virgo tacta.} (1915,
p. 302)
\end{quote}

Freud entende que Wittels representa um ponto de vista juvenil e censura
sua extrema falta de delicadeza, interpretando que a censura e o
desprezo pelas mulheres aparece sempre que ele ``descobre o segredo'' de
que elas não têm aversão ao sexo, sendo sua misoginia decorrente do
desprezo inconsciente que é feito contra a própria mãe.

\begin{quote}
A mulher, a quem a civilização impôs o fardo mais pesado (sobretudo o da
procriação), deve ser julgada com tolerância e generosidade nos aspectos
em que ficou atrasada em relação ao homem. Além disso, falta ao artigo o
senso de justiça: ele demonstra ceticismo em face do novo, mas não
contesta o que é antigo. Ainda que seja repreensível. Os desarranjos da
profissão médica não foram introduzidos pela mulher, mas existem há
muito tempo. (2015, p. 303)
\end{quote}

Freud prossegue, então, dizendo que: ``É correto que as mulheres não
ganharão nada com os estudos e que seu destino também não mudará para
melhor com ele. As mulheres também não se comparam ao homem no tocante à
sublimação da sexualidade''. (1915, p. 304). Na sequência lemos nos
registros dos comentários de Freud sobre a necessidade de que, uma vez
revelada a origem sexual da energia que mobiliza as maiores conquistas
da cultura, aprendamos a usar de forma consciente a repressão em
benefício e acordo com ela:

\begin{quote}
O ideal de cortesã é inútil para nossa cultura. Esforçamo"-nos para
revelar a sexualidade; mas uma vez que isso se estabeleceu, exigimos que
todo esse recalque sexual se torne consciente e que se aprenda a
subordiná"-lo às necessidades da cultura. Nós substituímos o recalque
pela repressão normal. A questão sexual não deve ser dissociada do
social, e quando se prefere a abstinência às condições miseráveis em que
o sexo é vivenciado, não o faz sem protesto. A consciência do pecado que
impede a sexualidade é muito disseminada, e mesmo aqueles que são livres
sexualmente se sentem grandes pecadores. Uma mulher que não é confiável
em matéria de sexo, como a cortesã, não vale nada, ela é uma miserável.
(2015, p. 304-305).
\end{quote}

Se há uma tolerância de Freud com as ``mulheres bem instruídas'' que
sacrificam sua satisfação sexual para progredir de alguma forma na
cultura, a satisfação sexual das mulheres parece ainda ter a conotação
de algo deplorável, já que o fator que torna as cortesãs miseráveis
parece ser a incapacidade de sublimação e não o contexto econômico e
social. As demais mulheres continuam relegadas à função de mães, esposas
e objeto do desejo dos homens, o que será ainda utilizado na construção
do seu mito de criação da sociedade, o Totem e Tabu (1913).

As discussões sobre a origem social e política do patriarcado retomam
diferentes histórias hipotéticas, ligando"-o geralmente a características
universais da sociedade humana. ``A gênese da família (patriarcal) é
frequentemente entendida como sinônimo da origem da vida social
propriamente dita, e tanto a origem do patriarcado quanto a da sociedade
são tratadas como sendo o mesmo processo'' (Pateman, {[}1988{]} 1993,
p.43). De acordo com a autora, esta confusão entre a origem das
histórias hipotéticas do surgimento do patriarcado com as histórias de
origem da sociedade humana ou civilização é compartilhada tanto por
teóricos clássicos do contrato, quanto por psicanalistas como Freud. Não
é difícil perceber a aproximação entre a afirmação de Pateman e a
descrição da ligação entre o Complexo de Édipo e criação das leis
sociais de interdição do incesto e parricídio.

Para Carole Pateman (1988), o patriarcado moderno e a dominação dos
homens sobre as mulheres é estabelecido pelo contrato sexual, que foi
sistematicamente recalcado no contrato político original (referência
aqui é a Locke). A suposta liberdade individual seria, na verdade, uma
ficção política da origem dos direitos que, no fim, garantem a
perpetuação de relações de dominação e subordinação.

\begin{quote}
A dominação dos homens sobre as mulheres e o direito de acesso sexul
regular a elas estão em questão na formulação do pacto original. O
contrato social é uma história de liberdade, o contrato sexual é uma
história de sujeição. O contrato original cria ambas, a liberdade e a
dominação. A liberdade do homem e a sujeição da mulher derivam do
contrato original e o sentido de liberdade civil não pode ser
compreendido sem a metade perdida da história, que revela como o direito
patriarcal dos homens sobre as mulheres é criado pelo contrato.'' (p.
16-17)
\end{quote}

Mesmo não entrando no mérito de defender ou não o contratualismo, é
interessante trazer Pateman (1988) para o debate pois a teoria freudiana
sobre a origem da sociedade parece partir dos mesmos pressupostos do
contrato originário, como veremos mais adiante. Para justificar o uso
deste conceito criticado por muitos autores, Pateman afirma que
({[}1988{]} 1993) o ``patriarcado refere"-se a uma forma de poder político
mas, apesar de os teóricos políticos terem gastado muito tempo
discutindo a respeito da legitimidade e dos fundamentos de formas de
poder político, o modelo patriarcal foi quase que totalmente ignorado no
século \versal{XX}'' (p.38). Segundo a autora, o conceito foi praticamente
suprimido pelas teorias do contrato social que utilizam o próprio
patriarcado como mecanismo de análise devido à não separação entre as
interpretações patriarcais e seu significado. Dentro do próprio
movimento feminista algumas criticam o uso do conceito por o
considerarem a"-histórico e universalista, incapaz de abarcar as diversas
especificidades da experiência de ser mulher. Segundo Butler (2018)

\begin{quote}
Enunciar a lei do patriarcado como uma estrutura repressiva e reguladora
também exige uma reconsideração a partir dessa perspectiva crítica. O
recurso feminista a um passado imaginário tem que ser cauteloso, pois,
ao desmascarar as afirmações autorreificadoras do poder masculinista,
deve evitar promover uma retificação politicamente problemática da
experiência das mulheres (Butler, 2018, p. 72)
\end{quote}

Entretanto, diversas autoras concordam com Pateman (1988) que a
manutenção da noção de patriarcado é importante, pois representa o
``único conceito que se refere especificamente à sujeição da mulher, e
 que singulariza a forma de direito político que todos os homens exercem
 pelo fato de serem homens. Se o problema não for nomeado, o patriarcado
 poderá muito bem ser habilmente jogado na obscuridade, por debaixo das
 categorias convencionais de análise política'' (Pateman, {[}1988{]} 1993,
p.39).

Para a feminista e ativista negra estadunidense bell hooks --- responsável
pelo importante debate sobre a interseccionalidade dos diferentes
sistemas de opressão, discriminação e dominação envolvidos na construção
das identidades sociais ---, o patriarcado tem sido silenciado e
considerado como passado na nossa cultura. Em suas palestras, ela
percebe que quando descreve o sistema político como ``patriarcado
capitalista imperialista de supremacia branca'' a audiência geralmente
ri, e esta risada a lembra de que ela corre o risco de não ser levada a
sério se desafiar o patriarcado abertamente. Piadas, risos,
interrupções, acusar as mulheres de loucas, dar menos crédito ao que
dizem e produzem, justificar suas conquistas por sua aparência ou
sexualidade, olhá"-las apenas como objetos de admiração física e
hiper"-sexualizar seus corpos são todas formas de silenciamento ou
descrédito usadas como estratégias que mantêm a estrutura de poder e a
produção de saber sobre as mulheres sem revisão. Mesmo que não seja
possível perceber através do conteúdo, a forma do mesmo modelo ainda se
mantém certamente presente, hooks define: ``o patriarcado é um sistema
político"-social que insiste que os homens são inerentemente dominantes,
superiores a tudo e a todos considerados fracos, especialmente mulheres,
e dotados do direito de dominar e governar os fracos e manter esse
domínio através de várias formas de terrorismo psicológico e violência
'' (bell hooks, online, p.1). Levantar a relação entre o patriarcado e o
surgimento da teoria freudiana através da análise da atas e dos textos
pode não só possibilitar que a psicanálise repense suas bases teóricas
como também fazer com que a política recalcada atrás das bases
biologicistas e universalistas da teoria possa tomar forma de discurso,
e não só de sintoma.

\section{Mas ``o inconsciente não tem sexo''\ldots{}: aproximações do
patriarcado com a psicanálise }

O complexo de Édipo é um processo do desenvolvimento sexual infantil
descrito por Freud primordialmente marcado pela emergência de desejos
amorosos e hostis da criança em relação a seus progenitores. Este
conceito descreve um processo psíquico que tem uma função importante na
socialização humana e na estrutura política da cultura. Em Totem e Tabu
(1913), Freud defende uma clara ligação entre a origem da vida em
sociedade e a causalidade psíquica da neurose. A ideia de civilização é
calcada na diferenciação dos sexos, que irá determinar as formas de
organização dos papéis e funções exercidas no núcleo familiar e entre os
outros grupos. O que apresenta aproximações com a estrutura descrita a
seguir por Pateman (1988) do contrato original. Segundo Freud, a neurose
diz respeito à necessidade de refreamento dos impulsos de satisfação
sexual, e o mal"-estar seria o preço a ser pago para a entrada na cultura
e suas leis. A ambivalência dos irmãos em relação ao pai da horda
primeva toma corpo no ódio ao pai por ser ele um obstáculo ao anseio de
poder e desejos sexuais, mas também fonte de amor e admiração por ser
aquele que dispõe de todas as mulheres. O filhos unem"-se para matar pai
e após o banquete canibal realizado para a incorporação da força e poder
do pai, surge o sentimento de culpa e remorso que faz com que seja
restaurado seu lugar simbólico, tornando"-se mais forte internamente do
que fora vivo, já que não pode ser morto. Os irmãos instauram assim a
fraternidade, proibindo o assassinato e que alguém ocupe o lugar
paterno, limitando o poder e o gozo sobre as mulheres. ``Assim criaram,
a partir da consciência de culpa do filho, os dois tabus fundamentais do
totemismo, que justamente por isso tinham de concordar com os dois
desejos reprimidos do complexo de Édipo'' (Freud, 1913, p. 219). Neste
paralelo que Freud faz entre a internalização da lei na economia
psíquica da criança --- que será explicado através do complexo de
castração ---, com a internalização da lei social no Totem e Tabu, o
complexo de Édipo é a forma de transmissão da internalização da lei
social patriarcal na criança. Nesse modelo, o amor pelo objeto primário,
a mãe, e o ódio ao pai"-rival fariam parte da ``natureza humana'', já que
nossa condição antropológica universal é o desamparo. A conquista
civilizatória de renúncia dessas tendências viria através da
identificação com o pai que só é tornada possível, na leitura freudiana,
pela ideia de uma inferioridade do corpo feminino, aquele que teria
sofrido a punição que toda criança gostaria de escapar a todo custo: ser
mulher.

A criação da teoria sobre a origem política não pode ser separada do seu
ponto de partida, e nesse sentido a referência à ``civilização'' já se
refere à ``sociedade civil'' em uma forma histórica e culturalmente
específica de vida social. Alguns teóricos entendem a família patriarcal
como a forma social originária e natural de onde surgiria o patriarcado
tradicional. Supor que o funcionamento político da comunidade teria como
único fundamento o parentesco sanguíneo é um pressuposto falso, segundo
Pateman (1988). O que sustentava a família patriarcal seria o que ela
chamou de ``ficção legal'', na qual as famílias, que também absorviam
estranhos e agregados, eram mantidas unidas pela obediência ao chefe
patriarcal, direito sustentado por uma ficção e não pela natureza.

Mais tarde, nas ``sociedades progressistas'' sobre as quais os
contratualistas (Locke, Hobbes e Rousseau) escrevem, o patriarcado passa
a ser considerado um direito paterno sustentado pelo contrato do
casamento, no qual mulheres e filhos trocam obediência por proteção.
Esse modelo se amplia para a política, do mesmo modo que um pai,
responsável pela sua família, controla o comportamento de um filho e da
mulher para protegê"-los, também o Estado protege os cidadãos ambos
justificando o uso da violência para obter obediência. Segundo a teoria
do contrato, no modelo patriarcal da ordem política, o dominante"-pai ``é
assassinado por seus filhos, que transformam (a dimensão paterna do) o
direito patriarcal paterno no governo civil. Os filhos transferem essa
dimensão do poder político para os representantes, O Estado'' (Pateman,
1988, p. 56) --- o que nos leva a reconhecer na descrição que Freud faz
em seu mito do pai da horda (1913) o mesmo processo que os
contratualistas chamam de ``modelo patriarcal da ordem política''.

Stephen Frosh (1987) defende que o sujeito entra na cultura por meio do
complexo de Édipo, pois é a primeira vez que as figuras externas são
internalizadas no sujeito, tanto representantes da cultura como da
política, que ordena o funcionamento dos corpos. Quando o menino se
apaixona pela mãe contra a autoridade do pai ele se percebe em risco, e
o medo da punição o faz abrir mão de seus desejos, provocando sua
``castração psíquica''. Assim, o menino obedece a essa ameaça e reprime
seu desejo. Este não é apenas um modelo individual para o
desenvolvimento psíquico, é também o encontro entre sujeito e sociedade.
O super"-eu como ``herdeiro do complexo de Édipo'' é, ao mesmo tempo, uma
instância que cria o sujeito e aquela que irá julgá"-lo. Em ``O Eu e o Id''
(1923) Freud explica que a agressividade e tirania são decorrentes do
enraizamento dessa instância no Id, o que lhe confere acesso direto às
ambivalências dos impulsos desejosos que são censurados pelas exigências
de ordem social que são internalizadas.

\begin{quote}
A ideia chave aqui é que a internalização da autoridade como uma
resposta ao efeito do complexo de Édipo resulta em haver uma espécie de
``estado'' dentro da mente, com o super"-eu atuando como uma espécie de
representante para as atividades repressivas do mundo externo social.
Assim como a sexualidade infantil excessiva é canalizada através dos
processos de desenvolvimento para a genitalidade, a agressão é dominada
pela incorporação ao super"-eu punitivo; a civilização avança em
detrimento da felicidade individual. (Frosh, 2010, p.61 tradução nossa)
\end{quote}

Consequentemente, é possível entender o super"-eu como a incorporação do
poder familiar, que é a estrutura que garante a manutenção do poder
social dentro de um governo específico. O problema aqui é que o super"-eu
julga a partir de leis que ele mesmo criou, pois tanto a criação do
conceito do complexo de Édipo quanto o processo de desenvolvimento que
ele descreve estariam implicados na internalização do patriarcado na
criança, feito através da autoridade conferida ao pai pela estrutura
familiar patriarcal.

Relembramos Sedgwick (2003) nesse momento: a dimensão política está
ligada diretamente à questão da performatividade do conhecimento que é
transmitido e de como as formas de produzi"-lo têm efeitos sobre aquilo
que produz. Portanto, é fundamental pensar como a teoria psicanalítica
tem dimensionado uma possível normatização, uma vez que seus conceitos
tratam e reproduzem a estrutura patriarcal.

\section{Mulheres questionando A Psicanálise }

Juliet Mitchell (1974) foi uma das pioneiras a apontar que a psicanálise
estaria marcada pela cultura patriarcal. Seu trabalho problematiza o
papel da ideologia nos processos inconscientes, principalmente a
transmissão da diferença sexual. Ela mostra como, para além dos fatores
constitucionais biofísicos, químicos e anatômicos, as mulheres aprendem
os comportamentos que evidenciam a diferença sexual e são socializadas
para assumirem a posição do segundo sexo. Podemos pensar que essa
tendência é expressa quando se analisa a sexualidade feminina sempre a
partir do corpo e dos desejos masculinos. Como se não houvesse saída,
qualquer comportamento ou diferença tendia a ser interpretado pelo grupo
das quartas"-feiras como desvantagens, falta, atraso ou debilidade moral
o que fica claro na ata do dia 19 de outubro de 1910 (ainda não citada),
quando Adler apresenta um caso de uma menina de vinte anos com sintomas
de enurese. Ela relata gostar de mentir para os homens e sonha estar
fazendo sexo por cima. O sonho é interpretado por Adler como ``protesto
masculino'', pois tanto a masturbação quanto urinar são considerados como
marcas do homem. Stekel argumentar que pode ser também um protesto
feminino e uma discussão é instaurada. Na reunião anterior, de 12 de
Outubro, o grupo havia discutido a educação feminina, e Freud argumentou
que a coeducação na América falhou, retomando a opinião de Hall de que
``as meninas se desenvolvem mais rapidamente que os meninos, sentem
 superiores a eles em tudo, e perdem o respeito pelo sexo masculino. A
 isso deve ser acrescentado o fato de que na América o pai ideal aparenta
 estar soterrado, então a garota Americana não pode manter a ilusão que é
 necessária para o casamento'' (12 Outubro de 1910, p.14 tradução nossa).
Como apareceu na ata das jovens médicas, fica claro que para o grupo a
mulher tem a função restrita a ser esposa e mãe, e sua educação e
sexualidade deveriam guiá"-la para essas funções.

Se seguirmos Mitchell (1974) podemos pensar que a psicanálise não só
descreveu processos inconscientes neutros e sem sexo, mas criou um
inconsciente desde sempre marcado pelo patriarcado e cujas
interpretações sofreram da sua influência, gerando uma clínica que
recolhe e trata os efeitos desta forma de poder no sujeito.

\begin{quote}
Foi na análise do ``patriarcado'' --- a lei do pai (mais específica que o
sexismo dos homens) --- que a psicanálise foi transformada em uma
possível fonte de compreensão. A questão da ideologia e a transmissão de
idéias inconscientes e os efeitos da diferença sexual ligaram"-se à noção
de patriarcado. (Mitchell, 1974, p.~\textsc{xxvi})
\end{quote}

Ou seja, como uma teoria patriarcal, não há como separar o que a
psicanálise diz do sujeito e seu inconsciente do que diz respeito ao
próprio patriarcado e suas consequências. Léa Carneiro Silveira (no
prelo) argumenta que as fantasias também podem remeter a história e
contingência dos traços de memória daquilo que aconteceu com nossos
antepassados, ``como essa herança é constitutiva da cultura, não há como,
pois, sustentar --- nem que seja por essa razão --- que a teoria da
cultura seja algo externo à psicanálise'' (p.6).

A psicanalista estadunidense da relação de objetos, Jessica Benjamin,
defende que a acentuada ênfase na diferenciação dentro da psicanálise
reflete a experiência masculina. Em \emph{Bonds of Love} (1980),
Benjamin escreve que nossa cultura somente conhece uma forma de
individualidade, a saber: a posição masculina de superdiferenciação (ou
uma falsa diferenciação), esta depende de uma negação de todas as
tendências que indicam a semelhança, a fusão e a capacidade de resposta
recíproca. Tal maneira dualística de estabelecer e proteger a
individualidade encaixa"-se na lógica da racionalidade ocidental, isto é,
tornar"-se um sujeito só seria possível através da transformação do outro
em objeto. Mas essa separação que busca uma identidade reconhecida como
diferente e individual faz com que o sujeito isolado procure resolver
sua dependência ao outro --- considerada um traço feminino --- através
da dominação. Os meninos conquistam sua identidade distinta através do
repúdio da mãe que pode ser mais ou menos violento, já que todas as
crianças se identificam e querem imitar suas mães, então no complexo de
Édipo, eles ``descobrem'' que eles só podem tê"-la, não ser como ela. Se
a criança passa a ver a mãe como um objeto, para conquistar uma
subjetividade que tem capacidade de agência, a criança entende que
precisa diferenciar"-se dela, e não reconhecê"-la, numa lógica do dar ou
tomar, e não da reciprocidade.

Benjamin mostra como essa perspectiva edípica tradicional na psicanálise
ignora que a descoberta de si e a descoberta do outro andam juntas, que
tais processos são interdependentes. Ela afirma que há uma negação em
nossa cultura da \emph{nurturance} (alimentação, nutrição, estimulação,
cultivo) e acompanhado ao desejo do reconhecimento daquele que nos cuida
há também a negação do reconhecimento mútuo. É justamente a negação da
interdependência --- que de certa maneira tenta ser traduzida através da
noção de ambivalência --- e o desejo de onipotência e de controle do
outro que acabam formatando a maneira como pensamos os fenômenos dentro
da nossa cultura. Os limites e a ``separabilidade'' são constantemente
reforçados e a aspiração de reciprocidade e reconhecimento mútuo são
insistentemente minados pelo medo de se perder os próprios contornos e a
onipotência mental: ``O medo da intrusão, o desejo de ser
auto"-suficiente, de ser inamovível, são formas familiares de evitar um
confronto cru com a realidade de que o outro existe separado de mim, e
de quem eu posso vir a precisar'' (Benjamin, 1980, p. 163, tradução
nossa). A ideia de Benjamin é de que toda violência é uma falha na
diferenciação e o limite do outro é buscado como uma forma de proteção
aos próprios impulsos sádicos e masoquistas. Para ela, ``o fim
intolerável para o masoquista é o abandono, e pro sádico, é a morte do
outro que ele destrói. Quando o outro é drenado de sua resistência, ela
só pode ser eliminada com a morte'' (Benjamin,1988, p. 166).

O problema é que quando ser mãe é a principal tarefa social conferida à
mulher, ela também passa a tornar"-se mais dependente da criança para
conferir sentido, significado e formas de gratificação. A mulher fica
vulnerável a um modelo onde o reconhecimento passa a ser uma forma de
aprovação de seu desempenho --- que é julgado pelo comportamento da
criança que reflete seu sucesso e seu fracasso ---, o que torna mais
difícil encorajar e tolerar o processo de diferenciação do filho que
poderia ser incentivado através da assertiva da mãe sobre suas próprias
necessidades.

Benjamin (19080) revê os conceitos contemporâneos de independência e
autonomia e compreende que ele acabam por gerar tentativas de dominação,
estratégia criada para lidar com o horror da dependência. Essa crítica
cria um novo paradoxo de forças contraditórias entre desejo de
reconhecimento e independência, o que a leva a pensar nas formas de
sustentação do reconhecimento entre sujeitos iguais. Essa nova lógica
propõe o mútuo reconhecimento não como algo estanque e que nunca irá
faltar, mas como uma ética de relação com o outro, um horizonte a ser
perseguido.

A desnaturalização do papel da mãe que passa as ser intrinsecamente
relacionado ao papel social da mulher, demonstra como as premissas
sociais, evidentes na leitura das atas, foram usadas por Freud para a
construção do Complexo de Édipo, que representa como a psique responde e
se organiza em uma sociedade patriarcal. Tomando como modelo o que
acontece nos consultórios, a autora sugere que assim como na análise
existe uma dinâmica entre colapso e recriação (Benjamin, 2004), também o
reconhecimento poderia ser um processo dinâmico de troca de lugares de
poder, tanto nas relações pessoais quanto na busca por transformação
social.

\section{Conclusão: manutenção da forma através da exploração do
conteúdo}

Não se pode remover do histórico da psicanálise o que Freud disse em
1907: ``É correto que as mulheres não ganharão nada com os estudos e que
seu destino também não mudará para melhor com ele. As mulheres também
não se comparam ao homem no tocante à sublimação da sexualidade''.
(1915, p. 304). Mas é possível pensar que essa origem poderia ter sido
diferente, que fez parte de um momento histórico, e trabalhar para mudar
as consequências desse pensamento na psicanálise que estudamos e
praticamos hoje. Repensar a Sociedade das Quartas"-feiras através da
leitura de mulheres é colocar o patriarcado em diálogo com o feminismo,
assumindo que o destino não é dado pela anatomia, como dizia Freud, mas
pela história, criadora de políticas normatizadoras internalizadas pelos
processos psíquicos. Ao voltarmos à sala de Freud não pretendíamos
somente desvelar o patriarcado da psicanálise feita por homens do século
passado, o que seria o retorno ao já sabido, mas poder falar de um dos
componentes do jogo de poder da produção teórica que se mantém
silenciado pela desautorização. Apesar de o patriarcado ser essa crença
na superioridade masculina, ele é transmitido não só por homens mas por
todos --- objetos e ao mesmo tempo agentes do processo.

Pode"-se dizer que desde a sociedade das Quartas"-feiras, tanto o lugar
privilegiado de saber quanto a quem irá produzir ou organizar a teoria
sobre o patriarcado e temas correlacionados, continua a ser
reiteradamente dado a homens por homens. De uma forma travestida, o
conteúdo produzido por feministas, teóricas de gênero ou das
masculinidades é objeto de exploração do patriarcado, que mais uma vez o
usa para se colocar dentro da discussão contemporânea, mantendo ao mesmo
tempo o lugar de produtor do saber sobre o corpo e psique da mulher e
das outras subjetividades não hegemônicas.

A história parece nos mostrar como aquilo que em algum momento
acreditou"-se ter sido superado subsiste: o Estado, afinal, nunca
conseguiu ser laico, a teoria de Darwin tem perdido espaço para o
criacionismo, a Terra para muitos voltou a ser plana e o corpo feminino
nunca deixou de ser considerado menos que humano. Da mesma forma, dentro
da história do pensamento psicanalítico, aquilo que já foi alcançado e
discutido pelos desdobramentos da teoria psicanalítica pós"-freudiana e
suas críticas também não garante que antigos pressupostos sejam
retomados como verdadeiros, ou que eles tenham sequer desaparecido em
algum momento. Tendo isso em mente, nosso pensamento hegemônico que
cultiva a fé no ideal de linearidade do progresso precisa ser revisto à
luz (e à sombra) dos retrocessos que lhe são inerentes, entendendo não
só como as ideias são capazes de conferir efeitos de visibilidade e
invisibilidade, mas também como tais ideias não morrem no inconsciente
da cultura e podem ser reativadas a qualquer momento. Partimos do
princípio de que a psicanálise e sua estratégia arqueológica de
investigação são recursos potentes para o entendimento e revelação dessa
dinâmica que subverte os critérios da racionalidade consciente. Mas, até
que ponto ela é capaz de elaborar a própria lógica inconsciente que
sustenta a teoria que busca descrevê"-lo? E, em que medida sabemos quando
as premissas contaminadas por moralismos de época acabaram
reinfiltrando"-se na maneira como interpretamos e damos inteligibilidade
aos acontecimentos do presente? A retomada dos primórdios da psicanálise
revela"-se frutífera quando o complexo de Édipo passa a ser lido como uma
crítica das formas como se dão a organização, as regulações sociais e os
modelos de normatização. Assim, ele é um conceito que nos permite
recolher os efeitos de como a fantasia e as teorias incidem no corpo,
como o social toma forma na psique, tornando impossível a separação
desta de sua esfera política, social e cultural (Butler, 2010).

Mantendo a complexidade entre patriarcado e psicanálise, podemos
entender a causalidade psíquica com a analogia do espectro produzido
pela luz branca que incide em um prisma. Por mais que não seja possível
ver a olho nú, ela é composta por diferentes tipos de ondas, que seriam
como linhas de poder. Não é possível ter luz sem que as diferentes cores
estejam presentes, mas saber que as usamos para ver, nos permite
perceber quando a luz está mais para azul ou para amarelo. Nesse
sentido, a análise teria também a função de prisma, promovendo a
difração das linhas de força que compõem a própria psicanálise, das
quais somos ao mesmo tempo agentes e determinados. Há que se sustentar a
aposta de que o destino não está fadado a repetir as narrativas
violentas das histórias de origem que nos formam.

\section{Bibliografia}

Atas da Sociedade Psicanalítica de Viena 1906-1908/ Tradução de Marcella
Marino Medeiros. Org. Checchia, Marcelo \& Torres, Ronaldo e Hoffmann,
Waldo. São Paulo: Scriptorium, 2015.

Minutes of the Vienna Psychoanalytic Society (1908-1910). v. 2. Org.
Herman Nuremberg and Ernst Federn. Nova York: International University
Press, 1967.

Beauvoir, Simone. O Segundo sexo -- fatos e mitos; tradução de Sérgio
Milliet. 4 ed. São Paulo: Difusão Européia do Livro, 1980.

Benjamin, Jessica. Beyond Doer and Done to. Psychoanalysis Quartery. 73
(1):5-46, 2004)

\_\_\_\_\_\_\_\_\_\_\_\_\_\_. The bonds of love: psychoanalysis,
feminism, and the problem of domination. Pantheon books: \versal{NY}, 1988,

\_\_\_\_\_\_\_\_\_\_\_\_\_\_. (1998) Shadow of the other:
intersubjectivity and gender in psychoanalysis. New York: Routledge.

Butler, Judith. (1990) Problemas de gênero: feminismo e subversão da
identidade. Tradução de Renato Aguiar. Rio de Janeiro: Civilização
Brasileira, 2018.

Butler, Judith. Conversando sobre psicanálise: entrevista com Judith
Butler. Em \versal{KNUDSEN}, Patrícia Porchat Pereira da Silva. Rev. Estud. Fem.,
Florianópolis, v. 18, n. 1, p. 161-170, Apr. 2010. Available from
\textless{}\emph{https://bit.ly/1CI0wqr}\textgreater{}.
access on 05 Mar. 2019.

bell hooks. Understanding patriarchy. Disponível
em: \textless{}\emph{https://bit.ly/2m2cOuE}\textgreater{}.

Freud, Sigmund. (1913). Totem Tabu. In: Obras Completas Volume 11: Totem
e tabu, Contribuição à história do movimento psicanalítico e outros
textos (1912-1914) / tradução Paulo Cesar de Souza. São Paulo: Companhia
das Letras, 2013.

\_\_\_\_\_\_\_\_\_\_. (1923). O Eu e o Id. In: Obras Completas Volume
16: O eu e o id, ``autobiografia'' e outros textos (1923-1925)/ tradução
Paulo César de Souza. São Paulo: Companhia das Letras, 2011.

\_\_\_\_\_\_\_\_\_\_. (1924). Dissolução do complexo de Édipo. In: Obras
Completas Volume 16: O eu e o id, ``autobiografia'' e outros textos
(1923-1925)/ tradução Paulo César de Souza. São Paulo: Companhia das
Letras, 2011.

Frosh, Stephen. The politics of psychoanalysis: an introduction to
Freudian and Post"-Freudian Theory. London: Macmillan, 1987.

\_\_\_\_\_\_\_\_\_\_\_\_. Psychoanalysis outside the clinic:
interventions in psychosocial Studies. London: Palgrave macmillan, 2010.

Mitchell, Juliet. (1974) Psychoanalysis and feminism: radical
reassessment of freudian psychoanalysis. England: Penguin, 1990.

\_\_\_\_\_\_\_\_\_\_\_\_\_. (1999) Introduction. In Psychoanalysis and
feminism: radical reassessment of freudian psychoanalysis. England:
Basic Books, 2000.

Pateman, Carole. (1988) O contrato sexual. Tradução Marta Avancini, Rio
de Janeiro: Paz e Terra, 1993.

Roudinesco, Elisabeth. Freud em sua época e no nosso tempo. tradução:
André Telles. Rio de Janeiro: Zahar, 2016.

Sedgwick, Eve Kosofsky. ``Paranoid Reading and Reparative Reading, or,
You're So Paranoid, You Probably Think This Essay Is About You''. In:
Touching Feeling: Affect, Pedagogy, Performativity. Durham: Duke
University Press, 2003.

Silveira, Lea Carneiro. Sexualidade feminina e herança filogenética:
Sobre como Juliet Mitchell elude premissas que Freud precisa assumir
para a tese da inferioridade da mulher. Em Misogenia na Psicanálise. No
prelo.

\chapter*{\emph{Bêtes Noirs}: Uma releitura interseccional do texto freudiano
sobre homossexualidade feminina\footnote{Taduzido do inglês por Léa
  Silveira.}}
\addcontentsline{toc}{chapter}{\emph{Bêtes Noirs}: Uma releitura interseccional do texto freudiano
sobre homossexualidade feminina, \footnotesize\emph{por Marita Vyrgioti}}
\hedramarkboth{\emph{Bêtes Noirs}: Uma releitura interseccional do texto freudiano\ldots{}}{}


\begin{flushright}
\emph{Marita Vyrgioti}
\end{flushright}

\epigraph{\emph{Shall I never again, hand in hand with you.\\
Walk over green and flowery meadows\\
In the sunshine?\\
Shall I never again, hand in hand with you,\\
Endure life's burdens willingly\\
Because we are together?\\
One more time, hand in hand with you,\\
On your shoulder let me lament to you\\
How bitterly alone I am.}}{Poema escrito por Melanie Klein em torno de 1920}

\section{Fantasmas lésbicos: Homossexualidade na história da psicanálise}

O convite para contribuir com um volume sobre \emph{Freud e o
patriarcado} corresponde a uma oportunidade para voltar à questão da
homossexualidade feminina e perguntar o que o desejo lésbico tem a dizer
sobre a psicanálise. Autoras feministas têm mostrado que, nos escritos
psicanalíticos, a homossexualidade em mulheres é um desejo que não pode
ser teoricamente representado a não ser como um repúdio à diferença
sexual e ao sujeito masculino (O'Connor e Ryan 1993), como uma regressão
(Fuss 1999), como uma identificação masculina (Worthington 2011) ou como
perversão (De Lauretis 1994). Essas críticas feministas localizam a
homossexualidade de mulheres como um ponto cego na psicanálise de modo
intimamente ligado à problemática teorização da sexualidade feminina.
Nessa medida, sua principal intenção é procurar formas de simbolizar a
homossexualidade em mulheres para além do âmbito da patologia e
caracterizar a psicanálise como um território paradoxal: se, no
consultório, desejos marginais, perversos, assassinos, homossexuais e
impopulares podem ser livremente expressos, na articulação teórica, em
contrapartida, eles encontram pouco espaço\footnote{Os exemplos mais
  conhecidos são Teresa de Lauretis, Parveen Adams e Joyce McDougall.}.
Neste capítulo, procuro formular uma pergunta ligeiramente diferente;
uma que aborda a dificuldade de falar do desejo homossexual feminino
como uma questão sócio"-política entrelaçada com a epistemologia da
psicanálise. Meu objetivo é situar esse violento silenciamento nos
efeitos de uma constelação mais vasta de poder patriarcal e colonial,
bem como de uma sexualidade policialesca, repressiva e racialista,
emblemática da Europa imperial. Ao escrever a partir da perspectiva do
desejo lésbico e suas interseções com outras formas de silenciamento e
apagamento, questiono o que a subjetividade lésbica pode nos ajudar a
ver sobre a psicanálise freudiana e seus enredamentos com o patriarcado.
Dada a patologização sistemática dos desejos homossexuais femininos na
representação psicanalítica, essa questão torna"-se uma questão a
respeito de como os ``fantasmas lésbicos'' manifestam a sua presença na
tradição da psicanálise freudiana e de como explorar sua capacidade de
assombrar.

Falar de ``fantasmas'' na psicanálise é falar do inconsciente da própria
disciplina. Como Stephen Frosh observa, a psicanálise está enraizada na
personalidade de Freud, nas suas crenças, nas suas amizades e
colaborações pessoais e apaixonadas, nos seus sonhos e desejos
inconscientes. (Frosh 2017) Essa estreita ligação da disciplina com a
``personalidade'' do seu fundador é o seu ``calcanhar de Aquiles'' no
sentido de que, em primeiro lugar, as reavaliações teóricas
contemporâneas da cumplicidade da psicanálise com as constelações
coloniais patriarcais remontam necessariamente a Freud e à sua
teorização idiossincrática. Mas, em segundo lugar, e talvez mais
crucialmente, no sentido de que os desejos inconscientes, os conflitos e
as dificuldades das figuras"-chave da psicanálise ``são susceptíveis de
emergir em seu trabalho''. (Frosh 2017) Significa isso, então, que a
questão dos ``fantasmas lésbicos'' da psicanálise também diz respeito aos
desejos homossexuais silenciosos, não reconhecidos, escondidos e
inconscientes dos seus principais teóricos? Haveria algo nos trabalhos
psicanalíticos em torno da homossexualidade feminina que ainda não foi
explorado --- juntamente com outras formas de ``ambivalência inescapável''
em sua complexa história psicossocial --- e que produz ``algo
fantasmagórico e melancólico'' em suas revisões contemporâneas? (Frosh
2013, 5) Ao proceder às leituras para a escrita deste capítulo, fiquei
impressionada com uma das obras mais conhecidas no campo do lesbianismo
e da psicanálise, escrita em 1993 por Noreen O'Connor e Joanna Ryan,
intitulada \emph{Wild desires and mistaken identities: Lesbianism and
Psychoanalysis}. O que me intrigou foi a referência à ``selvageria'' do
desejo lésbico. Certamente, pode"-se argumentar, os títulos estão
sujeitos a pressões editoriais; mas, existiria a possibilidade de estar
em jogo aqui uma ênfase colonial inexplorada, que toma os desejos
lésbicos como indomáveis e não domesticados? Alternativamente, será que
o título indica que o desejo lésbico é ``erroneamente'' visto como uma
``identidade selvage'' e, se assim for, por que o livro deixa
inexploradas as implicações coloniais de tal associação? Por último, mas
não menos importante, na mesma medida em que O'Connor e Ryan forneceram
uma das mais sistemáticas críticas às ausências e deturpações de
mulheres homossexuais por parte de figuras"-chave dentro da tradição
psicanalítica mais ampla (incluindo Carl Jung e Jacques Lacan), uma
ausência conspícua no livro é a contestada e alegada homossexualidade da
filha do fundador, Anna Freud. Vou argumentar que tanto a questão da
interseccionalidade, nomeadamente as formas como as deturpações da
homossexualidade em mulheres estão enredadas nas economias coloniais
alicerçadas na ideia de raça, quanto a ambiguidade e o secretismo em
torno dos desejos (femininos) homossexuais das pioneiras psicanalíticas,
estão no centro dos ``fantasmas lésbicos'' da disciplina e marcam a sua
relação ambivalente com o patriarcado (branco).

Este capítulo é, portanto, um retorno a esses primeiros fantasmas
psicanalíticos. Em particular, procuro voltar ao único ``caso'' de
Sigmund Freud que se refere a uma mulher homossexual como um caso
``fantasmagórico''. Antes de fazer isso, vou primeiro indicar como a
historiografia da psicanálise excluiu a possibilidade de desejos
homossexuais femininos, condenando"-os ao estatuto de histórias
marginais, indignas de exploração, como ``rumores'' e ``fofocas''. Na
sequência, mostrarei que, na conjuntura colonial, a sexualidade só pode
ser compreendida a partir de uma perspectiva interseccional. Isso
significa não só que a forma como a homossexualidade é vista depende das
imagens sociais da racialização, mas também que as relações homossexuais
(femininas) que podem ser socialmente simbolizadas e pensadas dependem
da substituição da falta de diferença sexual por uma diferença racial
imaginária. Portanto, voltando ao único caso freudiano de paciente
homossexual mulher, pretendo mostrar que a ambivalência da psicanálise
em relação aos desejos lésbicos depende fortemente do fato dela forcluir
questões em torno do social, nomeadamente questões de diferenças raciais
e de classe. Em outras palavras, o que esses primeiros fantasmas
lésbicos enterrados nas primeiras histórias da psicanálise expõem é que
é preciso proceder a uma compreensão interseccional deles para que
certos pontos cegos da teoria possam ser postos de lado.

\section{A homossexualidade em mulheres como tabu do psicanalista}

Na historiografia da psicanálise, a homossexualidade feminina aparece
como uma história de secretismo, desconforto e ambiguidade. Tomemos o
exemplo da obscuridade em torno da relação de Anna Freud com Dorothy
Burlingham. Na biografia de Anna Freud escrita por Elisabeth
Young"-Bruehl, a relação entre as duas mulheres aparece como algo que
satisfaz o desejo de maternidade de Anna, mas ao mesmo tempo preserva
seu status de herdeira assexuada do legado psicanalítico. Apesar do
relato detalhado da intimidade pessoal e da colaboração profissional
entre as duas mulheres, Young"-Bruehl rejeita categoricamente os rumores
de sua parceria lésbica com base na dedicação amorosa de Anna ao
trabalho de seu pai: ``{[}Anna{]} tinha uma vida familiar rica e plena,
embora não tivesse, na década de 20 ou mais tarde, uma relação sexual
com Dorothy Burlingham ou com qualquer outra pessoa.'' Young"-Bruehl
continua: ``ela permaneceu uma `vestal'" --- uma palavra que sinaliza
``a virgindade de Anna Freud e seu papel como a \emph{principal guardiã}
da pessoa de seu pai e de sua ciência, a psicanálise'' (137, grifo meu).
O argumento subjacente a essa passagem coloca o desejo de Anna por (uma
forma estranha de) maternidade, e a herança do trabalho psicanalítico do
seu pai como em oposição a, ou incompatível com isso. O desconforto de
Young"-Bruehl com o misterioso relacionamento de Anna Freud com
Burlingham é ilustrado mais claramente numa passagem posterior, em que
ela observa que Anna Freud ``podia supervisionar e apoiar
altruisticamente o interesse de Dorothy por homens, desde que este
permanecesse platônico e não ameaçasse a amizade entre elas''. (137) Há
um sentido em que Young"-Bruehl reconhece a atração de Anna por Dorothy
(afinal ela ``altruisticamente'' apoia seu interesse por homens) mas, ao
mesmo tempo, a historiógrafa garante ao leitor que a abstinência e
pureza de Anna, seu auto"-sacrifício altruísta, lhe valeram a posição de
principal guardiã da psicanálise.

É impossível não nos perguntarmos: o que Young"-Bruehl está tentando
proteger? E de quem? Parafraseando O'Connor e Ryan, acaso seriam os
``desejos selvagens'' de Anna Freud, cuja realização poria em perigo a sua
herança? Ou será que Young"-Bruehl está tentando proteger Anna de uma
``psicanálise selvagem'' e das interpretações voyeurísticas e carregadas
de bisbilhotices realizadas por historiadores e críticos, que poderão
achar pouco convincente a sua co"-habitação \emph{platônica} com Dorothy
Burlingham, os telefonemas recíprocos tarde da noite quando Dorothy
residia no primeiro andar do apartamento de Berggasse 19? A quem
pertence a vergonha pela homossexualidade de Anna Freud --- se não a toda
a tradição psicanalítica, freudiana, e à fantasia de qual seria a
implicação para a disciplina se se descobrisse que a filha de Freud
poderia ter sido uma lésbica (não assumida)? Afinal, foi Sigmund Freud
quem se debateu durante anos com as raízes judaicas da psicanálise e sua
desvalorização em uma conjuntura social vienense anti"-semita que
descartou como escandalosa a centralidade da sexualidade, da
bissexualidade e da sexualidade infantil. (Frosh 2009) Assim, defendo
que parece haver um véu sutil e homofóbico que cobre as articulações
ambíguas entre a vida pessoal de Anna e as firmes declarações a respeito
de sua castidade. A representação historiográfica de sua sexualidade
está entrelaçada com a vergonha e o sigilo que estão ligados a uma
hegemonia europeia colonial e patriarcal que não permite que suas
principais teóricas e colaboradoras falem de seus desejos homossexuais.

De modo parecido, em seu recente trabalho magistral sobre a vida de
Sigmund Freud, Elizabeth Roudinesco passa ao largo do problema de
representar a relação entre Anna e Dorothy, conduzindo"-a ao status de
mimetismo e ``semelhança''. Em suas mãos, a sexualidade da filha de
Sigmund Freud mantém um componente homossexual, mas apenas na fantasia;
sua relação é ``como se'' fosse homossexual e, como tal, a possibilidade
de sua intimidade material e física é apagada. Roudinesco escreve que
elas ``desenvolveram relações íntimas \emph{que se assemelham muito às
das lésbicas}''\footnote{Na versão original da biografia escrita por
  Roudinesco, o trecho está redigido do seguinte modo: ``(\ldots{}) tout
  en nouant des relations d'intimité qui ressemblaient fort à celles de
  deux lesbiennes.'' (Roudinesco, E. \emph{Sigmund Freud, en son temps
  et dans le nôtre}. Paris: Le Seuil, 2014, p. 313) Note"-se a ausência
  da ideia de semelhança, fundamental para o argumento de M. Vyrgioti,
  na versão em português: ``(\ldots{}) suas relações de intimidade
  sugeriam as de duas lésbicas''. (Roudinesco, E. \emph{Freud na sua
  época e em nosso tempo}. Rio de Janeiro: Zahar, 2016, p. 288). Essa
  ausência comprometeria o argumento de M. Virgioti, razão pela qual
  optamos por não seguir a versão da editor Zahar. {[}N.T.{]}}.
(Roudinesco 2016, 249, grifo meu) No entanto, ao contrário de
Young"-Bruehl, Roudinesco associa o desconforto e a ``hostilidade'' de Anna
para com a sua homossexualidade (ela a via como uma patologia que
obstruía a sua prática clínica) a seu pai e à sua relação analítica.
Nesta representação biográfica, é a própria Anna que nega qualquer forma
de relação sexual com Dorothy, fazendo assim uma afirmação mais firme do
que a de Young"-Bruehl, que atua como intérprete confirmando a amizade
entre ambas. Roudinesco justifica o desejo de abstinência de Anna do
seguinte modo: ``Anna negou categoricamente a existência de uma relação
sexual com sua nova amiga --- uma maneira de permanecer fiel ao único
homem que amou, o seu pai.'' (Roudinesco 2016, p. 249, grifo meu) Ao ler
essa passagem, entendo ser preciso perguntar de que forma uma relação
sexual de Anna com Dorothy constituiria um ato de infidelidade ao seu
pai. Por que a realização sexual de uma filha constituiria uma ameaça ou
uma traição ao seu pai --- se este ato de traição não estiver ligado à
escolha de objeto homossexual? Por outras palavras, teria a sexualidade
de Anna em geral ameaçado a sua relação com o seu pai, ou a sua
homossexualidade em particular?

É importante notar que, antes desta passagem, Roudinesco nota a
incapacidade de Freud de se separar da sua amada filha --- a quem ele
chama Antígona --- e parece perceber a sua homossexualidade como um ato
de confirmação narcisista da sua autoridade patriarcal. Na
co"-parentalidade por parte de Anna com relação aos filhos de Dorothy,
Freud ``viu"-se mais do que nunca como o feliz patriarca de uma família
reconstituída submetida à erosão da antiga autoridade patriarcal, o
próprio processo que deu origem à psicanálise''. (Roudinesco 2016, p.
250) De fato, Roudinesco cita uma carta escrita por Freud para seu amigo
de longa data, Ludwig Binswanger, em janeiro de 1929, na qual ele
sublinha seu contentamento por ter em sua família Dorothy e seus filhos.
``Nossa simbiose com uma família americana (sem marido) cujos filhos a
minha filha está criando com uma mão firme do ponto de vista
psicanalítico, está cada vez mais estabelecida, motivo pelo qual nossos
planos para o verão são compartilhados.'' (Carta de Freud a Biswanger,
citada em Roudinesco, 2016, 250) Como veremos mais adiante, o que leva
Freud a fazer esse esclarecimento de parentesco nos oferece um ponto de
entrada a partir do qual revisitar seu trabalho analítico com Margarethe
Csonka, paciente até recentemente anônima, em seu artigo de 1920,
\emph{A psicogênese de um caso de homossexualidade em uma mulher}, ao
qual voltarei na seção final deste capítulo. Isto porque o que
caracteriza esta família é não apenas a ausência de uma figura
patriarcal, masculina. São dois os parâmetros que Freud estabelece para
seu contentamento: ele ainda é o único patriarca (família sem marido), e
a companhia de sua filha distingue"-se por sua respeitabilidade social:
trata"-se de uma família ``americana''\footnote{De acordo com Peter Gay,
  Freud não gostava muito da América desde sua viagem com Jung, em 1909.
  (Gay 2006) No entanto, deve"-se notar aqui que Dorothy Burlingham era
  uma nova"-iorquina abastada e neta do fundador da Tiffany and Co.
  (Roudinesco, 2016, 249)}. Portanto, o narcisismo de Sigmund Freud
permanece ileso à homossexualidade de Anna por motivos de ``raça'' e
classe, bem como por motivos de dominação patriarcal.

Essa perspectiva é apresentada por Suzanne Stewart"-Steinberg na
exploração que faz do caso freudiano de uma paciente homossexual
feminina, de 1920, caso que ela lê em relação à teoria da ``rendição
altruísta'' de Anna Freud. Stewart"-Steinberg mostra como, no universo
freudiano, a homossexualidade (feminina) toma a forma de um pacto
inconscientemente acordado entre pai e filha para proteger tanto o
narcisismo do pai quanto o da filha. De acordo com Stewart"-Steinberg,
Anna Freud torna o conceito de ``rendição altruísta'', como retira"-se em
favor de outra pessoa, a premissa fundamental de sua prática analítica:
``a atribuição a si mesma da condição de soldada'', o ``afrouxamento dos
limites do ego'', a ``rendição emocional'' (Stewart"-Steinberg 2011, 46).
Teorizada em relação ao complexo de masculinidade, ``à inveja do pênis e
a fantasias masculinas de ambição'' (Stewart"-Steinberg 2011, 91), a
rendição de Anna assinalava o abandono do poder político e social
masculino, a emasculação inerente à construção social da feminilidade.
Porém, enquanto de uma perspectiva sociológica, ``rendição'' é sinônimo de
retirar"-se do poder, em termos psicanalíticos, impotência
{[}\emph{powerlessness}{]} não é simplesmente sinônimo de incapacidade.
Como Jacqueline Rose disse, ``a subserviente, paradoxalmente, não
desobedece menos e não desfaz menos {[}à/a herança do pai{]} no momento
mesmo de sua rendição''. (Rose 1993, 193) Dito de outro modo, o poder de
Anna corresponderia precisamente a seu ato de se retirar
masoquisticamente do seu desejo (homossexual). É por isso que, como
observa Stewart"-Steinberg, enquanto Anna teoriza a etiologia da
homossexualidade em relação ao poder fálico e à masculinidade, seu pai
teoriza a etiologia da homossexualidade (masculina) em relação a ``raça'':
``O próprio Freud é levado numa longa nota de rodapé a ligar o processo
de `retirar"-se em favor de outra pessoa' não só a uma causa explicativa
da homossexualidade, mas também a seu \emph{retorno regressivo à horda
primordial}.'' (Stewart"-Steinberg 2011, 91) Para Sigmund Freud, a
homossexualidade de Anna repara seu narcisismo (no sentido de que ela
não faz nenhuma reivindicação contra sua autoridade). Para Anna, o
entendimento de Freud sobre a homossexualidade repara o dela; ela não
faz nenhuma regressão à ``horda primitive'', pois está numa relação
ambígua, obscura, com uma senhora altamente estimada e respeitável. Não
é isso o que se passa com a paciente de Freud --- e, como vou argumentar,
classe e ``raça'' marcam os limites da sua análise do caso.

Antes de concluir esta seção, gostaria de me referir a mais uma vinheta
marginal da história da psicanálise que se encontra na biografia de uma
pioneira para ilustrar a impossibilidade de deslocar a ``raça'' da
impossibilidade inicial de falar sobre homossexualidade. Diferentemente
do caso de Anna Freud, a outra teórica que eu gostaria de mencionar aqui
não é tão conhecida pela sua abstinência de relações íntimas obscuras;
em vez disso, ela é conhecida pela ousadia, constrangimento e
intensidade de suas interações com outras mulheres, entre elas Anna
Freud. Na biografia de Melanie Klein escrita por Phyllis Grosskurth, a
amizade íntima que Klein tinha com a irmã do marido de sua cunhada,
Klára Vágó, fica sob escrutínio. (Grosskurth, 1986, 84-85) De acordo com
Grosskurth, as duas mulheres passaram férias de verão em Rosenberg,
depois das quais Klein escreveu o poema que consta na epígrafe deste
capítulo, sua biógrafa argumentando que ele teria sido destinado a Vágó,
cujo retrato Klein também guardava em seu consultório. Vágó era
divorciada e é descrita como uma ``mulher inteligente, educada e
independente''. (Judith Sekatz Weisz, 8) Para Grosskurth, foi a criação
de Klein --- ela dedica várias páginas à forma como sua mãe a controlava
e policiava durante as férias em Rosenfeld --- que constrangeu sua
sexualidade. Em outras palavras, foi a mãe e sua ambivalência dirigida à
relação da filha com Vágó que, \emph{inadvertidamente}, impediram Klein
de prosseguir uma relação sexual com ela:

\begin{quote}
``Melanie tinha sido \emph{condicionada} a ser dependente, a cultuar os
homens e o amor ideal. Sua \emph{criação tinha"-a ensinado a reprimir a
sua própria energia sexual}. É possível que, através de Klára, ela tenha
começado a questionar se suas fantasias de romance eram falsas, hostis à
sua \emph{verdadeira natureza} e desenvolvimento, e, portanto, se um
casamento baseado em tais equívocos não seria uma farsa.''(Grosskurth
1986, 85)
\end{quote}

Grosskurth considera a homossexualidade de Klein em termos
essencialistas como uma ``verdadeira natureza'' indiscutível, que é
dramaticamente reprimida por uma mãe autoritária, ambivalente e
controladora. Como ela escreve em uma carta para sua filha, Libussa
achou inapropriado e muito ``imprudente você dividir um quarto com ela
em Abbazia. Acho que você pode dizer a ela que não seria bom para seus
nervos, que você precisa de tranquilidade total, e que você não deve se
submeter a nenhuma pressão no sentido de se comprometer com outra
pessoa.'' (Grosskurth 1986, 50) No entanto, Libussa também apoiou
perversamente a intimidade de sua filha com Vágó, a quem ela igualmente
viu como uma figura de apoio emocional que compensaria a fragilidade
psicológica de sua filha:

\begin{quote}
``É realmente óbvio que os seus nervos só precisam de paz e
tranquilidade, e que nada os perturbe, para se tornarem mais fortes e
gradualmente saudáveis. Estou imensamente feliz pois Frau Klara, com sua
calma, suavidade, seu caráter amável e gentil, só pode exercer uma
excelente influência sobre sua mente agitada {[}\ldots{}{]}'' (Grosskurth
1986, 84).
\end{quote}

A observação mais importante que Grosskurth faz sobre a relação de Klein
com Vágó remonta a questões de ``raça'' e homossexualidade. Em sua vasta
biografia de Klein, sua relação com Vágó ocupa apenas algumas páginas
dedicadas à visão de Klein sobre judeus que, devido ao colapso do
Império Austro"-Húngaro e à violência do ``terror branco'' anti"-semita,
converteram"-se ao catolicismo romano. Uma dessas famílias era a de Vágó.
Grosskurth sugere que Klein se opôs, de um modo que a incomodava, aos
membros de sua família que abandonaram o judaísmo, enquanto que, por
outro lado, seria ao seu carinho e amor por Vágó que ela deveria a
presença de traços de idéias cristãs em sua teoria posterior sobre a
culpa primária e a inveja constitucional --- uma visão também
compartilhada por outra biógrafa de Klein, Julia Kristeva. (Kristeva
2001, 22; Grosskurth 1986, 84) Em muitos aspectos, Grosskurth pinta o
retrato de Klein como uma homossexual não assumida e como uma judia não
assumida: ``Melanie nunca \emph{divulgou} que em Budapeste a família
Klein aderiu à Igreja Unitária e que todos os seus filhos foram
batizados''. (Grosskurth, 83) Talvez, então, Grosskurth sugira que
consideremos a impossível homossexualidade de Klein em relação a seu
impossível judaísmo. O argumento de Grosskurth sugere que foi a criação
de Klein --- a abordagem enlouquecedora de sua mãe, proibindo e ao mesmo
tempo aprovando seu relacionamento com Vágó --- que não permitiu que ela
realizasse seu desejo por essa mulher, conduzindo"-a, em vez disso, a
expressá"-lo em poemas e cartas românticas. Mais crucial ainda é o fato
de que Grosskurth vincula a relação lésbica não satisfeita de Klein com
a impossibilidade de seu judaísmo, também melancolicamente suprimido
pelo terror anti"-semita, forçando"-a a converter"-se e a batizar seus
filhos. Um segredo, algo que Klein não assume --- que ela '\emph{nunca
divulgou}'. Grosskurth, assim, lê Klein de uma maneira que implica que
as formas anti"-semitas e homofóbicas de opressão a levaram a uma
renúncia melancólica de seu desejo; como ela declara em seu poema
escrito após o intervalo de férias com Vágó ``deixe"-me lamentar com
você/o quão amargamente sozinha estou'', um sentimento espelhado na
exploração autobiográfica de seu judaísmo:

\begin{quote}
``Sempre detestei que alguns judeus, independentemente dos seus
princípios religiosos, tivessem vergonha da sua origem judaica e, sempre
que a questão surgiu, fiquei contente por poder confirmar a minha
própria origem judaica, embora eu não tenha \emph{nenhuma crença
religiosa}.'' (Klein 2013, 133, grifo meu).
\end{quote}

\section{Homossexualidade feminina, promiscuidade e pele}

Até o momento, explorei modos pelos quais os desejos homossexuais
femininos eram tabu para os psicanalistas; de certa maneira, eles
informam o inconsciente da disciplina. Nesta seção, situo a
homossexualidade feminina no contexto colonial para mostrar que sua
supressão não poderia ser alcançada a não ser dentro de uma economia de
desejos racializados. O objectivo desta seção, portanto, é duplo: em
primeiro lugar, mostrar como as mulheres homossexuais eram vistas como
estando numa posição estranha às estruturas patriarcais e coloniais e,
em segundo lugar, ilustrar como a internalização da homofobia (tal como
nos casos de Anna Freud e Melanie Klein) não é independente dos
processos sociais racializadores.

No relato histórico bem documentado de Robert Aldrich sobre colonialismo
e homossexualidade (\emph{Colonialism and Homosexuality}), a
homossexualidade masculina surge como uma prática que, ``para horror dos
moralistas caseiros'', encontra um lugar na colônia. Os viajantes
europeus procuravam ``aventuras coloniais'' eróticas, promovidas pelas
hierarquias coloniais entre colonizador e colonizado, bem como as
fantasias que acompanhavam a alegada natureza selvagem do Oriente.
(Aldrich 2003) Apesar da perpetuação do poder colonial europeu que
policiava, perseguia e prendia homossexuais masculinos, a colônia
permitiu a emergência de subculturas homossexuais e relações
privilegiadas de ``companheirismo'' (Aldrich 2003, 41) ou ``camaradagem''
(Aldrich 2003, 101) que ``tinham algumas lacunas em comparação com {[}as
práticas homossexuais masculinas na{]} Grã"-Bretanha''. (Aldrich 2003,
239) Dado o seu interesse na homossexualidade (masculina), Aldrich
observa que ela ``desempenhou um papel muito mais significativo e
diverso no colonialismo do que muitos admitiriam.'' (Aldrich 2003, 6) No
volume editado por Philippa Levine, \emph{Gender and empire}, Barbara
Bush também evoca a homossexualidade como uma prática exclusivamente
masculina que subverte a ``domesticidade branca heterossexual'', que
transgride ``as fronteiras raciais das quais dependia a estabilidade do
Império, enfraquecendo assim os fundamentos do colonialismo''. (Bush
2007, 84) O que emerge desses relatos é que a colônia se torna tanto o
lugar em que a masculinidade europeia branca transcende as fronteiras da
respeitabilidade quanto aquele em que a masculinidade desviante precisa
ser controlada (a homossexualidade masculina era ilegal na maioria das
colônias britânicas). Ao mesmo tempo, conferir privilégio ao estudo da
homossexualidade masculina como endêmica das colônias implica tanto a
ausência quanto a insignificância das práticas homossexuais femininas na
historiografia colonial. Aldrich justificou sua ênfase nos homens como
um ``alerta'' necessário. (Aldrich 2003, 8) Estendendo esse ponto,
argumento que, em virtude de a homossexualidade feminina ter uma relação
diferente com o imaginário patriarcal, ela parece estar ausente do
imaginário imperial europeu.

Na economia colonial, os desejos homossexuais femininos não são
percebidos nem compreendidos independentemente de outras categorias
sociais de desvio sexual. Jill Suzanne Smith mostra como as mulheres
que, através da sua ambiguidade sexual, atravessam perfeitamente as
fronteiras da respeitabilidade, tanto no sentido moral quanto por
refrência a classes sociais, foram rotuladas como coquetes\footnote{A
  razão pela qual estou me voltando para o conceito de ``coquete'' aqui é
  porque, como veremos mais tarde, o termo ``\emph{cocotte}'' (ou
  \emph{kokotte} em alemão) é aquele que Freud usa para descrever a
  amante de sua paciente homossexual feminina. De acordo com o
  \emph{Oxford Dictionary}, ``\emph{cocotte}'' realmente significa
  ``prostituta'', enquanto ``\emph{coquette}'' implica uma mulher
  namoradeira, com moral sexual branda. Ambos os termos regulamentaram e
  policiaram a sexualidade feminina na virada do século e, como veremos,
  também carregam subtextos raciais. {[}Não é possível encontrar
  paralelo em português para o primeiro desses termos (\emph{cocotte}),
  já que --- segundo o Houaiss --- ``cocota'' tem o significado de
  ``menina pré"-adolescente e bonita''. (N.T.){]}} ou ``prostitutas
    burguesas''. (Smith 2013, 62) Ao borrar as fronteiras da prostituição, as
``coquetes de Berlim'' buscaram saídas alternativas ao casamento que lhes
concedessem benefícios financeiros ao mesmo tempo em que lhes
permitissem visualizar ``novas identidades sociais e sexuais para as
mulheres, tais como a mulher solteira namoradeira, a mãe solteira, a
divorciada, a viúva e a lésbica'' (Smith 2013, 21). Enquanto a leitura de
Smith não se envolve com a racialização implícita desses novos papéis
sexuais femininos, as leituras interseccionais de raça e gênero nos
contextos coloniais ilustram como a prostituta burguesa feminina se
enquadra nos limites da branquitude. Sander Gilman, por exemplo, mostra
como na França a prostituição é representada na iconografia popular como
``sexualidade lasciva'', baseada em representações racistas de mulheres
sul"-africanas (Gilman 1986). Na mesma direção, Ann Stoler argumenta que,
na Inglaterra, a mulher promíscua da classe trabalhadora foi
interpretada como uma ``relíquia primitiva de um período evolucionário
anterior'', {[}\ldots{}{]} uma ``mulher selvagem'' que contrastou com o ``modelo
moral de restrição sexual e civilidade da classe média'' (Stoler 2000,
128). Assim, a prostituta, a coquete, entre outras, tornam"-se
representações sociais e estereotipadas da promiscuidade, ilustrando o
que Ann McClintock argumenta como sendo a rejeição da diferença sexual
através do policiamento das sexualidades não matrimoniais e da sua
associação com a diferença racial. (McClintock 1995)

É nesses discursos sobre a sexualidade feminina transgressora que a
homossexualidade feminina é encontrada. Assim, é impossível considerar a
homossexualidade feminina na cultura imperial europeia fora das
associações entre promiscuidade e excesso, bem como do seu subtexto
racializado, que, como veremos, representa uma grande ameaça tanto para
o patriarcado como para a família burguesa branca e a moral de classe ---
e contribui para os escritos obscuros sobre as próprias figuras lésbicas
da psicanálise.

Antes de concluir, consideremos duas vinhetas que ilustram os efeitos
materiais do silenciamento e da destituição do desejo homossexual
feminino em relação às experiências concernentes à pele. Argumentei que
as sexualidades femininas desviantes eram articuladas em termos
racializados, e através de escritos pessoais de vida e desejo lésbicos
também discernimos esta estranheza da ``heterossexual aberrante'' contra
``o seu papel social como objeto do desejo masculino'' expresso em termos
raciais. (McClintock 1995, 195) No romance de Radclyffe Hall \emph{The
Well of Loneliness} (1928) --- proibido sob alegação de obscenidade no
momento de sua publicação --- acompanhamos Stephen, uma jovem aristocrata
inglesa, autora e heroína de guerra, através de uma série de relações
sexuais com mulheres, e lemos sobre a experiência alienante de ser
tomada como outro:

\begin{quote}
`Por toda a sua vida ela tem que arrastar este seu corpo como um
monstruoso grilhão imposto ao seu espírito. Este corpo estranhamente
ardente, porém estéril, que deve adorar, mas nunca ser adorado em troca
pela criatura de sua adoração. Ela ansiava por mutilá"-lo, porque isso a
fazia sentir"-se cruel; era \emph{tão branco}, tão forte e tão
auto"-suficiente; contudo, mesmo assim era uma coisa tão pobre e infeliz
que seus olhos se enchiam de lágrimas e seu ódio se tornava piedade. Ela
começou a se lamentar por isso, tocando seus seios com dedos
deploráveis, acariciando seus ombros, deixando suas mãos escorregarem ao
longo de suas coxas retas --- Oh, pobre e tão desolado corpo! (Hall 1981,
186-87, grifo meu)
\end{quote}

O corpo de Stephen não é um corpo desejável, mas um ``corpo imperfeito,
faltoso, e defeituoso, despossuído, inadequado para carregar e
significar desejo'' (De Lauretis 1994, 212) No entanto, é também um
corpo racializado. Ao ser hiperbolicamente branco (``tão branco''), excede
a marca racial apropriada da feminilidade; sua monstruosidade é sua
brancura, sua força e auto"-suficiência. É um corpo tornado indesejável,
deplorável precisamente devido à sua excessiva participação no
patriarcado burguês branco. Pelo contrário, e no contexto das
subculturas lésbicas em Nova Iorque em meados dos anos 50, a negritude e
a homossexualidade não se articulam em termos de invisibilidade como
transparência, mas em termos de invisibilidade como morte. Audre Lorde
escreve:

\begin{quote}
``A maioria das lésbicas negras não assumiu o lesbianismo, reconhecendo
corretamente a falta de interesse da comunidade negra em nossa posição
bem como as diversas ameaças mais imediatas à nossa sobrevivência como
negras em uma sociedade racista. Já era difícil o suficiente ser negro,
ser negro e mulher, ser negra, mulher e gay. Ser negra, mulher, gay e
assumir a homossexualidade em um ambiente branco {[}\ldots{}{]} foi
considerado por muitas lésbicas negras \emph{como algo simplesmente
suicida}.'' (Lorde 1982, 151, grifo meu)
\end{quote}

É a partir desta conjunção de invisibilidade, apagamento e morte
inerente às estruturas coloniais e patriarcais que procuro rever o caso
da análise de Margarethe Csonka, de Freud. Meu objetivo é investigar
como os efeitos de uma supressão sistemática e violenta dos desejos
homossexuais femininos são operados no contexto da análise de
Margarethe. Em outras palavras, o que pode este caso de outro fantasma
lésbico da psicanálise nos dizer sobre o funcionamento da teorização de
Freud.

\section{Freud interseccional: Revisitando a psicogênese de um caso de
homossexualidade numa mulher }

No único caso analisado por Freud de uma mulher homosexual, publicado em
1920 e intitulado \emph{A psicogênese de um caso de homossexualidade
numa mulher}\footnote{O título registrado em inglês pela autora é
  \emph{The psychogenesis of a case of homosexuality in a woman}, e o
  título original do texto de Freud, em alemão, consultado na
  \emph{Studienausgabe}, é \emph{Über die Pcyhogenese eines Falles von
  weiblicher Homosexualität}. Para a versão em português dos trechos
  citados desse texto, optamos pela versão da Editora Autêntica na qual
  o título é \emph{Sobre a psicogênese de um caso de homossexualidade
  feminina}. {[}N.T.{]}}, de alguma forma a jovem homossexual é esboçada
como um sujeito paradoxal que odeia tanto homens quanto mulheres. (De
Lauretis 1999, 43) ``Furiosamente ressentida e amargurada'' com seu pai
porque ele não realiza seu desejo edipiano de que ele lhe desse um
filho, ela se comporta como muitos homens que, após uma primeira
experiência dolorosa, viram as costas para sempre ao sexo feminino
infiel e se tornam misóginos'' (Freud 1920, 157). A conclusão de Freud é
muito surpreendente, dada a bissexualidade originária que ele discutiu
nos \emph{Três ensaios} e o impacto de seu reconhecimento da natureza
polimorfamente perversa da sexualidade (Freud 1905, 191), que, como
observou a crítica feminista Jacqueline Rose, está estruturada em torno
das noções de impossibilidade e de fracasso da identidade sexual. (Rose
2005{[}1986{]}) Esta forclusão de qualquer figuração positiva da mulher
homossexual na psicanálise, no entanto, conduz a uma perda de
resiliência teórica; conduz a uma rigidez intelectual que espelha o
impacto das doutrinas coloniais e patriarcais que contaminam as
abordagens psicanalíticas da subjetividade com formas de poder que
silenciam ou suprimem a dimensão da ação.

\emph{A psicogênese de um caso de homossexualidade em mulheres} é um
caso amplamente negligenciado na história psicanalítica\footnote{A
  recente e imponente biografia de Élisabeth Roudinesco, \emph{Freud na
  sua época e em nosso tempo} (Zahar Editora), fornece, com base em
  depoimentos da própria paciente cujo nome era Margarethe Csonka, um
  panorama mais amplo a seu respeito (1900-1999). (Roudinesco 2016, p.
  475 da edição em inglês).}. Se alguma atenção lhe foi dedicada, isso
deve"-se à pesquisa gay, lésbica e \emph{queer}, que se dirigiu aos
pressupostos patriarcais e heteronormativos e à patologização da
homossexualidade tanto em sujeitos masculinos quanto femininos. É também
um caso inacabado e ambivalente. Embora não completamente elaborado
(Freud terminou a análise da jovem sugerindo que ela procurasse uma
analista mulher) e sem um nome fictício adequado como ocorria em seus
trabalhos anteriores (De Lauretis 1999, 39), este caso apresenta uma
conclusão bastante rígida e incontestada. Por trás do desejo homosexual
da jovem, Freud argumenta que existe uma atitude de desapontamento e
ódio para com ambos os pais, uma inveja dos pais, o casal procriador
heterossexual, que é a razão pela qual ela teria se afastado tanto dos
homens quanto da maternidade. Quando Margarethe compartilhou a
explicação que Freud esboçou para ela sobre sua homossexualidade com sua
amante, ela disse: ``isso é revoltante''. (Roudinesco 2016, 246)

Para Freud e o pai da menina, a escolha homossexual é uma escolha que
indica uma traição e uma rejeição da respeitabilidade patriarcal e
burguesa. ``O amor lésbico ameaça profundamente a ordem patriarcal'',
observa Pérez Foster. (Pérez Foster 1999, 130) Mas talvez não seja
apenas a ordem patriarcal. Nas fronteiras imperiais, Freud e o pai da
menina estão tomando a homossexualidade feminina como uma ``rejeição
narcisista'', já que a predisposição psíquica a uma sexualidade
heterossexual está ligada ao conformismo obrigatório às injunções da
classe média que estão essencialmente e obrigatoriamente alinhadas à
branquitude. Isso torna"-se mais evidente se nos concentrarmos na escolha
de objeto --- na pessoa amada escolhida pela moça e no modo pelo qual sua
família e seu analista a percebiam. O que levou Margarethe ao divã de
Freud, em primeiro lugar, foram as pressões de seu pai, já que fazia
vários anos que ela buscava o afeto de ``uma atriz de cinema'', de uma
``coquete'' e agora de uma ``\emph{cocotte}''\footnote{Optamos por manter o
  termo no original, sem tradução. Cf. nota 6, acima. {[}N.T.{]}}.
(Freud 1920, 161) Todas as três mulheres, argumenta Diana Fuss, ``ocupam
uma classe abaixo da menina, mas elas também representam coletivamente
uma classe de mulheres que ganham a vida independentemente, fora do
casamento e do contrato heterossexual.'' (Fuss 1995, 69) Dito de outro
modo, a moça apaixona"-se no domínio do seu gênero (biológico), mas fora
do domínio da sua classe social e da respeitabilidade racial. Como tal,
a homossexualidade de Margarethe insta os dois pais (Freud como o pai da
psicanálise e o pai de Anna, assim como o pai de sua paciente) a
renegociar a relação entre sua masculinidade e sua ``raça'', para se
reposicionarem contra a ``branquitude''. É por isso que Freud observa que
o pai nega o que ele, Freud, já sabe: que a psicanálise participa do
mesmo grau de ``menosprezo'' que a homossexualidade feminina em Viena ---
ela alia"-se aos desditosos --- e, ainda assim, isso não o ``deteve'' e
ele, então, voltou"-se para a nova prática ``em busca de auxílio''
(Freud, 1920, p. 160).

Ambos, no entanto, forcluem qualquer possibilidade de entendê"-la. Uma
das lacunas mais intrigantes neste caso é a falta de atenção a uma
transferência mortal que Margarethe põe em cena, mas que nenhum dos dois
``pais'' aceita. O desejo de seu pai é ver o desejo lésbico da filha
perecer, já que sua exibição desinibida de homoerotismo feminino em
público constitui uma ameaça ao narcisismo masculino. O desejo de Freud
não é curar sua homossexualidade --- ele declara isso impossível ---, mas
negar que qualquer forma de sofrimento lhe pudesse ser vinculada. Essa
dupla exclusão desses dois pais em relação à filha lésbica é melhor
delineada através do incidente que eventualmente a conduziu à análise em
primeiro lugar:

\begin{quote}
``Certo dia acabou ocorrendo o que, de fato, nessas circunstâncias teria
de acontecer, o pai encontrou pela rua sua filha em companhia daquela
dama de quem já havia tomado conhecimento. Ele passou por elas com um
olhar furioso \emph{que não anunciava nada de bom}. Imediatamente a
jovem correu e jogou"-se por cima do muro em direção à linha de trem
urbano que passava ali perto. Ela pagou por essa tentativa de suicídio
indubitavelmente séria com uma longa convalescença, mas, por sorte, sem
lesões permanentes.'' (Freud, 1920, p. 158-9, grifo meu)
\end{quote}

Apesar da ``tentativa de suicídio'' da garota (Freud, 1920, p. 148),
Freud diagnostica que a jovem não ``(\ldots{}) era doente --- ela não
sofria por razões internas, não reclamava sobre seu estado (\ldots{}).''
(Freud, 1920, p. 162) Parece"-me bastante surpreendente que a tentativa
de suicídio não seja sequer uma indicação de que a menina não esteja bem
do ponto de vista psíquico. Em vez disso, sua boa saúde psicológica é
justificada por seu desejo homossexual desinibido e desprovido de
remorso --- em outras palavras, se há algo errado com essa jovem, é sua
homossexualidade: ``(\ldots{}) a tarefa solicitada não era a de
solucionar um conflito, mas a de converter uma variante da organização
sexual para outra.'' (Freud, 1920, p. 162) Diante da recusa de Freud,
sentimo"-nos compelidas a formular a seguinte pergutna: teria ele
minimizado a tentativa de suicídio da menina precisamente porque
(inconscientemente) entendeu que ela não era o resultado de algum
conflito interno, mas de um olhar paternal assassino; um olhar que
desejava sugar a própria vitalidade de sua vida amorosa? Afinal, Freud
reconhece que ``a homossexualidade de sua filha tinha algo que
despertava a sua {[}do pai de Margarethe{]} completa amargura.'' (Freud,
1920, p. 160)

Desde seu artigo de 1910 sobre o distúrbio psicogênico da visão, Freud
estava bem ciente da força e do caráter punitivo do olhar invejoso.
Diferentemente de Lady Godiva, que percorreu nua uma cidade vazia, cujos
habitantes se esconderam atrás de venezianas fechadas, (Freud 1910) a
paciente de Freud aparece ``nas ruas mais frequentadas na companhia de
sua indesejável amiga'' (Freud 1920, 148) e é recebida não por um
voyeurista, mas por um olhar capaz de ferir. Esse olhar, como diz
Stephen Frosh, tem o efeito de ``sugar a vida de uma pessoa, de fazer
perder sangue, de torná"-la morta"-viva; de, sob a influência do desejo de
outro, transformá"-la em pedra, incapaz de fugir'' (Frosh 2013, 78). O
próprio Freud estava bem ciente do poder desse tipo de olhar. Em 1912,
durante um congresso de psicanálise em Munique e depois de uma discussão
animada com Carl Jung, Freud experimentou um dos seus ``desmaios'' (Freud,
1912, 523-25). Freud interpreta sua perda temporária de consciência como
resposta a um desejo de morte inconscientemente transmitido por seu
antigo e amado discípulo, sobre quem ele investiu --- e desinvestiu ---
suas expectativas relativamente à continuação da psicanálise. Mas, na
atitude ``um pouco neurótica'' de Freud, como ele a caracteriza mais tarde
em carta a Jung, é o filho assassino que deseja a morte de seu pai.
(Freud 1912, 524) O que impede Freud de seguir uma linha de análise
semelhante no caso da garota homossexual, sua paciente? Por que o fato
de ela ter se jogado ``por cima do muro'' ao se deparar com o ``olhar
raivoso'' de seu pai não é, afinal de contas, a resposta a um
\emph{desejo de morte} similar por parte de um patriarca heterossexual e
dirigido a sua filha rebelde e desviante? Dito de outro modo, o que
alimenta a cegueira que impede Freud de reconhecer a inveja no olhar
patriarcal? E ainda mais, em uma reviravolta interessante, ele até
inverte os efeitos do olhar paterno invejoso, como narra Roudinesco; no
final do tratamento, Freud diz à garota: ``Você tem olhos tão astutos
que eu não gostaria de encontrar você na rua se eu fosse seu inimigo.''
(Roudinesco 2016, 246)

\section{Conclusão}

No início deste capítulo, argumentei que a posição irresoluta da
homossexualidade feminina dentro da psicanálise é assombrada por
histórias obscuras, silenciadas e não ditas de desejos lésbicos entre
alguns dos membros fundadores da disciplina --- especialmente, a da
relação de Freud com sua filha Anna, e sua homossexualidade repudiada.
Considerando a relação ambígua de Anna Freud com Dorothy Burlingham
através da historiografia psicanalítica, dois grandes pontos emergem. Em
primeiro lugar, que é impossível considerar os desejos lésbicos apenas
em relação ao poder patriarcal. Em vez disso, precisamos considerar a
supressão da homossexualidade feminina através de uma perspectiva
interseccional. E, em segundo lugar, que o silenciamento e a
indizibilidade dos desejos lésbicos na psicanálise nos ajudam a dar
corpo ao modo como opera o narcisismo colonial e patriarcal. Revisitando
assim o caso de 1920 de Freud, podemos indicar o que tem sido o trabalho
dos fantasmas lésbicos reprimidos na psicanálise: reservar, salvaguardar
e naturalizar o lugar da masculinidade branca e ocultar a
vulnerabilidade de seu surgimento. Nas melhores palavras possíveis: ``Um
indivíduo feminino que se sente masculino e amou de maneira masculina
dificilmente se deixará forçar no papel feminino, se tiver de pagar pela
transformação, nada vantajosa em todos os aspectos, com a renúncia à
maternidade.'' (Freud, 1920, p. 189)

\section{Referências bibliográficas}

Aldrich, Robert. 2003. \emph{Colonialism and Homosexuality}. New York:
Routledge.

Bush, Barbara. 2007. ``Gender and Empire: The Twentieth Century.'' In
\emph{Gender and Empire}, edited by Philippa Levine, 77--111. Oxford:
Oxford University Press.

De Lauretis, Teresa. 1994. \emph{The Practice of Love: Lesbian Sexuality
and Perverse Desire}. Bloomington and Indianapolis: Indiana University
Press.

\_\_\_\_\_\_\_\_\_. 1999. ``Letter to an Unknown Woman.'' In \emph{That Obscure
Subject of Desire: Freud's Female Homosexual Revisited}, edited by
Ronnie C. Lesser and Erica Schoenberg, 37--53. New York and London:
Routledge.

Freud, Sigmund. 1905. ``Three Essays on the Theory of Sexuality.''
\emph{The Standard Edition of the Complete Psychological Works of
Sigmund Freud, Volume \versal{VII} (1901-1905): A Case of Hysteria, Three Essays
on Sexuality and Other Works}, 123--246.

\_\_\_\_\_\_\_\_\_. 1910. ``The Psycho"-Analytic View of Psychogenic Disturbance
of Vision.'' \emph{The Standard Edition of the Complete Psychological
Works of Sigmund Freud, Volume \versal{XI} (1910): Five Lectures on
Psycho"-Analysis, Leonardo Da Vinci and Other Works}, 209--18.

\_\_\_\_\_\_\_\_\_. 1912. \emph{The Freud/Jung Letters: The Correspondence
Between Sigmund Freud and C. G. Jung}.

\_\_\_\_\_\_\_\_\_. 1920. ``Sobre a psicogênese de um caso de homossexualidade
feminina'' (Trad. M. R. S. Moraes). Em: \emph{Neurose, psicose,
perversão}. Belo Horizonte: Autência, 2016. 157-192. {[}Versão em
ingles, consultada pela autora: ``The Psychogenesis of a Case of
Homosexuality in a Woman.'' \emph{The Standard Edition of the Complete
Psychological Works of Sigmund Freud, Volume \versal{XVIII} (1920-1922): Beyond
the Pleasure Principle, Group Psychology and Other Works}, 145--72.{]}

Frosh, Stephen. 2009. \emph{Hate and the Jewish Science: Anti"-Semitism,
Nazism and Psychoanalysis}. 2nd ed. New York: Palgrave Macmillan.

\_\_\_\_\_\_\_\_\_. 2013. \emph{Hauntings: Psychoanalysis and Ghostly
Transmissions}. Studies in the Psychosocial. \versal{UK}: Palgrave Macmillan.

\_\_\_\_\_\_\_\_\_. 2017. ``Primitivity and Violence: Traces of the Unconscious
in Psychoanalysis.'' \emph{Journal of Theoretical and Philosophical
Psychology} 37 (1): 37--47.

Fuss, Diana. 1995. \emph{Identification Papers: Readings on
Psychoanalysis, Sexuality, and Culture}. 1st ed. London and New York:
Routledge.

\_\_\_\_\_\_\_\_\_. 1999. ``Fallen Women: `The Psychogenesis of a Case of
Homosexuality in a Woman.''' In \emph{That Obscure Subject of Desire:
Freud's Female Homosexual Revisited}, edited by Ronnie C. Lesser and
Erica Schoenberg, 54--75. New York and London: Routledge.

Gay, Peter. 2006. \emph{Freud: A Life for Our Time}. Revised. New York
and London: Norton.

Gilman, Sander. 1986. ``Black Bodies, White Bodies.'' In \emph{Race,
Writinf and Difference}, edited by L. Henry and Jr Gates. Chicago:
Chicago University Press.

Grosskurth, Phyllis. 1986. \emph{Melanie Klein: Her World and Her Work}.
Cambridge \versal{MA}: Harvard University Press.

Hall, Radclyffe. 1981. \emph{The Well of Loneliness}. New York: Avon
Books.

Klein, Melanie. 2013. ``The Autobiography of Melanie Klein.'' Edited by
Janet Sayers and John Forrester. \emph{Psychoanalysis and History} 15:
127--63.

Kristeva, Julia. 2001. \emph{Melanie Klein}. Translated by Ross
Guberman. Vol. 2. Female Genius: Life, Madness, Words-\/-Hannah Arendt, %%%\/??
Melanie Klein, Colette vols. New York: Columbia University Press.

Lorde, Audre. 1982. \emph{Zami: A New Spelling of My Name}. Toronto:
Crossing Press.

McClintock, Anne. 1995. \emph{Imperial Leather: Race, Gender and
Sexuality in the Colonial Contest}. New York and London: Routledge.

O'Connor, Noreen, and Joanna Ryan. 1993. \emph{Wild Desires and Mistaken
Identities: Lesbianism and Psychoanalysis}. London: Virago Press.

Pérez Foster, RoseMarie. 1999. ``Las Mujeres: Women Speak to the Worf of
the Father.'' In \emph{That Obscure Subject of Desire: Freud's Female
Homosexual Revisited}, edited by Ronnie C. Lesser and Erica Schoenberg,
130--40. New York and London: Routledge.

Rose, Jacqueline. 1993. \emph{Why War? Psychoanalysis, Politics, and the
Return to Melanie Klein}. The Bucknell Lectures in Literary Theory.
Oxford \versal{UK}\& Cambridge \versal{USA}: Blackwell Publishers.

\_\_\_\_\_\_\_\_\_. 2005. \emph{Sexuality in the Field of Vision}. London: Verso.

Roudinesco, Élisabeth. 2016. \emph{Freud: In His Time and Ours}.
Translated by Catherine Porter. Cambridge \versal{MA}: Harvard University Press.

Smith, Jill Suzanne. 2013. \emph{Berlin Coquette: Prostitution and the
New German Woman, 1890-1933}. Ithaca and London: Cornell University
Press.

Stewart"-Steinberg, Suzanne. 2011. \emph{Impious Fidelity: Anna Freud,
Psychoanalysis, Politics}. Ithaca and London: Cornell University Press.

Stoler, Ann Laura. 2000. ``Cultivating Bourgeois Bodies and Racial
Selves.'' In \emph{Cultures of Empire\,: Colonizers in Britain and the
Empire in the Nineteenth and Twentieth Centuries}, edited by Catherine
Hall, 87--119. Manchester: Manchester University Press.

Worthington, Anne E. 2011. ``Female Homosexuality: Psychoanalysis and
Queer Theory.'' Middlesex University.

\chapter*{Fricção entre corpo e palavra: crítica ao \emph{Moisés} de Freud e Lacan}
\addcontentsline{toc}{chapter}{Fricção entre corpo e palavra: crítica ao \emph{Moisés} de Freud e Lacan,\\ \footnotesize\emph{por Alessandra Parente}}
\hedramarkboth{Fricção entre corpo e palavra}{}

\begin{flushright}
\emph{Alessandra Parente}
\end{flushright}

\epigraph{\emph{Em todo traje hei de sentir as penas,\\
Da vida mísera o cortejo.\\
Sou velho, pra brincar apenas,\\
Jovem sou, pra ser sem desejo.\\
Que pode, Fausto, o mundo dar"-te?\\
Deves privar"-te, só privar"-te!\\
É o eterno canto, este, que assim\\
A todo ouvido vibra e ecoa,\\
Que a vida inteira, até o seu fim,\\
Cada hora, rouca, nos entoa.\\
Só com pavor desperto de manhã,\\
Quase a gemer de amargo dó,\\
Ao ver o dia, que, em fugida vã,\\
Não me cumpre um desejo, nem um só;\\
Que até o presságio de algum gozo\\
Com fútil critiquice exclui,\\
Que as criações de meu espírito audacioso\\
Com farsas mil da vida obstrui.}}{Fausto, Goethe}

Uma lupa na parte \versal{III} de \emph{O homem Moisés e a religião monoteísta} é
capaz de iluminar tópicos centrais para o debate do patriarcado na
psicanálise. Em grande medida, as relações entre feminismo e psicanálise
desenham"-se ali e nada têm de desprezíveis para os dias de hoje. Freud
contrapõe ao matriarcado o patriarcado, ressaltando a superioridade
deste sob vários aspectos. Olhando de modo mais fino, porém, a questão é
complexa e exige acuidade na análise \emph{nachträglich} que se pode
empreender do embate entre os dois sistemas. Quando se observa mais
detidamente, são três as estruturas sociais colocadas em seu
\emph{Moisés}:

\begin{enumerate}
\def\labelenumi{\arabic{enumi})}
\item
  A conhecida versão do Pai primevo que goza e governa sua horda pela
  violência e pela força.
\item
  O poder das mulheres numa ordem matriarcal que teria, em seu fundo,
  reminiscências nostálgicas do Pai protetor e forte.
\item
  A ordem dos irmãos com a religião totêmica, logo depois substituída
  por um Deus com feições humanas, ocupando o lugar simbólico do Pai.
  Essa versão é aquela que daria consistência à forma mais evoluída aos
  olhos de Freud: a abstrata regulação divina do monoteísmo judaico.
\end{enumerate}

A proibição de representar imagens divinas e conceder nome pronunciável
à \versal{YHWY} delineiam os traços do povo judeu. Nele, a percepção sensória
recebe lugar secundário e a abstração do pensamento torna"-se valor a ser
cultivado. Em suma: judeus seriam mais afeitos às qualidades
intelectuais, superiores. O resultado seria, regozija"-se Freud em seu
\emph{Moisés}, o ``triunfo da intelectualidade sobre a sensualidade, ou,
estritamente falando, uma renúncia instintual''.

Num salto inesperado, Freud associa essa suposta superioridade da
intelectualidade, ligada à religião mosaica, ao caráter patriarcal do
monoteísmo:

\begin{quote}
Sob a influência de fatores externos que não precisamos acompanhar aqui,
e que em parte também não são suficientemente conhecidos, aconteceu que
a ordem social matriarcal foi substituída pela patriarcal, ao que
naturalmente estava ligada uma reviravolta das relações jurídicas até
então existentes. Acredita"-se perceber o eco dessa revolução ainda na
\emph{Oréstia}, de Ésquilo. Mas essa mudança da mãe para o pai indica,
além disso, uma vitória da espiritualidade sobre a sensualidade, ou
seja, um \emph{progresso cultural}, pois a maternidade é demonstrada
pelo testemunho dos sentidos, enquanto a paternidade é uma suposição
construída com base numa conclusão e numa premissa. A tomada de partido
que eleva o processo de pensamento acima da percepção sensível dá provas
de um passo com sérias consequências. (Freud, 1939/2014, p. 157, grifo
meu).
\end{quote}

Ora, é forçoso admitir que o ``progresso cultural'' da passagem ao
patriarcado cumpriu muito mal sua promessa. Para Freud, além de a
abstração derivar do fato de que ``a maternidade é demonstrada pelo
testemunho dos sentidos'', diferentemente da paternidade que é ``uma
suposição construída com base numa conclusão e numa premissa'', ela
adviria da ausência do nome e da imagem de Deus que impõe a negatividade
no processo de simbolização. Ela também seria uma libertação dos judeus
da ``escravidão dos sentidos''. Desmaterializado, Deus alimentaria o
interesse espiritual, convertido na literatura sagrada compartilhada por
esse povo disperso pelas terras do deserto. Os efeitos dessa tradição
intelectual seriam o controle da brutalidade e da tendência à violência,
ligadas à força muscular como ideal.

É interessante avançar nessas conexões freudianas. Se recorrermos a uma
passagem reveladora na \emph{\versal{X} Conferência introdutória sobre
psicanálise}, Freud interpreta a madeira como ``símbolo feminino
materno'' (Freud, 1916/1996, p. 161) e indica a conexão etimológica das
palavras \emph{matéria} e \emph{mãe}. \emph{Matéria}, explica, deriva de
\emph{mater}, mãe. A matéria (\emph{Stoff}), da qual algo sai vitorioso,
é igualmente sua participação materna. Os rumos dessas analogias e
cruzamentos avança nos seguintes termos:

\begin{quote}
E, por falar em madeira, é difícil compreender como esse material veio a
representar o que é materno. No entanto, nisso a filologia comparada
pode vir em nosso auxílio. Nossa palavra alemã \emph{Holz} parece provir
da mesma raiz da ``υλη {[}hulé{]}'' grega, significando material,
matéria"-prima. Esse parece ser um exemplo da ocorrência não rara de um
nome genérico de um material vir a ser, afinal, reservado a algum
material determinado. Ora, existe no Atlântico uma ilha chamada
``Madeira''. Este nome lhe foi dado pelos portugueses quando a
descobriram, porque naquela época estava toda recoberta de florestas.
Pois na língua portuguesa ``madeira'' está relacionada a ``floresta''.
Os senhores observam, porém, que ``madeira'' é apenas uma forma
ligeiramente modificada da palavra latina \emph{materia}, que, mais uma
vez, significa ``material'' em geral. Contudo, \emph{materia} é derivada
de \emph{mater}, ``mãe'': o material do qual tudo é feito, por assim
dizer, a mãe de tudo. Esse conceito antigo da coisa sobrevive, portanto,
no uso simbólico de madeira como ``mulher'' ou ``mãe'' (Freud,
1916/1996, p. 161).
\end{quote}

Entre \emph{Moisés} e essas formulações da conferência, expostas acima,
temos em Freud uma clara antítese: de um lado a abstração, ligada à
lógica patrilinear, de outro a matéria, associada à estrutura
matrilinear. No fundo, essas são raízes de formulações metapsicológicas
sofisticadas em Freud e pós"-freudianos: no complexo de Édipo, trata"-se
do terceiro elemento --- abstrato --- que se interpõe na relação tida como
fusional mãe"-bebê --- carnal ou material. O Nome"-do"-Pai e objeto fálico
--- vazio --- são apenas poucos exemplos daquilo que se arma a partir
dessas articulações freudianas no arcabouço teórico psicanalítico e,
consequentemente, no repertório que se configura na escuta clínica desde
então.

Por mais que se contemporize o embate\footnote{Essas questões são parte
  dos debates de feministas e de psicanalistas mulheres há bastante
  tempo. (Cf., por exemplo, Arán, M. (2003), Fraser, N. (2017),
  Federici, S. (2004/2017), Silveira, L. (2017)).}, o maior problema
concentra"-se ainda no fato de que sistemas um tanto questionáveis estão
fundados exatamente na antítese entre matéria e abstração e na estrutura
social moldada na combinação desses ingredientes antagônicos. Nos dias
atuais, elucubrações teóricas desse teor apontam para problemas
urgentes, como: será mesmo que a abstração --- e aqui entram a
racionalidade instrumental, tal como desenhada por frankfurtianos, a
virtualidade dos algoritmos, o poder fundado nas leis jurídicas, a
exploração da natureza pela técnica, o desprezo pelas questões
climáticas em nome do capital, a hierarquia mente e corpo e suas
consequências opressivas etc. --- deve prosseguir reivindicando
superioridade frente à materialidade --- o corpo, a coletividade
auto"-reguladora, a natureza? Por outro lado, seria o caso de
simplesmente invertermos os termos, colocando imediatamente a
materialidade como superior à abstração? Seria pertinente manter tal
subdivisão? São essas as questões que orientarão minhas ponderações
neste capítulo.

\section{Das tripas, coração}

\begin{flushright}
\footnotesize
\emph{A diferença sexual\\
é uma heterodivisão\\
do corpo na qual\\
a simetria não é possível}\\
Paul Preciado
\end{flushright}

Em \emph{As formações do inconsciente}, Seminário 5, Lacan diz: ``É
claro que, nos dados iniciais da utilização do bezerro de ouro, a noção
da \emph{matéria} está implicada''. Bezerro de ouro de um lado,
inscrição pelo dedo de Deus --- isto é, sem mediações corpóreas --- dos
mandamentos na tábua da Lei, de outro. Arão contra Moisés. Idolatria da
imagem contra negatividade. ``Recusa {[}\ldots{}{]} {[}ao{]} que dá valor ao
bezerro de ouro'' em nome de um ``para"-além'', lembra"-se Lacan.

Indo agora direto ao ponto: na psicanálise, onde estaria o ``para"-além''
da ``matéria''? Lacan titubeia em seu O seminário, livro 7 \emph{A ética
da psicanálise} --- a sublimação, afinal, gira em torno da \emph{Coisa}
ou do \emph{Vazio}? \emph{Carne} ou \emph{Nada}? Equivalentes?
Complementares? Entrelaçados? Apartados? Depois de debates densos entre
feministas da terceira onda e membros da comunidade \versal{LGBTQI}+, ``a
identidade sexual não é a expressão instintiva da verdade pré"-discursiva
da carne, e sim um efeito de reinscrição das práticas de gênero no
corpo'' (Preciado, 2004/2014, p. 29). Para Preciado, ``o gênero é, antes
de tudo, prostético, ou seja, não se dá senão na materialidade dos
corpos.''. Em suma: ``seria puramente construído e ao mesmo tempo
inteiramente orgânico'', escapando das ``falsas dicotomias metafísicas
entre o corpo e a alma, a forma e a matéria''. (Preciado, 2004/2014, p.
29).

De Freud a Lacan, o objeto fálico torna"-se substrato da tensão entre
carne e palavra. Resultado: nada do pênis, aquele entumecido e
penetrante. Evidentemente não ignoramos, nesse contexto, outras
considerações como as de Preciado, para quem: ``os órgãos sexuais não
existem em si. Os órgãos que reconhecemos como naturalmente sexuais já
são o produto de uma tecnologia sofisticada que prescreve o contexto em
que os órgãos adquirem sua significação (relações sexuais) e de que se
utilizam com propriedade, de acordo com sua `natureza' (relações
heterossexuais)'' (Preciado, 2004/2014, p. 31).

Entretanto, resgatar o raciocínio ideológico que ronda inconscientemente
uma época e reconstituí"-lo como o fundo das articulações teóricas da
psicanálise servirá justamente para reconhecer certos limites de
elaborações psicanalíticas que podem e devem ser ultrapassadas. Falo,
objeto \emph{a}, ideais do eu, Nome"-do"-Pai. Toda uma estrutura teórica
que ainda persiste --- apesar das ressalvas, apesar das evasivas --- para
subscrever uma lógica patriarcal.

De \emph{Totem e tabu}, no qual a mulher não tem lugar político, às
familiares construções metapsicológicas, que colocam na mãe as mais
diferentes razões da loucura --- a velha falta do Édipo ---, é possível
pensar algo inusitado: os alicerces mais arcaicos da psicanálise foram
erguidos por alguns punhos quiromaníacos, que depositavam tintas no
papel como rastros da retração, recuo à volúpia efetiva ante o corpo da
mulher.

Para também incrementar nossa verdade, recorramos ao velho expediente
freudiano, citando as palavras de Mefistófoles no \emph{Fausto} de
Goethe: ``Gris, caro amigo, é toda teoria,/ E verde a áurea árvore da
vida.''. (Fausto, 1808/2003, p. 195). Verso antológico que, como explica
Mazzari, ressoa do \emph{Gênesis} (2:9). Ali a ``árvore da vida'' foi
obra divina no meio do jardim do Éden e só o fruto proibido, oferecido
por Eva a Adão, conduziu ao conhecimento do bem e do mal --- já sabemos,
de cor e salteado, o roteiro dessa história: da indecorosa sedução de
Eva à desventurosa entrega carnal da Margarida de \emph{Fausto}, as
mulheres e seus corpos conduzem a todo o mal das civilizações. Histórias
de pauladas na carne, no corpo, nas curvas da sedução, na
sensorialidade.

Lacan, sem ser o único, é bom frisar, leva adiante algumas das premissas
colocadas na parte \versal{III} do \emph{Moisés} de Freud. ``Antes de mais
nada'', diz ele, no velho estilo, ``{[}o pai{]} interdita a mãe. Esse é
o fundamento, o princípio do complexo de Édipo''. (Lacan, 1957-8/1999,
p. 174). Interdita"-a por qual razão? Ela é \emph{perigo}, carne
tentadora que convida ao incesto. Seduz de modo englobante o corpo do
filho\footnote{A título de exemplo, vejamos o que Lacan diz em O
  seminário, livro 17 \emph{O avesso da psicanálise}: o papel da mãe é o
  desejo da mãe. É capital. O desejo da mãe não é algo que se possa
  suportar assim, que lhes seja indiferente. Carreia sempre estragos. Um
  grande crocodilo em cuja boca vocês estão - a mãe é isso. Não se sabe
  o que lhe pode dar na telha, de estalo fechar sua bocarra. O desejo da
  mãe é isso. (1969-0/1992, p. 118).}, caso a interdição paterna, ``base
de nossa relação com a cultura'' (Lacan, 1957-8/1999, p. 180), não o
salve de suas garras.

Sob a aura do \emph{falo}, a carne da mulher jaz esquartejada nesta
cultura de tons patriarcais ou, em casos menos piores, rendida aos
sintomas histéricos. Daí que não seja tão absurda a seguinte imagem: sob
o ar cinzento das baforadas de charutos, dadas entre entusiasmadas
reflexões culturais do século passado, na Bergasse 19, depositavam"-se
apetitosas fibras de carne, que aguardavam ansiosas ao mais fogoso ardor
nos quartos da Viena --- pecado?

Dito de outro modo: talvez pairassem sobre corpos prostrados e gozos
sintomáticos das histéricas de Freud, ruminações, nuvens de fumaça e
pensamento que emanavam de corpos mofados de obsessivos, dispostos em
torno do discurso teórico infindável sobre a sexualidade. Nas
quartas"-feiras, num longínquo Império, arquiteta"-se, numa atmosfera
densa e empesteada, uma das mais importantes filhas da cultura moderna:
a psicanálise.

Cena que permite levantar uma pergunta singela e palpável: esvaziar o
pênis corpóreo, sem dúvida desejado por muitas mulheres em sua mais
absoluta concretude entumecida, e revesti"-lo de \emph{penetrante} poder
pelo termo \emph{falo}, que ainda remete ao órgão genital masculino, mas
agora ocupando lugar eminente, não seria estratégia psíquica um tanto
defensiva?

Explico: inseguranças de toda sorte sobre medidas, sobre potência, sobre
fracasso, sobre tempo de ereção, sobre desempenho, sobre satisfazer
mulheres sedentas, entregues aos gozos convertidos no corpo por falta de
vida desejosa, não explicariam também um tantinho a escolha tão curiosa
pelo termo \emph{falo}\footnote{A este respeito Cf. Silveira, L. (2017)
  \textless{}\emph{https://bit.ly/2lZiUf4}\textgreater{}}?
Explico ainda melhor aos desentendidos, tim tim por tim tim: o intenso
trabalho intelectual na elaboração de teorias sobre o \emph{poder do
falo} não seria o anverso sombrio de corpos languidos e histéricos,
ávidos por sexo e à espera de um pênis robusto e vívido, que nunca
chega?

É bom lembrar como a psicanálise observa as estruturas histérica e
obsessiva. Emblema vivaz do desejo, o corpo enfático do sujeito
histérico se oferece ao Outro, carregando o cerne de seus impasses. Do
corpo retraído e mortificado do obsessivo, por outro lado, apaga"-se
qualquer tipo de rastro do corpo sexual. Como demanda de amor dirigida
ao Outro, a histérica apela pelo reconhecimento de sua existência
através de seu corpo, ao passo que o corpo do obsessivo é envolvido por
uma couraça intransponível e purificada, que, contudo, imprevisivelmente
o \emph{trai}, o faz tropeçar naquilo que mais quer esconder.

Sendo breve: a hipótese que agora se desenha de modo mais preciso é a de
que o uso do termo \emph{falo} trai as melhores intenções daqueles
senhores e estremece alguns dos mais fundamentais pilares da
psicanálise. Um lastro de sujeira deixado atrás da eliminação do órgão.
Corpo histérico que \emph{falA}: ``quero com você!''
\emph{FalO}\footnote{Importante lembrar que esse tom jocoso só pode ser
  feito se consideramos os termos no idioma português.}, o pênis
abstraído das reflexões intelectualizadas, que responde: ``Com você, tão
apetitosa? Agora, não! Ai, que medo, sozinho é melhor\ldots{}''. Diante dessa
pequena tese da traição, do tropeço do obsessivo, a tradução para o
português da frase lacaniana ``\emph{Moi, la Verité, Je parle}'' vem a
calhar: ``Eu, a verdade, \emph{falo}''. Que verdade do eu
{[}\emph{moi}{]} o eu {[}\emph{Je}{]} fala{[}o{]}?

Negação do sexo, da fome, da carne. Por medo? Do que exatamente? A
cavernosa e misteriosa escuridão? Os dentes que podem arrancar o membro?
Nas atas das reuniões de quarta"-feira, o exemplo --- entre vários outros
que poderiam ser citados --- é de um caso atendido por Adler: ``À noite,
uma paciente acorda de um sonho e percebe que havia mordido o dedo até
sangrar. De acordo com a interpretação, o dedo representa o pênis (como
em Orestes), e o ato sintomático sugere uma defesa contra a perversão
oral.''. (Ata da reunião de 10 de outubro de 1906/2015).

A eterna insatisfação das histéricas não resultaria também de recuos
obsessivos e amedrontados ante o buraco, a vulva --- o \emph{Vazio} --- ou
a \emph{Coisa}, de onde corre sangue e gozo? A Coisa {[}\emph{das
Ding}{]} --- kantiana ou lacaniana, pouco importa --- pode ter mais carne
e sangue do que volteios teóricos sobre o desejo ou sobre a verdade
filosófica gostariam de admitir. Fugir dela implica constantemente
trombá"-la de frente. Domesticá"-la pela palavra só faz abri"-la mais sob
forma de ferida. É certo que Lacan, diferentemente dos filósofos
modernos, escuta a voz da carne, dá a ela lugar de existência, contanto
que ela permaneça cativa enquanto resto, resíduo de códigos bem
estabelecidos pela lógica estrutural e linguística.

De todo modo, o que se tem é um claro recuo das vias de fato, capaz de
realizar parcialmente o desejo, para uma punhetagem com cores obsessivas
do intelectual, que nunca chega ao ponto (G?). Como manifestação em ato
de uma subjetividade, o discurso, tal como pensado por Lacan, a tudo
envolve --- nada lhe escapa\footnote{Aqui seria necessário contrapor a
  teoria da linguagem de Lacan e de Walter Benjamin (1916/2011),
  trabalho que terei que desenvolver em outra oportunidade. Vale
  adiantar que em Benjamin há uma linguagem das coisas e dos animais que
  é transfigurada em discurso pelos humanos, mas permanece como
  existente, apesar de não capturada pela linguagem humana --- a
  linguagem em geral não está num para"-além, mas nas coisas --- inclusive
  nossos corpos --- que também se expressam.}. Preciosismos linguísticos
lustram o \emph{falo} até que ele fique reluzente. Desde o Levítico
(15:1), Deus diz a Moisés e a Arão para que exijam dos homens sua
purificação. Impurezas que brotam do toque nas mulheres em seu período
menstrual, de fluxos que exalam de seus corpos, do sêmen que
eventualmente escapa de seu pênis. Ainda estamos às voltas com o gesto
de expurgar os sinais de existência de um corpo que fala para além do
discurso? Enaltecer o \emph{falo} não seria justamente uma maneira de
repetir tal gesto?

Sob novo ângulo a hipótese reformula"-se ainda mais uma vez: os homens
pensam, escrevem, burilam, escavam, retirando carne e sangue do pênis e
afastam"-se da cavidade vaginal, que pode conter todas as impurezas.
Arquitetam em conjunto o \emph{falo}, pênis livre de incômodas ou
insuportáveis máculas. Elevado às alturas, os membros dos moços
amedrontados mantêm a potência imaginária. Lógica compensatória, que não
é minha. Também está registrada nas Atas de quarta"-feira e sai da boca
de Adler, mas recebe estímulos e reiterações de grande parte dos
ouvintes, principalmente de Freud: ``{[}\ldots{}{]} tornar funcional o órgão
inferior {[}\ldots{}{]} acontece por meio da compensação: a inferioridade do
órgão {[}alguma insuficiência ou deficiência nele{]} é compensada por
maior atividade cerebral''. (Ata da reunião de 07 de novembro de
1906/2015).

Seria, pois, compensação a invenção da potência inveterada do
resplandecente \emph{falo}? Atividade cerebral que o aperfeiçoa até o
mais almejado objeto, aquele que causa o desejo e em torno do qual todos
giram? Um delírio de grandeza às avessas, como sugerem os colegas das
quartas"-feiras, em atas subsequentes à citada acima? Temores sobre a
insuficiência do órgão não seriam proporcionais às articulações
fantasmáticas capazes de torná"-lo grandioso em seu caráter justamente
\emph{Vazio} --- traindo sua protuberância pela memória inescapável do
órgão oco da mulher, agora metamorfoseado? Não resultaria também daí um
acúmulo de substância tóxica, oriunda dos desvios da meta sexual
sublimatória? Aliás, é desse veneno que padece Fausto, o personagem que
espelha o drama da modernidade europeia. A entrega de seu cultivadíssimo
e sublimado espírito a Mefistófeles, em nome de um pouco de corpo, um
pouco de sensorialidade, um pouco de sexo, um pouco de êxtase, embotados
ao longo de anos de sábias pesquisas intelectuais, não denunciaria os
limites da sublimação e os resquícios de investimentos exacerbados em
objetos ideais, fálicos?

Sem esse viés aqui assumido, Piera Aulagnier, em seu discurso na lição
\versal{XVIII} de O seminário, livro 9, \emph{A identificação}, descreve assim o
obsessivo:

\begin{quote}
O sadismo está longe de ser sempre desconhecido, ou sempre controlado,
no obsessivo. O que ele significa no obsessivo é, sim, a persistência
daquilo que se chama de relação anal, ou seja, uma relação onde se trata
de possuir ou de ser possuído, uma relação onde o amor que se
experimenta, ou do qual se é o objeto, só pode ser significado, para o
sujeito, em função dessa possessão que pode, justamente, ir até à
destruição do objeto. O obsessivo, poderíamos dizer, é, de fato, aquele
que castiga bem porque ama bem; é aquele para quem a surra do pai
permaneceu como a marca privilegiada de seu amor, e que busca sempre
alguém a quem dá"-la ou de quem recebê"-la. Mas, tendo"-a recebido ou dado,
tendo"-se assegurado de que o amam, é num outro tipo de relação com o
mesmo objeto que ele buscará o gozo {[}\ldots{}{]}.
(Aulagnier, 1961-2/2003,
p. 287-8).
\end{quote}

Não ressoa na história do patriarcado algo de familiar desse sadismo do
obsessivo, que persiste na relação anal? Não é novidade que na história
do patriarcado o amor insiste em ser jogo de possessão até a destruição
do objeto ilusoriamente conquistado, possuído. Do castigo pelo bem e em
nome do amor assistimos até mesmo aos mais atrozes fenômenos sociais
como inquisição, escravidão, colonialismo, supremacismo.

Voltando às teses apresentadas no \emph{Moisés} de Freud: aquele Deus
inominável e irrepresentável, que alimentaria a tradição intelectual e
que controlaria a brutalidade e a tendência à violência, ligadas à força
muscular como ideal, acaba por reverter"-se em\ldots{} mais violência, só que
trasvestida, então, em leis jurídicas abstratas, em teorias elevadas, em
objetos fálicos. Desvios e desvios da meta sexual, desvios e desvios do
alvo, contornos e contornos do vazio --- o vaso, a dama, \emph{das Ding}
--- para subjugar o corpo a outra lógica. Bordejar, fazer curvas
moebianas rumo ao infinito de voltas e o alvo nunca alcançado: um falo
retumbante nos fantasmas e na ordem psíquica.

Salve"-me gozo, livre"-me desse mais"-de"-gozar!

\section{\versal{YHWY} e o Nome-do-Pai}

\begin{flushright}
\footnotesize
\emph{{[}\ldots{}{]} nós {[}psicanalistas{]} sempre nos distinguimos\\
por manipular os corpos através das palavras.\\
No entanto, no limite, agora já não haveria\\
mais nenhuma razão para acreditar \\
que não devamos fazer o contrário: \\
manipular as palavras em função \\
dos corpos ou através da imagem.}\\
Tania Coelho dos Santos
\end{flushright}

Recorrer ao gozo como expediente de acesso ao corpo não é saída sem
consequências --- lembremo"-nos de que, na lógica psicanalítica, quem goza
é apenas o Pai primevo em toda sua arbitrariedade. Goza, aliás, de todas
as mulheres. Infelizmente não se resolvem décadas ou séculos de acúmulo
tóxico sublimatório de maneira direta, isto é, restituindo ao corpo e à
materialidade seu lugar central de importância. Mesmo porque, dentro da
tradição filosófica, a psicanálise representa o esforço revolucionário
de trazer, para dentro do pensamento europeu e universal, o caráter
decisivo do desejo erótico, da sexualidade em suas mais amplas
variações, das pulsões, dos objetos parciais. Não é o caso, então, de
espezinhar, de maneira raivosa e com métodos hermenêuticos
psicanalíticos, que, diga"-se, existem graças a esses senhores um pouco
acanhados, nas eventuais fraquezas ou nos possíveis sintomas que
eventualmente carregavam no outro século. É preciso ter em mente que
eles ergueram, talvez justamente em virtude desses traços frágeis, um
dos mais significativos edifícios da cultura, que, embora reitere alguns
de seus preconceitos, é sobretudo capaz de resistir a diferentes modos
de opressão nela existentes.

Entretanto, se temos em Jacques Lacan\footnote{Renunciamos desde já às
  distinções feitas pelos lacanianos entre as diferentes partes de seu
  ensino. Seguiremos sua letra de forma a perseguir os caminhos que aqui
  interessam --- nenhuma lealdade ao seu próprio percurso será, por
  conseguinte, assumida e deixamos essa tarefa aos comentadores mais
  aptos à tarefa da fidelidade.}, e em alguns de seus herdeiros, níveis
acentuados de abstrações de questões, cujas raízes tentamos apontar, é
ainda importante desemaranhar nós mal dados que se proliferam de maneira
irrefletida nos dias de hoje.

\asterisc

%\begin{flushright}
%\footnotesize
\epigraph{\emph{a contrassexualidade aponta para a substituição\\
desse contrato social que denominamos\\
Natureza por um contrato contrassexual.\\
No âmbito do contrato contrassexual,\\
os corpos se reconhecem\\
a si mesmos não como homens ou mulheres,\\
e sim como corpos falantes,\\
e reconhecem os outros corpos como falantes}}{Paul Preciado}
%\end{flushright}

Ao menos nos primeiros anos de seu ensino, quando ainda não estava às
voltas com o ``corpo falante''\footnote{Cf. Miller, J"-A (2016)
  \textless{}\emph{https://bit.ly/2OUbhjW}\textgreater{}}
(Miller, 2016), Lacan reiterava e acentuava certa supressão corpórea.
Sabemos que, no fundo, tentava formular respostas à crítica que tecia em
relação aos rumos tomados pelos pós"-freudianos, que conduziam as
análises de seus pacientes fomentando uma adesão irrefletida à imagem
egóica do analista. Exatamente aí ganharam força suas formulações sobre
o Nome"-do"-Pai e a articulação do desejo pelo \emph{falo}.

Em seu O seminário, livro 5 \emph{as formações do inconsciente,} Lacan
(1957-58/1999), articula a \emph{lei} no nível do significante e o que a
sustenta é o Nome"-do"-Pai, como estrutura simbólica. É dele que se forma
o \emph{falo}, como objeto imaginário. Em torno deste a mãe vai e vem ---
a imagem é o \emph{fort"-da} --- e o sujeito projeta um para"-além que o
descola do objeto"-mãe. Atrás da mãe e, portanto, fora dela, além dela,
está toda a ordem simbólica e imaginária de que ela mesma depende e é
refém, sem ser parte.

Ainda em 1957-8, a ligação inextrincável entre \emph{falo} e \emph{pai},
nos campos imaginário e simbólico, joga"-nos diretamente na
dialética do complexo de Édipo. Ao \emph{falo}, constituído no plano
imaginário ``como objeto privilegiado e preponderante'', subordina"-se o
desejo da mãe, desenhado imaginariamente no Outro. Na negação da posição
simbólica, o imaginário vela e fixa no eu {[}moi{]} um objeto específico
e especular. No plano simbólico, por sua vez, Lacan coloca muito
perfeitamente o que está em jogo, ao dizer: ``O que se produz no nível
das diferentes formas do Ideal do eu não é, como se supõe, efeito de uma
\emph{sublimação}, no sentido de esta ser a neutralização progressiva de
funções enraizadas no interior. Muito pelo contrário, sua formação é
sempre mais ou menos acompanhada por uma \emph{erotização da relação
simbólica}''. (Lacan, 1957-8/1999, p. 274-5), pelo objeto causa do
desejo, o objeto fálico.

Formulação capaz de retomar nosso ponto e responder, ainda que de forma
bastante insuficiente, à questão colocada pelo próprio Lacan: ``por que
Freud precisou de Moisés?'' (1969-70/1992, p. 144). Vimos que, do lado
do Nome"-do"-Pai, enreda"-se toda a ordem simbólica e imaginária fálica,
além do Ideal de eu, oriundos da lógica totêmica freudiana que insiste
na morte do Pai e no advento de sua Lei. O Moisés de Freud ainda
persiste, até certo ponto, por essa via. Lembra Lacan (1969-70/1992, p.
122) que, no enredo mosaico, o lado das mulheres é o da prostituição,
moeda de troca como em \emph{Totem e tabu}. Numa sociedade na qual a
transmissão da propriedade e do \emph{status} era patrilinear, não era
confortável assumir a paternidade de filhos de uma prostituta --- daí
toda a ênfase nas leis inscritas diretamente por Deus nas tábuas da Lei.
Não à toa, a metáfora da apostasia é a prostituição, que rompe com a
relação exclusiva com Deus, reinscritas pelas leis do casamento.
Adultério e promiscuidade indicam traição a \versal{YHWH} e se aproximam da
idolatria.

Essas ideias arcaicas do enredo mosaico, de viés freudiano, casam"-se bem
com a crítica lacaniana ao pós"-freudismo, que teria traído os princípios
essenciais da psicanálise de um modelo transferencial livre da sugestão
e, por conseguinte, descolado da imagem idolatrada do analista.

Num salto aparentemente estratosférico, vale mencionar aqui o belíssimo
texto ``Mutações materialistas da \emph{Bilderverbot}'', de Rebecca
Comay. Partindo de uma passagem da \emph{Dialética Negativa} para
reintroduzir de maneira totalmente original os embates sobre idolatria e
negatividade em Theodor Adorno e Walter Benjamin, ela tece observações
cruciais para o debate que estou propondo aqui. Eis o trecho central
delas:

\begin{quote}
{[}\ldots{}{]} só sem imagens seria possível pensar o objeto plenamente. Uma
tal ausência de imagens converge com a interdição teológica às imagens.
O materialismo a seculariza na medida em que não permite que se pinte a
utopia positivamente; esse é o teor de sua negatividade. Ele está de
acordo com a teologia lá onde é maximamente materialista. Sua nostalgia
seria a ressurreição na carne; para o idealismo, para o reino do
espírito absoluto, essa nostalgia é totalmente estranha. (Adorno, 1966,
p. 205 \emph{apud} Comay, 2005, p. 32/ Adorno (trad. Casanova),
1966/2009, p. 176)
\end{quote}

Não cabe fugir ao nosso escopo para tratar de todos os prismas da
disputa Benjamin"-Adorno, já bem esmiuçada na literatura\footnote{Cf.,
  por exemplo, Bretas, A. (2007)}. De todo modo, convém insistir no
texto de Comay, já que há, para ela, um paralelo entre o atrito
Benjamin"-Adorno e a tensão Arão"-Moisés, apresentada na ópera \emph{Moses
und Aron} de Arnold Shoenberg. Isto é, mais uma vez estamos às voltas
com o monoteísmo judaico e a tensão entre imagem e negatividade,
corporeidade material e abstração, instinto e espírito, ato/crime
(\emph{Tat}) e sublimação.

Do lado de Benjamin, como se sabe, há a imagem"-dialética, tensão
antitética na própria figurabilidade. Pelo prisma de Adorno, há uma
negatividade irredutível, que amplia e exige tradução de afetos e
vibrações do corpo. O que interessa destacar, na passagem da
\emph{Dialética Negativa} citada por Comay (2005), é um ```acordo'
forjado precisamente onde a antítese parece mais intratável''. (p. 33).
Isto é, o `casamento profano' entre teologia e materialismo não
desenharia, como ingênua e rapidamente costuma"-se concluir, uma versão
secular da teologia mosaica. Mais exatamente: não se trata, é o que
insiste a autora, de tornar compatível naquele nó a teologia judaica e o
marxismo. A proibição monoteísta, ou a aparente mortificação dos
sentidos, não condiz com a ``passagem sublime da cegueira física para a
percepção espiritual (Édipo, Tirésias)''. (Comay, 2005, p. 33). O trecho
de Adorno indicaria, diferentemente, que o impacto da recusa sensória
seria uma espécie de ``vindicação do próprio corpo no ponto exato de sua
desfiguração mais irreparável'' --- lá onde está a lei negativa mosaica,
houve a libertação do corpo escravizado.

Por isso, a visão adorniana considera que ``o materialismo absorve ou
reinscreve a teologia precisamente ao falar de uma restituição além de
toda compensação idealizante''. Contra o idealismo, Adorno recupera
Moisés, defensor de \versal{YHWH}, o inominável, o irrepresentável. Com ele, quer
uma redenção pelo imperativo iconoclasta, o que exclui qualquer
possibilidade de apelo a representações não dialéticas, qualquer apelo à
corporeidade estática e imagética, que se pretende histórica. O recurso
da iconografia mosaica é sua forma de resistir a uma mera troca de
abstrações comensuráveis, a equivalências de valores que integram, não é
novidade, a lógica marxista da mercadoria. Quase trocar seis por meia
dúzia, ou pior. Não é o caso, em suma, de trocar facilmente o espírito
intocável pelo corpo cambiável no mercado.

Esse desvio é importante na medida em que por ele cruzam"-se o
patriarcado mais subterrâneo --- o monoteísmo judaico --- e o
materialismo"-dialético, conjunção que se pretende resistente às formas
da abstração próprias ao capitalismo. De outra perspectiva, porém, é
impossível negar que os alicerces abstratos do modelo capitalista, desde
suas origens --- leis jurídicas abstratas, o mercado despersonalizado
como regulador social, a mercadoria - estão fincados no mesmo pavimento
da tradição monoteísta.

Contrariando seu famoso axioma ``não há relação sexual'', Lacan mostra
como entre o povo eleito, descrito em Oséias, havia, sim, relação
sexual. Contudo, ela é, para \versal{YHWH}, prostituição a ser condenada, sendo a
mulher, como sempre, o principal alvo. Se com a morte do Pai
estrutura"-se a renúncia ao gozo absoluto em nome da lei, só com ela
também se torna possível o tempo trágico, no qual se reitera
compulsivamente o Ideal de eu, que não deixa de estar marcado por traços
identificatórios com um pai tirano que dispunha das mulheres, gozava
sozinho e era autor de violências desmedidas. Que esse seja o contorno
ou a borda de um vazio no qual se desenha o desejo não é sem
consequências, especialmente para as mulheres que vivem sob o
patriarcado impressos nessas linhas.

Nesse mesmo horizonte, Lacan rende"-se:

\begin{quote}
Como sabem, por mais longe que esteja dos outros deuses, o Deus de
Moisés diz simplesmente que não se deve ter relações com eles, mas não
diz que não existam. Diz que não se deve correr para os ídolos mas,
afinal, trata"-se também de ídolos que o representam, a Ele, como era
certamente o caso do bezerro de ouro. Esperavam um Deus, fizeram um
bezerro de ouro --- é muito natural. Vemos aí que há uma relação
completamente diferente, uma relação com a verdade. Já disse que a
verdade é a irmãzinha do gozo. (Lacan, 1969-0/1999, p. 123).
\end{quote}

Em O Seminário, livro 17 \emph{O avesso da psicanálise}, voltamos ao
gozo, agora como par da verdade. É pelo elemento trágico que ambos ---
verdade e gozo --- se entrelaçam. ``O grosseiro esquema \emph{assassinato
do pai - gozo da mãe}'' (p. 123) não se refere, adverte então Lacan, ao
fato de que, matando Laio, Édipo acessou livremente Jocasta. Esta só se
une a Édipo depois que este triunfou numa prova de verdade, na qual
confronta a monstruosa esfinge.

Ou seja, se o assassinato do pai edifica a interdição do gozo em sua
versão primária (Lacan, 1969-0/1999, p. 126), por outro lado, é condição
de gozo (Lacan, 1969-0/1999, p. 126), associada então à função
simbólica, transmitida pela castração. Édipo não sofre a castração pela
interdição de seu pai; toda tragédia de Sófocles representa a própria
castração --- os olhos de Édipo caem"-lhe de seu rosto na defrontação com
a verdade. Ao traçar seu destino, o herói escolhe o trono, ainda que
este lhe tenha sido recusado como herança natural. Casa"-se com a mulher
proibida, querendo saber da verdade sobre suas decisões. No seu crime,
desenha"-se a autoria trágica de seu caráter heroico, inarredável às
linhas do desejo.

Do lado do gozo não castrado, sem a inscrição histórica própria à lógica
trágica, há o Pai tirano, cuja imagem de acesso a todas as mulheres,
transfigura"-se em algo derrisório: ``aquele que goza de todas as
mulheres, imaginação inconcebível, posto que é normalmente bem
perceptível que \emph{já é muito} \emph{dar conta de uma}'' é o real
como impossível --- não há sujeito do gozo sexual (Lacan, 1969-0/1999, p.
130) ou, no famoso axioma, ``a relação sexual não existe''. Trocando em
miúdos: para Lacan, não há traço capaz de fundar a relação sexual, só
havendo o falo como indicativo do gozo sexual impossível, fora do
sistema, isto é, enquanto para"-além absoluto. Sem significante sexual, o
Outro emerge como lugar de fala e o inconsciente, que dele advém,
estrutura"-se como linguagem. Solução não tão distante daquela
vislumbrada por Theodor Adorno\footnote{Cf. Safatle, V. 2006.} e sua
insistência na noção de não"-idêntico, resto que resiste e que se mantém
além de toda a gramática cooptada pelo sistema moderno capitalista ---
uma solução cujas raízes seriam moisacas, isto é, o corpo livre do
trabalho escravizado, como já dissemos.

\section{Da faxina geral ao gozo possível}

\epigraph{{[}\ldots{}{]} o que faltaria ao ensino de Lacan\\
foi não ter ido até lá,\\
não ter dado um pequeno passo\\
fora da cultura onde nós estamos,\\
onde para nós a psicanálise é uma\\
prática que tem uma significação cotidiana,\\
significação essa na qual nos banhamos\\
sem pensar nisso senão para estruturá"-la,\\
logificá"-la, complexificá"-la.\\
Faltaria ao ensinamento de Lacan dar esse passo,\\
efetivamente, esse passo fora da psicanálise.}{Jacques"-Alain Miller}

Na configuração fálica, o desejo aparece ``limpo do gozo'' (Viltard,
1996, p. 222) --- da sujeira corpórea, lembremo"-nos mais uma vez, ligada
tradicionalmente à mulher, sempre fora dos horizontes da linguagem,
mesmo no último Lacan. A essa altura, vale introduzir novas camadas a
serem também destrinchadas aqui. Seria possível encontrar modelos de
resistência ao capitalismo como estrutura patriarcal (Federici,
2004/2017) na versão lacaniana da psicanálise, como retorno ao Freud de
\emph{Totem e tabu}\footnote{O debate que se desdobra aqui remonta à
  discussão entre Hobbes (contrato social) e Rousseau (bom selvagem);
  nossa posição, porém, pretende não recair em nenhum desses dois polos,
  mas defender um campo de tensão entre corpo e palavra.} e de
\emph{Moisés e o monoteísmo}? A questão, agora colocada de forma mais
precisa, aparece nos seguintes termos: onde hoje ainda persiste o
problema da lógica fálica, destituída de matéria? Ou: qual é o submundo
desses impasses, afinal?

Em outra ocasião\footnote{Cf. Parente, \versal{AAM} (2019).}, tentei demonstrar
como a ideologia hierárquica, que coloca em lugar elevado o
espírito"-homem e em posição rebaixada o corpo"-mulher, reproduzida por
Freud em sua parte \versal{III} de \emph{O homem Moisés e a religião monoteísta},
revela uma estrutura de poder já denunciada por Marx e Engels e na qual
as mulheres aparecem oprimidas por condições absolutamente desfavoráveis
do ponto de vista social e político --- delas resultam a acumulação
primitiva e a acumulação por expropriação (David Harvey), que junta às
mulheres os povos colonizados. Recapitulando o tópico anterior, podemos
recolocar o problema nos seguintes termos: depositar o ponto de saída no
resto\footnote{Corpos considerados dejetos ou restos aparecem nas
  análises de Franz Fanon (\emph{apud} Ratele, K. 2004), que enfatiza a
  construção histórica de uma fantasia sobre a superioridade sexual de
  homens e mulheres negros. Seus genitais seriam melhores, mais firmes,
  mais flexíveis, maiores ou mais selvagens. O tamanho do pênis ou a
  elasticidade das vaginas e a maneira de usar diferentes partes do
  corpo tornariam negros corporalmente superiores. Superioridade que
  remete à natureza, à biologia, à corporeidade. Conjunto que, de todo
  modo, é oposto à civilização e, por outro lado, está associado à
  animalidade e à selvageria não cultivada. O poder sexual da mulher e
  do homem negros centra"-se no corpo. Entretanto, como a disputa se dá
  entre homens, não são os corpos das mulheres, mas as preocupações com
  o tamanho grande do pênis africano e a superioridade sexual dos
  africanos e outros povos colonizados que, em geral, alimentam
  fantasias dos brancos que se afirmam econômica e culturalmente. Subjaz
  à história colonial e racista um discurso supostamente científico, de
  meados do século \versal{XIX}, que sustenta o lado selvagem de homens africanos
  e, por outro lado, sua incapacidade de exercer o poder soberano,
  ligado à inteligência. Nesse cenário, africanos aparecem como
  genitalmente bem"-dotados por sua inferioridade e seu lado animalesco.},
como vimos pela articulação Lacan"-Adorno, não acaba por subscrever
ranços patriarcais e colonialistas, ainda que não"-intencionalmente?
Vejamos esse ponto.

Ainda na exposição da estrutura fálica e edípica em O seminário, livro
5, Lacan traz o humor dos chistes como \emph{resto} que escapa à rede
linguística e revela o mais"-de"-gozar. Esclarece Viltard (1996) que, no
retorno de Lacan (1968-9/ 2008) ao Freud de ``Os chistes e sua relação
com o inconsciente'', o objeto \emph{a} emerge não apenas como objeto
causa do desejo, mas também como ``objeto perdido na relação do gozo com
o saber''. Essa dedução de Lacan vem a partir de sua leitura de \emph{O
capital} na seção \versal{III}, capítulo 5 ``O trabalho e sua valorização'' na
qual há, de acordo com ele, uma curiosa observação de Marx: na exposição
de seus argumentos para demonstrar ao trabalhador que o mercado é
honesto, o capitalista deixa escapar um sorriso de esguelha\footnote{Na
  versão de Marx: ``{[}O capitalista{]} já recobrou com um sorriso
  alegre sua fisionomia anterior. Ele troçou de nós com toda essa
  ladainha. Não daria um centavo por ela. Ele deixa esses e semelhantes
  subterfúgios e petas vazias aos professores da Economia Política,
  expressamente pagos para isso. Ele mesmo é um homem prático que nem
  sempre pensa no que diz fora do negócio, mas sempre sabe o que faz
  dentro dele.'' (Marx, 1897/1996, p. 152).}. Em outras palavras: na
velha história contada pelo capitalista de que seria justo o acordo
entre ele, que fornece os meios de produção, e o trabalhador, que vende
sua força de trabalho, sobrevém um \emph{riso}, sinal de seu êxito em
obter a ``doação'' da mão"-de"-obra em seu benefício.

O capitalista \emph{sabe} que o trabalhador, além de lhe vender algo
pelo qual será pago, também lhe concederá um a"-mais, pelo qual ele nunca
será recompensado: a mais"-valia subtraída do trabalho como \emph{resto}
não pago e não precificado sob forma de salário. O júbilo, para Lacan,
expressaria também esse secreto e decisivo \emph{saber}, que daria
sustentação ao seu discurso, articulador da engrenagem do mercado e
alheio à verdade do gozo e do mais"-de"-gozar, como expressão sintomática
da contínua frustração. Atrás do riso esconder"-se"-ia, em suma, o a"-mais
que lhe rende riquezas e o torna capaz de acumulá"-las\footnote{Deixemos
  isso de lado, para assinalar algo mais concreto --- material --- ainda
  na seção \versal{III} do capítulo \versal{V} de \emph{O capital}. Numa breve passagem
  para explicar que a mesma coisa, a depender da perspectiva de trabalho
  pela qual se olha, pode ser ora matéria"-prima, ora outro meio de
  produção. Assim descreve, dentre outros exemplos, o seguinte: ``o
  mesmo produto pode no mesmo processo de trabalho servir de meio de
  trabalho e de matéria"-prima. Na engorda do gado, por exemplo, o gado,
  a matéria"-prima trabalhada, é ao mesmo tempo meio de obtenção de
  estrume''. (Marx, 1897/1996, p. 152). Com essa passagem, o que se vê
  claramente é que nem mesmo o que é tido como resto da civilização
  escapa à lógica do capitalista.}.

Em Lacan\footnote{Cf. Oliveira, C. (2008).}, a homologia entre o campo
descrito por Marx e o por ele proposto está na estrutura, sustentada
entre mais"-valia e mais"-de"-gozar. Como vimos, tal estrutura remete ao
gozo, o real enquanto impossível --- a referência aqui, não esqueçamos, é
o gozo do pai primevo de \emph{Totem e tabu}. Sob forma de repetição
compulsiva da frustração, o mais"-de"-gozar presentifica uma perpétua
renúncia ao gozo, um tempo de promessa jamais cumprida. Ou seja:
eternamente indisponível enquanto tal, o tempo de desfrutar ou gozar
projeta"-se num futuro nunca realizável --- o gozo está barrado pela Lei,
após a morte do pai.

Nessa disposição inacessível do gozo, o saber abstrato cumpre papel
fundamental. Sustentada pelo capitalismo no campo do Outro, agora sob
forma equiparável de valores e preços, a mais"-valia retém os meios de
gozar e interdita o gozo pela ordenação do saber numa linguagem em que
se articula sempre uma falta"-a"-gozar. O gozo sexual que não existe
coloca a linguagem no plano de um gozo fálico, destituído, portanto, de
corporeidade. Como esclarece Viltard:

\begin{quote}
Como o gozo sexual é marcado pela impossibilidade de estabelecer, no
enunciável, o \emph{Um} da relação sexual\footnote{Não convém
  desenvolver, nesse ponto, toda a fórmula da sexuação lacaniana com as
  referidas distinções entre o homem e a mulher --- ou o universal em
  contraposição à inexistência da mulher.}, uma vez que não há
significante do gozo sexual, deduz"-se que o gozo é fálico, isto é, não
tem relação com o Outro como tal, sendo `\emph{gozo da fala, fora do
corpo}'. (Viltard, 1996, p. 223).
\end{quote}

Na lógica fálica, como tentamos desenvolver no início deste trabalho,
estamos num registro que suprime a materialidade corpórea do desejo ---
estamos aqui de volta à cena das noites de quarta"-feira entre baforadas
de charuto e onanismo cerebral. A essa descoberta subterrânea soma"-se a
abstração inerente à maquinaria capitalista\footnote{Cf. Benjamin, W.
  (1916/2011).}. Não é necessário demasiado esforço para lembrar que a
existência da mais"-valia depende de uma base abstrata de valor do
trabalho, que se dá com a extração de um valor médio para a produção de
mercadorias. Na compra"-e"-venda do trabalho, própria ao sistema
capitalista, toda atividade humana transfigura"-se em valor de troca. Sem
a `absolutização do mercado' --- a expressão é de Lacan --- no campo do
Outro, a mais"-valia, sugada do trabalho não pago, nem poderia integrar o
discurso capitalista. Opera"-se, então, na passagem da força de trabalho
em valor"-de"-troca e mercadoria, um processo ideológico de
desmaterialização tanto da atividade humana, como dos lugares de poder.

O tempo de gozo, tragado pela maquinaria capitalista, é substância
incalculável no plano da existência, mas passa a ser matematizado
ideologicamente no interior da lógica do mercado. Como a abstração do
valor médio de trabalho, o saber torna"-se homogeneizado para se
configurar num bem adquirido no mercado. O sintoma refere"-se, então, à
maneira pela qual cada sujeito sofre pela supressão do gozo e pela sua
conversão irremediável à verdade social média, abstrata.

Nesse ponto será importante distinguir o gozo sexual como inteiramente
inacessível no horizonte do desejo/lei e o gozo, tal como descrito por
Piera Aulagnier em sua fala em O seminário, livro 9, \emph{A
identificação}:

\begin{quote}
O que diferencia, no plano do gozo, o ato masturbatório do coito,
diferença evidente, mas impossível de explicar fisiologicamente, é que o
coito, por mais que os dois parceiros tenham podido, em sua história,
assumir sua castração, faz com que, no momento do orgasmo, o sujeito vá
encontrar, não como alguns disseram uma espécie de fusão primitiva
{[}\ldots{}{]} mas, ao contrário, o momento privilegiado em que, por um
instante, ele atinge essa identificação sempre buscada e sempre fugidia
em que ele, sujeito, é reconhecido pelo outro como o objeto de seu
desejo mais profundo, mas em que, ao mesmo tempo, graças ao gozo do
outro, pode reconhecê"-lo como aquele que o constitui enquanto
significante fálico. Nesse instante único, demanda e desejo podem, por
um instante fugaz, coincidir, e é isso que dá ao eu esse desabrochamento
identificatório do qual o gozo tira sua fonte. O que não se deve
esquecer é que, se nesse instante demanda e desejo coincidem, o gozo
traz, todavia, em si a fonte da insatisfação mais profunda, pois, se o
desejo é, antes de mais nada, desejo de continuidade, o gozo é, por
definição, algo de instantâneo. (Aulagnier, 1961-2/2003, p. 282)
\end{quote}

Vedar os olhos e tapar os ouvidos para termos retrógrados, como
\emph{coito} e significante \emph{fálico}, usados de maneira infeliz
pela psicanalista, permite vislumbrar a força dessa passagem, que reúne,
a um só tempo, dois aspectos decisivos: corpo e negatividade. Para a
Sra. Aulagnier, como se refere a ela o sr. Lacan, há sexo, há orgasmo,
há encontro. Não do tipo fusional, evidentemente. O que há é um breve
\emph{instante} no qual o desejo é reciprocamente entrevisto ---, ainda
que aí não haja nenhum tipo de equiparação mensurável e quantitativa de
afetos entre os envolvidos na relação sexual, deve"-se admitir que esta,
nesse quadro, \emph{ex"-iste}.

Vê"-se aqui que não se trata de simplesmente restituir ao corpo lugar
desconectado da linguagem, como pura natureza. Entretanto, contra um
gozo impenetrável e inacessível --- o real como impossível do velho pai
daquela obsoleta horda ou a Lei dos irmãos"-homens que dela advém ---,
trata"-se, isso sim, de advogar em defesa de uma tensão permanente entre
corpo e palavra. Nela, o corpo ganha estatura equânime à da palavra e
ambos operam numa constante força de fricção. Recuperado repetidamente
por Lacan, o ``isto'' hegeliano, figura abstrata recoberta de linguagem
dialeticamente construída, é substituído por um instante no qual
fragmentos do corpo são tocados --- há algo que \emph{ex"-iste} pelo
toque, pelo olhar, pelo som, pela respiração do Outro. Aparentemente
diminuta, tal mudança gera impactos de grande magnitude. Daquilo que
Aulagnier chamou orgasmo, e que entendo, de modo amplo, como encontro
efetivo de corpos, proliferam"-se, é o que quero sustentar, reverberações
e abalos sucessivos que se fazem sentir como frêmitos no corpo e no
pensamento. Diferentemente do ``isto'' hegeliano --- o \emph{nada} que
remete a toda uma estrutura de linguagem implicada por esses velhos
jogos de poder, demasiadamente desgastado semântica e sintaticamente ---,
o impacto gerado pela fricção de corpos"-palavras leva a oscilações
próximas da desordem revolucionária, única capaz de abalar os discursos
patriarcais --- até mesmo de alguns psicanalistas ---, agora convertidos
na abstração do discurso do capitalista.

\asterisc

Vale considerar, então, que, o aprisionamento ao mais"-de"-gozar só pode
ocorrer porque em algum lugar sabemos que existe uma espécie de gozo que
parece valer a pena. Tal gozo se dá num encontro desencontrado, mas
efetivo, de corpos esburacados pela linguagem. Tal corporeidade fugaz,
efêmera e entremeada em suas fissuras pela estrutura de linguagem, é o
que se constrange com o advento do discurso abstrato do capitalista.

Seja como for, se Lacan é radicalmente crítico ao discurso do
capitalista, ele aloca na verdade da frustração, eternizada no
proletário, o elemento de resistência perdido --- o mais"-de"-gozar. É
neste \emph{resto sofrível} que parece se concentrar um resquício de
verdade sempre reservada num para"-além, fora do corpo.

\versal{YHWH}, o Nome"-do"-pai, o falo, o gozo impossível são figuras de uma
renúncia inesgotável à materialidade corpórea. Até mesmo para Éric
Laurent (2016), que persegue junto com Jacques Alain"-Miller o último
Lacan, o \emph{corpo falante} parte ``da radicalização do que falha na
experiência sexual, por exemplo: gozar do corpo do outro é impossível''.
Insiste na ideia de que só ``há gozo do próprio corpo'' ou, mais
precisamente: ``do próprio corpo atrelado também ao incorporal de seus
fantasmas'' (Laurent, 2016, p. 41). Não se trata, é evidente, de gozar
do corpo do outro --- o que configuraria uma imagem de abuso ---, mas
também não se trata de gozar sozinho e com os próprios fantasmas no
encontro com o Outro --- a velha fórmula lacaniana de que ``não há
relação sexual''. Contra práticas quiromaníacas de teor obsessivo, que
fazem do corpo resíduo, a colisão de parcelas fragmentárias de corpos
parece ser o que efetivamente rompe velhas fórmulas. São partes
materiais esburacadas por palavras que se tornam acessíveis num encontro
em que o desejo pode ser reciprocamente reconhecido. Ali, nesse encontro
desencontrado, a materialidade corpórea, fragmentada e parcial, une"-se à
negatividade abstrata fantasmática do desejo. Atentar para a figura
bíblica de Moisés, desviando da tradição judaica tal como ela se
estabeleceu no Ocidente, leva também à possível junção tensa entre corpo
e palavra, que ganharam forma nas fagulhas trocadas entre Moisés e Arão,
representadas na ópera de Schoenberg.

Anatomia não é destino, mas supressão do gozo sexual também não!

\section{Referências bibliográficas}

Adorno, T. \emph{Dialética Negativa}. Rio de Janeiro: Zahar, 2009.

Aran, M. ``Lacan e o feminine: algumas considerações críticas''. In:
\emph{Revista Natureza Humana}. 5(2): 293-327, jul.-dez. 2003.

São Paulo, 2003

Aulagnier, P. (1961-2) ``Exposição da Sra. Aulagnier: Angústia e
identificação''. In: Lição \versal{XVIII} de O seminário, livro 9 \emph{A
identificação}. Recife: Centro de estudos freudianos do Recife, 2003.

Benjamin, W. ``O capitalismo como religião'' In: \emph{O capitalismo
como religião.} São Paulo: Boitempo, 2013.

\_\_\_\_\_\_\_\_ (1916) ``Sobre a linguagem em geral e sobre a linguagem
do homem''. In: Escritos sobre mito e linguagem. São Paulo: Editora
34/Duas Cidades, 2011.

Bretas, A. ``Pensar ao mesmo tempo dialética e não"-dialeticamente:
Adorno, leitor de Benjamin''. In: Revista Controvérsia, v. 3, n. 1,
(jan"-jun. de 2007), pp. 1-11.

Comay, R. ``Materialist Mutations of the Bilderverbot''. In: Benjamin,
A. Walter Benjamin and Art.London/New York: Continuum, 2005.

Federici, S. (2004) \emph{Calibã e as bruxas}, São Paulo: Editora
Elefante, 2017.

Fraser, N. ``Contra o `simbolicismo': Usos e abusos do `lacanismo' para
políticas feministas''. Em: \emph{Lacuna: Uma revista de psicanálise}.
São Paulo, n. -4, p. 9, 2017.

Freud, S. (1914) \emph{Totem e tabu.} In: Obras completas brasileiras,
Rio de Janeiro: Imago, 1996.

\_\_\_\_\_\_\_\_ (1916) ``\versal{X} Conferência introdutória sobre psicanálise''
In: Obras completas brasileiras, Rio de Janeiro: Imago, 1996.

\_\_\_\_\_\_\_\_ (1939) \emph{O homem Moisés e a religião monoteísta}.
Porto Alegre: L\&PM, 2014.

Goethe, J. W. (1808) \emph{Fausto \versal{I}}. São Paulo: Editora 34, 2013.

Harvey, D. ``O ``novo imperialismo'': ajustes espaço"-temporais e
acumulação por desapossamento''. In:
\textless{}\emph{https://bit.ly/2BkmUL3}\textgreater{},
acesso em 10/04/2019.

\_\_\_\_\_\_\_ ``O ``novo imperialismo'': acumulação por desapossamento
(Parte \versal{II})''. In:
\textless{}\emph{https://bit.ly/2kISnma}\textgreater{},
acesso em 10/04/2019.

Lacan, J. (1957-8) O seminário, livro 5 \emph{As formações do
inconsciente}. Rio de Janeiro: Zahar, 1999.

\_\_\_\_\_\_\_ (1959-0) O seminário, livro 7 \emph{A ética da
psicanálise}. Rio de Janeiro: Zahar,1995.

\_\_\_\_\_\_\_ (1968-9) O seminário, livro 16 \emph{de um Outro ao
outro}. Rio de Janeiro: Zahar, 2008.

\_\_\_\_\_\_\_ (1969-0) O seminário, livro 17 \emph{O avesso da
psicanálise.} Rio de Janeiro: Zahar, 1992.

Laurent, E. ``O corpo falante'' Entrevista com Éric Laurent por Marcus
André Vieira. In; Revista Cult, 211, ano 19, abril de 2016, pp. 38-41.

Marx, K. (1897) \emph{O Capital}, v. \versal{I}. In: Coleção \emph{Os
Economistas}. São Paulo: Editora Nova Cultural Ltda, 1996.

Miller, J"-A \emph{O inconsciente e o corpo falante.} In: Apresentação do
tema do \versal{X} Congresso da \versal{AMP}, no Rio de Janeiro, 2016
\textless{}\emph{https://bit.ly/2OUbhjW}\textgreater{}, acesso em 10/04/2019.

Nunber, H \& Federn, E. (orgs.) \emph{Atas da Sociedade Psicanalítica de
Viena}. São Paulo: Scriptorium, 2015.

Oliveira, C. ``O chiste, a mais"-valia e o mais"-de"-gozar -- ou o
Capitalismo como piada''. In: Revista Estudos Lacanianos, Belo
Horizonte: \versal{UFMG}/Scriptum Editora, ano \versal{I}, no. 1 (jan"-jun/2008), pp.
57-68.

Parente, A. \emph{Freud como grão"-burguês e o patriarcado na
psicanálise.} In: Revista Peixe"-elétrico (no prelo).

Preciado, P. (2004) \emph{Manifesto Contrassexual}. São Paulo: n"-1
Edições, 2014.

Ratele, K. ``Kinky Politics''. In: Arnfred, S. \emph{Re"-thinking
Sexualities in Africa}. Sweden: Almqvist \& Wiksell Tryckeri \versal{AB}, 2004.

Safatle, V. \emph{A paixão do negativo}. São Paulo: \versal{UNESP}/\versal{FAPESP}, 2006.

Silveira, L. ``Assim é a mulher por trás de seu véu? Questionamento
sobre o lugar do significante falo na fala de mulheres leitoras dos
\emph{Escritos}''. \emph{Lacuna: Uma revista de psicanálise}, v. 3, p.
8-8, 2017.

Viltard, M. Verbete: Gozo. In: Kaufmann, P., \emph{Dicionário
enciclopédico de psicanálise.}Rio de Janeiro: Zahar, 1996.
