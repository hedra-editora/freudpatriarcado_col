\chapter{Apresentação}

De acordo com C. Delphy\footnote{Delphy, Christine. ``Patriarcado
  (teorias do)'' (Trad. N. Pinheiro). Em: Hirata, Helena {[}et al.{]}.
  \emph{Dicionário crítico do feminismo}. São Paulo: Unesp, 2009.}, a
palavra ``patriarcado'' significa literalmente ``autoridade do pai'',
derivando dos termos gregos ``\emph{pater}''\footnote{O verbete
  ``πατήρ'' do dicionário Bailly indica sua ocorrência no sentido de
  ``pai'' {[}\emph{père}{]} em Homero, Ésquilo e Sófocles e, na
  sequência, discrimina os usos do termo como expressando o sentido de
  pais {[}\emph{parents}{]}, ancestrais ou fundadores e ainda o emprego
  por analogia no sentido de título de respeito dirigido aos idosos e o
  emprego por extensão no sentido de fonte {[}\emph{source}{]}. (Bailly,
  Anatole. (1894) \emph{Le grand Bailly -- Dictionnaire grec/français}.
  Paris: Hachette, 2000, p. 1498.)} (pai) e ``\emph{arkhe}'' (origem,
comando\footnote{Vale mencionar que os sentidos de origem e comando
  convergem também para a acepção de ``princípio'' carregada por άρχή
  (Cf. Bailly, \emph{op. cit.}, p. 281).}). ``\emph{Pater}'', no entanto
-- que comparecia como termo em sânscrito, em grego e em latim --, não
correspondia ao sentido contemporâneo de ``pai'', para o qual se
empregava a palavra ``genitor''. O termo não carregava referências à
filiação biológica e se referia ao homem que, ao exercer autoridade
sobre uma família ou sobre um domínio, não dependia de nenhum outro
homem. É no final do século XIX, no contexto de teorias preocupadas em
discutir a (suposta) evolução das sociedades humanas, que a palavra
``patriarcado'' adquire um novo significado: ``São Morgan e Bachofen'',
escreve Delphy, ``que lhe dão seu segundo sentido histórico (\ldots{}). Eles
postulam a existência de um direito materno que teria sido substituído
pelo direito paterno, explicitamente chamado por Bachofen de
patriarcado. Ele é seguido por Engels e depois por Bebel'' (p. 174).
Esse significado ainda não é, porém, o significado adotado pelo
pensamento feminista, que só vem a se gestar, segundo Delphy, na década
de 70 do século XX. Segundo Delphy, a circunscrição do terceiro sentido
do termo ``patriarcado'' é atribuída a Kate Millet. Com sua obra
\emph{Sexual politics} é que teria sido aberto o uso da palavra para
designar sistemas que oprimem as mulheres, isto é, sistemas de dominação
masculina, mesmo que isso ultrapasse a questão do poder do pai.

O uso do termo ``patriarcado'' no título deste livro não significa que
esteja aqui assumido o pressuposto de uma plena univocidade de sentido
em torno dele -- supor tal unidade seria negligenciar a multiplicidade
de formas históricas assumidas pelo exercício masculino do poder; seria,
portanto, de algum modo, já sucumbir ao alvo que se pretende para a
crítica. No que diz respeito ao campo da teoria psicanalítica, no
entanto, é fato facilmente constatável que ela põe em jogo uma forma de
conceber o psíquico -- ou a subjetividade -- como algo que se constrói a
partir de um modelo que assume, em seu centro, uma equivalência
generalizada entre cultura e masculinidade. Seja atravessando a
argumentação de ``Totem e tabu'' e, com ela, o conceito de complexo de
Édipo, seja mobilizando noções como ``Nome-do-pai'' ou ``gozo Outro'', o
lugar das mulheres (e do feminino?) é reiteradamente remetido, de
maneiras que não deixam de ser complexas e profundamente ambíguas, aos
limites da cultura e da civilização, visivelmente consideradas em termos
patriarcais.

Daí que a inspiração de partida para este livro não se pretenda neutra.
Temos assistido a significativos retrocessos políticos, culturais e
sociais ao redor do mundo. Todos eles ligam-se aos ideais conservadores
patriarcais, alinhados a um capitalismo feroz. Por outro lado, as
mulheres estão em estado de alerta. Levantes feministas e diferentes
versões do debate proliferam ao redor do mundo num grande movimento para
que os avanços em relação à igualdade de gênero e a liberdade não
retrocedam e sigam seu curso de maneira cada vez mais intensa e
profunda. É nesse curso de avanços e resistências aos retrocessos que
esse livro pretende inscrever-se. E aí não cabe mais fazer vistas
grossas aos compromissos patriarcais de Freud e de outros psicanalistas
pós-freudianos. Formulações psicanalíticas que porventura compactuam ou
alimentam visões obtusas precisam, mais do que nunca, de respostas
contundentes. Isso não significa anular o pensamento de Freud ou a
psicanálise, nem muito menos colocá-la em risco, declarando sua
invalidade. Ao contrário: revisitar certas elaborações metapsicológicas
ou orientações clínicas que pareçam obsoletas para repensá-las hoje é
justamente o caminho que permite manter viva a força atual da
psicanálise. Sem essa circulação encarnada no presente, ela morre por
asfixia. Aliás, a psicanálise mantém sua potência justamente por sua
capacidade de invariavelmente se reinventar.

Partimos do entendimento de que precisamos enfrentar, seja em que
direção for, a suspeita, assim expressa por Butler, de que a teoria
recorra ``à própria autoridade que descreve para reforçar a autoridade
das suas próprias reivindicações descritivas.'' (J. Butler, Undoing
gender, Nova York e Londres: Routledge, 2004, p. 47). De modo geral,
podemos dizer que a psicanálise descreve a constituição do sujeito a
partir do acontecimento psíquico de inscrição da lei ao mesmo tempo em
que entende que essa lei é algo vinculado ao pai ou à função paterna.
Ora, ao assumir esse tipo de encaminhamento, a psicanálise não coloca,
no centro da construção de seus modelos teóricos, algo que deveria ser
explicado, em vez de ser tomado como dado? Parte da legitimidade
descritiva reivindicada pela psicanálise não se alicerça nesse
território impensado em que à cultura, à vida pública e às realizações
teóricas se atribui uma caracterização, de saída, masculina?

Foi na direção de pensar sobre essa trilha que convidamos as autoras e
os autores desse livro a tecerem suas considerações. Algumas irão
explorar a legitimidade e a preservação dos modelos descritivos
psicanalíticos ancorados nas inspirações originárias de Freud e em seus
desdobramentos, buscando atribuir-lhes uma potência própria, que
ultrapassaria mesmo um diagnóstico de presença de premissas patriarcais
ou ainda encontrando, nos próprios textos de Freud, elementos que
permitiriam vislumbrar modelos distintos; outras irão problematizá-las,
apostando mais diretamente na necessidade de repensá-las.

\asterisc

Como toda subdivisão, a que foi feita para compor este livro carrega
certa arbitrariedade. Isto é, fica claro como os textos se conectam, se
sobrepõem, se articulam ou criam tensões entre si. Apartá-los por um
subtítulo pode dar a ilusão de uma intencionalidade prévia e de uma
restrição do conteúdo a determinados temas específicos, o que certamente
não corresponde à verdade. De todo modo, convém sempre traçar eixos de
leitura que nos ajudem a compreender o material e a integrá-lo a uma
linha mais ampla de raciocínio. Foi assim que, com o material em mãos,
traçamos aqui quatro eixos centrais para pensar os diferentes prismas a
partir dos quais Freud foi relido: I. Revisitando os pilares de Freud;
II. Interstícios do texto freudiano: acerto de contas; III.
Extraterritorialidades: o olhar lançado de fora na análise de Freud; IV.
Contra o mestre: Freud em atrito com as ideias de seus contemporâneos;
V. De Freud aos debates atuais: psicanálise e feminismo.

Em ``Aquém do pai? Sexuação, socialização e fraternidade em Freud'',
Pedro Ambra parte do pilar que sustenta a fundamentação clássica da
teoria freudiana -- o lugar do pai e a universalidade do complexo Édipo
-- como ``fio de Ariadne'' na orientação de elaborações teórico-clínicas
para traçar um pequeno, mas decisivo, deslocamento em direção ao
complexo fraterno, explorando, nas fraturas dos textos freudianos e de
outros autores, como Juliett Michell e Jacques Lacan, algo capaz de
sustentar um complexo fraterno, desenhado horizontalmente. O eixo do
comentário sobre Freud é o texto \emph{Alguns mecanismos neuróticos no
ciúme, na paranoia e na homossexualidade}, de 1922. Nesse quadro, o
complexo fraterno é anterior ao Édipo, conduzindo ao problema em torno
do semelhante nas identificações subjetivas. Trata-se, mais
especificamente, de reconhecer um laço estabelecido em um período de
expulsão dos irmãos e no qual se desenvolvem, de acordo com Freud
(1914), \emph{sentimentos e atividades homossexuais}. No exílio
{[}\emph{Vertreibung}{]} --- e aqui Pedro Ambra mostra o jogo dos
radicais \emph{Vertreibung} e \emph{Trieb} para abordar a pulsão em
contraste com uma ex\emph{pulsão} -- consolidam-se as possibilidades
afetivas para o retorno dos irmãos, o assassinato do pai e a instauração
do tabu do incesto. Resgatando um tempo zero da tragédia de Édipo com a
paixão de Laios por Crísipo, Pedro Ambra estabelece um paralelo com
``Totem e tabu'', dando destaque ao laço homossexual no contexto de uma
ex-\emph{pulsão}. Assim delineia toda uma sorte de laços horizontais
capazes de regular outra modalidade de \emph{socius} anterior ao modelo
identificatório familiar com a primazia do pai -- ainda que eles sempre
se deem em situação de exílio.

Embora parta da mesma materialidade de Pedro Ambra ao retomar os
meandros de \emph{Totem e tabu} em ``Desamparo e horda primitiva'',
Janaína Namba observa o que dali teria conduzido Freud na elaboração da
constituição da sexualidade feminina, tal como descrita em \emph{Sobre a
sexualidade feminina}, de 1931. Tratando da castração como algo que
ressoa de forma análoga à expulsão dos irmãos pelo pai, o texto de
Janaína conversa intimamente com o de Pedro Ambra, mas trazendo as
mulheres ao debate. Em sua perspectiva, elas não teriam vivido essa
espécie de trauma do exílio, já que teriam sido, ao contrário, amadas
pelo pai. Entretanto, isso lhes confere um lugar submisso a ele e não
aquele que dará início à cultura pelo seu assassinato. Dessa
perspectiva, as mulheres sempre sentiriam uma espécie de nostalgia da
horda, onde eram amadas e cuidadas pelo pai, ao mesmo tempo em que
manteriam entre si uma certa rivalidade pelo seu amor. O saldo, porém,
é, para a autora, interessante. Menos aterrorizadas com a castração, as
mulheres são mais aptas a suportarem vulnerabilidades e o próprio
desamparo -- papel que as retiraria, então, de sua sempre enfatizada
passividade.

Já saindo desses pilares mais estruturais para mergulhar nos poros
deixados pelos textos freudianos, temos o belo texto ``Dentro do sonho''
de Stephen Frosh. Nessa imersão naquilo que não virou efetivamente letra
freudiana, mas que não deixa de estar nos interstícios, o húmus que
alimenta as teorizações do pai da psicanálise ganha um caráter híbrido e
pouco afeito a categorias precisas e identitariamente circunscritas. A
tônica masculina das identificações de Freud com personagens bíblicos
não exclui traços de identificações femininas. Com base na
\emph{Interpretação dos sonhos}, obra fundante da psicanálise,
especialmente numa delicada análise do sonho da injeção de Irma, Frosh
vai diluindo os contornos exatos de identidades de gênero, pontuando
questionamentos sobre a polaridade masculino/feminino, ainda que recorra
e empregue esses termos como importantes por serem determinados cultural
e socialmente.

Com tom provocativo, Aline Martins e Lívia Moreira destacam em ``A
origem do destino criado para as mulheres pela psicanálise: por uma
leitura reparadora através das atas da Sociedade das Quartas-feiras'' o
ranço claramente patriarcal das discussões ali conduzidas, seja no que
diz respeito à construção do conceito de complexo de Édipo, seja, mais
especificamente, naquilo que concerne à sexualidade feminina. As autoras
promovem uma ``leitura reparadora'' das atas, expressão cujo sentido
buscam em Eve Sedgwick. Partindo dessa referência, mobilizam uma crítica
fina da presença do patriarcado na teorização psicanalítica, que se
respalda nas obras de importantes teóricas, como Carole Pateman, Juliet
Mitchell e Jessica Benjamin. Elas assumem a potência da psicanálise para
a elaboração daquilo que transcende a consciência e, exatamente por
isso, defendem a necessidade de situar o complexo de Édipo como
modulação psíquica do social: ele seria capaz de nos mostrar como ``o
social toma forma na psique''.

O corajoso ``\emph{Bêtes Noirs}: uma leitura interseccional do caso
freudiano de homossexualidade feminina'' de Marita Vyrgioti busca
compreender como o reconhecimento da estrutura patriarcal do complexo de
Édipo reverbera na compreensão da homossexualidade feminina. A autora
elege \emph{A psicogênese de um caso de homossexualidade feminina}
(Freud, 1920) para enfrentar o problema por ela formulado e consegue
identificar a interdependência da questão patriarcal e da questão racial
como eixos que atravessam a leitura freudiana do caso. Resulta de sua
análise uma interessante e surpreendente interseção entre o afastamento
da referência fálica por parte da homossexualidade feminina e o
afastamento da hegemonia da branquitude. Compreender esse ponto
específico nos permitiria, então, questionar a interpretação fornecida
por Freud ao caso, na medida em que ela excluiria a possibilidade de
figurar positivamente o desejo homossexual feminino. O que, afinal,
teria impedido Freud de tomar o olhar furioso do pai de Margarethe
Csonka como sendo um olhar carregado de inveja? Não seria essa uma
interpretação legítima?, pergunta-se Marita Vyrgioti.

Talvez num tom que Donna Haraway reconheceria como blasfêmico, ``Fricção
entre corpo e palavra: crítica ao Moisés de Freud e Lacan'' busca as
raízes afetivas, de teor defensivo, que tornaram possíveis tanto a
criação da interpretação do termo \emph{falo} pela psicanálise, como a
atribuição de sua mais extrema relevância para a teoria psicanalítica,
que, segundo Alessandra Parente, ainda se mantém presente em Lacan.
Rastreando a tensão entre corpo e palavra no \emph{Moisés} de Freud e
Lacan, a autora procura extinguir qualquer tipo de hierarquização entre
esses campos, mostrando como é recorrente o recurso de enaltecer um dos
termos em detrimento do outro. Recusando o aforisma lacaniano de que ``a
relação sexual não existe'' e também observando uma forma longe da
ilusória complementariedade amorosa, a autora conclui suas análises com
a assertiva de que a ``a anatomia não é destino, mas a supressão do gozo
sexual também não!''. Com ela, pretende colocar em pé de igualdade corpo
e palavra, sem deixar de reconhecer um campo de tensão entre eles.

Já num território estrangeiro que, exatamente por isso, alcança pontos
cegos e naturalizados para aqueles olhos impregnados pelo hábito, Filipe
Ceppas traz em ``Oswald contra o patriarcado: antropofagia, matriarcado
e complexo de Édipo'' a inventiva crítica de Oswald de Andrade a Freud.
O ruído brasileiro antropofágico apresenta o matriarcado como
alternativa ao patriarcado, no qual reconhece a psicanálise freudiana,
modulada pelo conceito de complexo de Édipo. Ao cacoete burguês, Oswald
oferece a transversalidade antropofágica. Ceppas transita por extensa
bibliografia antropológica e filosófica, trabalhando especialmente com o
caráter problemático da tese do matriarcado e a influência de Beauvoir
sobre Oswald. Aponta, ademais, para os problemas inerentes à pretendida
universalidade do complexo de Édipo à luz dos sistemas de parentesco,
trazendo aproximações e distanciamentos entre Freud e Lévi-Strauss.

Em ``Escrever: mulheres, ficção e psicanálise'', as integrantes do Grupo
de Estudos e Trabalho em Psicanálise e Feminismo, Julia Fatio
Vasconcelos, Manuela Borghi Crissiuma, Mariana Facanali Angelini, Renata
de Lima Conde acompanham Virginia Woolf em ``Um teto todo seu'', visando
destacar a ambiguidade da questão da realidade na condição feminina e
então explorar, não sem antes atravessarem alguns pontos levantados por
Simone de Beauvoir, o modo pelo qual tal ambiguidade se expressa no
pensamento psicanalítico. O grande problema que as mobiliza como autoras
aqui é tentar pensar, ou colocar em jogo, esses dois aspectos: a
psicanálise encontra o fulcro do sintoma no conflito entre sexualidade e
moral, mas em que medida não reproduziria elementos dessa mesma moral a
cujo diagnóstico procede? Ponderações de Juliet Mitchell, Gayle Rubin,
Emilce Dio Blecihmar, Maria Rita Kehl e Collete Soller são convocadas
para lidar com o tema e promover avanços na discussão.

Inspirada pelos escritos de Deleuze e Guattari e lançando seu olhar a
partir desse horizonte, Ana Carolina Minozzo apresenta uma releitura do
lugar da angústia na obra freudiana em ``Freud e o
conhecimento-do-corpo: viajando pelos limites da linguagem através da
angústia?''. Com ela, resgata os primórdios das elaborações do pai da
psicanálise em consonância com a perspectiva deleuzeguattariana que
acaba por oferecer um viés feminista a Freud. É na natureza excedente da
angústia, tal como caracterizada na metapsicologia -- excesso tanto com
relação ao sentido quanto com relação à libido -- que a angústia,
desprovida de sentido, seria capaz de convocar a materialidade do corpo
para a energia libidinal e, consequentemente, colocar em marcha um tipo
de singularidade que pode ser vinculado à criação de modos de
subjetivação distintos daqueles que são hegemonicamente veiculados pelo
patriarcado.

Já na tensão de Freud com seus contemporâneos, Marcelo Amorim Checchia
traz um momento delicado da história da psicanálise em ``O patriarcado
entre Sigmund Freud e Otto Gross'', ainda conseguindo remexer, por meio
dessa operação de regaste, com os nossos próprios tempos. Expondo o
lugar espinhoso ocupado por Gross na relação entre Freud e Jung,
Checchia apresenta partes sombrias daquilo que figura hoje como discurso
oficial da psicanálise, isto é, como discurso dos vencedores da história
-- num sentido benjaminiano. Tomando como ponto de partida a noção de
patriarcado tal como ela aparece em Freud, passando pela relação de
Gross com seu próprio pai e alcançando uma problematização a respeito da
segregação sofrida por Gross, Cecchia indica como os trabalhos desse
psicanalista esquecido atravessaram centralmente a oposição
patriarcado/matriarcado e colocaram de modo singular problemas que
tentamos rearticular nos dias atuais.

Outra abordagem de tal viés em atrito é a de Paula Peron que em
``Apontamentos ferenczianos para a atualidade da psicanálise'' explora
os escritos pré-psicanalíticos de Sándor Ferenczi, mostrando uma faceta
pouco convencional do psicanalista contemporâneo de Freud. Em muito
tópicos e à luz do presente, Ferenczi emerge como um autor muito mais
progressista do que o pai da psicanálise. Paula Perón deixa claro um
espírito independente e anti-patriarcal do psicanalista húngaro ao
destacar elementos por ele abordados ainda no começo do século XX. Tais
elementos reverberam nos dias atuais, tanto quando se pensa na luta
feminista como quando se analisa a luta da classe oprimida de modo mais
amplo. Em sua vasta apresentação do autor, Paula Peron traz vários
aspectos que tensionam com as perspectivas freudianas ou que, ao menos,
dão visibilidade a temas que não eram publicamente assumidos por Freud:
ocultismo, resistências à abstração do pensamento, crítica ao
intelectualismo, inseparabilidade entre psiquismo e corpo, modelos
políticos horizontais. Tudo isso leva Ferenczi ainda mais longe, numa
luta concreta contra a homofobia e declaradamente favorável às causas
das prostitutas. Não só isso: ele se engaja profundamente na defesa de
melhores condições de trabalho para os jovens médicos. Trata-se, em
suma, de um rico panorama de tópicos abordados pelo psicanalista húngaro
que são extremamente ousados.

Os debates feministas sacudiram a psicanálise de forma decisiva e ela
definitivamente não pode passar incólume ao crivo dessa onda potente de
mulheres insurgentes e questinadoras. Marco fundamental nesse horizonte
é, sem dúvida, a Introdução à \emph{Feminine sexuality}, escrita em 1982
por Jacqueline Rose e agora traduzido para o português. À época em que
foi publicado pela primeira vez, o texto integrava uma coletânea
dedicada à sexualidade feminina no pensamento de Lacan e na \emph{École
Freudienne}. O volume fora editado pela Macmillam Press com o intuito de
trazer alguns dos trabalhos centrais para os tensos diálogos entre
psicanálise e feminismo. Alguns foram traduzidos para o inglês por
Jacqueline Rose e ela, então, tratou de pensar aquilo que lia em Lacan
de seu lugar de feminista e intelectual. Seguindo os passos do
psicanalista francês, Jacqueline Rose observa a necessidade de um
afastamento da biologia e da ideia de uma impostura envolvida na
reivindicação de sua posse. Sua análise nos permite apreender, e
acompanhar em seus desenvolvimentos conceituais, as diretrizes
mobilizadas no aforismo ``Ⱥ mulher não existe''. A autora mostra que
elas põem em jogo as formas masculina e feminina de posicionamento como
lugares de discurso diante de uma impossibilidade inscrita tanto na
sexualidade quanto na linguagem, impossibilidade que remete
simultaneamente à castração e à diferença sexual e que situa em seu eixo
próprio a fantasia da complementariedade. Defende, assim, que esses
pontos tornam possível situar como equivocada a ideia de que a
psicanálise lacaniana assumiria o privilégio do masculino de modo não
problematizado.

Beatriz Santos, por sua vez, põe em jogo em ``Imposições sexuais e
diferenças entre os sexos -- bruxas, \emph{femmes seules,} solteironas e
Sigmund Freud'' o questionamento a respeito do caráter compulsório da
heterossexualidade em mulheres e da concepção, atrelada a isso, da
vigência de um certo inatismo em tal determinação. Vale-se do trabalho
de Adrienne Rich para dirigir, aos textos freudianos sobre anatomia e
diferença sexual, a pergunta sobre a vinculação possível entre o
patriarcado e o pressuposto da heterossexualidade, e então insistir na
necessidade de a psicanálise pensar não apenas o lesbianismo como algo
independente de uma referência à experiência dos homens, como também a
homofobia. É preciso reconhecer que a imposição da heterossexualidade
como instituição envolve a mesma necessidade de diagnóstico que, por
exemplo, o capitalismo e o racismo. Há, em qualquer dos casos, a
mobilização de normatividades que são contingentes. À luz desse
argumento, Beatriz Santos avança em análises promovidas por Sabine
Prokhoris, com quem ela pode estabelecer a seguinte indicação: a noção
freudiana de diferença sexual talvez nos permita, ela mesma, ultrapassar
o registro insistente de algo que se supõe simplesmente constatável como
binário (e então operante a partir daí) na direção da ideia de
``negociação com a experiência''.

Em ``Simbolicismo e circularidade fálica: em torno da crítica de Nancy
Fraser ao ``lacanismo'', Léa Silveira toma como ponto de partida certas
declarações de Camille Paglia e de Julia Kristeva para, reconhecendo
pontos compartilhados por ambas, discutir o texto de Nancy Fraser,
\emph{Contra o ``simbolicismo'': Usos e abusos do ``lacanismo'' para
políticas feministas} e então acompanhar a filósofa em seu diagnóstico
de que o lacanismo estaria marcado por uma circularidade entre a
atribuição de um caráter falocêntrico à cultura e a atribuição de um
caráter falocêntrico à própria constituição do sujeito. Pontua, no
entanto, que o reconhecimento dessa circularidade -- que, afinal, esteia
a tese do repúdio do feminino pela cultura e do caráter masculino desta
-- não tem como consequência necessária o desinvestimento de interesse
na psicanálise por parte de um debate preocupado com as pautas
feministas. Pelo contrário, uma teoria psicanalítica que preservasse o
anti-psicologismo e o anti-biologicismo avançados por Lacan e que, ao
mesmo tempo, renovasse sua compreensão da diferença sexual de modo tal a
não apagá-la parece à autora algo fundamental para o encaminhamento da
reflexão sobre a agência política.

Virginia Costa parte de Horkheimer e Adorno em ``Sobre o declínio da
autoridade paterna: uma discussão entre teoria crítica e psicanalistas
feministas'' e observa a correlação entre tal declínio e a exacerbação
do autoritarismo na sociedade. Donde parece ser decisivo enfrentar a
tese de Horkheimer relacionada à duplicidade do papel do pai e de sua
autoridade -- nela haveria tanto aspectos progressistas, voltados para a
promoção da autonomia, quanto aspectos regressivos, referentes a
atitudes de adaptação e subserviência. Problematizando essa discussão,
Virginia Costa apresenta os termos de um debate entre, de um lado, C.
Lasch, e, de outro, J. Benjamin e N. Chodorow. Lasch lamenta o declínio
da autoridade paterna na família, inscrevendo-o no cerne de um
diagnóstico que se direciona para a ideia de que a ausência do pai na
família é um dos fatores pelos quais uma sociedade tem o privilégio
masculino absorvido pelo sistema capitalista. A emancipação da mulher na
família corresponderia, assim, a uma maior opressão social. Em
contraposição, Benjamin e Chodorow dirigem críticas ao enaltecimento da
autoridade paterna, representada na perspectiva de Lasch, questionando
sobretudo a vinculação entre mulher e maternagem. O que Virginia Costa
se pergunta, porém, é: até que ponto os posicionamentos dessas
psicanalistas feministas não implicariam a subordinação da
especificidade da pulsão a determinações sociohistóricas?

Por fim, em ``Sequelas patriarcalistas em Freud segundo Luce Irigaray:
sexualidade feminina e diferença sexual'', Rafael Cossi contextualiza
historicamente a crítica de Irigaray a Freud, indicando os principais
elementos dessa crítica em torno do falogocentrismo da abordagem
freudiana do complexo de Édipo. Seu texto nos permite compreender como a
rejeição, por parte de Irigaray, da centralização do falo e das teses a
respeito da inferioridade feminina desembocam em três resultados
interdependentes: a sustentação da \emph{mimesis} e da paródia como
práticas, a necessidade de simbolizar os lábios genitais e o interesse
em promover uma prática de escrita que se volte para o corpo da mulher e
para o caráter indefinido de seu gozo.

São esses alguns dos diversos prismas que a obra freudiana ainda nos
permite entrever. Seja contra o mestre da psicanálise, seja seguindo
seus passos ou cruzando seus percursos, não parece ser possível escapar
de sua influência para pensar os limites do patriarcado -- a partir e
contra sua obra. Esperamos que o livro ofereça uma parcela desse
entusiasmado campo da reflexão crítica e que tal entusiasmo possa
contagiar práticas de intensa liberdade e igualdade entre todos.
