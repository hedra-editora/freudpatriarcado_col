\part{Revisitando os pilares de Freud}


\chapter*{Aquém do pai? Sexuação, socialização e fraternidade em
  Freud\footnote{Os desenvolvimentos que seguem são parte de uma
    pesquisa de doutorado financiada pela CAPES e desenvolvida entre a
    Universidade de São Paulo e a Université Paris Diderot. Ambra, P.
    (2017) \emph{Das fórmulas ao nome: bases para uma teoria da sexuação
    em Lacan.} Tese de Doutorado. Disponível em \textless{}
    http://www.teses.usp.br/teses/disponiveis/47/47134/tde-12012018-174515/pt-br.php
    \textgreater{}}}
\addcontentsline{toc}{chapter}{Aquém do pai? Sexuação, socialização e fraternidade em
  Freud,\\ \footnotesize\emph{por Pedro Ambra}}
\hedramarkboth{Aquém do pai? Sexuação, socialização e fraternidade em Freud}{}

\begin{flushright}
\emph{Pedro Ambra}\footnote{Psicanalista, doutor pela Universidade de São Paulo
  e pela Université Paris Diderot é autor de livros, capítulos e artigos
  sobre psicanálise, gênero e sexualidade. É pesquisador do Laboratório
  de Teoria Social, Filosofia e Psicanálise da USP e professor do
  Mestrado em Psicossomática da Universidade Ibirapuera.}
\end{flushright}

Não seria exagerado dizer que a interpretação clássica da teoria
freudiana tenha como um de seus pilares fundamentais a noção de pai. O
pai, bem entendido, tomado não como aquele que, concretamente, gera e
cuida dos filhos, mas, sobretudo, um representante psíquico em certa
medida primordial e incontornável à estruturação do sujeito enquanto
tal. Em outras palavras, pareceria haver para Freud uma sorte de
necessidade em centralizar o pai em sua teoria, seja ao sublinhar a
universalidade do complexo Édipo, ao refletir sobre a sociedade ou nas
direções clássicas de tratamento por ele defendida em seus casos
clínicos.

Tendo a achar que o pai se tornou --- talvez mais a partir
pós-freudianos do que a partir da própria letra de Freud, aliás --- uma
espécie de fio de Ariadne que permitiria reencontrar o caminho
verdadeiro da teorização e da clínica psicanalíticas. E, de fato, não é
impossível reencontrá-lo em textos aparentemente tão distintos quanto
\emph{Totem e Tabu}, \emph{O Homem dos Ratos}, \emph{O Homem Moisés e a
religião monoteísta} e \emph{A feminilidade}.

No entanto --- e aqui se evidencia uma semelhança metodológica entre a
postura do analista em sua práxis e em sua reflexão teórica --- me
parece sempre frutífero desconfiar das universalidades hegemônicas que
parecem explicar a totalidade de um sistema de pensamento. Se é verdade
que a teoria freudiana seria patriarcal no sentido de ter na figura do
pai um importante ordenador clínico, teórico e, no limite,
epistemológico, isso não nos impede de buscar frestas, rachaduras ou
perspectivas aparentemente contraditórias ao cânone.

O presente texto, assim, é uma tentativa de negritar um aspecto pouco
comentado da teoria freudiana referente à importância não da
verticalidade paterna, mas da horizontalidade do semelhante na
estruturação subjetiva, notadamente do que tange à sexuação ---
compreendida aqui como as séries processuais de identificações que
articulam sexo e gênero na constituição psíquica. Partindo do texto
\emph{Alguns mecanismos neuróticos no ciúme, na paranoia e na
homossexualidade}, de 1922, buscaremos alguns outros indícios que
demonstram haver uma espécie de polaridade no que tradicionalmente
compreendemos como patriarcado em Freud: talvez haja uma importância não
negligenciável das suposições, fantasias e laços presentes entre
semelhantes que não chega a eclipsar, mas, no mínimo, se articula às
concepções clássicas do Édipo e do pai em Freud.

\section{O complexo fraterno}

No já citado texto freudiano de 1922, diferentemente dos três tipos de
identificação classicamente retomados do Capítulo 7 de ``Psicologia das
massas e análise do eu'', temos aqui a identificação como um mecanismo
definidor da \emph{homossexualidade}. Ainda que chegue a reconhecer
causas orgânicas da homossexualidade, Freud foca seus esforços em
refletir sobre os processos psíquicos que estariam em sua origem,
chegando mesmo a descrever o chamado ``processo típico'', no qual o
jovem rapaz, intensamente fixado à sua mãe --- e notamos aqui a
silenciosa assunção do masculino como modelo ---, encontrará uma
``mudança'' alguns anos após a adolescência.

O reconhecimento do fator orgânico da homossexualidade não nos dispensa
da obrigação de estudar os processos psíquicos envolvidos na sua gênese.
O processo típico, já verificado em inúmeros casos, consiste em que,
alguns anos após o fim da puberdade, o adolescente, que até então se
fixava fortemente à sua mãe, realiza uma mudança: identifica-se com a
mãe e busca objetos amorosos em que possa reencontrar a si mesmo, que
gostaria de amar como a mãe o amou. Característica desse processo é que
normalmente, por muitos anos, uma condição necessária do amor será que
os objetos masculinos têm que ter a idade em que nele ocorreu a
transformação. (Freud, 1922/2011, p. 221, trad. Souza)\footnote{Sublinhemos
  Jacques Lacan fará, em 1932, uma tradução (bastante questionável,
  aliás) deste texto de Freud. O que César traduziu, por exemplo, por
  ``mudança'' {[}Wendung{]} e ``transformação'' {[}Umwandlung{]} Lacan,
  traduzirá por \emph{crise} e \emph{bouleversement} {[}perturbação{]}.
  De toda forma, a questão discutida no texto será central em boa parte
  do ensino de Lacan a partir do estatuto do semelhante junto ao
  imaginário, cujo modelo é, precisamente, o \emph{irmão}/\emph{irmã}, e
  gozará de relativa permanência conceitual por, no mínimo, duas
  décadas, antes da consolidação estruturalista em seu ensino.}

Freud afirmará que a fixação à mãe evitaria a passagem a um outro objeto
feminino e que a identificação com a mãe --- aqui sinônimo de
``fixação'' --- é o que, no fundo, permitiria que o indivíduo continue
fiel a esse primeiro objeto --- tendendo, em seguida, a uma escolha
narcísica de objeto. Mais à frente, afirmará que ``a pouca estima pela
mulher, a aversão e até mesmo horror a ela, procedem geralmente pela
descoberta, feita bastante cedo, de que a mulher não possui pênis''. Tal
ideia da homossexualidade masculina como se constituindo a partir de um
desgosto ou horror do feminino perdurará durante muitas décadas no
pensamento psicanalítico --- inclusive nas mais diferentes filiações
teóricas.\footnote{É curioso notar que a homossexualidade feminina
  dificilmente é descrita a partir de uma ``repulsa do masculino''. No
  caso de Csillag, a jovem atendida por Freud cuja queixa dos pais seria
  seu relacionamento com outra mulher, o psicanalista busca (inclusive
  já no título) uma psicogênese de sua homossexualidade. No entanto,
  logo vai deixando claro no texto que qualquer explicação universal
  seria psicanaliticamente imprecisa.}

Supor que uma certa modalidade pretensamente estável de escolha objetal
tem como correlato o desprezo de outra nos levaria à questão inversa,
que se demonstra ser falsa: um homem heterossexual não irá desprezar o
sexo masculino. Ao contrário, o desprezo pelas mulheres é inclusive uma
das características mais marcantes do machismo, ou até mesmo da própria
masculinidade. E aqui a passagem rápida feita por Freud entre
identificação com a mãe --- sinônimo automático de mulher heterossexual
--- e desprezo pela mulher não é explicada, senão pela dificuldade em
psicanálise de separar identificação e escolha objetal.

Vejamos como o problema é abordado por Butler, em ``Le transgenre et les
`attitudes de révolte''' {[}O transgênero e ``as atitudes de
revolta''{]}, visando utilizar a homossexualidade como um paradigma para
se pensar a questão \emph{trans.} Reproduzo aqui a passagem, que, apesar
de longa, apresenta bem uma crítica aplicável ao núcleo da argumentação
de Freud nesse primeiro tempo de seu texto sobre a homossexualidade
masculina:

\begin{quote}
Por exemplo, alguns psicanalistas, incluindo as feministas, poderiam
argumentar que os intensos vínculos homossexuais entre meninos indicam
que esses meninos repudiaram sua mãe. Isso é então entendido como uma
ruptura na capacidade relacional em si. A pressuposição é a de que a
relação com a mãe é primária e que qualquer violação dessa relação é uma
violação em qualquer capacidade relacional. O psicanalista Ken Corbett
refutou esse ponto de vista, sugerindo que quando os meninos curtem seu
prazer fálico juntos, são tomados por um modo de relação distinto
daquele que podem ter com meninas ou mesmo com o materno. Não há nenhuma
razão para inferir imediatamente que o deslocamento em direção à
homossexualidade masculina é um repúdio da mãe. Não é porque tais jogos
fálicos, particularmente entre os meninos, geralmente não envolvem
meninas ou mulheres --- embora seja possível --- que eles têm por
princípio um repúdio às meninas ou às mulheres. Na verdade, uma coisa é
que o desejo seja dirigido a um gênero, e não a outro, e uma outra é
fundar seu prazer sobre a exclusão motivada pela agressividade ou mesmo
pelo ódio em relação a esse gênero. Ao considerar a relação com a mãe
como sendo a relação primária, arrisca-se, estranhamente, acabar por
explicar tais jogos entre meninos fazendo referência ao materno, que é
uma forma de se afastar de seu modo de relação próprio. Na verdade,
corre-se o risco de acabarmos tendo uma teoria da sexualidade masculina,
concebida para proteger o narcisismo da mãe presumidamente
heterossexual. (Butler, 2009, p. 15)
\end{quote}

A argumentação de Butler segue no sentido de usar esses mesmos
questionamentos para se pensar uma constituição da identidade transexual
para além do repúdio ao sexo. Notemos que há em Butler uma crítica que
se articula de duas formas distintas. Em primeiro lugar \emph{coloca-se
em causa a primazia da mãe como matriz das relações sexuais}, e mesmo
sociais, posto que qualquer outro vínculo deve remeter-se a esse
primeiro, nem que seja de maneira agressiva --- tal é a hipótese de
alguns psicanalistas evocados, ao que poderíamos incluir alguns
desenvolvimentos de Freud e Lacan. Assim Butler abre a possibilidade de
se pensar um vínculo para além, ou para aquém, do Édipo; ou, no mínimo,
como não tão radicalmente comprometido com uma alteridade tão
normalizadora como o que virá a ser o grande Outro. Uma outra, que se
articula com essa primeira, é a \emph{valorização desse vínculo entre
semelhantes}, fora de um regime de simples desdobramento edípico --- no
caso, aqui, entre jovens rapazes. Notemos, no entanto, que Butler
realiza tal movimento visando colocar de lado tendências agressivas
dirigidas em relação à mãe. Mas teria tal reflexão algum tipo de
embasamento no arsenal conceitual freudiano?

Voltemos ao texto ``Sobre alguns mecanismos''. Logo no início,
dirá Freud sobre o ciúme ``normal'':

\begin{quote}
{[}\ldots{}{]} é profundamente enraizado no inconsciente, dá continuidade aos
primeiros impulsos da afetividade infantil e vem do complexo de Édipo ou
do complexo fraterno {[}\emph{Geschwisterkomplex}{]} do primeiro período
sexual. No entanto, é digno de nota que algumas pessoas o experimentam
de forma bissexual, ou seja, pode haver no homem, além da dor pela
mulher amada e do ódio pelo rival, luto por causa do homem
inconscientemente amado e ódio pela mulher como rival, num ciúme
reforçado. (Freud, 1922/2011, p. 210, trad. modificada)\footnote{Ainda
  que não comente este texto, Butler empregará um argumento bastante
  semelhante a partir de \emph{O eu e o isso} para discutir sua proposta
  de uma \emph{melancolia de gênero,} no já clássico ``Problemas de
  gênero'' (1990/2014).}
\end{quote}

É a primeira e única ocorrência do termo \emph{Geschwisterkomplex}
{[}complexo fraterno, complexo de irmãos{]} em Freud; e não seria digna
de nota não fosse o tipo de argumento em jogo nesse artigo. A passagem
acima chama a nossa atenção não apenas pela presença desse estranho
complexo, mas igualmente pelo seu estatuto de anterioridade em relação
ao Édipo, sendo considerado como referente ao \emph{primeiro período
sexual}. Na sequência, uma outra pontuação sobre o ciúme sublinha seu
caráter bissexual: enraizado profundamente no inconsciente, o sentimento
de ciúmes em alguma medida ignora seja a identidade sexual do parceiro,
seja a fixação de escolha objetal (hetero ou homossexual). De toda
forma, essa observação nos leva à questão do estatuto do semelhante
junto às vicissitudes identificatórias e objetais da constituição
psíquica.

\section{Os irmãos}

Juliet Mitchell, conhecida por seu clássico \emph{Psicanálise e
Feminismo}, foi talvez quem mais sistemática e verticalmente discutiu a
importância dos irmãos na clínica, em especial na análise da histeria.
Um trabalho posterior da psicanalista, \emph{Loucos e medusas}
(2000/2006), organiza-se ao redor de uma articulação cuidadosa entre
três eixos. Um refere-se à dispersão epistemológica e diagnóstica da
categoria de ``histeria'' ao longo do século XX (Mitchell, 2000/2006, p.
143). Reconstrói-se aí a ideia de que essa categoria flertava, desde seu
nascimento, com a iminência de seu próprio desaparecimento, mas que uma
análise cuidadosa da natureza mimética de suas manifestações demonstra
que ``não há como a histeria não existir: ela é uma resposta particular
a aspectos particulares da condição humana de vida e morte. Através das
culturas e da História, suas modalidades são várias, mas serão todas
variações sobre o tema de uma forma particular de sobreviver''
(Mitchell, 2000/2006, p. 383). Um segundo eixo refere-se à consideração
da centralidade da pulsão de morte na compreensão da histeria --- fato
pouco explorado diretamente pelos pós-freudianos, talvez porque a pulsão
de morte toma força na obra do psicanalista num momento em que a
teorização da histeria já fora consolidada. ``Quando não se permite que
a histeria desapareça, há outra teoria de psicanálise a ser escrita ---
uma que assuma a total importância das pulsões conflituosas de morte e
vida construídas no contexto da condição de prematuridade do nascimento
humano'' (Mitchell, 2000/2006, p. 383).

Mas é um terceiro eixo de discussões que nos interessa mais aqui,
articulado intimamente com a consideração das pulsões destrutivas.
Trata-se do resgate da importância dos laços horizontais tanto na
histeria quanto na constituição do psiquismo. Na esteira da conceituação
freudiana que discutíamos acima, Mitchell sublinha o fato de que
diversos traços da constituição do eu advêm não de identificações
edipianas, mas de soluções ao lugar dado às pulsões dirigidas aos
semelhantes. Isso não quer dizer que Mitchell ignore ou secundarize o
Édipo, mas que o revisita a partir do lugar de destaque do semelhante na
histeria --- a análise dessa questão no caso Dora e na própria
autoanálise de Freud são preciosos nesse sentido.

Estaria em jogo, para a autora, uma espécie de regressão edípica na
histeria, ocasionada pelo questionamento do lugar do pequeno outro na
constelação do sujeito. Em outras palavras, é precisamente na entrada em
cena de um irmão --- tomado como rival --- que o drama edípico se
armaria da maneira classicamente apresentada. A pulsão de morte tem aí
um papel primordial, na medida em que a angústia em jogo refere-se a
esse horizonte de destrutividade total aportado pela presença do outro
lateral. Ao considerarmos a noção de complexo enquanto um quadro de
relações simbólicas a partir do qual a relação do sujeito com a
realidade é estruturada (Lacan, 1938/2003), temos que tal complexo
fraterno nos forneceria uma outra gramática de relações sociais e,
portanto, de constituição subjetiva, não mais exclusivamente pautada na
estruturação edípica, mas tecida junto a ela.

\begin{quote}
O relacionamento entre irmãos é importante porque, diferentemente da
relação parental, é nosso primeiro relacionamento \emph{social}. O modo
do tratamento psicanalítico obscurece este fato e a teoria o ignora. Com
o surgimento de um irmão mais novo ou com a percepção da diferença de um
irmão mais velho (ou substituto de irmão) o sujeito é desalojado,
deposto, e fica sem o lugar que era seu: ela ou ele devem mudar
totalmente em relação tanto ao resto da família quanto ao mundo externo.
Se a criança é uma menina mais velha, insistem para que se torne uma
``mãezinha'', se é um menino, para que se torne o ``irmãozinho'' (A
assimetria é notável aqui.) Para ambos, entretanto, o assassinato está
no ar. O desejo de matar o pai (parte do complexo de Édipo) que possui a
mãe e que, com ela, é responsável pelo usurpador, é secundário frente à
necessidade de eliminar aquele/aquela que tomou o seu lugar e o exilou
de si mesmo. (Mitchel, 2000/2006, p. 38)
\end{quote}

Mitchell critica em Freud a ignorância do caráter retroativo da
interpretação do Édipo, já que é ``a percepção inicial da presença dos
irmãos que produz uma situação psicossocial catastrófica de
desalojamento. E isso deflagra uma regressão aos relacionamentos
parentais anteriores que, até esse momento, estavam sem suas implicações
psíquicas'' (Mitchel, 2000/2006, p. 40).

Com essa inversão em mente, diversos detalhes da teoria psicanalítica
começam a se mostrar mais relevantes. Por exemplo, o desejo radical de
\emph{Antígona} é aquele de enterrar o \emph{irmão}, Polinice --- que
fora morto, por sua vez, por seu outro \emph{irmão}, Etéocles. Em
realidade, o marco zero da trilogia tebana se dá por meio de um impasse
entre semelhantes, Laios e Crísipo, como veremos à frente.

Da mesma maneira, a tragédia de \emph{Hamlet} --- tão importante para
Lacan em seu seminário \emph{O desejo e sua interpretação} --- tem como
início da trama um fratricídio. ``Na histeria vemos a assombração que
surge quando a encenação não se transforma em poesia nem é ritualizada
(como, pelo aparecimento do Fantasma, o assassinato de um irmão
atormenta a peça de Hamlet)'' (Mitchell, 2000/2006, p. 49).

Mitchell resgata, inclusive, diversos trechos da biografia freudiana
contemporâneos à redação dos \emph{Estudos sobre a histeria}, nos quais
há, na descrição que Freud faz sobre si mesmo a Fliess, uma passagem de
uma \emph{pequena histeria} a um estado mais ``normal''. De acordo com a
autora, esse processo que se entretece com o luto pelo pai refere-se, em
realidade, a uma elaboração sobre um irmão morto. Em meio a uma
racionalidade edipocêntrica,

\begin{quote}
ignora-se o fato de que mãe e pai são tão importantes e problemáticos
porque outros, além da própria pessoa, reivindicam-nos. Esses outros ---
no caso de Freud, Wilhelm Fliess como reencarnação emocional de seu
irmão morto --- são o efeito que fica quando algo capacita o histérico a
resolver sua histeria por meio da resolução do complexo de Édipo. Só
para continuar com Freud como caso exemplar de um problema generalizado,
embora ele tenha se recuperado, nunca foi capaz de tolerar
relacionamentos laterais com homens como colegas. (Mitchell, 2000/2006,
p. 71)
\end{quote}

Menos especulativa, mas igualmente precisa, é sua leitura sobre o caso
Dora. Apesar de Freud afirmar que havia em Dora uma histeria desde a
infância, sendo a manifestação sintomática --- analisada junto à
constelação formada por ela, seus pais e o casal K. --- uma atualização
de questões anteriores, o psicanalista não discute propriamente a
etiologia ou questões estruturais desse caráter mais estruturalmente
histérico do caso. ``O que não se vê é que a histeria de Dora
\emph{precede} o fato de ser um objeto de troca entre homens que são
mais velhos o bastante para serem seus pais'' (Mitchell, 2000/2006, p.
137). A autora condensa, então, sua interpretação: ``embora
posteriormente haja algumas indicações de que Dora fez uma identificação
paterna, a histeria emana na infância a partir do momento da interrupção
de sua identificação com o \emph{irmão}'' (Mitchell, 2000/2006, p. 137).

Assim, o relacionamento social que deflagrara os desejos edipianos de
Dora e o fracasso de sua solução era entre irmãos. Ela quisera
posicionar-se na família quando criança de forma igual ao irmão, só para
descobrir que não era igual a ele no sexo e que (provavelmente) ele,
sendo homem e primogênito, tinha o amor da mãe. (Mitchell, 2000/2006, p.
137)

A questão da importância dos irmãos, no entanto, talvez não fosse assim
tão desconhecida por Freud, como atesta o texto ``Sobre alguns
mecanismos neuróticos no ciúme, na paranoia e na homossexualidade'',
bem como ``A etiologia da histeria'', no qual --- ainda sob a égide da
teoria da sedução traumática --- Freud sublinha que o abuso (ou a
fantasia sobre o abuso) era cometido por pessoas que poderiam ser
divididas em três grupos: (1) desconhecidos; (2) conhecidos próximos,
como babás, governantas, tutores; e ``o terceiro grupo, finalmente'',
que ``contém relações infantis propriamente ditas --- relações sexuais
entre duas crianças de sexo diferente, em geral um irmão e uma irmã, que
se prolongam com frequência além da puberdade e têm as mais extensas
consequências para o par''. Ou seja, nesse momento de sua teorização, o
trauma ocasionado pelos irmãos teria, inclusive, consequências mais
graves do que aquele referente a adultos.

Um outro aspecto ligado a essa questão --- sumariamente ignorado por
Lacan na sua discussão sobre as psicoses, por exemplo --- é que Schreber
tinha um irmão que se suicidara com um tiro aos 38 anos, antes de sua
primeira internação. O que poderia ser um dado biográfico lateral
torna-se mais relevante por dois aspectos. O primeiro é que, após a
morte do pai, em 1861, Schreber --- ao contrário de sua predileção, na
adolescência, por Ciências Naturais --- resolve estudar Direito,
``seguindo as pegadas do irmão mais velho'' (Carone, 1984, p. 9). Se
lembrarmos que a interpretação clássica liga o desencadeamento do surto
de Schreber à sua \emph{nomeação} a Juiz Presidente da Corte de
Apelação, chama a atenção o fato de o suicídio do irmão ter se seguido,
precisamente, à sua \emph{nomeação} ao cargo de Conselheiro do Tribunal.
Um segundo aspecto tornaria essa possível identificação ao irmão ainda
mais digna de nota, na medida em que seu prenome era o mesmo de
Schreber, \emph{Daniel}. Prenome esse que era também o do pai e o do
avô, além de remetido a um suposto antepassado do Dr. Flechsig, Daniel
Fürchtegott Flechsig (Schreber, 1905/1984, p. 85).

Um último comentário referente à importância dos laços laterais vai nos
aproximar novamente das questões de sexuação. John Money --- além de ter
cunhado a expressão ``papel de gênero'', a partir da qual Stoller
desenvolveu o conceito de ``identidade de gênero'' --- foi o psiquiatra
responsável pela decisão de reassignação de gênero naquele que ficou
conhecido como o caso David Reimer, nos anos de 1960. Após complicações
em uma circuncisão aos sete meses de vida, os pais do jovem David
procuraram Money depois de ouvirem o ``especialista em gênero'' em um
programa de rádio. Baseado em seus estudos, que defendiam uma
neutralidade de gênero primeva, Money assegurou aos pais que seu filho
--- dada a mutilação sofrida --- seria mais feliz se criado como uma
menina, sugerindo uma cirurgia de construção vaginal. Lembremos que tal
proposta foi amparada pela ideia de que, tal como uma língua, o gênero
se aprende a partir de um social relativamente estável, conhecido e
controlável. Trata-se, portanto, de um papel, que pode ser exercido
independentemente do dado biológico constitutivo. Aos 14 anos, David
opta por assumir um papel de gênero (novamente?) masculino, por meio de
intervenções hormonais e cirúrgicas; casa-se com uma mulher, assumindo a
paternidade de três filhos. Após anos de severa depressão, suicida-se
aos 38 anos. (Ambra, 2018)

O caso Reimer é frequentemente citado por radicais de direita, em geral
religiosos, para atestar os perigos daquilo que denominam ``ideologia de
gênero'', nomeando o caso como uma das experiências mais monstruosas do
século XX, comparável a Auschwitz. Contudo, um detalhe que é pouco
discutido nesse caso refere-se ao fato de que o gêmeo de David, Brian,
tem um surto esquizofrênico justamente após a publicação da biografia de
David e comete suicídio aos 36 anos. David é devastado pelo suicídio do
irmão, fica responsável pela limpeza e cuidados de sua sepultura e
suicida-se apenas dois anos após o acontecido.

\section{O patriarcado e a expulsão}

Mas voltando à Freud, haveria, de fato, espaço para discussões dessa
natureza? Ou a questão da horizontalidade dos laços, em última
instância, sempre faria referência ao Édipo?

Uma discussão que Lacan localizará em Freud (mas que diga respeito mais
a seus desenvolvimentos) refere-se à \emph{anterioridade da
identificação ao pai em relação à mãe na estruturação do sujeito}.
Assim, uma argumentação a partir dessa ideia sublinharia o fato de que
tal laço entre semelhantes só é possível porque houvera uma
identificação primitiva ao pai, o que possibilitou esse tipo de ligação
aparentemente horizontal. Como em uma espécie de retomada ontogenética
do movimento mitológico de ``Totem e tabu'', uma economia de afetos
entre semelhantes só é possível dada a anterioridade de um laço com um
pai primevo. Bem entendido, a retomada que Butler faz de Corbett citada
acima sublinha o fato de que tal laço é diretamente sexual, posição
oposta à de Freud em ``Totem e Tabu'', para quem

\begin{quote}
A necessidade sexual não une os homens, ela os divide. Os irmãos haviam
se aliado para vencer o pai, mas eram rivais uns dos outros no tocante
às mulheres. Cada um desejaria, como o pai, tê-las todas para si, e na
luta de todos contra todos a nova organização sucumbiria. Nenhum era tão
mais forte que os outros, de modo a poder assumir o papel do pai.
\end{quote}

No entanto, uma passagem logo à frente no mesmo texto permite-nos
complexificar o quadro um pouco mais:

\begin{quote}
Assim, os irmãos não tiveram alternativa, querendo viver juntos, senão
--- talvez após superarem graves incidentes --- instituir a proibição do
incesto, com que renunciavam simultaneamente às mulheres que desejavam,
pelas quais haviam, antes de tudo, eliminado o pai. Assim salvaram a
organização, que os havia fortalecido e que pode ter se \emph{baseado
nos sentimentos e atividades homossexuais que teriam surgido entre eles
no tempo da expulsão}. (Freud, 1913/2012, p. 220, grifo nosso)
\end{quote}

Retomemos esse detalhe pouco explorado de ``Totem e tabu''. A ferocidade
do pai primitivo e sua gestão de corpos --- poderíamos dizer, até mesmo,
sua \emph{política sexual} --- não produziu diretamente uma revolta que
levou a seu assassinato. Houve, antes, um importante \emph{período de
exílio}. Nele os irmãos encontraram-se fora da horda, longe tanto da
ameaça do pai quanto das mulheres --- que, nesse momento, encontravam-se
indistintas entre ``aquelas da família'' e as ``de fora'', já que tal
divisão só terá espaço com a instauração do tabu do incesto, que depende
do assassinato do pai. Antes do Édipo, há, portanto, \emph{o exílio dos
semelhantes}. E é precisamente esse exílio --- no qual a identificação
entre os semelhantes mistura-se à prática sexual, no que tange ao laço
social --- que funcionará como \emph{base} da \emph{organização que os
fortaleceu}.

Ou seja, ao contrário de uma leitura clássica que coloca o assassinato
do pai como central e exclusivo, a leitura dessa passagem permite
apresentar um paradoxo, na medida em que a base da organização dos
irmãos se dá num período de exílio no qual não imperava uma lógica de
``partilha das mulheres'', tampouco de ódio ou culpa pelo assassinato do
pai. A instauração do incesto tem, assim, duas faces: uma ligada a uma
lei comum de aliança; outra referente ao desejo de estabelecer laços
horizontais --- na medida em que o incesto se mostra, na passagem, como
a única alternativa possível para os irmãos viverem juntos.

É nesse contexto de ``superação de difíceis incidentes'', portanto, que
aparece uma menção a ligações não heterossexuais. Com efeito, seu lugar
parece ser bastante revelador. Há, inicialmente, uma ideia de que tais
sentimentos e práticas precisam de um espaço específico, externo e não
regulado para quem possam ocorrer. No entanto, não estamos aqui em um
domínio de ``um limbo feliz de uma não identidade'', como pontuará
Foucault a respeito de Herculine Barbin; tampouco em um ambiente com
menor repressão pulsional, como insistirá Freud sobre a Antiguidade
clássica; e nem mesmo no campo de uma paixão entre iguais, como conta a
história da escola de Safo, em Lesbos.

Trata-se, propriamente de um exílio {[}\emph{eine Vertreibung}{]} --- e
lembremos que há em \emph{Vertreibung} o mesmo radical de \emph{Trieb},
pulsão. \emph{Vertreibung} é, literalmente, uma ex\emph{pulsão}.
Curioso notar como esse exemplo liga a homossexualidade não
necessariamente a um desvio da pulsão --- cujo objeto é necessariamente
contingente ---, mas à ideia de que há um centro pulsional, que é
heterossexual e familiarista, no qual aparentemente toda a organização
será provisória e fará referência à horda. Não há aqui possibilidade de
se constituir um funcionamento alternativo, mas antes o que parece ter
se passado no exílio da horda foi a constituição das possibilidades do
retorno, do assassinato do pai e da instauração do tabu do incesto.

No entanto, ao seguirmos a letra freudiana, percebemos que é
propriamente nos laços homossexuais que repousa qualquer possibilidade
de organização que levará à instauração da vida em civilização. Se
seguirmos a intuição de Freud de um estatuto análogo entre filogênese e
ontogênese, haveria, portanto, um \emph{tempo de compreender} horizontal
entre semelhantes, que se colocaria entre um \emph{instante de ver}
violento, no qual gozo e relações familiares não seriam geridos por uma
sexualidade pensada a partir do recalque, e um \emph{momento de
concluir}, no qual a instauração da lei se faz a partir do par
assassinato/pacto entre os irmãos. E, como sabemos, é tal pacto que será
o cerne do drama edípico.

Mas notemos que a horizontalidade não é um simples desdobramento da
relação concreta estabelecida entre irmãos, mas, sobretudo, uma
modalidade de reflexão sobre o laço que não necessariamente se pauta
pela verticalidade edípica. Um exemplo de tal ideia pode ser encontrada
no início de \emph{Psicologia das massas e análise do eu}, na qual Freud
--- logo após afirmar que haveria uma indistinção entre psicologia
social e individual --- dirá que as relações de um indivíduo com ``seus
pais e irmãos, com seu objeto de seu amor, com seu professor e seu
médico, isto é, \emph{todas as relações que até agora foram objeto
privilegiado da pesquisa psicanalítica, podem} reivindicar ser
apreciadas como fenômenos sociais {[}\ldots{}{]}'' (Freud, 1921/2011, p. 14,
grifo nosso).

Ou seja, Freud afirma que a pesquisa psicanalítica se debruça não sobre
a relação edípica, mas sobre as relações dos sujeitos com seus pequenos
outros próximos ou, nas palavras de Laplanche (2015, p. 167), ao
comentar a aquisição de gênero, seus \emph{socii}: ``quem insere {[}na
designação de gênero{]} não é o social em geral, é o pequeno grupo dos
\emph{socii} próximos. Em outras palavras, o pai, a mãe, um amigo, um
irmão, um primo etc.'' (Laplanche, 2015, p. 167)

Até agora estabelecemos a polaridade Édipo/vertical vs
Semelhantes/horizontal, mas mesmo ela pode ser subvertida a partir de um
exame mais detalhado da tragédia grega que origina a teoria freudiana. A
maldição narrada pelo oráculo de Delfos sobre o destino do filho de
Laios (matar o pai e esposar a mãe) tem um antecedente digno de nota:

\begin{quote}
Laios (o torto, em grego), de acordo com a mitologia grega, é o pai de
Óidipous ou Édipo, e filho de Lábdacos, rei de Tebas. Seu pai foi morto
por bacantes vingativas pela repressão ao culto a Dionísio. Como Laios
ainda era criança, a regência de Tebas foi entregue a Lico. Quando os
tiranos Anfião e Zeto mataram o regente e tomaram o poder na cidade, o
príncipe de Tebas foi exilado, ainda bebê, na Frígia, na corte do rei
Pélops.

Lá foi educado e cresceu. Mais tarde, Pélops teve um filho, Crísipo,
príncipe-herdeiro do trono frígio. Quando este se tornou adolescente,
Pélops pediu a Laios que fosse seu preceptor, e este se apaixonou pelo
menino.

Esse amor homossexual --- tolerado pelos costumes gregos enquanto
relação pedagógica/pedofílica --- deveria ser interrompido quando
Crísipo se tornasse adulto. Mas não foi o caso.

Para continuar a viver seu amor, Laios armou um plano: ofereceu-se para
escoltar o rapaz até os jogos de Neméia, onde ele iria participar como
atleta. Após as competições, em vez de retornar à Frígia, Laios raptou
Crísipo e fugiu para Tebas, onde pretendia recuperar o trono de seu pai,
Lábdacos.

Furioso, Pélops perseguiu-os. Por ter perdido o herdeiro, Pélops culpou
Laios e lançou sobre ele uma maldição: \emph{``Se tiveres um filho ele
te matará e toda tua descendência desgraçada será!''}
\end{quote}

Curiosa aproximação desse tempo zero da tragédia com ``Totem e tabu'':
um laço homossexual, entrelaçado no contexto de uma ex\emph{pulsão} ---
dado que se trata de uma paixão impossível na cidade --- cujo desfecho é
o paradigma da sexualidade freudiana (a constituição do tabu do incesto
edípico). É surpreendente o fato de nem Freud, nem Lacan --- nem mesmo a
própria Butler, em \emph{O clamor de Antígona} --- fazerem menção a esse
fato que não nos parece anódino, posto que, para muitos autores gregos,
o rapto de Crísipo por Laios teria sido a relação que inaugurou os
amores homossexuais na Grécia. Parece haver, portanto, antes das séries
de identificações familiares, uma sorte de laço --- e, presumidamente,
de identificação --- sexual que independe da verticalidade edípica,
ainda que pareça haver sempre esse um ao qual a horizontalidade se
refere em seu exílio (seja o pai primevo, em Freud; seja Pélops, na
tragédia).

Essa ``coincidência'' entre a anterioridade homossexual em ``Totem e
tabu'' e na tragédia edípica nos remete à discussão butleriana da
anterioridade do tabu contra a homossexualidade em relação ao tabu do
incesto. Butler entende que o próprio movimento de proibição do amor
pelo genitor de um dos sexos, em Freud, é aquele responsável por uma
identificação melancólica com esse (Butler, 1990/2002, p. 98). Essa
proposta encaminha bem, a nosso ver, a discussão referente ao destino da
bissexualidade primária, tão cara à Freud, para além das formações
sintomáticas na neurose, já que o amor ``homossexual'' --- no caso de
crianças futuramente ``heterosexuais'' --- é preservado em forma de uma
identificação. Tanto a culpa dos irmãos quanto a maldição tebana são as
marcas desse abandono melancólico e da constituição de uma identidade,
ainda que marcada por uma estrutura de relações de poder inscritas
historicamente.

Mas tanto em ``Totem e tabu'' quanto no romance de Laios e Crísipo, é
importante dizer que tal possibilidade de instauração de lei diz
respeito apenas aos \emph{homens}. Por mais que Freud (1913) afirme que
a morte do pai significou que as mulheres teriam sido libertadas,
deve-se compreender essa frase a partir da perspectiva desses homens ---
ou seja, as mulheres foram libertadas como objetos, mas não como
sujeitos. Mesmo que ao longo do texto haja alguns poucos exemplos de
agência em mulheres, ao descrever o mito fundador, Freud reservará às
mulheres exclusivamente o lugar de objeto --- o que, mais à frente em
sua obra, terá como polo complementar a aproximação da mulher à figura
da \emph{mãe}.

Retomemos, por fim, o ``Sobre alguns mecanismos\ldots{}''. Freud inicia sua
apresentação do problema com a retomada da seguinte explicação, à época
considerada clássica, da homossexualidade: uma identificação, tomada
aqui como sinônimo de fixação, à figura materna, associada a uma aversão
pelas mulheres. Não obstante, em determinado momento do texto, dirá:
``nós nunca acreditamos, porém, que essa análise da gênese da
homossexualidade fosse completa''. E aqui o texto começa a apresentar
algumas considerações interessantes, nomeadas pelo próprio Freud como um
``novo mecanismo'' na escolha homossexual de objeto: ``a observação me
fez atentar para alguns casos em que haviam surgido, na primeira
infância, impulsos de ciúmes particularmente fortes oriundos do complexo
materno, dirigidos contra rivais, geralmente irmãos mais velhos''.

Trata-se, assim, de um achado, fruto de uma observação clínica na qual,
evidentemente, a homossexualidade ainda não se encontrava desvinculada
de um quadro patológico, no sentido clássico do termo. Não obstante, ao
pensar a questão da patologia em psicanálise, uma certa inversão precisa
ser feita: nosso \emph{universal} é o patológico, na medida em que nossa
teoria do sujeito nele se baseia não como exceção, mas como regra. A
normalidade só pode por nós ser compreendida a partir do
\emph{singular}, na medida em que cada sujeito encontrará uma montagem
relativamente estável e normal, dadas as contingências de sua
constituição subjetiva. Freud tomará nesse texto a homossexualidade como
um paradigma dos laços sociais.

O argumento central seria que tais tendências agressivas contra os
irmãos, que podem até mesmo chegar ao desejo de assassinato, não
resistirão à ação do desenvolvimento e serão --- a partir da educação,
mas igualmente da impotência de seus desejos --- \emph{recalcadas}. O
resultado imediato de tal processo é que serão esses mesmos antigos
rivais, do mesmo sexo, os primeiros objetos de amor: eis seu novo achado
clínico sobre a gênese da homossexualidade. Freud vai além e pontua que
esse tipo de resultado do apego à mãe seria o mesmo da \emph{paranoia
persecutoria} --- em que primeiros objetos de amor tornam-se
perseguidores e pessoas odiadas tornam-se amadas. Donde se compreende o
motivo pelo qual é discutido nesse texto justamente o que haveria em
comum nos mecanismos do ciúme, da paranoia e da homossexualidade; ou
seja, há um traço comum entre todas essas manifestações, a saber: a base
de toda possibilidade de comportamento social, já que ``os sentimentos
de identificação afetuosos e sociais aparecem como formações reativas
contra os impulsos agressivos reprimidos''. (Freud, 1922/2011, p. 223.
trad. modificada)

Teríamos aqui, portanto, uma sorte de correspondente subjetivo de
aspectos mais gerais presentes em ``Totem e tabu'', como a agressividade
na base dos laços sociais, a partir de uma transformação de moções
hostis em amor. Há, igualmente, a ideia de que a base para a instauração
dos laços sociais passa pela homossexualidade como modalidade primeira
de relação, que sofre então a ação do recalque. Temos aí, num mesmo
movimento, uma teoria tanto psicológica quanto social, na medida em que
o recalque estruturaria tanto o desejo quanto a regulação do
\emph{socius}. Pode-se dizer que haveria, a partir desse texto, uma
teoria de sexuação em Freud segundo a qual uma dada identidade sexuada
advém, portanto, da superação de tendências agressivas em relação ao
semelhante.

Mais ainda, parece haver aí uma concepção subjacente de sociedade que
não necessariamente se descreve apenas pelo pilar da lei do Pai, me se
funda no atravessamento da agressividade contra o semelhante. Essa
espécie de horizontalização da relação do sujeito frente ao outro, que
articula o sexual e o social, talvez permitisse contrabalancear o pendor
exclusivamente patriarcal não só da teoria, mas da própria epistemologia
psicanalítica.

\section{Referências}

Ambra, P. (2018) A psicanálise no nascimento da identidade de gênero e a
recepção de Robert Stoller na teorização de Jacques Lacan. Em Porchat,
P. \& Mouammar, C. C. E., \emph{Psicanálise e interfaces}. 1ª Ed.
Curitiba: CRV

Butler, J. (1990/2014). \emph{Problemas de gênero.} (R. Aguiar, Trad.)
Rio de Janeiro: Civilização Brasileira.

Butler, J. (2009). Le transgenre et les ``attitudes de révolte''. Em M.
David-Ménard, \emph{Séxualités, genres et mélancolie: s'entretenir avec
Judith Butler} (p. 223). Paris: CampagnePrémière.

Carone, M. (1984). Da loucura de prestígio ao prestígio da loucura. Em
D. P. Schreber, \emph{Memórias de um doente dos nervos} (M. Carone,
Trad.). Rio de Janeiro: Graal.

Foucault, M. (1978/1982). O verdadeiro sexo. Em H. Barbin, \emph{o
diário de um hermafrodita} (I. Franco, Trad., p. 182). Rio de Janeiro:
F. Alves.

Freud, S. (1896/1980). A etiologia da histeria. Em S. Freud,
\emph{Edição standard brasileira das obras completas de Sigmund Freud,
volume III.} Rio de Janeiro: Imago.

Freud, S. (1905/2016). Análise fragmentária de uma histeria (``O caso
Dora''). Em S. Freud, \emph{Obras completas, volume 6: Três ensaios
sobre a teoria da sexualidade, Análise fragmentária de uma histeria (``O
caso Dora'') e outros textos (1901-1905)} (P. C. Souza, Trad., pp.
173-320). São Paulo: Companhia das Letras.

Freud, S. (1913/2012). Totem e tabu. Em S. Freud, \emph{Obras completas,
volume 11: Totem e tabu, Contribuição à história do movimento
psicanalítico e outros textos (1912-1914)}. (P. C. Souza, Trad., pp.
13-244). São Paulo: Companhia das Letras.

Freud, S. (1920/2011). Sobre a psicogênese de um caso de homosexualidade
feminina. Em S. Freud, \emph{Obras completas, volume 15: psicologia das
massas e análise do eu e outros textos (1920-1923)}. (P. C. Souza,
Trad., pp. 114-149). São Paulo: Companhia das Letras.

Freud, S. (1921/2011). Psicologia das massas e análise do eu. Em S.
Freud, \emph{Obras completas, volume 15: psicologia das massas e análise
do eu e outros textos (1920-1923)}. (P. C. Souza, Trad.). São Paulo.

Freud, S. (1922/2011). Sobre alguns mecanismos neuróticos no ciúme, na
paranoia e na homossexualidade. Em S. Freud, \emph{Obras completas,
volume 15: psicologia das massas e análise do eu e outros textos
(1920-1923)}. (P. C. Souza, Trad., pp. 209-224). São Paulo.

Gayla. (2006). \emph{Brian Henry Reimer.} Acesso em 28/02/2019,
disponível em Find a grave:
https://www.findagrave.com/cgi-bin/fg.cgi?page=gr\&GRid=16482516.

Lacan, J. (1938/2003). Os complexos familiares na formação do indivíduo.
Em J. Lacan, \emph{Outros escritos} (V. Ribeiro, Trad., pp. 29-90). Rio
de Janeiro: Jorge Zahar Editor.

Lacan, J. (1960-1961/1992). \emph{O Seminário, livro 8: a
transferência.} (D. D. Estrada, Trad.) Rio de Janeiro: Jorge Zahar
Editor.

Laplanche, J. (2003/2015). \emph{Sexual: a sexualidade ampliada no
sentido freudiano 2000-2006} (1ª ed.). (V. Dresch, Trad.) Porto Alegre:
Dublinense.

Mitchell, J. (2000/2006). \emph{Loucos e medusas: o resgate da histeria
e do efeito das relações entre irmãos sobre a condição humana.} (M. B.
Medina, Trad.). Rio de Janeiro: Civilização Brasileira.

Quinet, A. (2009). \emph{A maldição dos Labdácidas}. Acesso em
28/02/2019, disponível em Óidipous, filho de Laios:
http://oidipousfilhodelaios.blogspot.com.br/2009/01/maldio-dos-labdcidas.html.

Sófocles. (441 a.C./1990). \emph{Antígona} (15a ed.). (M. D. Kury,
Trad.) Rio de Janeiro: Jorge Zahar Editor.

Vorsatz, I. (2013). \emph{Antígona e a ética trágica da psicanálise.}
Rio de Janeiro: Jorge Zahar Editor.

Schreber, D. P. (1905/1984). \emph{Memórias de um doente dos nervos.}
(M. Carone, Trad.) Rio de Janeiro: Graal.



\chapter*{Vulnerabilidade fundamental}
\addcontentsline{toc}{chapter}{Vulnerabilidade fundamental, \footnotesize\emph{por Janaína Namba}}
\hedramarkboth{Vulnerabilidade fundamental}{}

\begin{flushright}
\emph{Janaína Namba}
\end{flushright}

Em \emph{Totem e Tabu} (1913), Freud descreve uma horda primordial
hipotética (pressuposta por Darwin, que buscava nos macacos um esboço do
comportamento social primitivo humano), que contava com um pai violento
e ciumento e reservava todas as mulheres da horda para si e expulsava os
filhos varões assim que cresciam. Como diz Freud, ``um dia os irmãos
expulsos se uniram, se aliaram, mataram e devoraram o pai. Unidos
ousaram fazer e levaram a cabo o que individualmente teria sido
impossível'' (Freud, 1913/2013, p. 207).

Após a morte do pai da horda primitiva, decorrente da conjuração dos
irmãos, fica instituída a proibição do incesto, para que nenhum deles
tenha novamente o privilégio que um dia tivera o pai. Com isso, aquele
que ousasse obter novamente o lugar do pai, se arriscaria a ser
assassinado como ele fora. Como diz Freud,

\begin{quote}
A necessidade sexual não une os homens, e sim os divide. Por mais que os
irmãos tivessem se aliado para subjugar o pai, cada um era rival do
outro quanto às mulheres. Cada um teria desejado todas para si, como o
pai. E na luta de todos contra todos, a nova organização teria
sucumbido. (Freud, 1913/2013, p. 210)
\end{quote}

A nova organização envolve, portanto, uma convivência de aliança entre
os irmãos, que renunciariam ao poder garantindo a vida de cada um do
grupo e das mulheres de seu próprio grupo. O parricídio e a proibição do
incesto tornavam necessária a busca de mulheres fora do clã primitivo.

Podemos dizer que a instituição da proibição do incesto é o fato mais
importante para a constituição da cultura. É considerada a primeira lei
negativa da exogamia. Uma proibição como essa implica uma busca
necessária das mulheres fora do clã, mas não de maneira aleatória, nem
individual, como na condição do estado de natureza. Segundo Dumont,
Lévi-Strauss considera a interdição do incesto como uma ``expressão
negativa'' e parcial de uma lei de troca universal, do princípio
universal de reciprocidade. Ela viria instaurar uma relação de alianças
entre famílias, ou seja, inaugurar as relações de alianças sociais.
(Dumont, 1997, p.117) Nas palavras de Lévi-Strauss,

\begin{quote}
O fato de a proibição do incesto ser vista de maneira inteiramente
independente das regras de suas modalidades, a constitui como uma regra,
por excelência. Pois a natureza abandona a aliança ao acaso e à
arbitrariedade, de tal modo que se torna impossível à cultura introduzir
uma ordem, seja qual for sua natureza. O papel primordial da cultura é
garantir a existência do grupo como grupo, e, portanto, substituir nesse
domínio, como em todos os outros, a organização ao acaso. A proibição do
incesto constitui determinada forma --- ou ainda, formas muito diversas
--- de intervenção. Mas antes de qualquer coisa ela é a Intervenção.
(Lévi-Strauss, 1967, p. 38)
\end{quote}

Ou ainda, segundo Keck, em \emph{Lévi-Strauss, uma introdução (2011)},
``não é porque há uma interdição que a proibição do incesto é uma regra,
mas ela é a regra mínima na qual é instaurada a interdição e da qual se
torna possível, por sua vez, todas as regras positivas.'' (Keck, 2011,
p.90) De modo que a característica universal da proibição do incesto,
não encerra um valor em si mesmo, ao contrário a universalidade seria
justamente uma forma vazia que permite a constituição de valores
diversos. Freud nos diz que embora

\begin{quote}
Esse estado primordial da sociedade não se tornou objeto de observação
em parte alguma. As organizações mais primitivas que encontramos, e, que
ainda hoje vigoram em certas tribos, são associações de homens,
constituídas por membros com os mesmos direitos e submetidas às
restrições do sistema totêmico. (Freud, 1913/2013, p.207)
\end{quote}

Em vista disso, é forçoso reconhecer que Lévi-Strauss tem razão: a
psicanálise de fato teria se voltado para as teorias sociológicas, de um
Robertson Smith, por exemplo, ao resgatar o chefe primitivo enquanto
``pai simbólico'', representado pelo totem. Segundo Freud,

\begin{quote}
A psicanálise nos revelou que o animal totêmico é realmente um
substituto do pai, e se harmonizava com isso a contradição de que
normalmente é proibido matá-lo e que sua morte se transforme em
festividade --- que o animal seja morto e, no entanto, pranteado. A
atitude emocional ambivalente que ainda hoje caracteriza o complexo
paterno em nossas crianças, e que muitas vezes prossegue na vida dos
adultos, também se estenderia ao animal totêmico substituto do pai.
(Freud, 1913/2013, p.206)
\end{quote}

Ainda que a antropologia estrutural e a psicanálise apresentem poucos
pontos convergentes, essas disciplinas concordam com o fato de que a
proibição do incesto seja o marco inaugural da cultura. De acordo com a
psicanálise essa proibição só ocorre por que houve o assassinato do pai
da horda primitiva, como anteriormente mencionado. A despeito do
sarcasmo levistraussiano com relação ao totemismo, noção que estaria
contaminada pelos preconceitos teóricos relativos à velha sociologia e
às voltas com um \emph{a priori} desmentido pela observação de campo
munida de conceitos pertinentes à etnografia, a psicanálise freudiana
está longe de ser uma fantasia, também ela apresenta um construto
teórico calcado e amparado numa clínica própria.

Antes de nos adentrarmos na psicanálise, podemos refletir um pouco sobre
a afirmação de Lévi-Strauss a respeito do resgate da velha antropologia
preconceituosa do século XIX, mais precisamente na ideia do totem como
substituto do pai. Independentemente das críticas feitas pelo etnólogo,
que foi um leitor cuidadoso da obra freudiana, perseguindo-a ao longo
dos anos como se fosse um obstáculo a ser ultrapassado pela antropologia
estrutural, de fato temos que Freud utilizar-se-á de uma ideia
patriarcal de Totem, baseando-se particularmente na principal obra de
Frazer, \emph{Totemismo e Exogamia}, de 1910, que irá permear Totem e
tabu do início ao fim, e em Robertson Smith no que diz respeito aos
ritos sacrificiais, à organização social em torno do totem e
principalmente, por mostrar em \emph{A religião dos Semitas}, que
``homicídio e incesto, ou ofensas desse tipo contra as sagradas leis do
sangue, são na sociedade primitiva os únicos crimes de que a comunidade
como um todo toma conhecimento.'' (Smith apud Freud, 1913, p. 150). Ora,
esses crimes seriam justamente os crimes de Édipo que comporiam, por sua
vez, o núcleo das psiconeuroses.

Renato Mezan em o \emph{Tronco e os Ramos} (2014) sugere que as
principais figuras tabu, no ensaio ``Tabu e ambivalência dos
sentimentos'' referem-se todas às figuras paternas, a saber, os
governantes, os assassinos e os mortos, de modo que ``os tabus seriam
então uma medida protetora contra o ódio velado dirigido a essas
figuras.'' (Mezan, 2014, p. 147) Ora, para Freud, o assassinato do pai
da horda fora um fato irremediável e sem retorno, só restava aos irmãos
e aos seus descendentes pranteá-lo e reatualizá-lo na forma de ritos, no
entanto, formava-se uma nova organização que para continuar existindo,
como discorreremos mais adiante, deveria impor uma lei de proibição ao
incesto que impedisse a relação sexual com as mulheres do mesmo clã.

Freud infere que essa nova organização assenta-se sobre laços fraternos
e teria uma base afetiva de cunho homossexual masculino, contendo ela
mesma os germes das instituições de direito materno, posteriormente
substituídas por uma ordenação patriarcal da sociedade.

A união dos irmãos expressaria, portanto, uma solidariedade e o vínculo
libidinal depositados no ódio comum contra o pai, que segundo Monzani,

\begin{quote}
eis o golpe de gênio de Freud: se é o ódio que transforma os seres
submetidos em irmãos, é seu assassinato que constitui o chefe da horda
em pai. Em outros termos, o pai não existe senão como ser mítico. O pai
é sempre um pai de morte.\footnote{Monzani, L. R. Rev. Filos., Aurora,
  Curitiba, v. 23, n. 33, p. 245, jul/dez 2011.}
\end{quote}

Se por um lado temos esse aspecto genial da teoria freudiana que remonta
a origem da sociedade em termos míticos, calcada no assassinato do pai
que viria justamente ser responsável pela vulnerabilidade estrutural a
qual vive o humano. Por outro, essa união exclusivamente masculina dos
irmãos em torno do assassinato do pai (homem) leva a um protagonismo
masculino e a uma identidade da psicanálise da cultura como sendo de
cunho predominantemente patriarcal. Podemos nos perguntar se seria essa
identidade apenas uma coincidência com os valores, a história e a
sociedade daquele tempo e do nosso, ou se esse viés já seria ele mesmo
uma escolha teórica?

Voltemos então à psicanálise às voltas com a narrativa totêmica.

Após a morte do pai primordial, os irmãos os devoraram e choraram a sua
morte. Ainda que os filhos tenham tido uma atitude hostil pelo fato de
odiarem o pai, também sentiam por ele admiração pelo seu poder e por sua
força. Apesar de ter seu lugar desejado, ninguém mais ousava ocupar o
lugar do pai. A atitude ambivalente surgiria então pelo ódio e pelo
desejo de ser como o pai.

É então a partir da ideia da refeição totêmica e do animal sacrificial
de Robertson Smith que Freud associou-os ao complexo paterno: ``A forma
mais antiga de sacrifício era {[}\ldots{}{]} o sacrifício animal, cuja carne
e cujo sangue, o deus e seus adoradores consumiam em comum.'' (Freud,
1913/2013, p. 198) Ou seja, o sacrifício animal exigia que todos
comessem e bebessem juntos, numa refeição em que se comemorava a
solidariedade mútua com a divindade. A refeição sacrificial, que se
tornou simbólica, baseava-se em antigas ideias de que comer e beber
junto com outra pessoa unindo-as numa comunhão social em que possuíam
obrigações recíprocas. (Freud, 1913/2013, p.199)

O animal sacrificado na refeição comunitária era o animal que dava
origem ao totem ou ao clã. O animal totêmico era um deus primitivo,
intocável, inviolável e só podia ser consumido num ritual coletivo entre
os membros do clã, pois nesse ritual celebrava-se a renovação e a
semelhança entre os membros do clã e seu deus. (Freud, 1913/2013, p.
201) Um ritual como esse envolve o lamento e luto perante a morte do
animal sagrado, mas esse é justamente o momento permitido para uma
comemoração festiva, em que ocorre a ingestão da substância do totem,
havendo consequentemente uma identificação com ele: ``nele há o
desencadeamento de todos os impulsos e a permissão para todas as
satisfações.'' (Freud, 1913/2013, p. 206) É nesse sentido, portanto, que
Freud aproxima o animal totêmico de um substituto do pai:

\begin{quote}
O violento pai primordial era certamente o modelo invejado e temido de
cada membro do grupo de irmãos. Agora no ato de devorá-lo, consuma-se a
identificação com ele; cada um se apropria de uma parte de sua força. A
refeição totêmica, talvez a primeira festa da humanidade, seria a
repetição e a comemoração desse ato memorável e criminoso com o qual
tantas coisas tiveram seu início, tais como as organizações sociais, as
restrições morais e a religião. (Freud, 1913/2013, p. 207-208)
\end{quote}

Se por um lado temos que somente com a união entre os irmãos é que a
morte do pai pôde ser consumada, fato fundamental para que a cultura se
constitua, sabemos também que em 1930, Freud aponta para esse ato
violento como um ato grávido de consequências para a cultura humana.
Esse seria o principal responsável pelo mal estar em que subjaz a vida
em sociedade, uma vez que se desenrola em torno do ódio ao pai. Nas
palavras de Freud já em 1913,

\begin{quote}
A sociedade se apoiava a partir de então na cumplicidade em torno da
culpa do crime cometido em comum; a religião, na consciência dessa culpa
e no arrependimento relativos a ele; a moralidade, em parte nas
necessidades dessa sociedade, e em outra parte nas expiações exigidas
pela consciência de culpa. (Freud, 1913/2013, p. 213).
\end{quote}

Entretanto, podemos dizer também que tanto a união quanto o ato
decorrente da união trazem consigo uma nova condição para os irmãos.
Esse laço entre os irmãos é inteiramente novo, uma ligação que passou a
existir, justamente para que o assassinato do pai pudesse ocorrer. Com
isso o lugar do pai, tal como estava ocupado lançava-os para uma
condição impossível de querer ser e estar como o pai:

\begin{quote}
A família foi uma restauração da antiga horda primordial, e também
restituiu aos pais uma grande parte de seus antigos direitos. Agora
havia novamente pais, mas as conquistas sociais do clã dos irmãos não
haviam sido abandonadas, e, a distância factual dos novos pais de
família em relação ao ilimitado pai primordial da horda era grande o
suficiente para assegurar a continuação da necessidade religiosa, a
conservação da insaciada saudade do pai. (Freud, 1913/2013, p. 217)
\end{quote}

Essa nova família, ainda que composta também pelo lugar pai, que viria a
ser ocupado por algum dos irmãos e não o mais forte, contava com uma
condição diferente, de modo que era então necessário buscar um novo
estatuto para o pai fora da horda e dentro da cultura que se instaurava.
Com isso esse lugar do pai contém também a inveja, a culpa e, sobretudo,
o vazio que representa esse lugar e essa nova condição. Um vazio que,
podemos pensar, contorna a cultura e se inscreve no indivíduo.

Assim, a morte do pai primevo pela união dos irmãos teria deixado
``traços indeléveis na história da humanidade'', os quais se expressam
em ``formações substitutivas.'' (Freud, 1913/2013, p. 224) E essas
formações não seriam outra coisa que processos psíquicos incorporados e
transformados ao longo do tempo, que na atualidade compõem nossas
disposições psíquicas. Como ressalta o próprio Freud, a
\emph{consciência de culpa} onipresente, imemorial, seja no âmbito da
neurose, seja no âmbito cultural, seria uma consequência desses crimes
primordiais. Ou seja, podemos reconhecer os traços deixados pela
humanidade não apenas nas incorporações das disposições psíquicas do
indivíduo, mas também na própria sociedade. Desse modo Freud anuncia
que,

\begin{quote}
no complexo de Édipo coincidem os inícios da religião, da moralidade, da
sociedade e da arte, em completa concordância com a constatação da
psicanálise de que esse complexo constitui o núcleo de todas as
neuroses, tanto quanto até agora elas cederam à nossa compreensão.
Parece-me uma grande surpresa que também esses problemas da vida
psíquica dos povos admitam uma solução a partir de um único ponto
concreto como é a relação com o pai. (Freud, 1913/2013, p. 226)
\end{quote}

Numa carta (julho de 1915) a Ferenczi, Freud diz que na época da
glaciação, a vida amorosa teria se tornado agressiva e egoísta e a
\emph{neurose obsessiva} seria uma defesa contra esse tipo de vida que
havia se configurado. Também como decorrência da renúncia às mulheres do
clã, se desenvolveria a demência precoce (esquizofrenias). A paranoia,
por sua vez se defende da organização homossexual da horda primitiva e a
mania-melancolia emergiria da identificação com o pai, e, do triunfo de
tê-lo vencido.

Constata-se assim, através dessas hipóteses freudianas, que, seja pelas
condições exteriores, seja pelos próprios atos humanos, ao longo da
filogênese emergiram alguns tipos de defesa contra a vulnerabilidade, ou
esse vazio outrora vivido.

Essa situação é repetida, ontogeneticamente, pela própria condição
humana, pela imaturidade e dependência do bebê que vive ``o desamparo
inicial como uma das fontes de todos os motivos morais'' (Freud,
1895/2003, p. 363). Dessa maneira, as defesas individuais também seriam
atualizações das defesas da espécie, ou seja, ainda que não haja uma
ameaça exterior ou uma situação de tamanha vulnerabilidade como aquela
que existira no período da glaciação, o indivíduo vive internamente essa
vulnerabilidade.

Ainda que o homem contemporâneo não esteja mais a mercê da natureza como
um dia estivera, carrega consigo a memória inconsciente dessa
vulnerabilidade ancestral. Reage como se seu desamparo fosse completo,
como se vivesse em meio a natureza hostil.

\begin{quote}
Compreendemos as primeiras prescrições e restrições morais da sociedade
primitiva como reação a um ato que deu a seus autores o conceito de
crime. Eles se arrependeram desse ato e decidiram que ele não deveria
mais ser repetido e que sua execução não poderia ter trazido ganho
algum. Essa criadora consciência de culpa não se apagou entre nós.
Podemos encontrá-la atuando de maneira associal nos neuróticos com a
finalidade de produzir novas prescrições morais e limitações contínuas
sob a forma de expiação pelas más ações que ainda serão cometidas. Mas
quando investigamos esses neuróticos em busca dos atos que despertaram
tais reações, ficamos decepcionados. (Freud, 1913/2013, p. 228-229)
\end{quote}

Mesmo sem saber do ato criminoso primordial, o desamparo e a
vulnerabilidade são vividos pelo neurótico contemporâneo sem que haja a
existência de uma realidade exterior e factual. As condições em que a
família reatualiza as relações um dia estabelecidas recuperam de maneira
inconsciente tanto o lugar do pai um dia almejado quanto o lugar dos
filhos que cometeram o crime primordial. Ou seja, a aliança e a culpa
são recompostas e não podem ser esquecidas. O desamparo é reiterado no
seio de cada família e com isso reproduz-se nessas relações familiares a
própria ambivalência, isto é, esse amor e esse ódio destinados ao pai.
Ambivalência, a culpa e o próprio desamparo são transmitidos a cada
geração, isso porque, nas palavras de Freud,

\begin{quote}
Em primeiro lugar, não poderá ter escapado a ninguém que sempre nos
baseamos na hipótese de uma psique de massa na qual os processos
psíquicos transcorrem como na vida psíquica do indivíduo. Supomos,
sobretudo, que a consciência de culpa referente a um ato pode sobreviver
por muitos milênios e permanecer eficaz em gerações que nada podiam
saber desse ato {[}\ldots{}{]} e se os processos psíquicos de uma geração não
continuam na seguinte, se cada geração tivesse de adquirir novamente sua
atitude diante da vida, não haveria qualquer progresso nesse campo e
praticamente nenhuma evolução. Surgem, então, duas novas questões: o
quanto se pode atribuir à continuidade psíquica nas séries de gerações,
e de que meios e caminhos uma geração se serve para transferir seus
estados psíquicos. (Freud, 1913/2013, p. 227)
\end{quote}

A partir desses questionamentos, temos que existe uma transmissão seja
de esquemas psíquicos, seja de conteúdos psíquicos inconscientes, sendo
esses últimos não necessariamente reprimidos. Nesse texto, Freud
menciona apenas que existem predisposições. Entretanto sabemos que são
os esquemas dados pelas fantasias das séries complementares apresentadas
na 23ª conferência de 1916. Essas predisposições da espécie, ou seja, a
própria estruturação do aparelho psíquico também depende daquilo que é
contingente, ou ainda, individual. O que o leva em Totem e tabu a
retomar as palavras de Goethe: ``o que herdaste de teus pais,
conquista-o, para que o possuas.'' (Goethe apud Freud, 1913/2013, p.
228)

Dessa maneira, temos que os processos psíquicos que anteriormente
aconteceram desde fora e posteriormente foram incorporados e
transmitidos por atos, cerimônias, relações que viriam disfarçar esse
inconsciente que se propaga, segundo Freud, foram concebidos como
citamos acima, ``as primeiras prescrições e restrições morais da
sociedade primitiva.'' (Freud, 1913/2013, p. 228) Isso fez com que a
consciência de culpa perdurasse no inconsciente ao longo de toda a
filogênese, repetindo-se ontogeneticamente. Segundo Monzani temos que

\begin{quote}
Por mais que rodeemos, há sempre um ponto cardeal de toda a civilização:
o complexo de Édipo. E, também, de cada indivíduo particular. Toda a
estrutura da sociedade depende desse fator. (Monzani, 2011, p. 245)
\end{quote}

Ou seja, ao considerarmos a cultura e a sociedade do ponto de vista da
psicanálise freudiana devemos também considerar as relações a partir do
complexo de Édipo, uma vez que desde o estabelecimento da exogamia a
partir da proibição do incesto, temos ``os elementos fundadores da
família e, portanto, das estruturas do próprio indivíduo.'' (Monzani,
2011, p. 246) Há, portanto, um problema da civilização que é uma questão
edípica e que deve ser tratada tanto no âmbito da psicologia individual
quanto no âmbito da psicologia social.

Dito isto temos que para Freud o complexo de Édipo se comporta de
maneira diferenciada para o menino e para a menina, mesmo que para ambos
haja uma combinatória entre as identificações e as escolhas de objeto de
amor, há para Freud o que se pode chamar de Édipo normal positivo e
Édipo normal negativo. Isto é, há um predomínio quanto à rivalidade
entre a criança e o progenitor do mesmo sexo e um predomínio quanto à
rivalidade entre a criança e o progenitor do sexo oposto
respectivamente.

No texto de 1931, \emph{Sobre a sexualidade feminina}, Freud diz que a
sexualidade da mulher se decompõe em duas fases, sendo a primeira
exclusivamente masculina e a segunda exclusivamente feminina e,
``portanto, no desenvolvimento feminino existe um processo de transporte
de uma fase a outra, sem análogo no menino.'' (Freud, 1931/2004, p. 230)
Esse processo de transporte diz respeito à mudança que ocorre durante
fase genital/fálica do clitóris, na infância para a vagina na puberdade,
no segundo tempo da fase genital. Essas mudanças teriam correspondência
no processo de escolha de objeto.

Será reiterado por Freud, nesse mesmo texto que trazemos conosco,
constitucionalmente, uma bissexualidade, que pode ser observada de modo
mais evidente na menina, por consequência da existência de um clitóris
(representante masculino do pênis na mulher) e uma vagina (representante
feminino por excelência). Com isso as etapas do desenvolvimento psíquico
envolveriam a mãe como objeto de amor originário, e, um objeto de amor
escolhido durante o complexo de Édipo, tanto para a menina quanto para o
menino. Particularmente na menina esse objeto secundário é o resultado
de uma escolha, passa por um processo de transformação e é deslocado
para o pai-homem, ou seja, há uma modificação na via sexual da mulher
que corresponde a uma mudança do sexo de seu objeto de amor. (Idem)

Freud acaba por colocar na forma de perguntas os possíveis caminhos e as
possibilidades de completude do desenvolvimento psíquico-sexual da
mulher, em outras palavras, sugere que para o desenvolvimento ser
completo há uma escolha objetal do sexo oposto ao dela, isto é, a
escolha do pai como objeto de amor:

\begin{quote}
Somente um terceiro desenvolvimento implica, sem dúvida, rodeios que se
encerram na configuração feminina que toma o pai como objeto e assim, a
forma feminina do Complexo de Édipo. Portanto, o complexo de Édipo na
mulher, tem como resultado final um desenrolar mais prolongado; não é
destruído pelo influxo da castração, mas criado por ele; escapa às
intensas influências hostis que no menino produzem um efeito destrutivo.
(Freud, 1931/2004, p. 232)
\end{quote}

Se já havíamos constatado os resultados freudianos quanto à configuração
do complexo de Édipo na menina, ainda não havia sido enunciado nesse
texto o processo de castração seja no menino, seja na menina. Freud se
refere então a esse processo de castração no menino como tendo um efeito
destrutivo, responsável pelo término do complexo de Édipo. Já a menina
testemunha uma coincidência entre o início do complexo de Édipo e a
castração.

Não é nosso propósito aqui adentrarmos na própria sexualidade feminina
individual, mas verificar, nos termos de Totem e tabu, o que poderia ter
sido o impulso para a constituição da sexualidade feminina tal qual
descrita por Freud nesse texto de 1931. Ou seja, tendo como principais
características esses dois tempos tanto corporais (passagem do clitóris
para a vagina) quanto psíquicos (mudança de escolha de objeto da mãe
para o pai).

Ao compararmos os processos individuais com os processos ocorridos desde
a horda, é possível estabelecer um paralelo, no menino, quanto à
castração, ou melhor dizendo quanto ao horror à castração. A castração
seria a própria expulsão dos filhos, que não foram escolhidos, da horda
primeva. Esses filhos passavam a viver sozinhos, caminhantes a esmo
lutando contra a natureza e contra outros homens. A identificação
amorosa com o pai da horda e o ódio por terem sido expulsos fez com que
esses filhos se unissem criando assim uma ligação fraterna inexistente e
retornassem para assassinar o pai opressor. Ora a castração viria marcar
o fim de um processo de desenvolvimento do menino dentro da horda, e, o
que significaria, muito provavelmente, o fim da vida do menino. Daí o
tamanho horror à castração. Se esse processo da união dos irmãos e o
assassinato do pai fazem a passagem da natureza para a cultura, nos
termos individuais encontramos uma dissolução do complexo de edípico em
que haveria uma consolidação da masculinidade no menino, que por sua vez
viria indicar a força para não deixar sucumbir a nova organização que os
deixou desamparados.

Se do ponto de vista masculino individual há uma consolidação da
masculinidade com a dissolução edípica, de maneira análoga haveria o
``reforço ou o estabelecimento'' de uma identificação com a mãe que
viria fixar o caráter feminino na criança. (Freud, 1923/2003, p. 35) No
que diz respeito às mulheres da horda, há uma aliança prévia entre elas
e a castração não é vista com horror, uma vez que não são expulsas, ao
contrário, têm a garantia do amor permanente que em contrapartida
devolvem na forma de obediência a esse pai. O complexo de Édipo
inicia-se com a castração do outro. A passagem do objeto primordial (a
mãe) para outro objeto de amor (pai) é dada dentro horda, daí podemos
extrair o que diz Freud a respeito de um prolongamento da fase
pré-edipica na menina. No entanto, o que vemos não é apenas uma
dissolução do complexo feminino, mas uma verdadeira dissolução da horda.
Disso decorreria que as mulheres se ateriam de maneira mais prolongada a
essa saudade do período de horda, onde suas garantias já estavam dadas.
Dessa aliança inicial podemos extrair algumas características: ainda que
seja uma aliança homossexual composta exclusivamente por mulheres, ela
depende do amor do pai, o que resultaria em uma rivalidade entre elas, o
que implicaria numa ambivalência nas relações entre as mulheres.

Se por um lado temos que a consciência de culpa propagou-se e ainda
podemos encontrá-la enquanto conteúdo psíquico do neurótico obsessivo
como se houvesse realmente cometido um crime, por outro temos a
propagação de uma nova aliança entre os irmãos enquanto fundamento
psíquico, isto é, essa nova aliança que foi incorporada seria o solo
psíquico relacional em que pôde surgir a própria consciência de culpa. A
união dos irmãos é que veio proporcionar uma nova \emph{Realidade} no
sentido de conceito metapsicológico tal como a definem Abraham e Torök.
(Abraham e Törok, 1987, p. 237-238) Essa realidade que assim como o
desejo, nasce do interdito, ``é comparável a um delito, até mesmo a um
crime.'' (Ibidem, p. 238) Isso porque ela mesma veio limitar futuros
atos criminosos após o crime primordial, ela já nasce ``com a exigência
de permanecer escondida'' uma vez que veio por fim a ao desejo de ser o
pai primevo. Com isso ela também exige outro estatuto para esse pai e
consequentemente acaba por velar e desfigurar o desamparo e a
vulnerabilidade que os irmãos e novos pais carregam consigo. Essa
vulnerabilidade é propagada na forma de um segredo, podemos dizer, na
forma de outra ``realidade metapsicológica'' que aquela outrora vivida.

Por outro a escolha freudiana em relegar às mulheres uma posição passiva
dentro da cultura ou associar tudo o que é feminino à passividade, ao
nosso ver decorre do próprio papel que essa exercia dentro da horda e
que carregara consigo para fora dela. Dessa maneira, ao formular o mito
da horda primitiva, a teoria freudiana parece ter legado aos homens a
determinação de conviverem com o constante horror à castração, unidos em
torno do ódio e, portanto, vulneráveis à própria sorte. Já para as
mulheres, ainda que rivais entre si e também vulneráveis tanto quanto os
homens, parece tê-las poupado de tamanho horror à castração.

\section{Referências Bibliográficas}

Abraham, N. e Törok, M. \emph{A casca e o Núcleo}. Trad. Maria José
Coracini. São Paulo: Escuta, 1987.

Dumont, L. Lévi-Strauss: les structures élémentaires in \emph{Groupe de
filiation et alliance de mariage}. Paris: Gallimard, 1997.

Freud, S. Proyecto de psicología (1950 {[}1895{]}) In: Publicaciones
prepsicoanalíticas y manuscritos inéditos en vida de Freud. Trad. José
Etcheverry. Buenos Aires: Amorrortu Editores, 2003

Freud, S. Tótem y tabú (1913). In: Tótem y tabú y otras obras:
1913-1914. Trad. José Etcheverry. Buenos Aires: Amorrortu Editores,
2003.

Keck, F. Claude Lévi-Strauss, une introduction. Paris:
Pocket/Agora, 2011.

Lévi-Strauss, C. Les structures élémentaires de la parenté. Paris:
Mouton\&Co, 1967.

Mezan, R. O tronco e os Ramos: Estudos de história da psicanálise. São
Paulo: Companhia das Letras, 2014.

Monzani, L. R. Revista de Filosofia Aurora, Curitiba, v. 23, n. 33, pp.
243-255, jul/dez 2011.
