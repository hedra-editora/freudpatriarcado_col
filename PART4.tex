\part{Contra o mestre: Freud em atrito com as ideias de seus contemporâneos}


\chapter*{O patriarcado entre Sigmund Freud e Otto Gross}
\addcontentsline{toc}{chapter}{O patriarcado entre Sigmund Freud e Otto Gross,\\ \footnotesize\emph{por Marcelo Amorim Checchia}}
\hedramarkboth{O patriarcado entre Sigmund Freud e Otto Gross}{}


\begin{flushright}
\emph{Marcelo Amorim Checchia}\footnote{Psicanalista, mestre, doutor e
  pós"-doutorando em Psicologia Clínica pela Universidade de São Paulo,
  autor de \emph{Poder e política na clínica psicanalítica} (2015),
  organizador de \emph{Combate à vontade de potência} (2016), um dos
  organizadores da edição brasileira das \emph{Atas da Sociedade
  Psicanalítica de Viena} (2015) e da edição brasileira das obras de
  Otto Gross, intitulada \emph{Otto Gross: Por uma psicanálise
  revolucionária} (2017).}
\end{flushright}

O tema do patriarcado é tão espinhoso para a psicanálise que é
impossível tocá"-lo sem que ela saia furada, arranhada ou mesmo
gravemente ferida. É como abraçar um cacto ou passar a mão, a
contrapelo, em um porco espinho. Abordar e analisar este tema, contudo,
é necessário, se quisermos tentar salvar a psicanálise do próprio
patriarcado. Para tanto, vale tomarmos a lição deixada por Walter
Benjamim em sua sétima tese sobre o conceito de história, ainda que ali
ele se refira ao materialista histórico: a tarefa deste consiste, diz
ele, em ``escovar a história a contrapelo''. O que ele quis dizer com
isso? Trata"-se, em suma, de abandonar a história ``oficial'' ou
``universal'' --- que adota sempre a perspectiva dos vencedores e
dominadores ---, de excluir seus elementos épicos e de recusar qualquer
identificação com seus ditos heróis.

Mas o que encontramos quando passamos a escova a contrapelo na história
do patriarcado na psicanálise? Por que este tema é tão espinhoso? São
inúmeros os problemas encontrados a cada escovada em diferentes níveis
de pelos desse animal pavoroso: no próprio uso do conceito de
patriarcado, na construção e transmissão de outros conceitos, na relação
interpessoal entre os psicanalistas e na política institucional das
sociedades psicanalíticas. Restringir"-me"-ei aqui a apontar alguns desses
problemas tomando como referência a história de como o patriarcado
incidiu nas obras de e na relação entre Sigmund Freud e Otto Gross.

\section{Primeira escovada: o conceito de patriarcado na obra freudiana}

O tema do patriarcado é abordado frontalmente e verticalmente por Freud
em apenas dois de seus textos: \emph{Totem e tabu} e \emph{Moisés e o
monoteísmo}. É verdade que \emph{Psicologia das massas e análise do eu}
também pode ser lido nessa chave, especialmente quando Freud analisa a
igreja e o exército e quando associa o líder do grupo ao pai da horda
primeva, mas patriarcado só foi nomeado enquanto tal nos primeiros dois
textos citados.

\emph{Totem e tabu} contém hipóteses bem interessantes, depois retomadas
em \emph{Moisés e o monoteísmo}, que passaram a ser discutidas também
por antropólogos e sociólogos. Apoiando"-se em Darwin e na antropologia
das tribos aborígenes da Austrália, Freud conjectura a origem da
sociedade patriarcal a partir do assassinato do pai da horda primeva. Há
muito insatisfeitos pela obrigatoriedade, imposta pelo pai, de renúncia
às mulheres, os irmãos teriam se unido para assassiná"-lo e devorá"-lo,
pondo fim à horda patriarcal. Para viverem juntos num novo laço fraterno
sem restituir a horda recém derrubada, foi preciso ``instituir a lei
contra o incesto, pela qual todos, de igual modo, renunciavam às
mulheres que desejavam e que tinham sido o motivo principal para se
livrarem do pai'' (Freud, 1913, p. 123). O tabu do incesto tornou"-se,
assim, o princípio sob o qual se erigiu uma nova organização social.

Mas associado a esse tabu, havia outro elemento fundamental: o totem.
Geralmente representado por um animal ou por um símbolo de alguma força
da natureza, cada totem designa um clã. Essa designação demarca a
ancestralidade de cada clã, evitando com que pessoas do mesmo clã se
casassem. A função do totem, portanto, era a de representar
simbolicamente a lei instituída pelo assassinato do pai, a interdição do
incesto, e a de regulamentar, assim, os casamentos. Ao longo do tempo,
porém, o totem foi ganhando outras funções. O poder de sua função
simbólica, associado aos supostos poderes mágicos do animal ou da força
da natureza que o totem representava, acabou fazendo do pai morto uma
divindade, uma autoridade, uma lei. E com ``a introdução das divindades
paternas, uma sociedade sem pai gradualmente transformou"-se numa
sociedade organizada em base patriarcal'' (Freud, 1913, p. 127). A
obediência, antes dirigida ao chefe da horda primeva, agora era
consagrada a um Deus. A religião totêmica tratou, assim, de reinserir a
obediência a um pai, mesmo que a um pai morto, restaurando a ordem
patriarcal.

A família, portanto, originada pela regulamentação dos casamentos e
submetida às leis de Deus, passou a garantir a manutenção da ordem
patriarcal ao simultaneamente continuar restringindo o acesso às
mulheres e ao instituir ao pai de cada família o poder antes exclusivo
ao pai da horda. Mas apesar de restaurar, numa nova configuração, o
patriarcado, ``a distância existente entre os novos pais de uma família
e o irrefreado pai primevo da horda era suficientemente grande para
garantir a continuidade do anseio religioso, a persistência de uma
saudade não apaziguada do pai.'' (Freud, 1913, p. 127).

Dessarte, conforme o tempo foi passando, o ritual de sacrifício do
animal totêmico foi perdendo seu valor sagrado, mas a representação de
Deus foi ganhando cada vez mais força, surgindo então a figura do
sacerdote como intermediário entre Deus e os homens. A própria relação
com Deus tornou"-se hierarquizada, disseminando o patriarcado entre os
próprios homens. Depois ainda surgiram os reis divinos ``na estrutura
social que introduziram o sistema patriarcal no Estado. Devemos
reconhecer que a vingança tomada pelo pai deposto e restaurado foi rude:
o domínio da autoridade chegou ao seu clímax.'' (Freud, 1913, p. 128).
Ao tratar da origem da sociedade patriarcal, Freud bem especifica,
portanto, como ela teria se difundido e ganhado diferentes configurações
sociais: na família, na religião e no Estado.

Entretanto, embora faça um uso bem apropriado do conceito de
patriarcado, ao supor que a instituição da cultura e da civilização é
concomitante à instituição do patriarcado, Freud parece considerar que o
patriarcado é inerente à cultura e a qualquer organização humana.
Estaríamos, deste modo, inexoravelmente fadados a viver sob regimes
patriarcais. Não haveria nada a fazer contra o patriarcado, a não ser
aceitá"-lo, compreendê"-lo e saber bem utilizá"-lo. Essa ideia é reforçada
ainda pelo fato de que em \emph{Moisés e o monoteísmo} --- escrito às
duras penas em sua velhice e no auge do hitlerismo --- Freud complementa
as teses de \emph{Totem e Tabu}, agora analisando a origem do povo
judeu, mas novamente sem questionar se a ordem patriarcal poderia ser de
alguma maneira destituída para dar lugar a um novo modo de organização
social.

Freud também pode ser alvo de objeção, especialmente da parte dos
antropólogos, por citar muito rapidamente Bachofen, importante jurista e
antropólogo que estudou comunidades matriarcais da antiguidade. Segundo
o antropólogo, as primeiras comunidades humanas teriam sido matriarcais
--- voltaremos a esse ponto ---, mas o psicanalista, sem entrar mais
profundamente na discussão, simplesmente situa essas comunidades como um
período de transição entre o patriarcado da horda primeva e o
patriarcado das comunidades totêmicas. Com exceção dessa brevíssima
citação de Bachofen, ao longo de toda sua obra Freud não tece mais
nenhuma reflexão acerca das sociedades matriarcais, o que parece
ratificar a inevitabilidade do patriarcado em qualquer organização
humana.

Por que Freud adota essa posição? Seria por desconhecimento de
antropologia? Ou seria por falta de reconhecimento dos princípios
patriarcais em seu próprio pensamento e personalidade? Talvez um pouco
dos dois, além de outros possíveis fatores desconhecidos? De qualquer
modo, essa posição não é sem consequências para a própria psicanálise.
Fato é que o patriarcado continuou sendo pouco abordado, principalmente
de maneira crítica, pelos psicanalistas. Quando foi enfrentado, foram
psicanalistas mais marginais às instituições psicanalíticas --- casos,
por exemplo, de Wilhelm Reich e Erich Fromm --- que se encarregaram de
trazer esse tema à tona, o que sinaliza que a psicanálise dita
``oficial'' não deixa espaço para essa investigação. O caso de Otto
Gross, psicanalista contemporâneo a Freud, mas segregado e depois
esquecido pelos psicanalistas, não só corrobora essa hipótese como
revela a violência do patriarcado no interior da própria comunidade
psicanalítica.

\section{Segunda escovada: quem foi Otto Gross? }

A vida e a obra de Otto Gross foram profundamente marcadas pelo
patriarcado.\footnote{Ainda que seja uma biografia resumida, muitos mais
  detalhes documentados sobre a vida de Otto Gross podem ser encontrados
  em ``Otto Gross, um psicanalista anarquista''.} Ele foi, ao mesmo
tempo, nas primeiras décadas do século \versal{XX}, o pioneiro e, até hoje, um
dos principais teóricos do patriarcado em psicanálise, e o maior ícone
europeu da luta pai \emph{vs.} filho. Seu pai, Hans Gross --- do qual
ainda falarei mais ---, foi um dos maiores patriarcas da Europa na virada
do século \versal{XIX} para o \versal{XX}. Jurista e professor, ainda hoje ele é
considerado o pai da criminologia moderna por ter instituído um método
de investigação científica para o crime. Ele foi um dos principais
defensores da teoria da degenerescência, segundo a qual todo criminoso,
homossexual, louco, mendigo, isto é, qualquer indivíduo que vivia às
margens dos valores morais seria uma espécie de pervertido e deveria ser
punido ou expulso da sociedade. Hans também se interessava pela
psicologia do criminoso, o que o fez se aproximar de Freud, convidando"-o
para uma das aula na Universidade e para publicar um texto na revista de
criminologia que dirigia. Provavelmente grato pelo espaço dado por uma
figura de tanto prestígio numa época em que ainda era duramente atacado
pelos médicos (o \emph{Três ensaios sobre a teoria da sexualidade} havia
sido recentemente publicado), Freud aceitou o convite e publicou
``Psicanálise e a determinação dos fatos nos processos jurídicos''.

Otto cresceu superprotegido e sob os valores paternos. Em 1899,
incentivado pelo pai, formou"-se em medicina e, em 1902, descobriu a
psicanálise, tendo conhecido Freud pessoalmente em 1904. Seus primeiros
escritos publicados foram mais na área médica e, mesmo aqueles que
tocavam em temas socias e éticos, ainda se aproximavam mais das ideias
do pai. Porém, a partir de 1905, quando passou um período em Ascona ---
uma pequena comuna suíça que abrigava anarquistas fugidos de vários
cantos da Europa --- Otto conheceu o anarquismo não só como campo de
ideias político"-filosóficas, mas como um modo de vida e de relação,
tornando"-se o primeiro psicanalista a articular psicanálise e política e
um dos poucos, até hoje, a sustentar essa associação com o anarquismo.
Desde então, a luta contra o patriarcado e todas as formas de
autoritarismo passou a ser um de seus principais objetivos de vida e
permeou toda sua prática clínica e sua produção teórica.

Seu primeiro texto nessa articulação da psicanálise com o anarquismo é
um perfeito retrato disso. \emph{Violência parental} é simultaneamente
um texto de apelo à libertação de sua analisante Elisabeth Lang --- que
foi internada compulsoriamente numa clínica psiquiátrica por seu próprio
pai, interrompendo assim o tratamento psicanalítico que caminhava bem na
elucidação de seus conflitos --- e de fundamentação dos efeitos psíquicos
do abuso do poder parental. O nexo decisivo do caso, segundo Gross, foi
a comprovação de que seu estado psíquico adoentado devia"-se justamente
às opressões continuamente exercidas pelo pai --- com a internação havia
o perigo de agravamento de sua condição, daí o apelo de Gross para sua
libertação. Gross ainda aproveita o caso para enunciar pela primeira vez
da ``verdadeira origem dos fatores conflitantes recalcados de efeito
patológico'', uma tese distinta a da etiologia sexual das neuroses de
Freud, embora revelada pela técnica freudiana: ``somente a revelação do
inconsciente pela técnica de Freud permite uma visada da psicologia do
conflito da infância e da tremenda importância patológica das sugestões
da educação como causa da neurose de recalcamento'' (\emph{ibid.}). A
educação, enquanto soma de todas as sugestões --- das pequenas seduções
para convencer ou manipular alguém às imposições mais diretas, acaba por
oprimir ou mesmo suprimir o que é mais próprio de cada um.

Por isso, o conflito psíquico primordial, Gross defenderá ao longo de
toda sua obra, se dá entre \emph{o próprio vs. o estrangeiro}, isto é,
entre a singularidade e disposições próprias de cada sujeito e as
arbitrariedades impostas pelo meio externo, advindas principalmente da
família patriarcal.\footnote{Remeto o leitor interessado em uma síntese
  mais desenvolvida sobre o pensamento grossiano ao texto ``Otto Gross e
  o combate à vontade de potência''.} O conflito de ordem sexual
revelado por Freud seria tão somente, ele dirá em \emph{Três ensaios
sobre o conflito interno}, um desdobramento do conflito primordial com
o princípio de autoridade. Se Freud havia abandonado a teoria do trauma
para dar relevo ao papel da fantasia na constituição dos conflitos
psíquicos, Gross de certo modo retoma a teoria do trauma sem descartar a
teoria da fantasia. As fantasias, assim como os sonhos e os sintomas,
estariam repletas de representações de violência --- possuindo assim um
potencial traumático --- justamente porque já haveria antes uma série de
violências decorrentes do princípio de autoridade na família patriarcal.
Espantosamente, mas não incoerentemente, a sociedade ainda permite e
encobre tal violência naturalizando"-a. Análises como as de Freud sobre a
instituição da sociedade patriarcal, embora finas na explicação sobre
seus princípios de funcionamento, pode contribuir para essa
naturalização. Já Gross, ao contrário, procura por todos os meios
denunciar a naturalização e a permissividade da sociedade com a
violência parental, apoiando"-se em sua clínica, na teorização e no caso
de Elisabeth Lang para isso: ``a significativa importância do caso, que
me parece merecer o mais elevado interesse da coletividade, reside na
prova das inconcebíveis possibilidades de violência parental abusiva,
contra os menores de idade, que ainda são admitidas pela sociedade''.

O caso de Lang, portanto, não é único. Infelizmente, longe disso. Era
preciso não só denunciar o que acontecia na família Lang, mas em toda
família, indistintamente da classe social: o patriarcado está tão
difundido na cultura que ``a evolução para a vontade de potência é igual
em todas as posições e classes, pois ela se dá imediatamente na
instituição da família, da família patriarcal, assentada no poder''.
Por conseguinte, Gross faz um estudo crítico, antropológico, político e
psicanalítico sobre a família e assevera: ``\emph{a família é a morada
de toda autoridade}; que o vínculo entre sexualidade e autoridade ---
tal como se manifesta na família, com o patriarcado ainda em vigor ---
agrilhoa toda individualidade.''.

A configuração familiar --- a costumeira e violenta imposição da
autoridade do pai, a dependência e submissão das mulheres em relação aos
homens --- simultaneamente oprime a individualidade e dá às crianças, por
si só, o quadro de referências de quem manda e quem obedece, quem domina
e quem é dominado. Ou seja, as relações de poder no interior da família
patriarcal são claramente associadas ao gênero: aos homens são dadas a
potência e a dominância; às mulheres, a passividade e submissão: ``temos
de ponderar que os tipos psíquicos `masculinidade' e feminilidade', tal
como hoje o conhecemos, são um produto artificialmente criado, o
resultado da adaptação às conjunturas existentes''. Além de enfatizar
que se trata aí de uma construção cultural, não biológica, Gross analisa
como essa configuração leva as crianças a diversos conflitos
relacionados à identidade e ao gênero sexual. Ser homem ou ser mulher é
definido por cada um mais pela relação de medo e/ou desafio à autoridade
do que pelo conjunto de experiências de contato afetivo com homens e
mulheres.

Ou, antes e pior, a busca de contato com o outro humano --- aspecto mais
amplo e fundamental da sexualidade na concepção de Gross --- é mediada
pelas relações de poder desde a mais tenra infância, quando o ser humano
se encontra ainda em situação de dependência extrema dos outros para
sobreviver, não apenas pelos cuidados básicos de alimentação e higiene,
mas igualmente pela necessidade de contato físico e psíquico. A privação
de amor tem um efeito devastador para o pequeno ser humano e, em níveis
mais extremos, ele pode até morrer. A angústia diante da solidão é ``um
verdadeiro e fundamentado medo da morte'' e é o que dá origem à pulsão
de contato sexual. É nesse estado de desamparo fundamental que já começa
a pressão à adaptação: ``a absoluta necessidade infantil de contato será
utilizada pelo entorno como meio coercitivo da educação; e a libertação
da solidão, a fabricação de contato, será vinculada à condição da
obediência, da adaptação, da renúncia à vontade própria''.

Essa é, segundo Gross, a primeira ``consequente e pavorosa instauração
da autoridade sobre a vida de cada um''. A imposição à renúncia do que
lhe é mais próprio instaura, em primeiro lugar, o conflito com o
estrangeiro. Na sequência, este conflito sofrerá ainda uma série de
desdobramentos a partir de engendramentos, imbricações e oposições entre
pulsões. A primeira fusão que ocorre é entre a sexualidade, enquanto
pulsão de contato físico e psíquico com o outro, e a submissão à
autoridade, dando à sexualidade, desde seus primórdios, um caráter
masoquista. Como, porém, a própria integridade do sujeito fica ameaçada
pela imposição de abdicação do que é próprio, a pulsão de
autoconservação passa a lutar contra a própria sexualidade, agora já
associada ao masoquismo. Entretanto, para se sobrepor à sexualidade, a
pulsão de autoconservação precisa ser superinvestida, resultando numa
pulsão do eu hipertrofiada. Essa pulsão do eu hipertrofiada é chamada
por Gross de ``vontade de potência'', que consiste numa ``necessidade de
fazer seu eu valer, a todo custo e por todos os meios'' e, portanto,
numa tendência à violação. Essa é a origem do sadismo enquanto
imbricação da vontade de potência com a sexualidade.

Essa é também a origem do conflito, agora todo ele no âmbito sexual,
entre masoquismo e sadismo, que vai incidir tanto no próprio psiquismo
como na relação entre os sexos, na medida em que a família se encarrega
de propagar uma cultura patriarcal que associa o feminino à submissão
(masoquismo) e o masculino à dominação (sadismo). Percebe"-se assim que
no pensamento grossiano o patriarcado está presente na formação da
subjetividade, na construção dos gêneros e na relação entre os sexos.
Por isso ele nos provoca constantemente a reconhecer em nós mesmos a
vontade de potência, nos incitando a combatê"-la em sua forma originária:

\begin{quote}
O revolucionário de hoje, que, com auxílio da psicologia do
inconsciente, avista as relações de gênero num futuro mais livre e mais
feliz, luta contra a violação em sua forma mais originária, contra o pai
e contra o patriarcado.
\end{quote}

Para combater o patriarcado em seu âmbito subjetivo, configurado sob
forma de a vontade de potência, Gross aposta na técnica psicanalítica.
Mesmo que não vise \emph{a priori} a luta contra o patriarcado, a
psicanálise, por suas próprias regras de funcionamento, tende a levar a
uma experiência diferente com o princípio de autoridade. Pelo simples,
mas incomum, fato de o psicanalista não utilizar de sua posição para
comandar a vida do paciente --- tal como ocorre na clínica médica
tradicional ---, o analisante, que costuma colocar o psicanalista no
lugar de autoridade, depara"-se com esse lugar vazio. O psicanalista não
responde desse lugar de comando e, com isso, as associações e o saber
construído pelo próprio analisante passam a ter mais valor que o
conhecimento prévio do psicanalista. Nessa perspectiva, a experiência
analítica é potencialmente uma experiência subversiva da relação com o
poder, e um de seus possíveis efeitos é a constatação que cada
analisante faz por si mesmo de suas próprias inclinações à submissão
e/ou à dominação e as consequências destas. Ao constatar os efeitos
deletérios da vontade de potência em suas diferentes configurações,
tornam"-se mais claras as perdas pela adaptação à autoridade,
possibilitando assim o fim do gozo com o poder, submisso ou autoritário.
A técnica psicanalítica pode então, ``de maneira lógica e sem
compromisso'', restabelecer a humanidade por meio da ``libertação da
influência da sugestão, da sedução e da constrição que alteram, deformam
e restringem''. A psicanálise implica,

\begin{quote}
num encadeamento lógico, a \emph{luta contra a adaptação em geral e, com
isso, contra o princípio da autoridade sob todas as suas formas --- pelo
menos sob as formas existentes em nosso tempo, no interior da família e
dos relacionamentos entre os seres humanos, bem como na relação com o
Estado, com o capital e com a instituição}.
\end{quote}

Portanto, pelas próprias características de funcionamento, a psicanálise
pode ter um papel importante numa revolução social contra o patriarcado.
A revolução precisa começar pela própria subjetividade, caso contrário
ela se reduzirá a mais uma sugestão para as massas e reproduzirá que
procura combater. Por isso,

\begin{quote}
\emph{O trabalho preliminar para essa revolução precisa promover a
libertação de cada indivíduo em relação ao princípio de autoridade que
ele carrega em si; em relação a todas as adaptações --- que nele se
formaram no decorrer de uma infância no seio da família autoritária ---
ao espírito das instituições autoritárias; libertação em relação a todas
as instituições que a criança recebeu do seu entorno, as quais tem
estado em eterna luta}, \emph{com ele e entre elas próprias, pelo
poder}; \emph{libertação, sobretudo, em relação a esse traço de caráter
servil que, invariavelmente, é herdado por todos de uma infância como
essa}: \emph{em relação ao próprio pecado original, a vontade de
potência.}.
\end{quote}

Mas Otto Gross não apostava restritamente na técnica psicanalítica, na
prática clínica, para combater o patriarcado. A psicanálise poderia
igualmente ajudar na formação intelectual do revolucionário ocupando,
inclusive, um lugar central nas ciências humanas. Ela pode servir de
base ao espírito da revolução contra o patriarcado familiarizando o
revolucionário com a psicologia do inconsciente e despertando, assim, o
interesse em cada um em descobrir em si mesmo a vontade de potência. No
entanto, ela não deve parar por aí. Pelo seu conhecimento adquirido por
meio da experiência clínica e sua decorrente teorização, a psicanálise
deve ser estendida à luta social, à luta contra todas as instituições
que disseminam e enraízam o patriarcado na cultura. Diferentemente de
Freud, que dissera em \emph{Moral sexual `civilizada' e doença nervosa
moderna}, que poderia caber ao psicanalista defender a necessidade de
reformas, mas jamais propor reformas, Gross expressamente defendia que a
psicanálise deveria propor reformas. No primeiro congresso internacional
em Salzburgo, Gross se pronunciou a esse respeito, mas obteve como
resposta de Freud: ``nós somos médicos e devemos permanecer médicos'' ---
mesma posição defendida diante de Reich, em 1932. C

Contudo, Gross não obedeceu ao dito pai da psicanálise e defendeu que
``a revolução por vir é a revolução pelo direito matriarcal.'' e que há
uma ``necessidade da desintegração da família patriarcal sob a
edificação do matriarcado comunista''. Essa revolução seria, na
verdade, uma restituição do lugar central que as mulheres já tiveram,
outrora, nas comunidades. Apoiando"-se também nos estudos antropológicos
de Bachofen sobre as sociedades matriarcais antigas e no estudo que ele
próprio faz do \emph{Gênesis}, Gross constrói uma tese radicalmente
diferente da de Freud com relação à instituição do patriarcado. Seu
texto \emph{A ideia de base comunista na simbologia do Paraíso} é o que
condensa suas principais ideias a esse respeito. Diria, aliás, que esse
texto está para Gross assim como \emph{Totem e tabu} está para Freud,
cada um, entretanto, defendendo pontos de vista bem diferentes.

Enquanto para Freud a instituição do patriarcado está na base da própria
constituição da cultura, para Gross as primeiras sociedades destinavam
às mulheres um lugar central: ``a antropologia moderna identifica como
instituição princeps o livre matriarcado, o dito matriarcado da horda de
tempos primevos''. Porém, não se tratava, neste matriarcado, de uma
organização social em que as mulheres comandavam os homens. Sua
configuração era completamente diferente da lógica de domínio e servidão
típica do patriarcado:

\begin{quote}
O matriarcado não apresenta obstáculos ou normas, nem moral ou controle,
no que concerne ao sexual. Não conhece o conceito de ``paternidade'' e
não conta com a sua constatabilidade em cada um dos casos individuais.
(\ldots{}).

Aqui jaz a diferença decisiva e crucial. \emph{A organização matriarcal
reparte o conjunto de todos os possíveis direitos, deveres,
responsabilidade e laço entre os indivíduos, de um lado, e a sociedade,
de outro. A instituição patriarcal desloca a ênfase para o laço legal
dos indivíduos entre si}.

\emph{No domínio do matriarcado, toda entrega de si só pode vigorar na
relação do indivíduo com a sociedade; e toda sensação de potência, de
modo coletivo.} Na mútua relação dos indivíduos entre si há espaço para
\emph{o desenvolvimento de relações que podem permanecer um fim em si
mesmo} \emph{e livres de traços de autoridade e motes de potência. O
matriarcado mantém a relação entre os gêneros isenta de dever, moral e
responsabilidade; de vinculações econômicas, jurídicas e morais; de
potência e submissão. Isenta de acordo e autoridade; isenta de
matrimônio e prostituição}''.
\end{quote}

Essa característica do matriarcado enquanto uma forma de laço social em
que prevalece o princípio de ajuda mútua, a preocupação com a
coletividade, deve"-se originariamente ao fato de que, nas primeiras
comunidades humanas, desconhecia"-se como as mulheres engravidavam. Com
isso, atribuía"-se a elas um poder mágico e toda a comunidade se
organizava para que elas pudessem gerir e cuidar dos pequenos. Não
havia, assim, o conceito de paternidade e, muito menos, ``razão alguma
para qualquer \emph{comprovação de paternidade} --- da qual, enquanto
chave da \emph{averiguação de um indivíduo responsável e imputável
financeiramente}, a sociedade patriarcal \emph{não pode prescindir --- e
é instada a fazer da condição indispensável dessa comprovação (em
primeiro lugar, pois, o imperativo da exclusividade sexual) o teor de
toda a sua moral e das suas instituições}''.

A ausência da ideia de paternidade possibilitava, então, que as relações
afetivas e sexuais fossem livres, não havendo necessidade da instituição
do casamento e da família. Foi com a descoberta da origem sexual dos
seres humanos que os homens teriam passado a reivindicar, pelo uso da
força, a posse das mulheres: ``a forma existente do casamento tem sua
origem no dito casamento por rapto; logo, (\ldots{}) o fundamento da família
patriarcal existente decorre do uso de escravas prisioneiras de guerra''
. Se até então as mulheres eram livres, a partir desse momento elas
passam a ser submetidas ou mesmo reféns dos homens. Até a maternidade,
antes assegurada por toda a comunidade, passa a ser dada exclusivamente
pelo marido, da qual a mulher passa a depender materialmente.

Com a assunção do patriarcado, as mulheres das antigas comunidades
matriarcais, por serem livres para os relacionamentos sexuais, passaram
a ser consideradas prostitutas. É assim que elas são descritas, por
exemplo, no \emph{Gênesis}, escrito já sob a perspectiva do patriarcado.
E mesmo já na nova ordem instituída, a mulher continua a ser descrita no
\emph{Gênesis} como o ser orientado por um princípio maligno e que
desvirtua o homem de espírito reto: é a mulher que anseia por vantagens
mesquinhas, é Eva quem é fraca e seduz Adão. Por isso Gross afirma que

\begin{quote}
no Gênesis o que está em pauta é essa catástrofe da civilização com a
qual o pensamento patriarcal tornou"-se o princípio dominante.

\emph{Essa é a grande reavaliação de todos os valores na qual a
humanidade concedeu à sua vida o caráter autoritário existente e criou
essas normas, as quais se revelam hoje, como sempre, inorgânicas e não
assimiláveis e expõem, com isso, a sua natureza de corpos estranhos ---
de modo que sempre, e por toda parte, são o foco inicial de
intermináveis conflitos internos e de todas as autodecomposições na
doença e na decadência}.
\end{quote}

O pensamento grossiano faz, assim, um importante contraponto à tese
freudiana sobre a origem e a perpetuação do patriarcado, ainda que Gross
concorde com a descrição freudiana da estrutura e do funcionamento da
sociedade patriarcal. Curiosamente, ambos citam como referência
Bachofen, que foi pioneiro nos estudos sobre as comunidades matriarcais
antigas, mas têm posições completamente distintas. Enquanto Freud
defende que as comunidades matriarcais ou, como ele também chama,
fraternais, teriam surgido numa espécie de ínterim entre o assassinato
do pai da horda primeva e a instituição do Totem e da lei do incesto,
para Gross o matriarcado estaria na origem das comunidades humanas.
Ademais, enquanto para o primeiro a organização patriarcal está na
origem da cultura, sendo assim uma forma de organização inevitável,
inerente à própria cultura, para o segundo o patriarcado não é uma ordem
intrínseca da humanidade. Muito pelo contrário, trata"-se de uma
organização violenta, que pode e precisa ser destituída. O matriarcado
da antiguidade, se não pode ser restabelecido, serviria ao menos de
referência para pensarmos em outras formas possíveis de organização
social e política.

As teses de Otto Gross, entretanto, jamais foram colocadas em debate no
cenário psicanalítico. Para sustentar essas ideias, Gross teve que
travar uma verdadeira guerra contra o patriarcado existente em sua
própria família e na psicanálise. Tragicamente, saiu perdendo. Para
contar essa história, é preciso, mais uma vez, passar a escova a
contrapelo.

\section{Terceira escovada: como e porque Otto Gross foi segregado e
esquecido?\protect\footnote{\uppercase{U}m pente fino a contrapelo foi passado nessa
  história em ``\uppercase{O}tto \uppercase{G}ross: um caso de segregação e esquecimento na
  história da psicanálise''.}}

Já na primeira década do século \versal{XX} Gross tinha um papel relevante para a
psicanálise. Defendia Freud nos congressos médicos, publicava textos de
divulgação da psicanálise apresentando ao mesmo tempo uma produção
original e tinha um número grande de pacientes e seguidores,
contribuindo assim para a expansão e internacionalização da psicanálise.
Freud considerava"-o, segundo suas próprias palavras, inteligente,
talentoso e um dos poucos capazes de dar uma contribuição original.
Nessa época, ele apostava em Gross, assim como em Carl Jung, como seu
possível e provável sucessor e herdeiro. O que aconteceu então para que
ele se tornasse uma figura desconhecida entre os psicanalistas das
gerações seguintes? Por que passou a ser considerado, pelos poucos que
ouviram falar a seu respeito, um psicótico?

Muito rapidamente, de 1908 para 1909, Otto Gross passou de uma figura
promissora para um risco à causa analítica. O evento decisivo para que
isso ocorresse foi um tratamento que Gross foi fazer com Jung. Desde
1900, quando era médico de bordo de navios que faziam o trajeto da
Europa para a América do Sul, Gross desenvolveu forte vício em ópio e
cocaína. Em 1902 já havia feito um tratamento de desintoxicação, mas
logo depois retomou o uso de drogas. Sua saúde, porém, se complicou e,
em 1908, chegou a um estado chamado por Freud de paranoia tóxica (Freud
entendia bem dos efeitos da cocaína). Para Hans Gross, Otto tinha se
tornado um problema maior desde 1905, pois seu estilo de vida ---
excêntrico para os valores da época, mas plenamente condizentes com seu
posicionamento político --- era justamente o combatido pelo renomado
criminologista: seu filho se tornara, a seus olhos, mais um degenerado.
Tentava, no entanto, ``salvá"-lo'' por todos os meios. Dava"-lhe dinheiro
para mantê"-lo próximo e buscava tratamentos para seus vícios.

Numa dessas tentativas de controlar Otto, em 1908, Hans entrou em
contato com Freud, com quem já tinha certa proximidade. Aproveitando que
Freud era ainda uma das poucas autoridades que Otto respeitava, Hans
literalmente implorou para que Freud o atendesse ao menos uma vez e
usasse alguma justificativa para interná"-lo. Hans também solicitou um
diagnóstico psiquiátrico, para assim ter um documento que justificaria
um pedido de tutela legal sobre a vida do filho. Freud não chegou a
receber Otto para essa conversa, mas prontamente entrou em contato com
Jung, solicitando que recebesse Otto para uma desintoxicação no
Burghölzli --- hospital em que Jung trabalhava, sob a direção de Bleuler
--- logo após o Congresso de Salzburgo. Em maio de 1908, Freud redigiu,
então, o atestado para a internação, pedindo a Jung para que o
mantivesse internado somente para desintoxicação até outubro, quando ele
mesmo poderia se encarregar do tratamento analítico de Otto.

Jung, porém, não cumpriu o combinado. Três dias após a internação de
Gross, escreveu a Freud dando notícias sobre o tratamento analítico já
iniciado e apresentando um diagnóstico de neurose obsessiva. Foi uma
experiência bastante intensa durante pouco mais de trinta dias. Jung
dedicou"-se integralmente ao tratamento de Otto, as sessões eram diárias
e duravam até doze horas. E quando Jung cansava, era Otto que assumia o
papel de analista, de modo que, como ele mesmo admitiu a Freud, sua
saúde se beneficiou. Mas Jung também ficou perturbado com essa
experiência. Nesse mesmo período, ele se encontrava em profundo conflito
em função de sua paixão por Sabina Spielrein --- sua primeira paciente
atendida sob o método psicanalítico e que no momento fazia formação em
medicina. Jung era casado e não conseguia admitir publicamente essa
paixão extraconjugal. De repente estava diante de um psicanalista
anarquista que vivia os preceitos do sexo e amor livres e que sustentava
tais princípios de maneira bem fundamentada na psicanálise, na
antropologia e no anarquismo. Jung identificou"-se tanto com Gross que
chegou a chamá"-lo de ``irmão gêmeo psíquico''.\footnote{Gross, inclusive,
  influenciou o pensamento deJung. Essa experiência de análise resultou
  no texto ``A importância do pai no destino do indivíduo''. A
  construção dos tipos psicológicos introvertido e extrovertido também
  foram baseadas em ideias de Gross.} Há, inclusive, uma carta de
Spielrein em que ela conta que Jung havia falado com profunda emoção de
Otto Gross e do conhecimento que adquirira com ele a respeito da
poligamia, prometendo que não iria mais reprimir seus sentimentos por
ela.

Contudo, Jung mostrava"-se incapaz de sustentar abertamente a poligamia e
indicava querer manter Gross por perto para resolver seus conflitos e
usar seu poder médico para isso. De alguma maneira, Gross percebeu que
Jung estaria disposto até a mudar seu diagnóstico para alcançar esse
fim. Escreveu, então, um telegrama para sua mulher pedindo que ela
entrasse em contato com Freud para obter indicação de um novo hospital
onde pudesse ficar internado. Antes, porém, que Freud lhe respondesse,
Gross fugiu do Burghölzli.

Para se justificar diante de Freud, Jung alterou o diagnóstico de Gross
--- mesmo expediente que utilizara em relação a Sabina Spielrein, quando
chegou ao conhecimento de Freud a notícia do envolvimento amoroso
existente entre eles.\footnote{Jung chegou a dizer a Freud que a história
  de envolvimento amoroso com sua então paciente não era verdadeiro, que
  se tratava de um delírio de uma psicose histérica.} Gross sofria, de
acordo com esse novo diagnóstico, de demência precoce --- um tipo de
psicose que posteriormente ficou conhecida como esquizofrenia. Freud
questionou o novo diagnóstico e reforçou a hipótese de uma neurose
obsessiva com uma transferência negativa devido às relações hostis com o
pai, dando a entender, portanto, que Jung estava se colocando de maneira
arbitrária diante de Otto. Ao mesmo tempo, no entanto, Freud colocou
panos quentes na situação e não prosseguiu no questionamento do que
teria acontecido com Gross. Pelo contrário, celebrou que ao menos agora
Jung se mostrava mais próximo de suas ideias.

Há diversos motivos possíveis para que Freud tenha adotado essa postura.
Primeiro, a relação de Otto com as drogas, especialmente a cocaína.
Sabe"-se o quanto Freud sofreu por ter sido um dos introdutores da
cocaína como uma forma de tratamento. Outro motivo era a relação de
Freud com Hans Gross: como defender Otto e não atender às demandas de
Hans se era benéfico à causa analítica manter relações com uma pessoa de
tanto prestígio? A vida excêntrica de Otto era outra razão para
excluí"-lo da causa psicanalítica. Ter como um de seus principais
representantes um psicanalistas envolvido em uma série de escândalos
seria um grande problema para expansão e consolidação da psicanálise. As
ideias e o posicionamento político de Otto era, do mesmo modo, um grande
problema, na medida em que Freud era contrário ao envolvimento de
psicanalistas em programas revolucionários. Tanto Hans Gross quanto
Freud, portanto, enquanto pais de um sistema de pensamento, buscavam
reprimir o filho rebelde.

Por fim, havia ainda a preocupação de Freud com sua sucessão e com a
institucionalização da psicanálise. Após o primeiro congresso
internacional, já havia mais condições de se constituir a International
Psychoanalytical Association (\versal{IPA}). Como ninguém do círculo vienense lhe
parecia adequado para assumir o cargo de presidente (ou de novo
patriarca da psicanálise), Jung e Gross eram os principais candidatos de
Freud. Com o desfecho do tratamento de Gross, Freud decidiu apostar
todas suas fichas em Jung. Antes um homem tão notável, inteligente, com
um espírito tão notável, Otto tornou"-se, aos olhos de Freud, um caso
perdido: ``infelizmente, não há nada a dizer dele; está viciado e só
pode causar um grande dano a nossa causa'' (Freud \& Jung, 1976,
30/06/1908). E de ``irmão gêmeo'', tornou"-se, para Jung, um ``doido
varrido'' e ``parasita'' (\emph{ibidem}, 19/04/1908). Com o apoio de
Freud, Jung, que almejava fervorosamente se tornar o filho e herdeiro de
Freud, parece não ter hesitado em selar a exclusão de Gross da
comunidade psicanalítica. Redigiu um atestado oficial confirmando o
diagnóstico de psicose, atestado que foi entregue a Hans Gross, que
buscava ter a tutela legal do filho.

As consequências desse ato foram nefastas para Otto. A partir de então,
passou a viver como foragido, perambulando por diversas cidades da
Europa. Como não podia ter endereço fixo, exercia a psicanálise nas
casas dos pacientes ou nos guetos anarquistas, em bares e restaurantes.
Tanto Freud quanto Jung prosseguiram acompanhando e elogiando a produção
teórica de Otto, mas permaneciam em silêncio e não faziam um gesto para
ajudá"-lo. Pelo contrário, Freud ainda pediu para que se retirasse o nome
de Otto da ata do congresso de Salzburgo e da citação que havia feito em
\emph{Os chistes e sua relação com o inconsciente}.

Mesmo vivendo como nômade, Otto continuou praticando a psicanálise e
publicado seu pensamento, incluindo suas críticas ao patriarcado.
Enquanto seu pai defendia o combate à imoralidade e a punição ao
degenerado, Otto via na tendência à imoralidade ``o grito ético de
afirmação da vida pela salvação da humanidade''. De um lado, portanto,
Hans levava ao paroxismo os valores da cultura patriarcal e autoritária,
de outro, Otto levava ao paroxismo os valores da cultura anarquista e
antiautoritária. Em 1913, quando Hans descobriu que Otto pretendia
escrever um texto que associaria o papel do pai ao sadismo, a tensão
chegou ao limite e Hans usou de todos os recursos que seu prestígio lhe
dava e conseguiu uma ordem judicial de internação de seu filho, obtendo
também a tutela que almejava há alguns anos.

No final de 1913, Otto foi então detido em Berlim e levado a um hospital
na Áustria. Tão logo ele foi detido, seus amigos, médicos, anarquistas e
artistas de diferentes cidades começaram uma campanha de libertação que
atingiu proporção continental. Diversos jornais e revistas de Berlim,
Munique, Viena, Paris e Praga publicaram depoimentos de pacientes
relatando os benefícios das experiências de análise e manifestos de
colegas denunciando a violência paterna e exigindo a libertação de Otto.
Com isso, ele se tornou em toda a Europa o ícone, ``o mártir e o profeta
da luta pai"-filho''. Graças a essa campanha, em seis meses a Justiça
austríaca concedeu liberdade condicional a Otto.

Uma das condições impostas por Hans era a da que Otto empreende"-se outro
tratamento psicanalítico, agora com Wilhelm Stekel --- outro psicanalista
dissidente da \versal{IPA}. Stekel foi o único psicanalista a apoiar Otto Gross.
Bem diferentemente das comunidades artísticas e anarquistas, a
comunidade psicanalítica não só silenciou diante da campanha
internacional de libertação como, em meados de 1912, foi denunciado às
autoridades por psicanalistas ortodoxos que o considerava herético ao
movimento. Ele ainda permaneceu rotulado como psicótico segregado dessa
comunidade até sua trágica morte, em 1920, dois dias depois de ser
encontrado caído em um beco, padecendo de fome, de frio e de sintomas de
abstinência de drogas.

\section{Quarta escovada: a estrutura e a violência do patriarcado está
disseminada nas comunidades psicanalíticas}

Essa triste história de segregação revela o quanto o patriarcado esteve
maleficamente presente na comunidade psicanalítica mesmo antes de sua
institucionalização. Freud, por vezes, queixava"-se de ser colocado no
lugar de pai por seus discípulos, mas por outras vezes não hesitava em
se colocar como um verdadeiro patriarca na relação com eles. Nas
correspondências com Jung isso era patente. Jung chegava a se referir a
Freud como pai e Freud, por sua vez, referia"-se a Jung como filho e
herdeiro. Jung, inclusive, lembrou"-o --- quando a relação entre eles
estava prestes a ser rompida --- que Freud teria admitido na viagem de
navio aos Estados Unidos que não queria perder sua autoridade diante de
seus discípulos.

Consciente ou inconscientemente, mas certamente preocupado com a escolha
do futuro líder da psicanálise, Freud acabou instaurando entre Jung e
Gross a rivalidade fraterna, em que ambos deveriam disputar o amor e a
herança paterna. Enquanto Gross, ao fugir do Burghölzli, deixou bem
clara sua escolha de que não responderia dessa posição --- não obstante
respeitasse Freud como autoridade médica, não o colocava na posição de
autoridade paterna ---, Jung usou de todas as armas a seu alcance para
afastar e prejudicar violentamente seu ``irmão gêmeo'', obtendo, com
isso, a conivência do pai Freud e a herança prometida.

Uma leitura, mesmo não muito atenta, das correspondências de Freud com
outros psicanalistas revela que Freud procurava usar da posição de
patriarca com ao menos a maior parte de seus discípulos mais próximos.
Não satisfeito com os desvios teóricos de Jung após sua nomeação como
presidente da \versal{IPA} em 1910, dois anos depois aceitou a sugestão de Ernest
Jones, a partir de uma ideia de Ferenczi, de formar um Comitê Secreto
que zelaria ``sua criação'' e controlaria os desvios junguianos.
Reuniram"-se, então, em torno do patriarca Freud, os filhos e discípulos
Hanns Sachs, Karl Abraham, Otto Rank, Max Eitinton, além claro de Jones
e Ferenczi.

Quem conta muito bem essa história é Phyllis Grosskurt, em \emph{O
círculo secreto}. Partindo sobretudo das correspondências trocadas
entre os membros do comitê, a autora conta como Freud criticava, para
cada um separadamente, os desvios teóricos e desvios de conduta dos
outros colegas e confiava, a cada um deles, a sucessão da psicanálise.
Freud não hesitou em se colocar como um chefe da horda primeva que, se
não restringia o acesso às mulheres, limitava o acesso à ``verdadeira''
e ``pura'' psicanálise para, desta forma, defender a causa psicanalítica
como para defender sua posição de criador da psicanálise. Grosskurt
também aponta como a postura de Freud contribuiu muito para que seus
discípulos adotassem uma postura passiva e submissa --- esperada pelo
próprio Freud, embora ele dissesse o contrário ---, deixando"-os
escravizados ao pensamento freudiano e à pessoa de Freud: ``a força da
personalidade e das ideias de Freud haviam gerado um culto da
personalidade em que Freud, como guru, exigira total lealdade pessoal e
profissional''. Com ela, podemos ver que algumas escovadas a contrapelo
na história da psicanálise revelam que ``o subtexto da história
psicanalítica é a história de como Freud manipulou e influenciou seus
seguidores e sucessores'' (\emph{Ibidem}), é a história de como a lógica
patriarcal esteve e ainda está presente nas sociedades psicanalíticas
não apenas em seu nível institucional, mas na própria construção e
transmissão conceitual. Como ela bem conclui: ``a história do Comitê
poderia servir de metáfora do próprio movimento psicanalítico''
(\emph{Ibidem}).

Isso porque, se continuarmos escovando a contrapelo, inúmeras outras
histórias surgirão, envolvendo de psicanalistas mais conhecidos ---
casos, por exemplo, de Adler, Ferenczi, Tausk, Rank, Lacan --- a
psicanalistas ``anônimos'' antigos e atuais, que fazem ou fizeram parte
do dia a dia institucional. E ainda assim, o patriarcado tende a
permanecer recalcado, ou mesmo reprimido, e em exercício pleno. Ou ao
menos tendia a permanecer recalcado. É uma felicidade fazer parte de uma
coletânea que recoloca o tema em discussão e poder deixar aqui a
mensagem de Otto Gross a Freud e seus seguidores:

``\emph{Só pode ter sido o recalcamento das últimas consequências
revolucionárias que impediu a breve iluminação desse axioma aos grandes
da nova disciplina, sobretudo ao genial inventor do próprio método em
desenvolvimento}''.

\chapter*{Apontamentos ferenczianos para a atualidade da psicanálise}
\addcontentsline{toc}{chapter}{Apontamentos ferenczianos para a atualidade da psicanálise,\\ \footnotesize\emph{por Paula Peron}}
\hedramarkboth{Apontamentos ferenczianos para a atualidade da psicanálise}{}

\begin{flushright}
\emph{Paula Peron}
\end{flushright}

Este texto revisita os chamados escritos pré"-psicanalíticos de Sándor
Ferenczi, em busca de ideias inspiradoras para diretrizes éticas,
politicas e terapêuticas para a Psicanálise atual, em tempos de
acentuação do declínio do patriarcado. O cenário já se alterou bastante
desde estas publicações do começo do século \versal{XX}, mas nelas ressaltarei a
presença de pressupostos antipatriarcais,\footnote{Reconheço os problemas
  de adotarmos um \emph{status} universal para o patriarcado, como
  aponta Butler (2018, p. 22), que pode anular ou reduzir ``expressões
  diversas da assimetria do gênero em diferentes contextos culturais''
  (idem, p. 72). Não problematizarei este ponto, mas reconheço as
  contradições e inconsistências deste suposto universal, justamente
  encarnadas em Freud e Ferenczi, mesmo em contextos culturais
  semelhantes.} ou seja, temáticas geralmente evitadas ou criticadas
pela Ciência da época, dominada por versões que valorizavam atributos
ditos masculinos: a razão, a consciência, a verticalidade nas relações
humanas, a ação, a falicidade, entre outros.

Sabemos que Freud é crítico da Modernidade e suas formações
institucionais, e trabalha em sua clínica e na teoria a partir dos
efeitos do patriarcado sobre as mulheres, homens, crianças e, como
chamava, os homossexuais. Nas análises dos malefícios e efeitos da
Modernidade e suas estruturas patriarcais, Freud apontou
impossibilidades e impasses gerados para os sujeitos, criticando, por
exemplo, a moral sexual, as relações familiares e o excesso de
restrições civilizatórias sobre os sujeitos. Por um lado, analisou o
desamparo derivado das rupturas da tradição patriarcal, forjadas pela
Modernidade e apontou suas consequências --- masoquismo, ilusões
defensivas frente à nostalgia do pai, mal"-estar contínuo, construções
religiosas reparadoras, entre outras. Por outro lado, podemos ver em
Freud repetições e resíduos da lógica patriarcal que expos, e uma
tentativa de reordenar o sujeito em direção do pai (Birman, 2006).

Minha intenção é mostrar como desde o início de sua obra, Ferenczi pode
fornecer acréscimos e contraposições a Freud, já que suas ideias
``compõem um ideário mais próximo das tendências políticas
contemporâneas do que o modelo de laço vertical com o Pai ou seus
derivados, apresentado por Freud'' (Reis e Gondar, 2017, p. 15). O que
pretendo é destacar Sándor Ferenczi, como figura contemporânea a Freud,
e fundamental para valorizarmos na Psicanálise as premissas na contramão
da lógica patriarcal e fálica, do ponto de vista teórico e terapêutico.
Seu texto de estreia na Psicanálise, em 1908, chamado O efeito na mulher
da ejaculação precoce masculina, problematiza os efeitos da prevalência
paterna como organizadora das subjetividades e, ao descrever a
frustrante vida sexual das mulheres casadas, com neurose de angústia,
afirma: ``Esse estado, quando se torna permanente, leva necessariamente
a um estado de tensão nervosa; só o egoísmo masculino, sobrevivência do
velho regime patriarcal, pôde desviar a atenção dos homens\ldots{} logo, dos
médicos, deste problema'' (1991/1908, p. 1).

Embora tais premissas estejam presentes e crescentes ao longo de toda a
obra de Ferenczi, escolhi recortar a seguir apenas alguns eixos de sua
produção inicial, que evidenciam principalmente jogos de poder diversos
e situações de assujeitamento. Frente a estes cenários, veremos Ferenczi
propor ideias e ações que não valorizam a prevalência da figura paterna
e seus representantes, colocando"-se contra a manutenção de versões
fálicas ultrapassadas de sustentação simbólica para os sujeitos em laço
social, seja na linguagem da Ciência, seja no cuidado aos excluídos,
desmontando a aura de superioridade do masculino, por vezes encontrada
em Freud, especialmente antes dos anos 20.

Sobre Ferenczi, cabe mencionar brevemente que viveu entre os anos de
1873 e 1933, morando em Budapeste quase toda sua vida (viveu também em
Viena para cursar Medicina aos 17 anos e, mais tarde, em Nova Iorque
para psicanalisar). Nascido em uma família judaica intelectualizada e
politizada, de doze filhos (Lorin, 1994), conheceu Freud em 1908, e
antes disto praticava como neurologista, trabalhando em hospitais, na
Caixa de Saúde de Budapeste e como perito judicial (Dean Gomes, 2016).
Roudinesco aponta que Ferenczi mostrou"-se, diferentemente de Freud,
``desde logo aberto aos debates promovidos pelas revistas de vanguarda a
respeito da art nouveau (\emph{Jugendstil}), da emancipação das
mulheres, da liberdade sexual e da expansão das novas ciências do
homem'' (Roudinesco, 2016, p. 24). Veremos como esta diferença deve ser
reconhecida para a ampliação a meu ver vantajosa do campo psicanalítico
atual.

Apesar da quantidade razoável de produção bibliográfica de e sobre
Ferenczi no mercado editorial brasileiro, os textos chamados
pré"-psicanalíticos de Ferenczi não constam e não são mencionados nas
Obras Completas brasileiras. Utilizarei aqui o livro \emph{Les Écrits de
Budapest}, organizado por Claude Lorin, com tradução livre. Os
interesses de Ferenczi são muito amplos: pela história, pelos conflitos
sociais, pelo Direito, pela política, pela literatura, e outros, em
textos de 1899 a 1907. Neste período Ferenczi lia Freud, mas ainda não o
conhecia e não fazia parte do campo psicanalítico. Isto ajuda"-nos a
construir uma versão de Ferenczi que não seja apenas efeito de
Freud.\footnote{Birman comenta como alguns autores --- Diane Chauvelot
  (1978) e Philippe Julien (1978) chegam a dizer que Ferenczi somente
  contribui para a Psicanálise após sua análise com Freud (entre 1914 e
  16) e antes seria apenas ``um mero divulgador das ideias do mestre'',
  com o que discordo completamente.} Nos escritos encontramos
observações de seu trabalho médico em hospital dos pobres, de idosos
(\emph{Erzsebet)} e em hospital militar, especialmente com população
carente, como moradores de rua, idosos desamparados e, prostitutas. O
tom que atravessa todos os artigos poderia ser resumido da seguinte
maneira: um médico clínico engajado, consciente dos atravessamentos
sociais e políticos participantes do adoecimento físico e psíquico,
enfim, um pensador politico (Gondar, 2012). Além disto, constantemente
desconfiado das capacidades integrativas, autônomas e racionais do eu ---
do paciente, do médico, do pesquisador.

A seguir examinarei o proposto acima, a partir de dois eixos de
destaque. O primeiro pretende evidenciar a presença das temáticas ditas
femininas, especialmente em suas defesas para ampliação e modificação da
Ciência, em particular a Ciência médica. Ferenczi pretendia abarcar
características e temas relativos aos afetos, às intensidades, ao
inconsciente e ao corpo, em um trabalho insistente de desmontagem das
dicotomias habituais: razão \versal{X} emoção, consciente \versal{X} inconsciente, mente \versal{X} corpo, Ciência \versal{X} campo das emoções, homem \versal{X} mulheres, entre outras. O
segundo eixo que destacarei trata do que Ferenczi chamou de ``politica
de exploração'' (Ferenczi, 1903, p. 191) dos assujeitados em relações de
poder e de formas de enfrentamento. Veremos como nos dois eixos Ferenczi
posiciona"-se de forma a aprofundar tendências que Freud reforçará apenas
tardiamente em sua obra, com as ideias de desamparo como base comum a
todos nós, e com a feminilidade como horizonte de cura. Nos textos
examinados, vemos em Ferenczi ``um polemista de talento, um espírito
profundamente criativo, um clínico sensível e perspicaz, um médico que
examina sua própria prática e seu campo com uma autocrítica
particularmente lúcida e severa'' (Lorin, 1983, p. 10). Neles proporá
soluções que fazem sobressalente a premissa de mais horizontalidade nos
destinos humanos, o que desenvolverá posteriormente com a ideia de
comunidade de destino (Ferenczi, 1932, p. 91), que não propõe filiações
ou garantias, e onde se constrói um porvir compartilhado, ``sem
lideranças nem certezas previas'' (Reis e Gondar, 2017, p. 10), a partir
de um solo comum de incertezas e vulnerabilidades.

\section{Muito além do principio da razão fálica }

Ferenczi inicia suas publicações explorando temáticas geralmente
reconhecidas como femininas\footnote{Em uma concepção em que ``(\ldots{}) a
  natureza é feminina e precisa ser subordinada pela cultura,
  invariavelmente concebida como masculina, ativa e abstrata. (\ldots{}) a
  razão e a mente são associadas com a masculinidade e a ação, ao passo
  que o corpo e a natureza são considerados como a facticidade muda do
  feminino, a espera de significação a partir de um sujeito masculino
  oposto'' (Butler, 2018, p.~74).} e defendendo que sejam integradas e
investigadas pela Ciência e não apenas consideradas resíduos laterais,
incômodos ou incompatíveis com nossas pesquisas. Insiste na ampliação
dos domínios da pesquisa médica e psicológica, e sugere a premissa de
uma constante desconfiança na racionalidade humana, recusando"-se a
objetificar a realidade ou reduzi"-la a uma linguagem estruturada
exclusivamente pela lógica paterna. De fato, desconstrói o mito da
objetividade e racionalidade das Ciências médicas, ressaltando a
presença dos elementos culturais e subjetivos que as permeiam, em uma
severa crítica a suposta separação entre sujeito e objeto de pesquisa.

No texto O espiritismo (1899), que escreveu aos 26 anos (Ferenczi
frequentou sessões de espiritismo, segundo Casonato, 1994), inflado no
combate ao materialismo científico, predominante na época, defende a
inclusão dos fenômenos chamados ocultos e espirituais nas pesquisas
científicas. Desde este texto, Sándor Ferenczi enfatiza a presença das
divisões no funcionamento mental, apontando que os fenômenos espirituais
poderiam ser entendidos como expressões do inconsciente e destas
divisões, e defende que a Ciência pesquise as dimensões inconscientes e
semiconscientes que participam do funcionamento psíquico --- do espirito,
como coloca. Ferenczi entende que a consciência humana progride em
zigzag, de forma caótica, a partir de crises e divergências (1899, p.
37), o que é bastante divergente da forma como Freud apresenta a
consciência humana, a intelectualidade e o pensamento, em algumas
passagens, como faculdades superiores (por exemplo, em Moisés e o
monoteísmo, 1939, p. 114).

Para Ferenczi, um apoio rígido no racionalismo, na objetividade, na
biologia e no empirismo deixa de lado fenômenos que nos interessam, e
defende constantemente que temas como o amor, o ódio, a raiva, a
memória, o esquecimento, o senso moral, a sensibilidade artística, ``a
psicologia das crianças e a psicologia das massas'' (O espiritismo,
1899, p. 40), e os fenômenos religiosos, sejam investigados pela Ciência
e pela Psicologia e não deixados somente aos romancistas, escritores de
ficção e fanáticos. O materialismo cego, segundo ele, escamoteia estudos
sobre nossas paixões mais profundas. Ferenczi tematiza detalhadamente o
amor, que seria capaz de produzir ações criativas e imaginativas ou
destrutivas que levam o homem ao topo de sua capacidade de ação e
deveria ser estudado não apenas em suas manifestações extremas, mas
também nas situações ditas normais. Ele constituiria uma ``zona de
fronteira'' entre o estado normal e o estado doentio da alma humana (O
amor na ciência, 1901, p. 103), e algumas de suas manifestações evocam
traços psicóticos maníaco"-depressivos, provocando alucinações e ilusões
em um sistema nervoso equilibrado (idem, p. 104), fazendo vacilar a
razão. O investimento do pensamento nas abstrações ficaria obstaculizado
na paixão amorosa, o que é vantajoso para o poeta e para o artista, mas
bloqueia o matemático\ldots{} (idem, p. 106), sendo comparado ao álcool: ``O
veneno responsável pela alteração das diferentes faculdades e mais
poderoso que o álcool chama"-se amor'' (idem, p. 106), mostrando"-nos como
este sentimento era visto como perturbador da ordem da razão. Dirá que
nos homens provoca poder, nas mulheres provoca submissão (idem, p. 106),
sinalizando os diferentes efeitos do amor sobre homens e mulheres em sua
época, e as autorizações sociais concedidas aos homens e as mulheres
sobre o destino de suas intensidades. No geral, há um tom crítico ao
intelectualismo, fazendo"-nos reconhecer que as categorias científicas
rígidas da Modernidade e do Patriarcado produzem efeitos de
inferiorização daquilo que não se molda aos seus enquadramentos
normativos.

Ferenczi interessa"-se muito pelo que chama de ``zona de fronteira''
(idem, p. 103) dos fenômenos humanos, ou seja, aquilo que remete aos
limites do controle do sujeito sobre si, aquilo que alude ao que o
escapa e ao mesmo tempo toma"-o, e este interesse leva Ferenczi a pensar
sobre a presença do corpo na dinâmica dos afetos. Ao perguntar"-se sobre
a condição de controle racional do amor, questiona se a inteligência e a
força moral teriam influencia inibitória sobre as manifestações do amor
e sobre as manifestações orgânicas no amor, e considera que ``o amor tem
uma influencia siderante sobre todos os processos do organismo'' (idem,
p. 107). Ferenczi inclui o corpo desde cedo em suas considerações sobre
os afetos e a psique, evidenciando outra incursão importante em
território de fronteira pouco explorado até então pela Ciência: o corpo
como \emph{locus} das manifestações psíquicas, subvertendo os esquemas
estritamente orgânicos, biológicos e deterministas.

No texto A região sensorial do córtex cerebral (1902) remete às
explicações psicológicas, afirmando que não há como entender certos
fenômenos corporais sem a Psicologia, ressaltando a independência entre
as localizações cerebrais e as manifestações psíquicas (idem, p. 177).
Desta forma, mostra"-nos como, contando com os nascentes conhecimentos
acerca do córtex cerebral, percebia toda a imprecisão ligada à intenção
determinista de redução do humano ao cerebral ou biológico. O corpo
aparece nestes inícios de seu percurso intensamente enlaçado pela
subjetividade e a dicotomia corpo --- mente não parece interessar a ele,
que enfatiza especialmente as reciprocidades e interferências entre
eles.

O texto de crítica aos excessos da leitura --- Leitura e saúde, 1901,
também pode ser examinado sob esta ótica: ali vemos Ferenczi buscar o
reconhecimento dos efeitos dos progressos humanos sobre o corpo. Neste
texto Ferenczi estuda as consequências dos excessos de leitura sobre as
crianças e pessoas em geral, e afirma que toda invenção humana carrega
malefícios na medida em que constitui um novo instrumento a disposição
da humanidade, cuja tendência é de abuso, ressaltando que as
intensidades e afetos são pontos"-chaves em sua compreensão dos sujeitos.
Considera que os avanços da civilização provocam problemas mentais e
nervosos de toda ordem, e que a chegada da imprensa teria nisto um
grande papel (Leitura e Saúde, 1901, p. 109). O excesso de tempo de
leitura poderia acarretar em problemas de postura corporal,
principalmente nas crianças e adolescentes, para quem Ferenczi defende
mais tempo livre e exposição à natureza, sendo bastante enfático nos
efeitos nocivos da leitura para os adolescentes, que poderiam ser
precocemente suscitados em sua vida sexual ou sofrer de cansaço por
estudos intensivos, e também para os adultos, que poderiam sofrer
efeitos morais ao tocar em fantasias íntimas e excitações sexuais.
Sugere equilíbrio na dedicação ao corpo tanto quanto ao espirito ---
estudar, repousar, distrair"-se\ldots{} É bastante interessante observar suas
intenções de questionar as invasões da cultura sobre os sujeitos e suas
fragilidades, tanto psíquicas quanto corporais.

O reconhecimento da dimensão afetiva aparece também em Consciência e
desenvolvimento (1900), onde aponta que as sensações e impressões
sensoriais são os átomos da psique, igualmente presentes no
recém"-nascido e no gênio, ressaltando a centralidade desta dimensão, que
não pode faltar nas teorizações da Psicanálise. No curso da infância
adquirimos nossa distinção em relação ao mundo animal --- a linguagem e o
pensamento. No primeiro ano, estamos sob o domínio dos instintos e
reflexos e, apenas posteriormente, diferenciando um eu em relação ao
mundo exterior, com suas próprias impressões sensoriais (1900, p. 68).
Neste texto, vemos operar importantes pressupostos: o afeto como
inicial, permanente e central no psiquismo, a infância e suas
sensibilidades no centro da questão, o destaque sobre a complexidade da
vida humana, em oposição a ideias simplificadas, mecanicistas e
deterministas, típicas da Modernidade e do Patriarcado. Neste mesmo
texto, aponta os limites do sujeito racional do conhecimento e a
inevitável participação da interioridade e subjetividade do pesquisador
e do médico em seus trabalhos. Defende um reconhecimento das fraquezas
de nosso intelecto, que redundaria em um ceticismo e uma reflexão
autocritica do homem em relação as suas afirmações e conhecimento (idem,
p. 70)

Seguindo neste campo, podemos dizer que a própria diagnóstica e
racionalidade médicas são questionadas em seus limites: Ferenczi critica
as ``doenças da moda'' (Sobre a neurastenia, 1905, p.256) e ``os
produtos da moda'' da Medicina (Da prescrição na terapia neurológica,
1902, p. 292), bem como a ideia de causas únicas para os sintomas (Dois
erros de diagnósticos, 1900, p. 86), do diagnostico \emph{a priori} e
preconceituoso (Complicações nervosas no curso de uma mielite, 1902).
Desta forma questiona"-se constantemente como médico e sobre a semiótica
médica no geral. Afirma, por exemplo, que as pesquisas médicas são
feitas sob o viés masculino, o que certamente impacta na descrição dos
sintomas subjetivos das mulheres, como no caso dos sintomas que
acompanham menstruação (Novo ensaio de explicação da menstruação, 1900,
p. 75).

\section{Estratégias de resistência e horizontalidade -- muito além do
princípio de organização fálica }

\epigraph{``(\ldots{}) a horizontalidade não é feminina nem masculina.
A questão mais premente hoje, em termos sociais e políticos,
não é uma nova distribuição do falo --- mesmo que pela
via mais sofisticada de sua superação, através do gozo feminino
--- e sim uma distribuição mais justa da vulnerabilidade.''}{(Gondar, 2012, p. 205)}

Vejamos agora como Ferenczi, a partir da ideia de ``politica da
exploração'', ressalta as relações de poder e seus pólos. Ferenczi
mostra"-se capaz de propor soluções em um sentido mais horizontal do que
verticalizado, sem retomar construções imaginárias que fortalecem
hierarquias, autoridades ou supostas posições fálicas. Em torno de duas
figuras que destacaremos agora --- o homoerótico e o médico, a quem
Ferenczi acrescenta muitas dimensões e complexidades subjetivas,
desenhando sujeitos vulneráveis, atravessados pelo inconsciente, por
afetos, poderes e injunções sociais. Há um reconhecimento das dimensões
de vulnerabilidade presentes em todos nós e, a partir disto,
diferentemente de Freud, que ``não parece dar conta das múltiplas formas
de potência produtiva de emancipação social'' (Safatle, 2015, p. 110),
Ferenczi apresenta caminhos de emancipação --- para o homoerótico, para o
médico estagiário maltratado, para o médico obrigado a escrever laudos,
como veremos. A partir do reconhecimento das fragilidades e impotências,
sugere condições de existência, a partir de uma comunidade de iguais e a
partir de resistências.\footnote{Gondar (2012) aponta como neste ponto há
  aproximação entre Ferenczi e Judith Butler, no reconhecimento da
  precariedade de todos os envolvidos numa relação, na superação da
  instância ideal e na proposta de laço social construído sobre o
  reconhecimento da vulnerabilidade, que conduz à potência.}

Como aponta Birman (2006, p. 320), ``a resistência também pode ser algo
do desejo, quando é o poder que está efetivamente em causa'' e, desta
forma, recoloca o registro da impossibilidade de adesão à ordem
patriarcal. Neste sentido, Ferenczi resiste a que o homoerotismo seja
compreendido como perversão, que seja desarticulado de suas dinâmicas
psíquicas, que sejamos separados em categorias puras para normatização,
ressaltando uma ética da diferença e da singularidade. Ferenczi
reconhece os assujeitamentos vividos pelos homoeróticos de seu tempo, a
desvalorização e humilhações que sofriam.

Ressalto que nos vários textos iniciais sobre a sexualidade, Ferenczi
mostra"-se liberal e, ao mesmo tempo, sustenta alguns postulados de sua
época. No texto Útero didelfo (uma afecção que provoca duplicidade do
útero), de 1899, ele coloca"-se contrário à ideia de uma degenerescência
das prostitutas (encontrada em Lombroso, Ottenghi e outros, conforme
Ferenczi), dada a alta frequência de malformações genitais. Defende um
maior cuidado médico para as ``mercadoras do prazer'' (Um caso de
retraimento retal, 1899, p. 59), criticando seu lugar marginalizado e
submetido. Estamos em 1899 e vemos Ferenczi opor"-se à ideia de uma
natureza degenerada de certas mulheres e uma convocação a que elas
ocupem um protagonismo maior em sua condição --- de mercadorias a
mercadoras do prazer --- e que tenham os direitos a saúde como outros
sujeitos.

Ao comentar mudanças de orientação sexual em pacientes, Ferenczi
ressalta os efeitos nocivos dos preconceitos relativos às diferenças
sexuais (Um caso de hipospadia, 1899), e da inflexibilidade moral dos
pais, defendendo que não se trata de estados psíquicos anormais e que
não devem ser somente estudados do ponto de vista da psicopatologia (A
homossexualidade feminina,\footnote{Ferenczi inventa o termo
  homoerotismo, em 1914 (Lorin, 1983).} 1902). Neste texto, enfatiza as
qualidades intelectuais gerais e musicais da Senhorita que apresenta
como Rosa K. Como não pensar na família K do caso Dora, contado por
Freud (1905), e na diferença entre Freud e Ferenczi no reconhecimento
das vulnerabilidades das pacientes em questão, da moralidade familiar,
das destinações sociais das mulheres? Freud patologiza Dora, enquanto
Ferenczi questiona o lugar social da erótica dissonante da Srta. Rosa K.
Pede a ela que escreva uma autobiografia, demonstrando uma convocação do
potencial do sujeito em sua autoria e no reconhecimento da legitimidade
de sua história, a ser narrada e compartilhada. Preocupa"-se com os
cuidados sociais às pessoas marginalizadas socialmente --- pergunta"-se,
por exemplo, onde protegemos, abrigamos e apoiamos pessoas como
Senhorita Rosa K., que não tem as mesmas chances que outras pessoas em
nossa sociedade (idem, p. 155). Afirma que os hospitais e prisões, para
onde frequentemente eram mandados, não são convenientes.

No texto Estados sexuais intermediários (1905), Ferenczi posiciona"-se
contra as sanções penais, injustas e inúteis a seu ver, que sofrem os
homoeróticos em vários países e convoca os médicos húngaros a se
associarem aos médicos alemães no movimento de oposição aos preconceitos
no campo psiquiátrico. O próprio Ferenczi tornou"-se representante do
Comitê Interhumanitário Internacional de defesa dos homossexuais em 1897
(Dean Gomes, 2016). Fala dos estados sexuais intermediários --- em
sujeitos que misturam traços masculinos e femininos. Esta mistura seria
um fator congênito desde a vida embrionária, restando elementos
femininos nos homens e vice"-versa, mesmo nas pessoas de sexo definido.
Estabelece também o caractere psicológico, para além dos hormonais.
Aponta que, grosso modo, os tipos femininos seriam receptivos e os tipos
masculinos produtivos, além de terem tendências a generalizar e a
abstrair. Os tipos femininos teriam pensamentos mais concretos e senso
de família, com mais ternura e menos agressividade que os masculinos.
Afirma que o espirito masculino tem algo de hipermetrope, enquanto o
espirito feminino seria míope, revelando humor e caricatura no desenho
dos tipos puros, para então afirmar que eles não existem neste estado,
mas misturados sempre.

Sua preocupação com o campo das condições sociais de certos grupos
inclui a problematização das condições de trabalho dos próprios médicos,
o que é muito interessante, porque indica que reconhece a importância de
pensar e reconhecer o sofrimento daquele que trata, desenhando menos
polarização entre médico e paciente. Novamente vemos Ferenczi
problematizando relações de poder e de vulnerabilidade. Como em relação
ao homoerotismo, Ferenczi reconhece a queixa de uma injustiça sofrida e
a necessidade de reparação, dá voz aos inferiorizados, e propõe
solidariedades, colocando"-se contra a recusa de reconhecimento de certas
precariedades, neste caso, dos jovens médicos, submetidos a hierarquias
de exclusão. Problematiza isto em Contribuição à organização do serviço
hospitalar do médico assistente (1903), e lembra"-nos que muitos nobres
projetos são antecedidos por situações trágicas ou opressivas, que
posteriormente são esquecidas, ressaltando que a politica de exploração
geralmente está na base das construções sociais fálicas. Defende os
médicos iniciantes, que trabalham sem direitos e que não poderia ficar
indiferente a isto. Chama de ``politica de exploração'' (idem, p. 191) a
forma como eles são tratados, visto que, apesar de serem bolsistas para
estudo, não há nada que os favoreça na situação para estudar, com
salários insuficientes para viver na capital ou em outras cidades. Não
há para o médico assistente uma biblioteca, um escritório, um gabinete
privado, eles não tem férias, seus superiores os esnobam e abusam de sua
disposição ao trabalho, e as perspectivas são opacas, dado que os
superiores permanecem em suas posições e não se criam cargos ou caminhos
para os médicos jovens serem empregados. Ferenczi sugere um comitê
exterior para gerir tais questões, o que a Sociedade Médica
acata.\footnote{Lembremos que em 1905 haverá na Hungria uma famosa greve
  geral de seis semanas, que refletiu a vasta insatisfação dos
  trabalhadores do país e culminou na proclamação de estado de sítio no
  país, em um acordo das classes dominantes (Lorin, 1983). Isto foi
  precedido por um clima social e político de insurreição, combatido
  militarmente pelo governo com uma represália contra os socialistas,
  sindicalistas e outros supostos traidores da pátria, inclusive os
  tidos como diabólicos e desordeiros homossexuais. O combate não foi
  apenas militar, mas também em um nível ideológico e médico, como
  aponta Lorin (1983), e Ferenczi alinha"-se aos agredidos e intolerados.}
Ferenczi comenta em 1907 (Instruções da lei de seguro dos trabalhadores
a respeito dos médicos) outra situação político"-institucional, onde os
médicos escreveriam laudos para empregados do mundo industrial e
comercial, para efeitos de seguro, a partir da decisão do Rei Francisco
Jose \versal{I} e do parlamento húngaro. Convoca que aos médicos seja dada
autoridade pública, sem o que certamente estariam expostos a insultos e
injurias por parte de assegurados insatisfeitos. Descrevendo pormenores
e criticando alguns pontos do novo funcionamento legal, Ferenczi defende
que todos os médicos deveriam examinar tais leis e formular opiniões
acerca delas, evidenciando que a proposição de proteção inclui a
ativação da dimensão pública, da autoria e do reconhecimento do
explorado, como no caso da Srta. K e sua autobiografia.

Ferenczi igualmente ressalta as relações de poder presentes entre médico
e paciente, criticando o lugar de saber e poder conferido ao médico ao
problematizar a utilização da hipnose (O valor terapêutico da hipnose,
1904) e os efeitos problemáticos derivados disto, como por exemplo
exigir excessivamente do paciente ou seduzir as pacientes. Em 1906,
volta a falar da hipnose, no texto Do tratamento pela sugestão
hipnótica, reforçando que há efeitos benéficos desde que os médicos
sejam honestos, e que a assimetria da dupla pode ser nociva ao paciente.
Revela que há pessoas no corpo médico que se beneficiam dos estados de
incapacidade dos indivíduos hipnotizados, cometendo malfeitos sexuais.
Desta forma, denuncia o abuso do poder médico, desidealiza"-o e antecipa
a importante problematização feita pela Psicanálise, inclusive por
Freud, acerca da influência do médico na terapêutica do sofrimento
mental. Ferenczi desde o início percebe que o médico pode abusar de sua
posição de poder, pode envolver"-se sexualmente com seus pacientes, ou
seja, também sofrerá toda ordem de impactos de seus próprios afetos e
inconsciente. A contratransferência aqui já está em questão, a partir de
um reconhecimento da impossibilidade da manutenção de um controle
estrito destes efeitos. A figura hierárquica do médico racional e em
controle cai por terra, surgindo a figura de alguém que navega em águas
psíquicas bem mais turbulentas.

\section{Palavras finais}

Nossas posições políticas e discursivas evidenciam as finalidades de
nossos atos e, no caso dos psicanalistas, de seus atos terapêuticos.
Assim, revelar o campo da política a partir da qual atuamos torna"-se
fundamental para nossa sobrevivência, dado que ``todo e qualquer
psicanalista está a principio implicado numa política, uma vez que esta
o remete a finalidade dos seus atos'' (Cecchia, 2015, p.20). Se
considerarmos que ``a prática clínica implica em ativação de forças, em
estratégias de ação para produzir transformações e, nesse sentido, é uma
prática politica'' (Reis e Gondar, 2016, p. 153), então necessariamente
temos que nos pronunciar sobre as políticas que nos conduzem.

Ferenczi trabalhou a partir de situações de desamparo discursivo
próximas às nossas, também em uma \emph{pólis} patriarcal sintomática,
mas não teorizou no sentido de recuperar o lugar do pai em decadência.
Sua obra não nos convida para segui"-lo como discípulos de um mestre, mas
abre o campo da Psicanálise e permite que cada membro da comunidade seja
psicanalista a seu próprio modo - analistas da psique humana, em todas
suas dimensões, menos conservadores, mais consonantes com nossa
atualidade, menos isomórficos com os pressupostos fundamentais do
patriarcado, convocando"-nos para relações horizontais. Nas palavras da
Gondar (2012, p. 204): ``Fundar as relações subjetivas (analíticas ou
não) sobre a precariedade de todos nós --- essa talvez tenha sido a sua
proposta maior''.

Voltolini afirma que, tragicamente, a Psicanálise está ameaçada de
extinção, enquanto na época freudiana estava ameaçada no reconhecimento
(2016, p.39). A dimensão sociopolítica do sofrimento dos sujeitos atuais
é evidente e tornou"-se inevitável implicarmo"-nos no reconhecimento das
resistências à lógica patriarcal e fálica, especialmente quando o lugar
fálico mostra"-se cristalizado. Para evitar nosso desaparecimento ou
irrelevância, Ferenczi é inestimável companheiro de percurso. Acredito
que, junto a todas as críticas e revisões feitas à obra de Freud e de
psicanalistas posteriores, permanece nossa tarefa de atualização da
Psicanálise, atuante em um mundo onde os princípios patriarcais vacilam
entre superados e convocados em letras maiúsculas.

\section{Bibliografia}

\textsc{Birman}, J. Arquivos do mal"-estar e da resistência. Rio de Janeiro,
Civilização Brasileira, 2006.

\_\_\_\_\_ (2014). Arquivo e memória da experiência psicanalítica:
Ferenczi antes de Freud, depois de Lacan. Rio de Janeiro, Contracapa,
2014.

\textsc{Butler}, J. P. Problemas de gênero: feminismo e subversão da identidade.
Rio de Janeiro, Civilização Brasileira, 2018.

\textsc{Checchia}, M. Poder e politica na clinica psicanalítica. São Paulo,
Annablume, 2015.

\textsc{Ferenczi}, S. Les ecrits de Budapest. Paris, Epel, 1994.

\_\_\_\_\_ (1932) Diário Clinico. São Paulo, Martins Fontes, 1990.

\_\_\_\_\_(1991) Obras Completas, São Paulo, Martins Fontes, 1991.

\textsc{Freud}, S. The Standard Edition of the complete psychological Works of
Sigmund Freud. Londres: The Hogarth Press, 1995.

\textsc{Gomes}, G. D. De Viena a Wiesbaden: o percurso do pensamento clínico
teórico de Sándor Ferenczi. Dissertação de Mestrado do Programa de
Pós"-graduação em Psicologia Clinica da Pontifícia Universidade Católica
de São Paulo. São Paulo, 2016, 294 páginas.

\textsc{Gondar}, J. Ferenczi como pensador politico. Cadernos de Psicanalise --
\versal{CPRJ}, Rio de Janeiro, v. 34, n.27, p. 193-210, jul/dez. 2012.

\textsc{Jorge}, M. A. C. Fundamentos da psicanálise de Freud a Lacan, vol.2: a
clínica da fantasia. Rio de Janeiro, Zahar, 2010.

\textsc{Koltai}, C. Apresentação. In: Rosa, M. D. A clínica psicanalítica em face
da dimensão sociopolítica do sofrimento. São Paulo, Escuta/Fapesp, 2016.

\textsc{Knobloch}, F. {[}Orelha do livro{]}. In: Reis, E. S. e Gondar, J. Com
Ferenczi: clinica, subjetivação, politica. Rio de Janeiro, 7 Letras,
2017.

\textsc{Lorin}, C. Introdução. In Ferenczi, S. Les écrits de Budapest. Paris,
Epel, 1994.

\_\_\_\_ Le jeune Ferenczi -- premiers écrits 1899-1906. Paris, Editions
Aubier Montaigne, 1983.

\textsc{Plon}, M. Da politica em O mal"-estar, ao mal"-estar na politica. In: J. L.
Rider, M. Plon, G. Raulet e H. Rey"-Flaud, Em torno de o mal"-estar na
cultura, de Freud. São Paulo, Escuta, 2002.

\textsc{Pujó}, M. Trauma e desamparo. In Clínica do Desamparo. Buenos Aires:
Revista Psicoanálisis y el hospital, vol. 17, p.29, 2000.

\textsc{Reis}, E. S. e Gondar, J. Com Ferenczi: clinica, subjetivação, politica.
Rio de Janeiro, 7 Letras, 2017.

\textsc{Roudinesco}, E. Sigmund Freud na sua época e em nosso tempo. Rio de
Janeiro, Zahar, 2016.

\textsc{Voltolini}, R. (org.) Crianças publicas, adultos privados. São Paulo, Ed.
Escuta, 2016.

\textsc{Safatle}, W. O circuito dos afetos: corpos políticos, desamparo e o fim
do indivíduo. São Paulo, Cosac Naify, 2015.
