\part{Extraterritorialidades:\\ o olhar lançado de fora na análise de Freud}

\chapter*{Oswald contra o patriarcado: antropofagia, matriarcado e
complexo de Édipo\footnote{Este texto não teria sido possível sem o
  apoio das seguintes instituições: Programa de Pós"-Graduação em
  Educação da Unicamp, que acolheu minha pesquisa de pós"-doutorado em
  2018 (com um agradecimento especial a Silvio Gallo, supervisor do
  projeto); \versal{CEDAE} (\emph{Centro de Documentação Alexandre Eulálio}) da
  Unicamp; Capes, que através do programa Capes"-Cofecub financiou minha
  bolsa de pesquisa na França; \versal{IMEC} (\emph{Institut Mémoires de
  l'Édition Contemporaine}) e Universidade de Paris 8 (com agradecimento
  especial aos professores do \emph{Laboratoire d'études et de
  recherches sur les logiques contemporaines de la philosophie}, Didier
  Moreau e Patrice Vermeren, assim como aos professores Hubert Vincent,
  da Universidade de Rouen, e Alain"-Patrick Olivier, da Universidade de
  Nantes); e Faculdade de Educação da \versal{UFRJ}, que me concedeu afastamento
  de um ano para a realização da pesquisa, da qual este texto apresenta
  pequena parte de alguns de seus resultados parciais.}}
\addcontentsline{toc}{chapter}{Oswald contra o patriarcado: antropofagia, matriarcado e
complexo de Édipo, \footnotesize\emph{por Filipe Ceppas}}
\hedramarkboth{Oswald contra o patriarcado}{}

\begin{flushright}
\emph{Filipe Ceppas}\footnote{professor da Faculdade de Educação da Universidade Federal do
Rio de Janeiro (\versal{UFRJ}), onde participa da formação de professores, em especial
de filosofia. É professor do Programa de Pós"-Graduação em Filosofia da \versal{UFRJ}
(\versal{PPGF}) e faz pesquisas sobre temas relacionados à filosofia francesa
contemporânea, ensino de filosofia, educação e antropofagia. Foi coordenador
do \versal{GT} Filosofar e Ensinar a Filosofar da \versal{ANPOF} de 2008 a 2012. Coordena
atualmente o Laboratório de Ensino de Filosofia Gerd Bornheim (\versal{LEFGB-FE/UFRJ}) e o Núcleo de Pesquisa em Filosofia Francesa Contemporânea (\versal{Nuffc-CNPq/PPGF-UFRJ}). Participou, de 2015 a 2018, como membro pesquisador do Projeto \versal{CAPES-COFECUB} ``Diferença, pluralismo e confiança na educação'' e é membro do laboratório \emph{Recherche sur la Philosophie Pratique et Appliquée} (L.R.Ph.P.A.) da University of the Aegean (Rhodes-Grécia).}
\end{flushright}

\begin{flushright}
\footnotesize
%\emph{à Léa Silveira}\\
``\versal{FREUD}\\
Diretor espiritual da burguesia''\\
Oswald, \emph{Dicionário de bolso} (1932)
\end{flushright}

\section{Apresentação}

Oswald de Andrade apresentou em sua obra algumas críticas a Freud a
partir da associação entre antropofagia e matriarcado. No
\emph{Manifesto antropófago} de 1928, Oswald opôs a antropofagia,
enquanto elemento"-síntese da ``revolução caraíba'' ou da ``idade de ouro
da América'', à dominação masculina, à culpa, aos ``males catequistas'',
a ``realidade social, vestida e opressora, cadastrada por
Freud'' (\versal{ANDRADE}, 1970, p.~18-19). Se Freud foi o herói que
``acabou com o enigma mulher e com outros sustos da psicologia
impressa'' (1970, p.~13), o ``matriarcado de Pindorama'' anuncia
a recusa da ``sublimação do instinto sexual'' pela ``escala termométrica
do instinto antropofágico'' (1970, p.~19). Em texto de 1950,
\emph{A Crise da Filosofia Messiânica}, seguindo Simone de Beauvoir em
alguns pontos, divergindo em outros,\footnote{\versal{BEAUVOIR}, 1976, p.~84s.
  Não consta que Oswald tenha escrito sobre as críticas de Beauvoir a
  Bachofen e sua tese do matriarcado, desenvolvidas em \emph{Le deuxième
  sexe} (idem, p.~123s), embora ele as tenha lido, conforme confirmam as
  anotações no exemplar existente em sua biblioteca, disponível no
  acervo do \versal{CEDAE} (Unicamp). Nas citações do texto de Oswald, mantenho a
  pontuação adotada pela edição de 1970.} Oswald escreve:

\begin{quote}
Nenhum sentido (\ldots{}) teria num regime matriarcal o que os freudistas
chamam de ``complexo de castração'', pois nenhuma diminuição pessoal da
mulher traria a constatação dela possuir um sexo diverso do homem.
Somente a ideia de domínio do irmão --- invenção patriarcalista ---
poderia, numa já complexa fase psíquica, trazer à criança qualquer
ligação do fenômeno doméstico de preponderância com o fato fálico. (\ldots{})

Evidentemente o freudismo se ressente dos resíduos de sua formação
paternalista. Falta a Freud e a seus gloriosos sequazes, a dimensão
Bachofen. Eles não viram que suas pesquisas se limitavam e sua
interpretação se deformava, na pauta histórica do patriarcado. (\ldots{})

Numa sociedade, onde a figura do pai se tenha substituído pela da
sociedade, tudo tende a mudar. Desaparece a hostilidade contra o pai
individual que traz em si a marca natural do arbítrio. No Matriarcado é
o senso do Superego tribal que se instala na formação da adolescência.

Numa cultura matriarcal, o que se interioriza no adolescente não é mais
a figura hostil do pai"-indivíduo, e, sim, a imagem do grupo social.

Nessa confusão que o patriarcado gerou, atribuindo ao padrasto --- marido
da mãe --- o caráter de pai e senhor, é que se fixaram os complexos
essenciais da castração e de Édipo.

Simone de Beauvoir, no \emph{Deuxième Sexe}, esse evangelho feminista
que se coloca no pórtico da nova era matriarcal, escreveu: ``\emph{Ce
n'est pas la libido féminie qui divinise le père}.'' É na luta doméstica
com a mãe e depois na luta com o ambiente, que cresce a divinização
possível do pai como socorro, poder mediador e alento sentimental.
Fenômeno do Patriarcado (\versal{ANDRADE}, 1970, p.~125).
\end{quote}

E, no texto \emph{Variações sobre o Matriarcado,} lemos:

\begin{quote}
Esse passado onde o domínio materno se institui longamente, fazendo que
o filho não fosse de um só homem individualizado, mas, sim, o filho da
tribo, está hoje muito mais atenta e favoravelmente julgado pela
Sociologia, do que no tempo das afrontosas progenituras que fizeram a
desigualdade do mundo. Caminha"-se por todos os atalhos e por todas as
estradas reais para que a criança seja considerada o filho da sociedade
e não como sucede tão continuamente, no regime da herança, com o filho
de um irresponsável, de um tarado ou de um infeliz que não lhe pode dar
educação e sustento. A tese matriarcal abre rumo (1970, p.~258).
\end{quote}

Nestas passagens, Oswald conjuga Beauvoir e ideias sobre o matriarcado
que ele retira dos autores mais importantes que escreveram sobre o tema,
sobretudo Bachofen, Morgan e Engels. A tese geral do matriarcado, de
natureza evolucionista, segundo a qual teria havido uma época em que as
mulheres dominavam, em função do desconhecimento da paternidade, há
muito caiu em descrédito científico.\footnote{Para uma boa exposição e
  análise panorâmica deste debate, cf. \versal{LYONS} \& \versal{LYONS}, 2004, p.~73s. Para
  uma abordagem alternativa, cf. \versal{KNIGHT}, 1991.} Separando o bebê da água
suja do banho, é importante dar crédito a Oswald por ter, em meados do
século \versal{XX}, percebido a importância de identificar na ``cultura
antropofágica'' dos indígenas uma fonte de crítica ao patriarcado, aos
complexos sociais que marcam a vida dos ocidentais urbanos, incluindo aí
a crítica à psicanálise. Para além da defesa de um certo comunitarismo,
onde o ``Supergo tribal'' prevaleceria sobre os indivíduos, vale notar
que Oswald imprime ênfase no matriarcado como perspectiva futura. Esses
aspectos da crítica oswaldiana nos instigam a revisitar alguns pontos do
debate contemporâneo sobre o patriarcado e o complexo de Édipo, na
interseção da filosofia com a antropologia e a psicanálise.

\section{Só a antropofagia nos une\protect\footnote{\uppercase{P}rimeiro aforisma do
  \emph{\uppercase{M}anifesto antropófago}.}}

A antropofagia é um elemento importante das sociedades ameríndias e de
outras sociedades autárquicas ao redor do mundo que nos ajuda a
reconhecer, num só relance, diferenças radicais entre estas e as
sociedades estatais hierárquicas. A antropofagia nas sociedades
autárquicas é um ritual que explicita uma compreensão diferenciada de
diversos aspectos da existência, incluindo ao menos dois princípios
fundamentais: (1) uma ontologia em que o que chamamos de ``alma'' e
``corpo'' são indissociáveis e (2) o fato de que as relações entre os
seres são sempre de aliança"-inimizade. Os dois princípios podem ser
exemplificados com o ritual da antropofagia funerária, ingestão de
restos incinerados dos mortos misturados à comida, como um purê de
banana, por exemplo. As cinzas são ingeridas, \emph{grosso modo}, para
que a alma do parente morto não ameace a comunidade.\footnote{\versal{MÉTRAUX},
  2013, p.~385s; \versal{CLASTRES}, 2014, p.~55s. Cf., ainda, \versal{VIVEIROS DE CASTRO},
  2015, p.~155s.} O tema é politicamente delicado, pois a atribuição de
``práticas canibais'' aos indígenas tende, conforme o espectro
ideológico e o grau de ignorância de cada um sobre o assunto, a reforçar
o estereótipo de ``selvageria'' atribuído a estes povos. Inspirado na
obra de Oswald de Andrade e sua visão antropófaga de mundo, entretanto,
caminho na direção contrária: só a antropofagia nos une, nos
distinguindo, em múltiplos aspectos.

Por um lado, tanto a antropofagia funerária como a guerreira (morte e
devoração do cativo, ritual extinto já no século \versal{XVII}, por obra da
dominação europeia) explicitam aspectos fundamentais das cosmovisões
ameríndias, entre eles uma concepção animista, disposicional e
fabulatória do universo, onde predominam as relações complementares de
aliança e predação entre os seres.\footnote{A bibliografia sobre o tema
  é vasta. Destaco \versal{BENSUSAN}, 2017; \versal{DESCOLA}, 2005; \versal{VIVEIROS DE CASTRO},
  2002, 2015; e \versal{CLASTRES}, 1972. Em minha tentativa de resumir alguns
  aspectos das teorias e dos dados antropológicos sobre a antropofagia
  e, mais adiante, sobre as questões de gênero ou da proibição do
  incesto, assumo o risco de formulações muito imprecisas ou mesmo
  equivocadas. O que significa dizer que, para ``os ameríndios'', as
  relações de aliança e de predação são complementares? Questões como
  essa parecem não fazer sentido fora de uma vasta cadeia argumentativa
  onde se conjuga um conjunto extremamente complexo e especializado de
  dados, modelos, teorias e embates do campo antropológico. Assim, o
  melhor que o não"-especialista pode esperar é que a distorção que
  imprime aos dados do campo alheio o obrigue a refazer, contínua, lenta
  e pacientemente, e de modo cada vez mais acurado, a sua pilhagem,
  evitando ao máximo torturar esses dados para fazê"-los falar aquilo que
  ele quer ouvir.} A antropofagia não é apenas uma metáfora cultural.
Ela foi e é um processo de incorporação, praticada em rituais em torno
dos quais o Ocidente construiu toda uma gama de
fantasmagorias.\footnote{Uma recente e significativa produção, acessível
  a um público não especializado, tem procurado descrever, analisar e
  desconstruir essa fantasmagoria, da qual destaco \versal{KILANI}, 2018;
  \versal{MONTANARI}, 2018; \versal{VILLENEUVE}, 2016.} Pensar a antropofagia para além da
metáfora, como um dado específico e importante das culturas ameríndias,
implica um esforço de aproximação a concepções dos seres e da vida que
nos são estranhas, a nós, seres urbanos dos trópicos que, entretanto,
carregamos alguma herança dessas culturas (carregamos?!).

Por outro lado, também ``a antropofagia nos une'' por não ser
exclusividade dos povos autóctones. As sociedades ocidentais (e também
as africanas, asiáticas, orientais\ldots{})\footnote{Cf. \versal{GUILLE"-ESCURET},
  2012; \versal{MALAMOUD}, 1989; e \versal{HEUSCH}, 1986.} estão fortemente marcadas por
ela, seja no registro mitológico, sobretudo grego;\footnote{\versal{DETIENNE} \&
  \versal{VERNANT}, 1979; e \versal{DETIENNE}, 1977.} seja em dogmas religiosos
fundadores, como a Eucaristia; seja na vasta produção literária e
filosófica que, ao longo dos séculos, revisita regularmente histórias e
imagens antropófagas;\footnote{Vale destacar a peça \emph{Pentesiléia},
  de Heinrich von Kleist, traduzida para o português por Jean Robert
  Weisshaupt e Roberto Machado, disponível em:
  \textless{}\emph{https://bit.ly/2mr6Gwd}\textgreater{}.}
seja em aspectos insuspeitos da nossa cultura contemporânea;\footnote{Ver
  \versal{LÉVI"-STRAUSS}, 2103.} seja, em especial, na teoria freudiana da
cultura, onde a ``cena primitiva'' do assassinato e devoração do pai é
um dos alicerces fundamentais convocados pelo ``pai da psicanálise''
(ele próprio, ao mesmo tempo, Édipo decifrador da Esfinge) na defesa da
tese da universalidade do complexo de Édipo.\footnote{Cf., além das
  referências da nota 7, acima, os textos da edição especial, intitulada
  ``Destins du cannibalisme'', da \emph{Nouvelle Revue de Pyschanalyse},
  nº~6, de 1972, de onde consta a primeira tradução para o francês do
  \emph{Manifesto antropófago} de Oswald. Entre 1989 e 1991, Jacques
  Derrida ministrou dois seminários sobre diversos temas relacionados ao
  canibalismo, com ênfase na eucaristia e na psicanálise freudiana,
  incluindo os trabalhos de K. Abraham e M. Klein (\emph{Manger l'autre}
  e \emph{Réthoriques du cannibalisme}). Esses seminários permanecem
  inéditos e estão disponíveis para consulta no \versal{IMEC}.}

Como característica central das culturas ameríndias, a antropofagia foi
reconhecida por Lévi"-Strauss como diferencial inequívoco das ditas
``sociedades selvagens'' frente às ``sociedades civilizadas''. O
antropólogo francês, que se esforçou, em toda a sua obra, por indicar
equivalências estruturais entre ambas as sociedades, reconhece na
antropofagia esse diferencial precisamente no que tange à relação com o
inimigo: as sociedades ocidentais são ``antropêmicas'', isto é, vomitam
seus elementos ameaçadores, afastando"-os do convívio social,
enclausurando"-os em instituições como as prisões e os manicômios;
enquanto as sociedades autoctones são antropófagas, isto é, devoram ou
incorporam os elementos externos ou internos que as ameaçam. Esta
definição permite expandir o uso do termo ``antropofagia'' para
caracterizar outras relações socioculturais estabelecidas pelos
ameríndios, que vão além dos rituais antropofágicos \emph{stricto sensu}
--- e Lévi"-Strauss toma como exemplo, em sua argumentação, a forma como,
em uma sociedade indígena norte"-americana, um infrator é reinserido no
corpo social.\footnote{Cf. \versal{LÉVI"-STRAUSS}, 2008, p.~463s. Esta definição da
  antropofagia é discutível, uma vez não ser verdade que uma infração
  implique, sempre ou na maioria dos casos (ou pelo menos não na
  superfície mais aparente dos eventos), em \emph{reincorporação} do
  infrator ao grupo social (cf., por exemplo, \versal{MELATTI}, 1994, p.~165-166).
  Penso que essa definição serve, entretanto, para indicar de modo
  simples e didático a complexa centralidade da ``transcorporação
  antropofágica'' no imaginário ameríndio, tal como analisada por
  Viveiros de Castro em suas obras. Cf., em especial, \versal{VIVEIROS DE CASTRO}
  2002, p~.163s. e 2015, p.~155s.} Com este sentido alargado, podemos
compreender a antropofagia como característica geral das culturas
ameríndias, para além de sua presença nos rituais e mitos de cada
sociedade em particular, e podemos falar de culturas ou sociedades
antropofágicas para nos referirmos globalmente às culturas ou sociedades
ameríndias.\footnote{Em outro texto, procurei contrastar esta
  antropofagia ameríndia com aspectos antropofágicos da cultura
  ocidental. Nesta, a antropofagia tende a funcionar, diferentemente,
  como princípio desmedido de aniquilação do outro e consumo devastador
  e autodestrutivo. Cf. \versal{CEPPAS}, 2019.}

Estabelecidas essas premissas, cabe perguntar em que medida as
sociedades ameríndias antropófagas podem ser concebidas, tal como queria
Oswald de Andrade, como sendo também ``sociedades matriarcais''; ou
quais seriam as relações entre antropogafia e matriarcado.

\section{Matriarcado}

No que se refere às sociedades ameríndias, o termo ``matriarcado''
parece totalmente inapropriado. Excetuando a lendária ``tribo das
mulheres guerreiras'' (as icamiabas ou amazonas), as sociedades
ameríndias não são sociedades em que as mulheres \emph{dominam}. Não
parece haver, entre os povos sul"-americanos, sociedades em que se
poderia reconhecer um predomínio das mulheres sobre os homens no que se
refere às deliberações políticas ou, de modo geral, à responsabilidade
quanto às questões espirituais.\footnote{Há, contudo, um extenso debate
  sobre configurações de tipo matriarcal em outras sociedades indígenas,
  o qual espero poder abordar em outro texto. Cf., por exemplo,
  \versal{GOETTNER"-ABENDROTH}, 2012, 2009 e 2008; e \versal{VAUGHAN}, 2007.} Porém, se a
literatura antropológica é unânime em afirmar que o poder político e, em
grande medida, também o poder espiritual nas sociedades ameríndias são,
via de regra, exclusividade dos homens, ela também nos leva a reconhecer
que são complexas e indissociáveis as relações entre o político, o
espiritual e outras dimensões da vida social, em muitas das quais
predomina o poder da mulher. A questão fundamental é que essas
sociedades tampouco podem ser chamadas, sem mais, de ``patriarcais'':
nelas (com exceção, talvez, das antigas sociedades andinas e
mesoamericanas) os homens também não dominam, pura e simplesmente.
Segundo esta hipótese, denominar as sociedades antropofágicas de
``matriarcais'', tal como Oswald insistia em fazer, é uma feliz
impropriedade, uma fértil provocação, uma vez que a própria ideia de um
``domínio de gênero'', quando referida às sociedades indígenas, parece
problemática e esconde uma grande variedade de questões complexas que
impedem que as identifiquemos com uma versão qualquer do patriarcado.

Assim, o uso do termo ``matriarcado'', embora neste sentido lato e
equívoco, tem uma vantagem inequívoca: com ele, somos desde o início
forçados a descortinar aspectos de interseções entre gênero, sexualidade
e relações sociais das sociedades autárquicas sul"-americanas
aparentemente incompatíveis com o patriarcado. Sociedades sem complexos
ou neuroses, como queria Oswald. Tal como a antropofagia, o matriarcado
pode, assim, ser visto como outro índice de diferenciação fundamental
entre as sociedades autárquicas ameríndias e as sociedades hierárquicas
e estatais. Vale dizer, a construção da sexualidade e das identidades de
gênero nas sociedades ameríndias não passa pelo \emph{domínio do Pater},
nem mesmo, \emph{stricto sensu}, pelo ``domínio dos homens'', tal como,
predominantemente, acontece nas sociedades ocidentais, hierárquicas e
estatais. O matriarcado, em minha leitura dos textos de Oswald de
Andrade (em parte \emph{malgré lui}), não é uma afirmação
filosófico"-científica a qual corresponderia um estado de coisas ou um
conjunto de fenômenos a serem explicados;\footnote{Cf., entretanto, o
  \emph{Post scriptum 1}, abaixo, onde apresento brevemente uma análise de
  Viveiros de Castro em que ele desenvolve uma forma positiva de se
  pensar uma certa centralidade do feminino nas culturas antropófagas.}
ele é antes um mito entre outros e, como tal, uma ferramenta de
luta, de denúncia do patriarcado, de negação da universalidade da
culpa, da castração e do complexo de Édipo como constitutivos e
determinantes da psiquê humana; negação, ainda, de que o ``domínio do
homem'' seria algo inerente a toda e qualquer cultura humana.\footnote{Vale
  ressaltar que, na teoria freudiana do complexo de Édipo e da
  castração, sobretudo a partir dos trabalhos de Lacan, a ``posição do
  pai'' não se confunde necessariamente com o pai biológico ou com o
  papel do homem num grupo social. Mas, como dizem Deleuze e Guattari,
  ``mesmo quando remontamos das imagens à estrutura,
  das figuras imaginárias às funções simbólicas, do pai à lei, da mãe ao
  grande Outro, estamos, na verdade, \emph{apenas adiando a questão}'' (2010, p.~115-116).
  Há, sem dúvida, muitas questões difíceis implicadas nessas afirmações
  e não pretendo enfrentar a maioria delas. Este texto pretende tão
  somente fornecer uma primeira formulação de alguns dos problemas
  centrais implicados na trilha aberta pela antropofagia oswaldiana.}

Os textos de Oswald acima citados partem, tal como o clássico livro de
Beauvoir, da ausência de qualquer diferença biológica determinante entre
os sexos: não há nada na anatomia sexual que leve, por si só, à
diminuição pessoal da mulher ou à preponderência do falo. Mas não
existiriam razões sociais, evolutivas, históricas, etc. que nos levariam
a reconhecer, também nas sociedades autárquicas, tais dimunição e
preponderância? Para responder esta questão, vale revisitar um texto de
Gayle Rubin, ``The traffic in women'', que analisa alguns
impensados essenciais no debate sobre a dominação masculina, no âmbito
da antropologia e da psicanálise, nos ajudando assim a compreender o
alcance das tiradas oswaldianas.\footnote{Agradeço à Léa Silveira a
  sugestão da leitura desse texto de Rubin, sem a qual este texto sequer
  existiria.} A respeito do princípio, fundamental na antropologia, da
``troca de mulheres'', Rubin afirma:

\begin{quote}
A ``troca de mulheres'' é (\ldots{}) um conceito problemático. Quando
Lévi"-Strauss argumenta que o tabu do incesto e o resultado de sua
aplicação constituem a origem da cultura, pode"-se deduzir que a anulação
(\emph{defeat}) mundial da mulher ocorreu com a origem da cultura, e que
ela é um pré"-requisito da cultura. (\ldots{}) Entretanto, seria no mínimo uma
afirmação duvidosa argumentar que, se não há troca de mulheres, não
haveria cultura, pela simples razão de que cultura é, por definição,
algo inventivo. É inclusive discutível que ``troca de mulheres''
descreva adequadamente toda a evidência empírica disponível sobre os
sistemas de parentesco (1975, p.~176).
\end{quote}

Viveiros de Castro, seguindo Lévi"-Strauss,\footnote{``\ldots{}l'ensemble des
  règles de mariage observables dans les sociétés humaines ne doivent
  pas être classées --- comme on le fait généralement --- en catégories
  hétérogènes et diversement intitulées: prohibition de l'inceste,
  types de mariages préférentiels, etc. Elles représentent toutes autant
  de façons d'assurer la circulation des femmes au sein du groupe
  social, c'est"-à"-dire de remplacer un système de relations consaguines,
  d'origine biologique, par une système sociologique d'alliance''
  (\versal{LÉVI"-STRAUSS}, 1958, p.~68).} argumentou, em texto de 1990, que a ``troca
de mulheres por homens'' não é essencial para as ``estruturas de
aliança'':

\begin{quote}
A noção de troca matrimonial não exige que homens troquem mulheres (em
contrapartida, as considerações sobre a frequência da patrilinearidade,
matrilocalidade, etc. implicam asserções deste tipo), mas apenas que
\emph{cada sexo apreenda o outro imediatamente como relação}. A oposição
estrutural que funda e organiza a aliança, e por ela a sociedade, não é
aquela meramente zoológica entre os sexos, mas aquela sociológica entre
\emph{termo} e \emph{relação}\ldots{} (\versal{VIVEIROS DE CASTRO}, 1990, p.~26; cf., ainda, 2015, p.~144s).
\end{quote}

Contudo, para além dos limites da forma como Rubin constrói seu
argumento (incluíndo o apelo às evidências empríricas, questão que
Lévi"-Strauss já havia bem elucidado em \emph{As estruturas elementares
do parentesco} {[}1982, p.~47s.{]}), ela acaba chegando
a uma conclusão semelhante:

\begin{quote}
Sistemas de parentesco não trocam simplesmente mulheres. Eles trocam
acesso sexual, \emph{status} genealógico, nomes e ancentrais de
linhagem, direitos e \emph{pessoas} --- homens, mulheres e crianças ---
em sistemas concretos de relações sociais. Essas relações sempre incluem
certos direitos para o homem, outros para a mulher (\versal{RUBIN},
  1975, p.~177).
\end{quote}

Necessário acrescentar que a própria noção de \emph{troca} é equívoca,
seja porque mulheres não são trocadas como objetos, seja porque, quando
o são, segundo o próprio pensamento nativo, objetos não são simples
objetos, mas veiculam aspectos importantes de \emph{pessoas} e valores
do clã. Como argumentou Strathern (2014), pensar na
circulação da mulher como a troca de uma propriedade implica, via de
regra, a projeção inadequada das categorias ocidentais de propriedade,
sujeito e objeto sobre a produção, reprodução e circulação das riquezas
e das pessoas nas sociedades tradicionais.\footnote{Strathern analisa
  sobretudo sociedades da Papua"-Nova Guiné, mas a noção de \emph{pessoa}
  também é fundamental na literatura sobre os indígenas sul"-americanos,
  adquirindo importância central nos estudos antropológicos a partir da
  segunda metade do século \versal{XX}. Cf. \versal{SEEGER, DA MATTA} \& \versal{VIVEIROS DE
    CASTRO}, 1979; e \versal{VIVEIROS DE CASTRO}, 2002, p.~181s.}

Na ideia de que são os homens que trocam mulheres, é também a noção de
\emph{escolha} que está em jogo. No casamento, se a escolha não é, via
de regra, da mulher, tampouco pode"-se dizer, sem mais, que seja do homem
que irá casar"-se com ela, por exemplo, do primo"-filho"-do"-irmão"-da"-mãe.
Por fim, a escolha é menos uma decisão dos homens do que decorrência das
regras que constituem o sistema de parentesco, dentre outros fatores
particulares e complexos de cada cultura.\footnote{Como Lévi"-Strauss
  afirma em \emph{As estruturas elementares do parentesco}, ``entre os
  sistemas que indicam o cônjuge e aqueles que o deixam
  indeterminado, há formas híbridas e equívocas'' (1982,
    p.~20).} Neste sentido, a seguinte passagem de Rubin é esclarecedora:

\begin{quote}
``Troca de mulheres'' é uma abreviação para expressar que as relações
sociais de um sistema de parentesco especificam que os homens têm certos
direitos sobre as parentes mulheres, e que as mulheres não têm os mesmos
direitos sobre elas mesmas ou sobre os seus parentes homens. Neste
sentido, a troca das mulheres é uma profunda percepção de um sistema no
qual as mulheres não têm total direito sobre si mesmas. {[}\emph{Mas}{]}
A troca de mulheres torna"-se um ofuscamento se ela é vista como uma
necessidade cultural, e quando é usada como única ferramenta com a qual
uma análise de um sistema de parentesco particular é
realizada (1975, p.~177).\footnote{Seguindo a análise de Viveiros
  de Castro, no texto acima citado, ``troca de mulheres'' seria um
  ``ofuscamento'' quando se confude ``a oposição termo/relação com uma
  dominância substantiva dos homens sobre as mulheres''.}
\end{quote}

Nas sociedades autárquicas, tanto o lugar da mulher como o dos homens,
no que diz respeito ao casamento, não obecedem a um princípio
substantivo de ``domínio dos homens'', mas orbitam em torno do sistema
de regulação das trocas matrimoniais, ainda que, na maioria dos casos, a
decisão sobre quem casa com quem seja ``realizada'' (ou, antes,
verbalizada?) por homens. Tudo isso pode soar como uma tentativa
desesperada de suavizar a tese da dominação masculina nas sociedades
autárquicas, mas a etnografia oferece inúmeras razões para levar a sério
essas ideias, sobretudo se consideramos que, parafraseando Beauvoir
\emph{passim} Oswald, os processos de determinação de conjugues não
parecem ser elementos suficientes para gerar a diminuição pessoal da
mulher, muito menos uma ``preponderância do falo''.

A diminuição da mulher e a preponderância do falo são parte de uma
história patriarcal. Rubin, mais uma vez, esclarece:

\begin{quote}
existem sistemas estratificados por gênero que não são adequadamente
descritos como patriarcais. Muitas sociedades da Nova Guiné são
cruelmente opressivas para as mulheres. Mas o poder dos machos nesses
grupos não se baseia em seus papéis como pais ou patriarcas, mas em sua
masculinidade coletiva adulta, incorporada em cultos secretos, casa dos
homens, guerra, redes de troca, conhecimento ritual e vários
procedimentos de iniciação. O patriarcado é uma forma específica de
dominação masculina, e o uso do termo deve ser restrito ao tipo de
nômades pastorais do Velho Testamento, de onde vem o termo, ou a grupos
como eles. Abraão era um patriarca --- um velho cujo poder absoluto sobre
as esposas, filhos, rebanhos e dependentes era um aspecto da instituição
da paternidade, conforme definido no grupo social em que ele
vivia (1975, p.~168).
\end{quote}

Ademais, muitas (a maioria?) das sociedades ameríndias, em seus
\emph{habitats originais}, parece estar longe de ser mais opressiva para
as mulheres do que para os homens. Ao contrário, os estudos
etnográficos, assim como os antigos relatos de missionários e de
viajantes, são pródigos em informar sobre uma certa distribuição em
geral igualitária na divisão sexual do trabalho e sobre o relativo grau
de liberdade com que os/as indígenas vivem a sexualidade.\footnote{O apelo
  a um ``habitat original'' pode parecer problemático, pois é evidente
  que o contato com ``a civilização'' gera um sem número de problemas,
  dentre eles a tendência à perturbação do aparente equilíbrio (por
  falta de um termo melhor) da organização social, da divisão sexual do
  trabalho, do prestígio de pessoas ou papéis sociais, etc. Mas, ainda
  assim, para além dos relatos por vezes contraditórios de viajantes e
  missionários, a etnografia contemporânea (de \versal{METRAUX}, 2013,
  p.~203-232, a \versal{VIVEIROS DE CASTRO}, 1986, p.~165s, passando por
  \versal{LÉVI"-STRAUSS}, 2008, p.~208s, e \versal{CLASTRES}, 2003, p.~118-145) nos oferece
  inúmeros relatos e análises que parecem corroborar uma tal percepção.}
O que não quer dizer que o lugar da mulher nessas sociedades seja isento
de tensão e subordinação, assim como também não o é o dos homens em
muitos dos seus papéis sociais. Qualidades atribuídas pelos próprios
indígenas ao feminino, ou a alguns de seus elementos, como o eventual
status socialmente inferior da mulher, enquanto lugar do profano, da
mácula ou da desordem, não podem ser analisados fora de uma complexa
rede de relações, mitos e práticas em torno de temas centrais, como a
menstruação, por exemplo, conjugando a análise do pensamento nativo com
os aportes de diversas disciplinas, como a arqueologia e a
biologia.\footnote{Cf. \versal{KNIGTH}, 1991.}

\section{Um mito entre outros}

A psicanálise complica bastante o meio de campo com conceitos como
complexo de Édipo e castração. A teoria de Freud possibilita uma certa
compreensão da subordinação das mulheres \emph{em geral} (tanto nas
sociedades ocidentais como nas não"-ocidentais) como decorrência de
inevitáveis tensões inerentes à experiência humana, e Gayle Rubin
adverte para a diferença entre, de um lado, ``racionalizar ou justificar
a subordinação'' e, de outro, conhecer seus mecanismos para lutar contra
eles. Os gêneros, como quase tudo o mais, entram na vida da criança
através da experiência corporal e do simbólico, e é bem cedo, a partir
das complexas relações entre os sexos já instituídas na sociedade que a
criança buscará resolver suas tensões afetivas e começará a criar sua
identidade. O pressuposto fundamental, e aparentemente razoável, é que é
a mãe o primeiro objeto de amor. Já a ideia de que abdicar deste objeto
envolveria, forçosamente, a repressão de dois direcionamentos pulsionais
indissociáveis, um incestuoso, voltado à mãe, e outro parricida, é o
núcleo problemático da triangulação edipiana cujas supostas validade e
universalidade são constantemente questionadas.

Perguntar como se desenrola, em geral, o ``amor pela mãe'', suas
eventuais relações incestuosas e sua negação, em curumins e cunhatãs;
analisar essas ligações afetivas e o processo de renúncia a ela em
famílias que em nada se parecem com a famíllia triangular edipiana, e
onde as manifestações e concepções de afeição são significativamente
diferentes das concepções ocidentais de ``amor'', em culturas totalmente
atravessadas por fortes laços comunitários, próprios a uma sociedade
autárquica; estas são, sem dúvida, questões importantes no debate sobre
a possível validade de se falar de complexo de Édipo entre os indígenas.
Mas são também questões ardilosas, porque pressupõem que a análise
antropológica se desenrola no mesmo nível epistêmico que a teoria
psicanalítica, o que não é o caso, ao menos quando se trata do
estruturalismo lévistraussiano.\footnote{O livro de Melford Spiro,
  \emph{Oedipus in the trobiands} (1982), por exemplo, desenvolve uma
  detalhada análise da suposta refutação da universalidade do complexo
  de Édipo, desenvolvida por Malinowski, procurando mostrar que as
  enormes diferenças entre as dinâmicas psicológica, sexual e afetiva da
  ``família triangular ocidental'' e as dos trobiandeses não excluem a
  pertinência, para estas últimas, de uma leitura edipiana. Impossível,
  aqui, entrar nos detalhes da argumentação de Spiro, alguns
  extremamente importantes para o nosso estudo. Mas este e outros textos
  sobre o tema trabalham, precisamente, sem levar em conta essa
  ``diferença epistêmica'' entre antropologia e psicanálise, da qual
  falarei a seguir. Seja como for, vale dar apenas um exemplo do quanto
  a argumentação de Spiro parece, em outros aspectos essenciais,
  problemática --- exemplo tanto mais relevante por dizer respeito a uma
  \emph{bandeira} fundamental de Oswald: a de que as sociedades
  indígenas são livres de ``complexos patriarcais'' e ``prostituições''.
  Spiro argumenta que, para Freud, os conflitos não resolvidos,
  subjacentes ao complexo de Édipo, são condição necessária, mas não
  suficiente para que haja neurose. Daí que, para ele, o fato de
  aparentemente não haver neurose entre os nativos não seria um dado
  significativo para uma refutação da universalidade da teoria
  freudiana. \emph{Mais quand même!}} Neste sentido, soa muito suspeito
o forte paralelismo que Gayle Rubin identifica nas obras de Freud e de
Lévi"-Strauss, no que concerne à influência do patriarcado:

\begin{quote}
A precisão no ajuste entre Freud e Lévi"-Strauss é impressionante.
Sistemas de parentesco exigem a divisão dos sexos. A fase edipiana
divide os sexos. Sistemas de parentesco incluem conjuntos de leis que
governam a sexualidade. A crise edipiana é a assimilação dessas regras e
tabus. Heterossexualidade compulsória é o produto do parentesco. A fase
edipiana constitui o desejo heterossexual. O parentesco descansa sobre
uma diferença radical entre direitos de homens e mulheres. O complexo de
Édipo confere direitos masculinos ao menino e força a menina a se
acomodar a menos direitos (\versal{RUBIN}, 1975, p.~200).
\end{quote}

A meu ver, há diversos problemas nessas sinopses, que distorcem aspectos
importantes tanto da teoria lévistraussiana como da freudiana. Não
pretendo entrar no mérito da adequação ou não dessas formulações. Elas
servem apenas como ponto de partida. Em primeiro lugar, há a advertência
de Viveiros de Castro, mencionada acima, de que os termos e relações que
compõem as trocas matrimoniais não devem ser identificados com
substâncias sexuais biológicas (advertência da qual, como vimos, Rubin
ela mesma se aproxima, apesar do substancialismo que por vezes assume).
Em segundo lugar, é preciso levar em conta a análise que o próprio
Lévi"-Strauss faz da hipótese freudiana da universalidade do complexo de
Édipo. As duas advertências se conjungam no fato de que a análise
estrutural lévi"-straussiana trabalha com a comparação de complexos
simbólicos relativamente independentes de seus contextos e conteúdos.
Esta análise não está preocupada com a vivência real dos sujeitos em
torno de suas representações, ainda que esta não lhe seja indiferente.
Mas, na análise estrutural, a premissa fundamental é a de que os
sujeitos agem constrangidos por (ou num \emph{corpo a corpo} com)
complexos simbólicos de tipos diversos, onde o conteúdo sexual não tem
prioridade.

Em textos seminais, reunidos em seção do livro \emph{Antropologia
estrutural}, de 1958, intitulada ``Magie et Religion'', mas sobretudo em
``La structure des mythes'', Lévi"-Strauss, pondo lado a lado o mito de
Édipo e o mito Zuni dos grupos Pueblo, procura demonstrar que só se pode
compreender bem um mito ao se analisar a estrutura combinatória de seus
elementos e, de preferência, a totalidade de suas versões, sendo
temerário querer encontrar o sentido para qualquer desses elementos em
uma atribuição de função (social, econômica, psíquica) e/ou
representação de universais, ainda que historicamente contextualizados,
tal como a expressão de polaridades do ser humano (feminino/masculino,
amor/ódio), etc. Para Lévi"-Strauss, antes do que indicar uma estrutura
universal ou histórica da psiquê humana, a teoria freudiana do complexo
de Édipo é apenas mais uma versão do mito grego, tão ``autêntica'' (o
termo é do autor) quanto qualquer outra. Essa questão será retrabalhada
em \emph{A oleira ciumenta}, de 1985, onde o antropólogo retoma o debate
com a psicanálise, argumentando que a genialidade de Freud está em
``pensar à maneira dos mitos'', aproximando a interpretação dos sonhos e
dos dramas psíquicos ao modelo mítico \emph{em chave estruturalista},
isto é, ao seguir com perspicácia a lógica de oposições e inversões
própria do mito.

As análises de Lévi"-Strauss nos levam, portanto, a recusar a hipótese de
Édipo como estrutura universal e fundamental da psiquê humana, tal como
proposto em \emph{Totem e tabu}, para além dos limites e problemas mais
evidentes da análise freudiana, como a pressuposição evolutiva de três
estágios do pensamento: animista, religioso e científico.\footnote{Cf.
  \versal{GAY}, 1989, p.~301-311; e \versal{GREEN}, 1975.} Em \emph{Totem e tabu}, Freud
postulou uma ``cena primordial'', o fenômeno primitivo e antropófago, do
assassinato e devoração do ``pai primordial'', doravante inscrito nas
sociedades e na estrutura da psiquê dos seres humanos como \emph{horror
ao incesto}, através do sentimento edipiano da \emph{culpa}. É porque
assume de saída que há uma base psíquica universal do ser humano, a unir
``primitivos'' e ``civilizados'', que Freud pode aproximar os totens e
tabus descritos pela etnografia às neuroses da sociedade europeia de
fins do século \versal{XIX} e início do século \versal{XX}.\footnote{Como nota
  Lévi"-Strauss, ``o próprio Freud reconhece, com candura, que (\ldots{}) a
  psicanálise nada mais faz do que encontrar no mito aquilo que ela lá
  pôs: `O material foi"-nos transmitido num estado que não permite fazer
  uso dele, para a resolução de nossos problemas. Ao contrário, deve ser
  submetido, antes, a uma elucidação psicanalítica.' Daí, a confidência
  melancólica a Jung a propósito de \emph{Totem e tabu}, que está a
  escrever, e das dificuldades que depara: `A isso, acrescente"-se que o
  interesse diminuiu, pela convicção de já se ter à partida os
  resultados que se pretende provar'. Não se poderia dizer melhor''
  (\versal{LÉVI"-STRAUSS}, 1986, p.~204).} Neste sentido, Freud não peca por adotar
um método empirista ingênuo. Coincidências entre representações, fobias,
prescrições, etc. não seriam nunca suficientes para garantir o movimento
inverso, isto é: a aproximação entre uma coisa e outra não é suficiente
para demonstrar Édipo como estrutura psíquica universal, tal como
gostaria Freud. O problema, segundo Lévi"-Strauss, não estaria tanto nas
aproximações em si, mas antes no princípio universal a elas atribuído,
sempre capaz de ser identificado em (e de explicar o sentido mais
profundo de) fenômenos aparentemente os mais diversos e
enigmáticos (\versal{LÉVI"-STRAUSS}, 1958, p.~228s). Lévi"-Strauss recusa a
centralidade de Édipo, ou de qualquer outro ``código único e
exclusivo'' (1986, p.~231), assim como a
identificação do sentido de um mito com sua origem, como suposto
acontecimento originário.

Lévi"-Strauss nos mostra, ainda, que, enquanto ``fundamento da cultura'',
a proibição do incesto é antes produtiva do que proibitiva: ela traduz o
estabelecimento de conjuntos de regras que conjugam a plasticidade das
trocas sexuais em um sem número de soluções diferentes. A diversidade
das complicadas alianças e regras matrimoniais não ocidentais, assim
como as representações mitológicas; sobretudo, o fato de que esta
diversidade permite a identificação de uma lógica combinatória mais
geral de oposições e inversões, capaz de dar sentido a um conjunto
aparentemente arbitrário e incompreensível de representações e práticas;
tudo isso parece demonstrar à exaustão o quanto a conjugação da
plasticidade sexual, enquanto condição para uma mínima coesão social,
não poderia estar restrita a um único complexo, a uma única
lei.\footnote{O que, evidentemente, não significa que a análise
  estrutural não trabalhe no registro da universalidade, que é,
  entretanto, formal e \emph{vazia}, como ``un pensar anónimo frente a
  la contrariedade del acontecimiento'', como formula Alejandro Bilbao:
  ``Que el pensamiento mítico sea la consecuencia de un procedimiento de
  distinción y diferencia entre series naturales y culturales al
  interior de una cultura (caso igualmente del totemismo), da cuerpo a
  la imposibilidad de un significado último para el mito, pero puede,
  por el contrario, dar lugar a una pronunciación por parte de este
  pensar, respecto del actuar de estas significaciones con relación a
  otras series culturales en estado de oposición frente a la naturaleza.
  Deste modo, se permite el encuentro de las semejanzas entre ambos
  grupos de oposiciones diferenciales, ya que al admitirlas en un
  pliegue de relaciones significantes, es posible estabelecer la
  existencia de producciones simbólicas universales, y por lo tanto,
  observar el trabajo que en un nivel de invarianza, el signo ejecuta
  sobre las producciones humanas'' (2009, p.~176-177).}

A crítica antropófago"-estrutural à absolutização do complexo de Édipo é
uma estratégia, dentre outras, de reavaliar a questão da dominação
masculina no que se refere aos povos autóctones. No mesmo passo, a feliz
impropriedade da tese matriarcal de Oswald de Andrade, um mito entre
outros, é uma fértil provocação para, com a ajuda da antropologia e das
vivências e concepções dos povos indígenas, avançar na desconstrução de
nossas concepções androcêntricas de sexualidade e relações de gênero.

\section{\emph{Post scriptum 1}}

Outro caminho para reavaliar a questão da dominação masculina no que se
refere aos povos antropófagos encontra"-se na obra de Eduardo Viveiros de
Castro. O autor nos indica uma série de evidências com relação à
centralidade do feminino na lógica de funcionamento das sociedades
ameríndias antropófagas. Para Viveiros de Castro, o ideal de matador dos
indígenas ``sugere que a posição de `comida dos deuses'
(\emph{Maï demïdo}, epíteto que descreve a condição humana) é feminina
--- que a condição de vivente humano é feminina, portanto'' (2002, p.~281). Ou, em outra passagem:

\begin{quote}
O canibalismo parece ter sido, entre muitas coisas, o método
especificamente feminino de obtenção da vida longa, ou mesmo da
imortalidade, que no caso masculino era obtido pela bravura no combate e
a coragem na hora fatal. Há mesmo indicações de que a carne humana era
diretamente produtora daquele aligeiramento do corpo que os Tupi"-Guarani
buscaram de tantas formas diferentes, pela ascese xamânica, a dança, ou
a ingestão de tabaco\ldots{} (2002, p.~257).
\end{quote}

Sobre esse predomínio do ``aligeiramento do corpo'' na ascese xamânica,
Deleuze e Guattari escreveram: ``\ldots{}é menos a mulher que é
feiticeira e mais a feitiçaria é que passa por esse
devir"-mulher'' (2007, p.~32). Para Viveiros
de Castro, as cosmovisões ameríndias (mediadas ou não pelos construtos
teóricos da antropologia) não devem ser encaradas como meras
representações de uma ``realidade do outro'' que espelham, no fundo,
nossos próprios problemas. Elas deveriam ser levadas à sério tal como
podemos levar à sério a filosofia ocidental, leibniziana, hegeliana,
marxista, ou qualquer outra, como compreensão e explicação do
real (\versal{VIVEIROS DE CASTRO}, 2015). E é como um campo de força de
experiências e visões de mundo radicalmente conflitantes com a(s)
metafísica(s) ocidental(is) que elas nos desafiam a repensar nossas
explicações mais usuais, incluindo a psicanálise, no que diz respeito às
diferenças de gênero. A imagem oswaldiana do matriarcado é, como foi
dito, uma fértil provocação para questionar nossa forma tradicional de
pensar a dicotomia masculino"-feminino, e pode perturbar mesmo as formas
mais tradicionais com que costumamos nos opor à dominação
masculina (\versal{DELEUZE} \& \versal{GUATTARI}, 2007, p.~68s).

\section{\emph{Post scriptum 2}}

Nada do que foi dito acima implica, obviamente, a rejeição em bloco da
teorização freudiana como sendo apenas um subproduto do patriarcado.
Importante ponderar que \emph{Totem e tabu} não se reduz, evidentemente,
à tentativa de comprovação da tese do assassinato e devoração da pai da
horda primitiva como fundamento da cultura e, portanto, da
universalidade do complexo de Édipo. Seria impossível, neste texto,
fazer justiça aos inúmeros méritos das análises de Freud acerca dos mais
diversos aspectos que ele, por caminhos mais ou menos legítimos,
identifica como passíveis de aproximar os totens e tabus dos povos
animistas e componentes dos comportamentos psicopatológicos modernos,
tal como interpretados pela psicanálise. Impossível igualmente, aqui,
sequer iniciar uma análise sobre a contribuição fundamental de Freud
para o debate acerca da antropofagia, cujos vestígios na obra de Oswald
de Andrade devem ser investigados. Como argumentou André Green, o
canibalismo na obra de Freud seria um falso problema se a devoração
totêmica do pai primordial fosse apenas um ``conceito teórico para uso
interno da psicanálise''. Como relembra o autor, ``dois anos após
\emph{Totem e tabu}, o canibalismo reapareceu na obra freudiana por meio
da clínica psicanalítica, em \emph{Luto e melancolia}. O estudo das
relações entre luto e melancolia permite descobrir neste estrutura as
particularidades da \emph{relação oral canibal}\ldots{}'' (1972, p.~38).
Já nos \emph{Três ensaios sobre a teoria da sexualidade}, escrito
em 1905, Freud subscrevia a opinião de ``alguns autores'', que
associavam a dimensão agressiva da pulsão sexual ao canibalismo --- o
que ele afirma, numa nota acrescida em 1915, ser confirmado por suas
considerações sobre as fases pré"-genitais do desenvolvimento sexual. As
teses desenvolvidas pela psicanálise, a partir desta associação entre
canibalismo e sexualidade infantil --- sobretudo nos trabalhos do próprio
Freud dos anos 1920 e de S. Ferenczi, K. Abraham e M. Klein --- abrem
outras múltiplas perspectivas de diálogo com a antropologia e a
filosofia, também elas sujeitas à crítica de elementos que permaneçam
eventualmente reféns do patriarcado, sobretudo no que diz respeito à
projeção do complexo de Édipo como princípio universal da psiquê humana.
\emph{Quanto a essa projeção}, na década de 1930, na esteira da rejeição
da psicanálise pelos comunistas, Oswald afirmou, em \emph{O rei da
vela}, pela boca de Abelardo \versal{I}, que Freud foi o último grande romancista
da burguesia. E essa tirada é um dos maiores elogios que se poderia
fazer ao pai da psicanálise.

\pagebreak

\section{Bibliografia}

\begin{Parskip}
\versal{ANDRADE}, Oswald. \emph{Obras completas 6. Do Pau"-Brasil à antropofagia
e às utopias}. Rio de Janeiro: Ed. Civilização Brasileira, 1970.

\versal{BEAUVOIR}, Simone de. \emph{Le deuxième sexe}. Paris: Gallimard, 1976.

\versal{BENSUSAN}, Hilan. \emph{Linhas de animismo futuro}. Brasília: \versal{IEB}, 2017.

\versal{BILBAO}, Alejandro. ``Lévi"-Strauss y la actitud frente al mundo como
lectura del acontecer: entre psicanálisis y antropología estructural''. Em:
\versal{BILBAO} et al. \emph{Claude Lévi"-Strauss en el pensamiento
contemporáneo}. Buenos Aires: Colihue, 2009.

\versal{CEPPAS}, Filipe. (2009) ``Aux marges de l'anthropophagie''. Em: \versal{LAGEIRA} et al.
\emph{Modernidade artística: conexões Brasil"-Europa} (no prelo).

\versal{CLASTRES}, Hélène. ``Les beaux"-frères ennemis. A propos du cannibalisme
tupinamba''. In: \emph{Nouvelle Revue de Psychanalyse}, nº~6. Paris:
Gallimard, 1972, p.~71-86.

\versal{CLASTRES}, Pierre. \emph{Arqueologia da violência}. Trad. P. Neves. São
Paulo: CosacNaify, 2014.

\_\_\_\_\_\_\_. \emph{A sociedade contra o estado}. São Paulo: CosacNaif, 2003.

\versal{DELEUZE}, Gilles \& \versal{GUATTARI}, Félix. \emph{Mil platôs. Capitalismo e
esquizofrenia Vol.~4}. São Paulo: ed.34, 2007.

\_\_\_\_\_\_\_ \& \_\_\_\_\_\_\_. \emph{O anti"-édipo}. São Paulo: ed.34, 2010.

\versal{DESCOLA}, Philippe. \emph{Par"-delà nature et culture}. Paris: Gallimard, 2005.

\versal{DETIENNE}, Marcel. \emph{Dionysos mis à mort}. Paris: Gallimard, 1977.

\versal{DETIENNE}, Marcel. \& \versal{VERNANT}, Jean"-Pierre. \emph{La cuisine du sacrifice en
pays grec}. Paris: Gallimard, 1979.

\versal{FREUD}, Sigmund. \emph{Obras completas, volume 11: Totem e tabu, Contribuição à história do movimento psicanalítico e outros textos (1912-1914)}. Trad. P. C. Souza. São Paulo: Companhia das Letras, 2012.

\versal{GAY}, Peter. \emph{Freud. Uma vida para o nosso tempo}. Trad. D.
Bottmann. São Paulo: Companhia das Letras, 1989.

\versal{GOETTNER"-ABENDROTH}, Heide. \emph{Matriarchal societies. Studies on
indigenous cultures across the globe}. New York: Peter Lang, 2012.

\_\_\_\_\_\_\_. \emph{Societies of peace.
Matriarchies, past, presence and future}. Toronto: Ianna Publications
and Education Inc, 2009, p.~55-69.

\_\_\_\_\_\_\_. ``Matriarchies as societies of peace:
re"-thinking matriarchy''. In: \emph{Off our backs}, Vol.~38, nº~1, 2008.

\versal{GREEN}, André. ``Le cannibalisme: réalité ou fantasme agi?''. In:
\emph{Nouvelle Revue de Pyschanalyse}, nº~6, p.~27-54, 1972.

\versal{GUILLE"-ESCURET}, Georg. \emph{Sociologie comparée du cannibalisme.
Vol.~2 La consommation d'autrui en Asie et en Océanie}. Paris, \versal{PUF}, 2012.

\versal{HEUSCH}, Luc de. \emph{Le sacrifice dans les religions africaines}. Paris: Gallimard, 1986.

\versal{KILANI}, Mondher. \emph{Du goût de l'autre. Fragments d'un discours
cannibale}. Paris: Éd. du Seuil, 2018.

\versal{KNIGHT}, Chris. \emph{Blood relations. Menstruation and the origins of
culture}. New Haven/London: Yale University Press, 1991.

\versal{LÉVI"-STRAUSS}, Claude. \emph{Antropologie Estruturale}. Paris: Plon, 1958.

\_\_\_\_\_\_\_. \emph{As estruturas elementares do parentesco}. Petrópolis: Vozes, 1982.

\_\_\_\_\_\_\_. \emph{Tristes Tropiques}. Paris: Gallimard, Bibliothèque de la Pléiade, 2008, p.~463-464.

\_\_\_\_\_\_\_. \emph{A oleira ciumenta}. Trad. B. Perrone"-Moisés. São Paulo: Brasiliense, 1986.

\_\_\_\_\_\_\_. \emph{Nous sommes tous de cannibales}. Paris: Éditions du Seuil, 2013.

\versal{LYONS}, Andrew P.; \versal{LYONS}, Harriet D. \emph{Irregular connections. A history of
anthropology and sexuality}. Lincoln: University of Nebraska Press, 2004.

\versal{MALAMOUD}, Charles. \emph{Cuire le monde. Rite et pensée dans l'Inde
ancienne}. Paris: Éd la découverte, 1989.

\versal{MELATTI}, Julio Cesar. \emph{Índios do Brasil}. São Paulo: Hucitec, 1994.

\versal{MÉTRAUX}, Alfred. \emph{Écrits d'Amazonie. Cosmologies, rituels, guerre
et chamanisme}. Paris: \versal{CNRS} éditions, 2013.

\versal{MONTANARI}, Angelica A. \emph{Cannibales. Histoire de l'anthropophagie en
Occident}. Paris: Arkhê, 2018.

\versal{RUBIN}, Gayle. ``Traffic in Women: Notes toward a Political Economy of
Sex''. In: \versal{REITER}, R. Rapp (ed.). \emph{Toward an Anthropology of Women}.
New York: Monthly Review Press, 1975, p.~157-210.

\versal{SEEGER}, Anthony; \versal{DA MATTA}, Roberto \& \versal{VIVEIROS DE CASTRO}, Eduardo. ``A construção da pessoa nas sociedades indígenas''. Em: \emph{Boletim do Museu Nacional.
Antropologia}, nº~32, 1979.

\versal{SPIRO}, Melford. \emph{Oedipus in the Trobiands}. Chicago: Univ. of
Chicago Press, 1982.

\versal{STRATHERN}, Marilyn. \emph{O efeito etnográfico}. São Paulo: CosacNaif, 2014, p. 109-133.

\versal{VAUGHAN}, Genevieve. \emph{Women and the gift economy. A radically
different worldview is possible}. Toronto: Inanna Publications, 2007.

\versal{VILLENEUVE}, Roland. \emph{Histoire du cannibalisme. De l'anthropologie
rituelle au sadisme sexuel}. Paris: Camion Noir, 2016.

\versal{VIVEIROS DE CASTRO}, Eduardo. \emph{Araweté: os deuses canibais}. Rio de
Janeiro: Jorge Zahar/Anpocs, 1986.

\_\_\_\_\_\_. \emph{Princípios e parâmetros: um comentário a L'Exercice de la parenté}. Rio de Janeiro: Comunicação do \versal{PPGAS}, 17, 1990.

\_\_\_\_\_\_. \emph{A inconstância da alma selvagem}. São Paulo: CosacNaif, 2002.

\_\_\_\_\_\_. \emph{Metafísicas canibais: elementos para uma antropologia pós"-estrutural}. São Paulo: Coasc Naify, 2015.
\end{Parskip}

\chapter*{Escrever: mulheres, ficção e psicanálise}
\addcontentsline{toc}{chapter}{Escrever: mulheres, ficção e psicanálise, \footnotesize\emph{por Julia F. Vasconcelos, Manuela B. Crissiuma, Mariana F. Angelini e Renata de Lima Conde}}
\hedramarkboth{Escrever: mulheres, ficção e psicanálise}{}

\begin{flushright}
\emph{Julia Fatio Vasconcelos}

\emph{Manuela Borghi Crissiuma}

\emph{Mariana Facanali Angelini}

\emph{Renata de Lima Conde}\footnote{Psicanalistas e integrantes do Grupo de Estudos e
  Trabalho em Psicanálise e Feminismo.}
\end{flushright}

``Quem era eu, então?''. Quase no final de sua vida, Virginia Woolf faz
essa pergunta a si mesma. Responde"-a dizendo que era Adeline Virginia
Stephan, a segunda filha de Leslie e Julia Prinsep Stephan, nascida em
25 de janeiro de 1882; em um mundo afeito a se comunicar, a escrever
cartas, a fazer visitas, a se expressar bem --- o mundo do final do século
\versal{XIX}.

Escritora e ensaísta britânica, Woolf, com sua escrita original, deixou
uma vasta obra literária. São notáveis livros como \emph{Mrs.
Dalloway} e \emph{Orlando}. Começou a escrever aos nove anos de idade um
jornalzinho da família e continuou se dedicando à escrita por toda a
vida. Foi autora de nove romances, diversos contos e uma quantidade
imensa de resenhas e ensaios críticos. Em 1918, fundou a editora Hogarth
Press junto com seu marido Leonard Woolf, por onde publicaram livros do
casal, de outros importantes autores e também a tradução britânica da
obra de Freud. Em 1910, Woolf se dedicou à campanha pelo voto feminino
e, durante boa parte da vida, à defesa da emancipação feminina,
registrando suas ideias em seus livros e ensaios.

Em 1920, Virginia Woolf foi convidada a palestrar sobre ``mulheres e
ficção'' em duas faculdades inglesas exclusivas para mulheres. Mais
tarde, em 1931, essas falas se tornariam o ensaio ficcional \emph{Um
teto todo seu}. Sendo ela escritora e ficção um tema literário, Woolf comenta que, no primeiro momento, mulheres e ficção soou como um convite para que ela falasse sobre seu ofício. A questão é que ``mulheres e ficção'' poderia
ter outros sentidos. Para ela, poderia ser também \emph{as
mulheres e a ficção que é escrita sobre elas, as mulheres e como elas
são, ou as mulheres e a ficção que elas escrevem}. Ou, poderia ainda
significar que essas três possibilidades estão inextricavelmente
emaranhadas. Virginia propõe, então, tratar do enlace dessas três. Para
isso, cria uma personagem, Mary Seton, que se confunde com a própria
autora, e que poderia ser também tantas outras mulheres, como ela mesma
diz ``\emph{chamem"-me Mary Beton, Mary Seton, Mary Carmichael, ou
qualquer outro nome que lhes agrade --- pouco importa}'' (\textsc{woolf}, 2014,
p.~13). Com esse enlace de personagens cria, então, uma atmosfera de
reflexão na qual realidade e ficção se confundem constantemente.

Para Woolf (ou Mary Beton), não se pode esperar a \emph{verdade} sobre
``mulheres e ficção''. No entanto, ela diz ser possível mostrar como se
chega a ter a opinião que se tem. Ao se tratar do tema das mulheres, a
autora destaca que esse assunto desperta todos os tipos de preconceitos
e paixões. E que esses estão presentes numa imensidão de livros escritos
por homens e presentes nas mais importantes bibliotecas.

\begin{quote}
Vocês têm noção de quantos livros sobre mulheres são escritos no
decorrer de um ano? Vocês têm noção de quantos são escritos por homens?
Têm ciência de que vocês, {[}mulheres{]}, são o animal mais debatido do
universo? (\textsc{woolf}, 2014, p.~43-44).
\end{quote}

Após uma longa pesquisa na biblioteca da universidade onde dará sua
palestra, Mary Seton fica impressionada com o conteúdo e a diversidade
de ficções que encontra nesses livros. Goethe honrava as mulheres;
Mussolini as desprezava. Alguns achavam que elas tinham um cérebro mais
superficial, outros que sua consciência era mais profunda. Havia os que
defendiam que as mulheres não tinham caráter; e ainda aqueles que
discutiam se elas tinham ou não alma. Muitos tomavam como certo que as
mulheres eram incapazes de aprender, e havia aqueles que achavam que
elas eram inferiores aos homens. Era impossível encontrar uma verdade.

No final do dia, depois de uma longa jornada, sente"-se com raiva do
que leu. Passa os olhos sobre o jornal da tarde e pensa: os homens
parecem controlar tudo, menos o clima. E ainda assim, quando escrevem
sobre as mulheres, parecem estar com raiva. Ela sabia disso,
pois quando os lia não podia acompanhar a linha argumentativa. Quando se
argumenta algo sem as paixões e preconceitos, é mais possível acompanhar
as ideias. Caso contrário, pensamos em quem está escrevendo. E, para
ela, esse era o caso:

\begin{quote}
Quando um argumentador argumenta sem paixão, ele pensa somente no
argumento, e o leitor não pode deixar de pensar no argumento também. Se
ele tivesse escrito sem paixão sobre as mulheres, se ele tivesse usado
provas indiscutíveis para construir seu argumento e não tivesse mostrado
indícios de que o resultado deveria ser um em vez do outro, também não
ficaríamos com raiva. (\ldots{}) Mas eu tinha ficado com raiva porque ele
estava com raiva. Ainda assim parecia absurdo, pensei, folheando o
jornal vespertino, que um homem com todo esse poder estivesse com
raiva (\textsc{woolf}, 2014, p.~52).
\end{quote}

Cansada de seu dia, considera que dificilmente alguém que tivesse acesso
a esses textos e jornais não notaria que se está vivendo sob as regras
do patriarcado. Fica intrigada ainda, então, com os motivos da raiva dos
homens quando falam das mulheres. Isso se destaca de modo particular
para ela em alguns textos que insistem de modo enfático na inferioridade
das mulheres. Avança então em sua reflexão e conclui ser possível que,
quando fizeram isso, estivessem não exatamente preocupados com a
inferioridade feminina, mas com a própria superioridade masculina. Era
isso que eles estavam tentando controlar e proteger ``de maneira um
tanto destemperada e com tanta ênfase, porque era para eles uma joia do
mais raro valor'' (\textsc{woolf}, 2014, p.~52).

\begin{quote}
Por isso, a enorme importância para o patriarcado de ter de
conquistar, ter de governar, de achar que um grande número de pessoas,
metade da raça humana, na verdade, é por natureza inferior. {[}Essa{]}
deve ser realmente uma das principais fontes de seu poder (2014, p.~53-54).
\end{quote}

Para ela, pensar as ficções que se criam sobre as mulheres passa, então
necessariamente, pelo entendimento dessas relações de poder que estão no
interior do patriarcado. Para destacar isso, Virginia recupera algumas
histórias e ao mesmo tempo inventa outras. Demonstra, com essa
estratégia, que, quando inventamos ou vivemos, reproduzimos uma ideia de
desigualdade na qual se impõem dificuldades e restrições para que as
mulheres adquiram os mesmos direitos que os homens e possam compartilhar
de espaços iguais. Pensar a estrutura das ficções que fixam essas
desigualdades é, portanto, fundamental. Se cada época inventa a sua, por
que continuamos lançando mão delas? Poderíamos falar de mulheres sem
alguma ficção? Mas, quais as consequências das ficções que se inventam e
por que algumas persistem tanto?

Nesse sentido, Virginia Woolf destaca a importância da entrada da mulher
na literatura: até Jane Austen, todas as grandes mulheres das obras
literárias tinham sido retratadas por homens. E, com isso, muita coisa
foi deixada de fora, pois a maioria das mulheres eram descritas não
somente por homens, mas apenas na sua relação com um homem. O amor era,
assim, o intérprete predominante para falar sobre a mulher.

\begin{quote}
Por isso, talvez, a natureza peculiar das mulheres na ficção, os
extremos impressionantes de beleza e horror, a alternância entre bondade
celestial e depravação demoníaca --- porque assim as enxergaria um
amante, conforme seu amor aumentasse ou diminuísse, é próspero ou
infeliz (\textsc{woolf}, 2014, p.~119-120).
\end{quote}

Da beleza e horror, da bondade e depravação; cada época, e cada um,
inventa, portanto, a sua maneira de dizer o que é uma mulher. Investigar
a estrutura desses discursos é fundamental para entender de que maneira
eles são tecidos, por que algumas significações insistem e quais são os
seus efeitos. As teorias feministas há tempos se debruçam sobre essa
questão. Gayle Rubin, por exemplo, em \emph{Tráfico de mulheres} (1975),
propõe uma longa reflexão através de uma análise dos trabalhos de Marx,
Lévi"-Strauss e Freud sobre a questão da gênese da opressão e
subordinação social das mulheres. Simone de Beauvoir, em \emph{O segundo
sexo} (1949), faz uma pesquisa inédita sobre os diversos
\emph{discursos}, ao longo da história, acerca das mulheres em
contraposição ao que ela descreve como a \emph{experiência} de ser
mulher.

Nesse sentido, conquistar a emancipação da mulher e, por consequência,
produzir também aberturas para os homens, é destrinchar a estrutura dos
discursos que pretendem encerrá"-los em algumas significações que são
tomadas como naturais.

Para isso, pretendemos abordar nesse texto as contribuições da
psicanálise e, ao mesmo tempo, implicá"-la nos efeitos de seu próprio
discurso. Se ela tem um caráter subversivo, isso não implica
necessariamente que sempre terá. Por um lado, são conhecidas as críticas
à teoria psicanalítica que contribui para a formulação de novas ficções
que alienam a mulher em discursos movidos por paixões e preconceitos;
por outro, ela é a possibilidade de subverter lógicas de dominação e
alienação, possibilitando que as mulheres se posicionem de outras formas
no discurso.

Aqui lembramos as seguintes palavras de Virginia: ``de qualquer forma,
quando o assunto é controverso --- e qualquer questão que envolve sexo é
--- não se pode esperar a \emph{verdade}\ldots{} Só se pode dar ao público a
oportunidade de tirar as próprias conclusões ao observar as limitações,
os preconceitos, as idiossincrasias do palestrante. É mais provável que
a ficção contenha mais verdade que o fato'' (2014, p.~13).

\asterisc

Com o tema ``mulheres e ficção'' acompanhamos, então, Lacan em
\emph{Diretrizes para um congresso sobre a sexualidade feminina}, de
1960. Nesse texto, encontra"-se uma afirmação bastante precisa: ``as
imagens e símbolos \emph{na mulher} não podem ser isolados das imagens e
símbolos \emph{da mulher}'' (\textsc{lacan}, 1995, p.~737). Ou seja, o que
se diz sobre as mulheres, os discursos e representações sobre a mulher,
recaem e têm efeitos na própria experiência de ser mulher. Anos depois,
ao retomar esse texto de Lacan, Colette Soler (2006) escreve que, quando
as feministas denunciam a coerção original que as imagens e símbolos de
uma cultura exercem sobre as mulheres, elas não estão erradas. Para ela,
a partir disso, é necessário entender que a mulher é uma invenção
\emph{histórica} da cultura, e que, portanto, muda de feição conforme as
épocas.

Dez anos antes de Lacan fazer essa afirmação, Simone de Beauvoir já
havia publicado o livro \emph{O segundo sexo}. A partir de uma extensa
pesquisa, Beauvoir divide sua obra em dois volumes. No primeiro,
intitulado \emph{Fatos e Mitos}, a filósofa retoma os discursos criados
sobre as mulheres através da história, mitos e produções teóricas. No
segundo volume, \emph{Experiência Vivida}, Beauvoir discorre sobre a
experiência de ser mulher a partir também dos efeitos desses discursos,
abordando desde a infância até a vida adulta. De sua obra, entre outras
conclusões, poderíamos chegar também à de Lacan: ``as imagens e símbolos
na mulher não podem ser isolados das imagens e símbolos da mulher.''

Com esse livro, de amplo valor para as lutas feministas, Simone de
Beauvoir desvela as construções sociais sobre as mulheres para
desnaturalizar esses discursos. Mais ainda, ela escreve sobre a
experiência de ser mulher: que não se reduz às construções sociais, mas
se realiza também através de seus efeitos. Para Beauvoir, os discursos
sociais reforçariam as dificuldades para as mulheres adentrarem os
espaços públicos. O que implicaria em maiores barreiras para conseguirem
experimentar as possibilidades de sua existência para além de uma
vivência alienada em pré"-determinações, contrapondo"-se ao que ela chama
de transcendência. Para a autora, as mulheres estariam situadas como
Outro, ou seja, como objeto e não como sujeito do discurso. Em relação a
isso, questiona:

\begin{quote}
(\ldots{}) como tudo começou? Compreende"-se que a dualidade dos sexos, como
toda dualidade, tenha sido traduzida por um conflito. (\ldots{}) Resta
explicar como o homem venceu desde o início. (\ldots{}) Por que este mundo
sempre pertenceu aos homens e só hoje as coisas começaram a mudar? Será
um bem essa mudança? Terá ou não uma partilha igual entre homens e
mulheres? (\textsc{beauvoir}, 1970, p.~15).
\end{quote}

Sobre esse tema, a psicanalista Juliet Mitchell escreveu o livro
\emph{Psicanálise e feminismo}, em 1979. Nele a autora afirma que:

\begin{quote}
A longevidade da opressão das mulheres não é trivial nem
historicamente transitória; para se manter de forma tão efetiva, ela
percorre a corrente mental e afetiva. Pensar que isto não deveria ser
assim não implica que já não seja mais assim (\textsc{mitchell}, 1979, p.~381).
\end{quote}

Em concordância com essa leitura, Michelle Perrot, uma das pioneiras do
movimentos de historiadores a propor uma historiografia das mulheres,
ressalta a dificuldade deste trabalho. Para ela, há muitos obstáculos.
Para se escrever uma história, são necessárias fontes, documentos,
vestígios e registros. No caso das mulheres, segundo Perrot, esses dados
foram frequentemente destruídos e apagados, permanecendo poucos
vestígios. Para ela, o silêncio das mulheres foi reiterado através dos
tempos pelas religiões, pelos sistemas políticos e pelos manuais de
comportamento. Como a autora destaca:

\begin{quote}
aceitar, conformar"-se, obedecer, submeter"-se e calar"-se. Este mesmo
silêncio, imposto pela ordem simbólica, não é somente o silêncio da
fala, mas também o da expressão, gestual ou escriturária (\textsc{perrot}, 2005, p. 10).
\end{quote}

Cabe então perguntar, sobretudo: como as mulheres responderam a isso?
Segundo Perrot: de forma mais ou menos silenciosa. Foi necessário ir aos
diários, procurar nos registros fora da história ``oficial'' as marcas e
vestígios deixados pelas mulheres. É justamente nesse contexto cultural
de silenciamento que é fundamental, então, compreendermos o que
significa o surgimento da psicanálise, que se deu no encontro entre
Freud e as pacientes histéricas.

``Fique quieto!'' ``Não diga nada!'' ``Você vê, realmente não durmo, não
sou hipnotizável''. Falas assim, de Anna O., Emmy von N., Lucy R.,
Katharina e Elizabeth von R., entre outras, questionaram o saber médico
de sua época, e sensivelmente, ouvindo o que suas pacientes lhe diziam,
Freud inventou um novo método --- a psicanálise --- na passagem do século
\versal{XIX} para o \versal{XX}, a partir da histeria e do que era dito por cada uma.

Mitchell afirma, em \emph{Loucos e Medusas} (2006), que a palavra
``histeria'', no entanto, é bem anterior à psicanálise. Ela surgiu no
século \versal{V} a.C. e, ao longo da história, significações diversas foram
formuladas a seu respeito, como doença no útero causada por viuvez,
abstinência sexual, loucura, bruxaria, e possessões demoníacas. Os
tratamentos incluíam fumigação de ervas na vagina, exercícios físicos
intensos, casamento e até mesmo fogueira e exorcismo.

Mais tarde, a partir do Renascimento, a histeria passou a ser
compreendida como uma desordem no cérebro. No século \versal{XIX}, com Charcot,
foi considerada doença neurológica. Havia a concepção de que a histeria
poderia ser compreendida e ``domada'' pelo saber médico.

Com a invenção da psicanálise operou"-se, então, uma inversão marcante:
foi proposto o saber no lugar da paciente/analisante, e não mais do
médico/analista. Para Freud, a partir disso, os sintomas de suas
pacientes não eram mais lidos como manifestações a serem domadas, mas
dizeres a serem escutados sobre os conflitos entre os desejos e a moral
sexual de cada época.

No entanto, para Mitchell, apesar dessa inversão, a palavra ``histeria''
ainda carrega o peso dos preconceitos e paixões de sua história e, nesse
sentido, é necessário estarmos advertidos, pois, segundo ela: ``a
histeria às vezes é um diagnóstico, mas às vezes é só um insulto''
(2006, p.~23).

Então, perguntamos: as diversas significações atribuídas à palavra
``histeria'' teriam sido deixadas de lado ou ainda produzem efeitos na
psicanálise?

\asterisc

Para Beauvoir, a teoria freudiana ainda é atravessada por uma história
de dominação masculina. A autora afirma que a psicanálise freudiana toma
o masculino como norma e o feminino como derivação deste, o que podemos
pensar com a pergunta de Freud ``O que quer uma mulher?'', ou mesmo
quando ele compara a mulher a um continente negro.

Sobre isso, Maria Rita Kehl comenta em seu livro \emph{Deslocamentos do
feminino}:

\begin{quote}
cada vez que um psicanalista, depois de Freud, sustentar que existe um
ponto impossível de se desvendar sobre o querer das mulheres, devemos
lhe responder, como Sócrates: ``indaga"-te a ti mesmo''\ldots{} pois só o que
um homem recusa a saber sobre o seu desejo é capaz de produzir o
mistério sobre o objeto ao qual ele se dirige, o desejo de uma mulher
(2008, p.~14).
\end{quote}

Sabemos que os impasses de Freud na produção de sua teoria foram muitos.
Acompanhamos o desenrolar de seu pensamento através de seus textos e,
especialmente, dos casos clínicos que publicou. Num tempo cuja norma era
o silêncio, Freud produziu uma ruptura com a invenção da psicanálise.
Criou um espaço e uma escuta para que as histéricas pudessem fazer falar
o seu sexo, seus conflitos. No entanto, ainda que ele tenha finalizado
sua obra deixando a feminilidade como uma questão em aberto, são
conhecidas as passagens em sua teoria sobre a ``inferioridade orgânica''
da menina, a inveja do pênis, o superego frágil, assim como as três
saídas da sexualidade feminina priorizando a maternidade como saída
normal, o que gerou grandes debates entre feministas e psicanalistas.

Sobre essas teses, há diversas leituras. Algumas feministas consideram
que, com elas, Freud disponibilizou ferramentas importantes para uma
análise do patriarcado, outras defendem que ele reproduziu a lógica
patriarcal ao desenvolver tais ideias. Para Gayle Rubin (2017), há uma
tensão na medida em que a psicanálise traz ferramentas conceituais
fundamentais para descrever a vida social e a opressão que incide sobre
as mulheres, ainda que Freud não tenha reconhecido as implicações nem as
críticas implícitas que sua obra poderia produzir.

Essa ambiguidade é bastante presente nas leituras atuais dos textos
freudianos. O conceito da inveja do pênis, por exemplo, retorna
constantemente para os debates. Por alguns, ele é lido como a
reafirmação de Freud de uma suposta superioridade masculina da qual as
mulheres teriam inveja, para outros, ao contrário, é o reconhecimento da
desigualdade de direitos e da dominação dos homens sobre as mulheres.
Para Rubin, a teoria freudiana mantém essa tensão. Ela pode ser lida
como uma teoria que racionaliza a subordinação das mulheres sem fazer
qualquer crítica, favorecendo os preconceitos e paixões. Por outro lado,
ela produz uma descrição valiosa dos processos de subordinação da mulher
e que são fundamentais para seu combate.

Desta forma, nos deparamos com um paradoxo que, como todo paradoxo, nos
faz trabalhar: a psicanálise que Freud inventou pode ser tomada ao mesmo
tempo como subversão e adequação. Aqui cabe, então, a questão que
Colette Soler (2006) propõe: em que medida a psicanálise escapa ou
participa aos preconceitos relativos ao sexo?

No texto \emph{Diretrizes para um congresso sobre a sexualidade
feminina}, já citado acima, Lacan nos aponta:

\begin{quote}
(\ldots{}) talvez esta seja uma oportunidade de distinguir entre inconsciente
e preconceito, quanto aos efeitos do significante. E de reconhecer, ao
mesmo tempo, que o analista está tão exposto quanto qualquer outro a um
preconceito relativo ao sexo, a despeito do que lhe revela o
inconsciente (1995, p.~740).
\end{quote}

É necessário, portanto, estarmos advertidas. O sujeito histérico, ao
falar de seu desejo, fala de si. Mas, não só e não tudo. Fala também do
que supõe ser a causa do desejo do Outro. Estaríamos, portanto, atentos
a essa dialética do desejo na clínica? É nesse sentido que Colette Soler
afirma que, ao escutar suas pacientes histéricas, Freud foi
instruído sobre a causa do desejo masculino e como elas se localizavam
nele. Segundo a autora, o que se diz da mulher é enunciado do ponto de
vista do Outro e se refere mais à sua aparência do que ao seu próprio
ser --- permanecendo este, portanto, elemento foracluído do discurso.

Virginie Despentes, autora de \emph{Teoria King Kong} (2016), é
precisa ao destacar, nesse sentido, que no campo político por muito
tempo o desejo foi de domínio masculino; e da mulher nada se quis ouvir.
Segundo ela, nessa lógica, aquilo que é verdade para o homem se desloca
para se impor como verdade à mulher --- estigmatizando sua sexualidade.

É a isso também que Colette Soler faz alusão ao trazer a questão do
masoquismo na mulher e indagar se essa seria uma fantasia do desejo do
homem:

\begin{quote}
(\ldots{}) a mulher é definida unicamente pelas vias de sua parceria com o
homem e a questão é saber quais são as condições inconscientes que
permitem a um sujeito consentir nisso ou não (2006, p.~26).
\end{quote}

Mas, trata"-se, então, de ser apenas objeto de desejo do outro na
histeria? Como isso é escutado? O que da histeria faz história? O que da
história retorna na histeria?

Aqui resgatamos um neologismo proposto por Lacan: \emph{hystoria} --- a
conjunção em uma palavra entre história e histeria. Sabemos que os
sintomas histéricos estão sujeitos à sua época, mas pouco reconhecemos o
que da história deve ser atribuído à histeria.

É fundamental, portanto, destacar a histeria como uma das formas de responder e lidar, ao longo dos séculos, com o mal"-estar da feminilidade. Uma das
formas de apontar as ficções criadas sobre a feminilidade, mas também de
buscar inventar outras. Emilce Dio Bleichmar (1988) defende que a
histeria é o sintoma da estrutura profundamente conflitiva da
feminilidade em nossa cultura, testemunhando o caráter desvalorizado de
sua identidade de gênero.

Tendemos a pensar que a histérica se interroga sobre se ela é homem ou
mulher. No entanto, é sobre o poder, a valorização e as formas de
reconhecimento que ela se interroga. Não é à diferença entre os sexos
que ela reage, mas à desigualdade.

Haveria então, segundo Bleichmar, um feminismo espontâneo próprio à
histeria que consiste numa reivindicação, ainda que turbulenta, de uma
feminilidade que não se reduza à sexualidade, a ser objeto do desejo do
Outro. Segundo ela, essa dimensão, o feminismo espontâneo da histeria,
permanece como ponto cego para a cultura, para o teórico, para a mulher
e para o próprio analista.

Ora, mas não é justamente sobre isso que opera uma análise, produzir
tensionamentos, desvelamentos, travessias e (re)invenções na posição do
sujeito e seus pontos cegos? No suposto encontro entre certa moral
sexual e a própria sexuação do sujeito?

Sabemos desde Freud que a sexuação é pulsional e perverso"-polimorfa, o
que implica, necessariamente, um desencontro de cada sujeito com a moral
sexual de cada época. Também sabemos que não é fora disso que os
sujeitos se constituem. Deixamos, então, como questão, para continuarmos
avançando: como trabalhar as tensões entre uma teoria que fura o suposto
encontro entre sujeito sexuado e a moral sexual de sua época, mas ao
mesmo tempo é atravessada por ela?

Para finalizar, lembramos aqui da sagacidade de Virginia Woolf quando
ela diz que uma das maneiras de se pensar a relação entre mulher e
ficção é falar da ficção que escrevem sobre elas\ldots{} Mas também das
ficções que elas mesmas podem escrever.

\section{Bibliografia}

\begin{Parskip}
\versal{BEAUVOIR}, Simone. \emph{O segundo sexo -- fatos e mitos.} Difusão
Européia de Livros: São Paulo, 1970.

\versal{BLEICHMAR}, Emilce Dio. \emph{O feminismo espontâneo da histeria}.
Editora Artes Médicas: Rio Grande do Sul, 1988.

\versal{BREUER}, Josef; \versal{FREUD}, Sigmund. (1893-1895) ``Estudos sobre a histeria''. Em:
\emph{Edição Standard Brasileira das Obras Psicológicas Completas de Sigmund
Freud,~v.~2}. Rio de Janeiro: Imago, 1996.

\versal{DESPENTES}, Virginie. \emph{Teoria King Kong}. n"-1 edições: São Paulo, 2016.

\versal{FREUD}, Sigmund. (1923) ``A organização genital infantil: uma interpolação
na teoria da sexualidade''. Em: \emph{Edição Standard Brasileira das Obras
Psicológicas Completas de Sigmund Freud,~v.~19}. Rio de Janeiro: Imago, 1996.

\_\_\_\_\_\_\_\_\_. (1924) ``A dissolução do complexo de édipo''. Em:
\emph{Edição Standard Brasileira das Obras Psicológicas Completas de Sigmund
Freud,~v~19}. Rio de Janeiro: Imago, 1996.

\_\_\_\_\_\_\_\_\_. (1925) ``Algumas consequências psíquicas da
distinção anatômica entre os sexos''. Em: \emph{Edição Standard Brasileira das
Obras Psicológicas Completas de Sigmund Freud,~v~19}. Rio de Janeiro: Imago, 1996.

\_\_\_\_\_\_\_\_\_. (1932-1933) ``Conferência \versal{XXXIII} Feminilidade''.
Em: \emph{Edição Standard Brasileira das Obras Psicológicas Completas de
Sigmund Freud,~v~22}. Rio de Janeiro: Imago, 1996.

\versal{KEHL}, Maria Rita. \emph{Deslocamentos do Feminino}. Rio de Janeiro: Imago, 2008.

\versal{LACAN}, Jacques. \emph{Escritos}. Rio de Janeiro: Jorge Zahar, 1995.

\versal{MITCHELL}, Juliet. \emph{Loucos e Medusas}. Rio de Janeiro: Editora Civilização
Brasileira, 2006.

\_\_\_\_\_\_\_. \emph{Psicanálise e Feminismo}. Belo Horizonte: Interlivros, 1979.

\versal{PERROT}, Michelle. \emph{As mulheres ou os silêncios da história}. Bauru:
Edusc, 2005.

\versal{RUBIN}, Gayle. \emph{O Tráfico de Mulheres}. São Paulo: Ubu Editora, 2017.

\versal{SOLER}, Colette. \emph{O que Lacan dizia sobre as mulheres}. Rio de Janeiro: Jorge Zahar, 2006.

\versal{WOOLF}, Virginia. \emph{Um teto todo seu}. São Paulo: Tordesilhas, 2014.
\end{Parskip}

\chapter*{Freud e o conhecimento-do-corpo: viajando pelos limites da
linguagem através da angústia\footnote{Tradução de Alexandre Cleaver e revisão
  de Alessandra Parente.}}
\addcontentsline{toc}{chapter}{Freud e o conhecimento-do-corpo: viajando pelos limites da
linguagem através da angústia, \footnotesize\emph{por Ana Carolina Minozzo}}
\hedramarkboth{Freud e o conhecimento-do-corpo}{}

\begin{flushright}
\emph{Ana Carolina Minozzo}\footnote{\versal{PhD} Researcher no Department of Psychosocial
Studies da Birkbeck, University of London e mestre e graduada em Psychosocial
Studies e Psychoanalytic Psychology, respectivamente, pela mesma
universidade. Sua pesquisa cruza os campos das humanidades médicas, filosofia
continental e teoria psicanalítica em relação aos diagnósticos e experiências de
ansiedade dentro e fora da clínica. Desde 2013 atua como professora de Estudos
Culturais e Teóricos na University of the Arts London além de contribuir com
diversas revistas e periódicos brasileiros e internacionais de artes, moda e
cultura. Também dedica"-se à formação clínica em Psicanálise no Centre for
Freudian Analysis and Research (\versal{CFAR}), em Londres, no Reino Unido, onde é
Trainee Psychoanalyst.}
\end{flushright}

Uma das principais críticas feita por Gilles Deleuze e Félix Guattari ao
projeto da psicanálise em sua influente obra de dois volumes
\emph{Capitalismo e Esquizofrenia}, de 1972 e 1980 respectivamente,
trata das políticas reacionárias da clínica psicanalítica, desde seus
arranjos institucionais até a política de seu pretenso sujeito.
Poderíamos argumentar, talvez, que ela não é uma intervenção direta ou
exclusivamente feminista no modelo psicanalítico e em suas ligações aos
arranjos patriarcais de poder no laço social e na vida psíquica;
\emph{O anti"-Édipo} consegue, porém, desafiar muito claramente um
problema que é relevante ao nosso debate: a questão do ``excesso'' à luz
da ``castração''. Excesso, nesse caso, entendido tanto como aquilo que
excede o significado e também como o que se acumula na forma de
``energia libidinosa'' e o mecanismo de seus destinos de acordo com
Freud, pós"-freudianos e Lacan. O que é descrito como um inconsciente
``molar'' e ``neurótico'', em forma de árvore, é celebremente
contrastado com um outro, ``molecular'' e ``rizomático'' e, portanto,
múltiplo e desprendido de uma versão lapidada pelo simbólico, que seria
assegurada pelo que Lacan chama de a ``função paterna'' do pai castrador
de Édipo.

A resolução do Édipo freudiano foi absorvida pelo estruturalismo
lacaniano. Portanto, ao invés de sua dissolução resultar no Super"-Ego,
Ego e Id, passa"-se a falar de linguagem e dos efeitos do significante no
mítico sujeito pré"-Simbólico --- aquele representado pelo delta no canto
inferior direito do Gráfico Lacaniano do Desejo.\footnote{Cf. \textsc{lacan}, 1960.} Dessa
forma, a entrada no Simbólico tem um efeito estruturante similar a um
convite à neurose. Em caso de falha, ou de qualquer alternativa a
participar da ``civilização e seus mal"-estares'', encontramos a psicose.
Todavia, em relação a esse ponto --- a primazia dessa entrada no
Simbólico para a formação do sujeito lacaniano, do inconsciente e de
suas estruturas relevantes ---, não podemos nunca nos esquecer a postura
claramente não patologizante, e por isso radical em seus próprios
termos, em relação a essas consequências, ou a relação não hierárquica
entre neurose e psicose. No entanto, essa investigação filosófica do
sujeito psicanalítico aqui nos leva à seguinte questão: há qualquer
coisa para além da psicose fora dessa concepção patriarcal das
possibilidades do sujeito?

E aqui, sem perder de vista o potencial radical de Freud (e de Lacan),
conforme Juliet Mitchell expôs de modo pungente e sucinto em
\emph{Psychoanalysis and Feminism} (1974), podemos considerar a
psicanálise simplesmente como uma ``descrição'' e não como uma
``prescrição''. Ela nos ajuda a identificar ``as ferramentas do
mestre'', por meio das quais podemos tentar desmantelar, ou pelo menos
sacudir, a casa do mestre. Assim, o que proponho neste capítulo é que,
sim, há muito para além da psicose, algo que escapa à linguagem, e
através da experiência do afeto da angústia estamos em contato com esse
excesso, que é tanto radical e expansivo quanto paralisante e penoso.
Esse ``excedente'' na angústia marca o ritmo do que Deleuze e Guattari
chamaram de ``devires''. Angústia, como veremos a seguir, é o que
persiste, insiste e se abre a uma possibilidade na experiência
subjetiva, emergindo diretamente do corpo enfrentando os limites da
linguagem.

Para chegarmos a esse entendimento, porém, devemos traçar as diferentes
ideias de Freud em relação à angústia em suas diversas obras, uma
leitura que busca resgatar um certo ``conhecimento"-do"-corpo'' das garras
da linguagem, Édipo e o Simbólico; portanto, do patriarcado e de seus
aliados ideológicos. Tal empreitada passa por recuperar o ``excedente''
afetivo de Espinoza, notavelmente uma influência importante para o
pensamento deleuziano e guattariano, nos primeiros escritos de Freud,
lançando a pergunta ``o que pode fazer um corpo'' a esses textos em
relação à angústia. Essa leitura também ilumina os ensinamentos
derradeiros de Lacan e, portanto, atribui um potencial político à ubíqua
e necessária experiência da angústia. Uma experiência central não apenas
para o desenvolvimento da própria psicanálise, mas também para o
diagnóstico psiquiátrico no século \versal{XX} e para o arranjo psicopolítico do
pós"-capitalismo contemporâneo.

\section{Um retorno ao Freud dos primórdios}

Em junho de 1983, Guattari participou de um colóquio em Cerisy, França,
sobre o trabalho do físico"-químico Ilya Prigogine, em que apresentou um
trabalho intitulado ``Energética Semiótica''. Tal estudo fez parte de
seu livro \emph{Cartografias Esquizoanalíticas}, de 1989, e marca o que
Watson (2011) apontou como sendo seu ``retorno a Freud'' via uma
formulação um tanto críptica da energética. Central ao seu argumento é o
entendimento de que os textos iniciais de Freud davam mais ênfase ao
fator ``energético'' de uma ``energia libidinal'' essencial, que foi
jogada para escanteio em sua segunda topografia. Nesse sentido, o que o
projeto de Freud vislumbrava, escreve Guattari, era ``estabelecer vias
de passagem entre a libido sexual e os efeitos de sentido significado
{[}\ldots{}{]} {[}em{]} sua hipótese inicial de uma energia cujos
efeitos seriam ao mesmo tempo físicos e psíquicos'' (\textsc{guattari} e \textsc{rolnik},
1996, p. 267). Contudo, tais metáforas de energia (que Guattari
encontrou nos textos pré"-psicanalíticos e nas cartas a Fliess) se
perderam no segundo modelo da psique, resultando no que Guattari
diagnosticou como ``o movimento psicanalítico não para mais de submeter
o conceito de energia libidinal aos mais diversos tratamentos, para
tentar dominar o escândalo teórico do qual ele é portador'' (1996, p. 267). Freud, pós"-freudianos e também o estruturalismo
lacaniano assim se comprometeram a ``nada mais, nada menos de sua
{[}energia libidinal{]} liquidação quase total sob forma de cadeia de
significantes'' (1996, p. 267). A ordem Simbólica,
encarregada de construções sociais, subjetivas e epistemológicas do
arranjo patriarcal colonial, dá consistência a uma clínica que é fundada
sobre um excedente --- o inconsciente --- porém articulada por meio do
próprio motor que absorve a ruptura, ou a potência do fluxo de devires
desse próprio excesso.

Todavia, nem tudo estava perdido. Quando o assunto é angústia podemos
perceber a forte presença desse fluxo libidinoso, com mais clareza nos
primeiros textos de Freud, mas ainda favorecendo a posterior
conceptualização da angústia e de sua função vis"-à"-vis à castração e do
``perigo'' --- em que ainda há a presunção de um ``caos'' para além da
linguagem. Essa investigação talvez tenha tornado Guattari um pouco
menos cético em relação ao projeto psicanalítico. Nas cartas de Freud a
Fliess escritas no século \versal{XIX}, começando por aquelas conhecidas como
rascunhos \versal{A} e \versal{B}, já podemos encontrar suas primeiras teorias sobre
angústia, especificamente as diferenças entre o afeto ``regular'' da
angústia e o caso da neurose de angústia. No rascunho \versal{A1} (1892),
sexualidade e repressão formam suas hipóteses. Libido e um limite
psíquico e corporal a essa ``energia'' estão em jogo quando se trata de
angústia. No rascunho \versal{B}, do ano de 1893, trabalhando em uma etiologia
para a neurose, um \emph{estado crônico} e um \emph{ataque de angústia}
são mencionados como duas manifestações diferentes de angústia, que
podem se combinar como sintomas que giram em torno do corpo (i.e.
hipocondria, agorafobia, etc.) e da \emph{neurose sexual} (\textsc{freud},
Rascunho \versal{B}, 1986, p. 39). Com isso, Freud está se referindo a
eventos ou circunstâncias que interrompem algum tipo de fluxo
``natural'' da satisfação sexual, ao fato de ele não poder ser
convertido em ideias (ou em um significante, como no caso da
hipocondria, por exemplo) ou ainda ao fato de a angústia ligar"-se a
derivativos somáticos. Será, porém, apenas um par de anos depois que
Freud irá elaborar em mais detalhes a equação da neurose, sexualidade,
repressão e angústia.

Em uma breve carta de 1894 intitulada Rascunho \versal{E}, Freud conduz seu amigo
através de seu raciocínio explorando especificamente a neurose de
angústia, que ele, primeiramente e durante as próximas décadas,
compreende como ligada à sexualidade, ou a esse ``acúmulo libidinoso''
que busca ser descarregado. Ele escreve: ``logo se tornou claro para mim
que a angústia de meus pacientes neuróticos tinha muito a ver com a
sexualidade; e, em particular, impressionei"-me com a precisão com que o
coito interrompido, praticado na mulher, leva à neurose de angústia''
(\textsc{freud}, 1985, Rascunho \versal{E}, p. 78). \emph{Coitus interruptus}, que
era uma prática comum à época, mais de meio século antes do surgimento
da pílula contraceptiva, causava uma angústia particular, tanto em
homens quanto em mulheres. No entanto, essa primeira observação logo
pediu uma revisão, já que ele notou que a angústia surgia mesmo em casos
em que não havia preocupação com uma possível gravidez. Um outro fator
emerge em suas observações iniciais que terá uma certa importância em
suas teorias sobre angústia, que é a de sua conexão com o corpo físico,
nesse ponto ainda ligado somente à satisfação sexual, ou descarga
``libidinosa''.

Freud acompanhou uma variedade de casos em que sexualidade e angústia
estavam conectados, coincidindo com ``uma questão de acumulação ou de
excitações físicas'' (\textsc{freud}, Rascunho \versal{E}, 1985, p. 79) que leva à
angústia via um ``desvio'' de tal acumulação e de seu descarregamento,
no qual a tensão acumulada se ``transforma'' em angústia. Dessa forma,
há desde o princípio do seu entendimento um caminho se formando por meio
de uma tensão física em excesso, um excedente, que é deixado
insatisfeito via, e isto é muito importante, uma relação com um
``outro'' e seu contentamento; também fisicamente, mas por motivos que
poderiam ou não ser físicos, e então acumulado e psiquicamente
transformado em algo diferente --- essa outra coisa seria a manifestação
dos sintomas de ansiedade.

A essa altura dos escritos de Freud, angústia é claramente o recurso
subjetivo para lidar com uma fonte interna de tensão que se encontra no
corpo, o fluxo energético/libidinoso --- energia sexual, fome, sede --- a
diferença sendo que apenas coisas bem ``específicas'' podiam matar e
satisfazer essas necessidades, prevenindo que voltassem a ocorrer nos
``órgãos envolvidos'' para cada necessidade. Muito antes de suas
formulações sobre pulsões e zonas erógenas, ao traçar essa rota para a
angústia, Freud ofereceu uma teoria interessante sobre o par
psique"-soma. Esse modelo conecta o corpo à psique por meio de um tipo de
``limiar'' que, quando alcançado, é capaz de empregar esse fluxo
libidinoso psiquicamente, iniciando, em suas palavras, uma ``relação com
certos grupos de ideias, que então se põem a produzir as soluções
especificas'' (\textsc{freud}, Rascunho \versal{E}, 1985, p. 80). A angústia surge
através ``da acumulação da tensão física e da prevenção do
descarregamento na direção psíquica'' (\textsc{freud}, 1894/ 1985, p. 82). Esse
modelo ``psicofísico'', como ele o chama nesse momento, assemelha"-se a
um tipo de \emph{conversão} em neurose de angústia comparável ao que
ocorre na histeria. Nessa etapa, a articulação de Freud entre angústia e
histeria sugere que na histeria uma excitação \emph{psíquica} toma o
caminho errado, adentrando exclusivamente o campo somático, enquanto na
neurose de angústia é uma \emph{tensão} física que ``não pode ser
transformada em afeto pela elaboração psíquica'' (\textsc{freud}, Rascunho \versal{E},
1985, p. 82). O que vemos é uma relação dinâmica entre o ``fluxo
libidinoso'' e ``representativos'', ou ideias, no modelo psico"-soma,
que Freud introduz nesses primeiros pensamentos. Uma ``conversão''
ocorre quando o excesso não consegue encontrar um terreno suficiente ou
adequado na estrutura que lhe envolve. No caso da histeria,
especificamente, há uma tradição estabelecida do pensamento feminista
que identifica esse modo de conversão sob a ótica da histeria e dos
sintomas da histeria como uma forma de protesto social contra o arranjo
patriarcal (de \textsc{cixous}, 1976 e \textsc{mitchell}, 2000 a \textsc{webster}, 2018 entre
muitos outros). Em linguagem lacaniana, seria um caso relacionado aos
limites explícitos do Simbólico em sua relação com o Real, tanto na
histeria quanto na angústia. Nesta última, ocorre uma ``conversão'' que
move o Real do corpo que não encontra lugar na experiência. Em outras
palavras, o corpo fenomenológico do sujeito na cultura como
experimentador de ressonâncias de um fluxo energético caótico e
excessivo é evidente nas descrições iniciais de Freud sobre a angústia.

Muito do conteúdo da carta mencionada acima deu origem a um artigo
expandido publicado mais tarde naquele mesmo ano, intitulado ``Sobre os
fundamentos para destacar da neurastenia uma síndrome específica
denominada `neurose de angústia'" (1894). O que ele adiciona nesse
artigo é que, em muitos casos de neurose de angústia, o desejo sexual
também diminui. A neurose de angústia, ele escreve, ``é o produto de
todos os fatores que previnem a excitação sexual somática de ser
trabalhada psiquicamente. As manifestações da neurose de angústia
aparecem quando a excitação somática que foi desviada da psique é gasta
subcorticalmente em reações totalmente inadequadas'' (\textsc{freud}, 1894, p.
109).\footnote{Tendo em vista que as \emph{Primeiras Publicações
  Psicanalíticas} de Freud ainda não tiveram, no Brasil, uma tradução
  feita diretamente do alemão, adotamos a tradução direta do texto de
  Ana Carolina Minozzo feita por Alexandre Cleaver que foi cotejada com
  o original alemão: ``{[}\ldots{}{]} zur Angstneurose aber führen alle
  Momente, welche die psychische Verarbeitung der somatischen
  Sexualerregung verhindern. Die Erscheinungen der Angstneurose kommen
  zustande, indem die von der Psyche abgelenkte somatische
  Sexualerregung sich subkortikal, in ganz und gar nicht adäquaten
  Reaktionen ausgibt'' (Cf. \textless{}\emph{https://bit.ly/2mihv3k}\textgreater{}).
  Podendo ser traduzido, por Alessandra Parente, da seguinte forma:
  ``chega"-se à neurose de angústia, entretanto, em todos os momentos nos
  quais a elaboração psíquica da excitação sexual somática está
  impedida. A aparição da neurose de angústia ocorre quando a excitação
  somática, desvia"-se subcoticalmente da psique, sendo gasta em reações
  inteiramente inadequadas.''} Ao parear, mais uma vez, os sintomas da
angústia e os aspectos físicos de interações sexuais, Freud aponta
precisamente para a função de um afeto ``regular'' de angústia, que
opera como uma ``proteção'' contra algo externo que não pode ser gerido
adequadamente, em oposição a uma forma mais ``problemática'' ou
paralisante da neurose de angústia.

Na parte final de suas \emph{Conferências Introdutórias}, Freud
oferece um relato atualizado e objetivo sobre a angústia. A angústia
realista está ``vinculada a um processo de fuga, e é lícito considerá"-la
manifestação do instinto de autoconservação'' (1917, p. 521),
assim explicando uma manifestação um tanto consciente de angústia do
corpo. Freud aponta que é o preparo para o perigo que aumenta o nível de
atenção e capacidade motora de alguém, preparando"-o para entrar em ação.
A angústia como ``sinal'' está dividida em dois momentos e ``adequado
naquilo a que chamamos angústia parece"-me ser a prontidão, e inadequado
o seu desenvolvimento'' (\textsc{freud}, 1917, p. 522). Nossa percepção dessas
manifestações de angústia, em lisura ou estriamento, leva Freud a tentar
desvendar suas agora complicadas respostas à questão central da
palestra: o que é exatamente a angústia?

E sua resposta é que a angústia é um afeto. Um afeto, por sua vez, na
visão de Freud, é um conceito complexo que ``compreende, em
primeiro lugar, determinadas inervações motoras ou descargas; em
segundo, certas sensações de dois tipos distintos: as percepções das
ações motoras ocorridas e as sensações diretas de prazer e desprazer que
dão o tom, como se diz, ao afeto'' (1917, p. 523). Afetos, a
partir desse ponto de vista, estão relacionados ao corpo e à psique,
aproximando"-se das percepções e sentimentos. O trabalho anterior de
Freud sobre a neurose de angústia forma a base para suas explicações
acerca de outro tipo de neurose de angústia, que ``nos coloca diante de
um mistério; nesse caso, perdemos de vista por completo a conexão entre
a angústia e o perigo ameaçador'' (1917, p. 430). Essa falta de
correlação com algum perigo conduz Freud a algumas hipóteses, ao tentar
conectar a angústia realista e a neurose de angústia; poderia haver algo
que de fato ``assusta'' o paciente no cerne da neurose de angústia? Esse
aspecto crucial, a ser desenvolvido mais tarde em seu texto de 1926
\emph{Inibições, sintomas e angústia}, encontra sua primeira explicação
aqui, resgatando as ideias anteriores de descarga da libido sexual. Sem
se distanciar muito drasticamente de seus textos anteriores, Freud
defende que ``não é difícil constatar que a angústia expectante ou
ansiedade {[}\emph{Ängstlichkeit}{]} geral tem estreita vinculação com
determinados processos da vida sexual, com certos empregos da libido,
digamos'' (1917, p. 531). E esse seria o caso nos mais variados
contextos; mesmo quando a sexualidade está ligada a diferenças
culturais, ele declara que ``por mais que essas relações sejam alteradas
e complicadas por influências culturais diversas, permanece válido para
a média das pessoas o vínculo existente entre angústia e restrição
sexual'' (1917, p. 533). No final, essas observações o levaram a
concluir que ``são duas as impressões que se adquire de todos esses
fatos: em primeiro lugar, a de que se trata de uma acumulação da libido
impedida de ter seu emprego normal; em segundo, a de que nisso nos
encontramos no terreno dos processos somáticos'' (1917, p. 533).

Embora as neuroses de angústia e realista como diferentes
``categorias'', propostas por Freud nesse texto, possam ter origens
diferentes, sendo a primeira relacionada à ``libido empregada de maneira
anormal'' e a última ``uma reação ao perigo'', no modo como tais
angústias são sentidas não há distinção, pois o que é ``real'' ou
``perigoso'' são categorias complexas ao se tratar do
inconsciente\ldots{} essa pergunta aberta é retomada nas décadas
seguintes, quando Freud trabalha com o conceito de ``angústia de
castração''. É necessário também introduzir outro fator que Freud
adiciona: o das oposições entre ego e libido. O ego está sendo
confrontado por um ``chamado'' libidinoso interno, e começa a surgir
como garantidor de certa estabilidade psíquica ao final da conferência
de Freud, que termina com o debate entre angústia e repressão. Ele
questiona ``o que se passa com o afeto vinculado à ideia reprimida''
(\textsc{freud}, 1917, p. 542) e responde que ``o destino imediato desse afeto é
ser transformado em angústia, qualquer que seja a qualidade que ele
mostre em sua evolução normal'' (1917, p. 542). Tal ``descarga''
em angústia daquilo que estava reprimido também segue uma rota
particular nas fobias, um pouco diferente daquela que acompanha os casos
de outras neuroses. Esse ``restante'' aparece descrito no caso mais
famoso de fobia diagnosticado por Freud, o do Pequeno Hans, publicado em
1909, que delineia que parte do excesso do ``fluxo libidinoso'' que não
foi capturado pela conversão em angústia não será desviado ao objeto
mesmo em casos de fobia. Nesse sentido, a angústia surge claramente para
nós como um ``excedente'' --- ou um excesso daquilo que Freud nomeou
``libido'', que não acha e não consegue achar espaço total e completo
para ser satisfeito ou canalizado no corpo (com sexo, comendo, bebendo,
ou outros aspectos do círculo de necessidade e desejo, o que mais tarde
será nomeado ``pulsão'') nem em representações, ou palavras e
Simbolização.

Ao invés de debater as possíveis ressonâncias que conectam a angústia e
a ideia de castração que irá marcar a teoria final de Freud para a
angústia, e também oferecer um ponto de partida para \emph{O seminário livro 10} de
Lacan sobre o tema, devemos, antes disso, desembaralhar suas ideias em
relação ao tópico do ``excesso'', lido através de suas formulações sobre
angústia. O excedente, como essa energia libidinosa excessiva que tanto
Freud quanto Guattari reconhecem estar no cerne de seus modelos
ontológicos, é o que está acumulado, descarregado, convertido ou
transformado, nos escritos de Freud sobre angústia que datam do final
dos anos 1890 até o início da década de 1920. Para responder por que
essa leitura do ``excedente'' de energia libidinosa em jogo no tema da
angústia é ``melhor'' ou mais interessante como desafio feminista às
vigas estruturais patriarcais da psicanálise, devemos viajar de volta à
ontologia de Espinoza, tão influente para Deleuze e Guattari.

\section{Um excesso afirmativo}

Os princípios de sua esquizoanálise e a conceptualização do desejo como
produção romperam com o foco no ``indivíduo'', favorecendo uma
``economia coletiva, de agenciamentos coletivos de desejo e de
subjetividade que, em algumas circunstâncias, alguns contextos sociais,
podem se individualizar'' (\textsc{guattari} e \textsc{rolnik}, 1996, p. 232). Essa visão
intrinsicamente política do desejo, do inconsciente e da subjetividade
foi frutuosa às pensadoras feministas. Acadêmicas feministas têm
mergulhado no trabalho de Deleuze e Guattari e em seu modelo
esquizoanalítico para desafiar a ideia psicanalítica de que a linguagem,
ou o Simbólico, seja estruturante. Bracha Ettinger (2006), por exemplo,
apresenta uma matriz afirmativa, ou generativa, para a variação
subjetiva em sua ``metramorphosis'', presente em suas
pinturas e prática clínica. Elizabeth Grosz e Rosi Braidotti, por outro
lado, dão corpo às bases filosóficas para uma compreensão afirmativa do
desejo em debates sobre os conceitos ontológicos, éticos e políticos que
permeiam a subjetividade, a materialidade, os discursos biológicos
``científicos'' e a tecnológica.\footnote{Cf. \textsc{grosz}, 2008, 2017; \textsc{braidotti},
2017.}

O que aqui proponho, como um pequeno gesto neste debate, porém, é mapear
as possibilidades que o monismo de Espinoza oferece e, ao mesmo tempo,
uma possível conexão transindividual presente em seu \emph{Ética},
publicada postumamente e pela primeira vez em 1677. Ou até mesmo um elo
político ou coletivo no entendimento dos afetos, sintomas e formação
subjetiva que está presente na ontologia de Espinoza --- conforme
discutido, por exemplo, por Chiara Bottici (2017) em relação ao
anarcofeminismo. Um ziguezague entre sujeito, afeto e condições de
subjetividade, estruturados pela ideologia presente no laço social, está
presente em leituras psicossociais ou críticas da psicanálise, nas quais
o entendimento de uma ``verdade subjetiva'' ocorre na formação do
sintoma e de sua singular função. Entretanto, para além de um foco
estrutural no significado e no desvelamento do sintoma, as bases
``energéticas'' para uma fonte corporal da angústia, delineadas por um
incipiente Freud, enxergam a angústia como um afeto do excedente: ela
emerge quando algo na experiência material do corpo, ou no reino das
``ideias'', limita o fluxo de energia libidinosa que caracteriza a vida
do corpo (sob uma ótica bergsoniana de uma ``vida'' ser uma tendência
que ```desembrulha' aquilo que está embrulhado na matéria'' {[}\textsc{grosz},
2007, p. 295{]}).

A concepção de Espinoza da natureza, da existência humana e da mente
está detalhada em seu \emph{Ética}, em que sua visão acerca de uma
substância infinita (que ele chama de Deus) que está constantemente se
modificando e que possui diferentes atributos abre caminho para um
debate sobre as possibilidades e o fluxo da dita substância e as
diferenças dos tais atributos. Na Parte \versal{I}, Proposição \versal{V}, Espinoza afirma
que ``não pode haver no universo duas ou mais substâncias que possuam a
mesma natureza ou atributo''. Nesse sentido, a natureza é compreendida
em seus valores distintivos, não de substâncias diferentes \emph{per se}
(conforme ele explica em Nota à Proposição \versal{X}, Parte \versal{I} ``há apenas uma
substância no universo, e ela é absolutamente infinita'') mas de seus
diferentes modos. Enquanto Deus é uma infinidade de possibilidades, um
corpo é um ``modo finito'' de expressão dessa substância (Parte \versal{II}, Def.
\versal{I}.). Esse foco em ``valores distintivos'' e, portanto, em um
desequilíbrio como uma necessidade estrutural, é o que permite a
Espinoza iluminar essa complicada relação entre ``mente'' e ``corpo''
(\textsc{kordela}, 2007). Seu monismo não tratou de simplesmente limpar o terreno
de qualquer diferença; na verdade, ele fala de ``pensamentos'' e
``corpos'' como diferentes em atributos e natureza, ou seja, em
``valor''. Um excedente, nesse arranjo distintivo, está evidente na
seguinte passagem da Parte \versal{I}, Definição \versal{II}: ``Diz"-se finita em seu
gênero aquela coisa que pode ser limitada por outra da mesma natureza.
Por exemplo, diz"-se que um corpo é finito porque sempre concebemos um
outro maior. Da mesma maneira, um pensamento é limitado por outro
pensamento. Mas um corpo não é limitado por um pensamento, nem um
pensamento por um.'' Nesse sentido, algo da existência do corpo não pode
ser capturado pelos pensamentos, da mesma maneira que os pensamentos não
encontram representação completa no corpo. Essa interpretação conceitual
bem simples de um ``excedente'' quando se trata do sujeito como um
``ser'' de valores concomitantemente dissonantes dialoga com a concepção
freudiana inicial da angústia como um excesso que não encontra terreno
nem no ``corpo'' e nem na ``mente''.

Esse conceito de excedente também adiciona uma outra camada de
complicações à noção de um ``excesso'' imanente à linguagem, ou ao
Simbólico nos trabalhos iniciais de Lacan. Acadêmicos lacanianos,
notadamente Žižek (1992) --- que, na realidade, também não é chegado à
filosofia deleuziana ---, dão ênfase a como a psicanálise não deveria
acatar a reclamação do paciente por seu valor aparente (um argumento
justo e radical, especialmente em se tratando de práticas terapêuticas
que rejeitam o inconsciente e servem bem à ideologia hegemônica do
capitalismo tardio contemporâneo). Ao invés disso, a psicanálise deveria
procurar pelo ``excesso'' de significado naquilo que o paciente veio
dizer, ou o ``excedente do que é efetivamente dito, não a mensagem
planejada, mas a mensagem em sua forma verdadeira, invertida'' (\textsc{kordela},
2007, p. 7). Essa versão do Real, nas palavras de Lacan, dispensando"-se a
contestação feita pela literatura contemporânea acerca de seus escritos
tardios,\footnote{Cf. \textsc{miller}, 2003; \textsc{soler}, 2014.} ainda atribui à psicanálise um
modo de interpretação dos sintomas que pode ser radical na forma de ``um
novo modo de semiotização da subjetividade'', inaugurado pelo trabalho
de Freud com pacientes diagnosticados como histéricos, mas que ainda
precisa de novas rupturas ``com os universos de referência'' (\textsc{guattari} e
\textsc{rolnik}, 2007). Ir para além da ``interpretação'' significa, para
Guattari, ir para além do ``poder'' de um analista e também das
``palavras'', significando embarcar em ``revoluções analíticas'' que
rompem com ``modos estratificados de subjetivação'' pré"-determinados, ou
pré"-gravados, que não estão apenas ligados ao encontro clínico. Ele
escreve sobre esse radical compromisso com o excedente como sendo parte
de ``modos de ruptura assignificante, que apareceram simultaneamente na
literatura, no surrealismo, na pintura, e aí por diante'' (\textsc{guattari} e
\textsc{rolnik}, 2007, p.~381).\footnote{Essa passagem não foi encontrada na versão
  brasileira da publicação, de modo que mantivemos a tradução direta
  feita por Alexandre Cleaver.} Ao invés de um ``resto'' ao que podemos
``pensar'' sobre, esse excesso na angústia poderia ser pensado em
relação ao que Guattari chamou de ``caos''.

Ser ``um corpo'' é uma realidade que se apresenta em constante tensão
entre a acumulação e fluxo ``caótico'' de energia libidinal e o que lhe
controla, ou permitindo que emerjam novas conjunções ou estabelecendo um
limite. O contorno de um corpo marcado por palavras; palavras de um
reino Simbólico estruturado dentro de um \emph{modus operandi} colonial
e patriarcal sugeririam uma circularidade das repetições sob a lógica da
pulsão de morte. Para que tal fluxo libidinoso, evidente nos primeiros
textos de Freud e tão estimado por Guattari, possa transmitir um caráter
afirmativo, o que deve ser redefinido é precisamente o mítico estado
pré"-subjetivo que Lacan --- e não Freud --- assentiu ser uma
``negatividade'' (ao menos nos trabalhos do princípio e metade da sua
vida). É a influência de Hegel na consideração do tempo e da história
que favoreceu a posição privilegiada de um Simbólico que não poderia se
alterar efetivamente, assim limitando as próprias noções de
criatividade, singularidade, potência e afirmação (Braidotti, 2017). Ao
contrário do que creem leituras superficiais, o elemento espinozista do
projeto de Deleuze e Guattari não estava oferecendo em seu lugar uma
visão do sujeito como tendo um ``reservatório de positividade'' no
começo, que então é ``perdido'' ao se deparar com a ordem
louca"-má"-triste edipal capitalista. Na elaboração de Guattari da noção
de ``caos'', encontrada em \emph{Caosmosis}, de 1992, e na coleção
\emph{Chaosophy}, de 1995, vemos essa ``energia libidinal'', que Freud
percebe estar flutuando pelo corpo em seus primeiros textos sobre
angústia, não como um ``início gerador'', mas como um meio, um fluxo que
rompe com a dualidade corpo/palavra e se foca no ``limiar''. Uma tensão,
um limiar, uma zona de inventividade, transformação; e a criatividade é,
nesse sentido, do nível do ``caos''.

Aqui, a escolha espinozista revela então que a ``afirmação'' é uma
questão de diferença, do excedente gerado no ``meio'', no decorrer da
vida, ao invés de um poder que ali estava e então é ``perdido'' pela
nossa entrada na cultura. Nesse sentido, meu foco na angústia como sendo
o afeto da afirmação (portanto, diferença e transformação ao invés de
repetição e resistência) dialoga com o tópico de encontrar na melancolia
e, portanto, no ``luto fracassado'', uma identificação com o que está
perdido como um modo de resistir ao poder. O trabalho de Butler (1997)
em \emph{A vida psíquica do poder} (que traça fronteiras entre Hegel,
Nietzsche, Freud e Foucault) pressupôs uma certa linearidade de tempo,
mesmo se em uma forma ``ideal''. Também radical em sua crítica à
opressão identitária, o foco na ``perda'' --- ou daquilo que ali estava e
foi perdido, ou até mesmo considerando a perda do ``poderia estar ali''
mas não foi permitido --- não rompe com a linearidade do tempo. Nesse
sentido, ele também não romperá com a preeminência da linguagem, ou o
Simbólico patriarcal e colonial. Então, na exploração do que existe para
além da lógica do patriarcado, um ``excesso'' que é produzido pela
diferença do ``meio'' que está vivo no afeto da angústia prova"-se mais
frutífero ao pensamento.

Para conectar essa produção distintiva do ``meio'' com a libido dos
fluxos ``energéticos'' do Freud incipiente, outro conceito central da
\emph{Ética} de Espinoza pode nos auxiliar: \emph{conatus}. Do latim
para uma tendência de ``se esforçar'', Espinoza o define como: ``Cada
coisa esforça"-se, tanto quanto está em si, por perseverar'' (Proposição
\versal{VI}, Parte \versal{III}). Não simplesmente seguir ``sendo'', ou uma
autopreservação, mas também tendo seu ``poder de ação aumentado''; é
assim que Espinoza define a qualidade conativa dos corpos. Essa
tendência ``afirmativa'' é necessariamente ``compartilhada'' ou
``coletiva'' uma vez que esteja relacionada ao aumento ou diminuição da
capacidade de ser afetado e de afetar outros corpos. Portanto, ``os
corpos conativos de Espinoza também são associativos ou (poderíamos até
dizer) corpos sociais, no sentido de que cada um está, por sua própria
natureza de corpo, continuamente afetando e sendo afetado por outros
corpos'' (Bennett, 2010:21). Sua ontologia, então, propõe que nós
compartilhamos a mesma substância que está no mundo em diferentes e
distintivas modalidades. De modo bem contrastante com a negatividade do
desejo (sem mencionar sua ligação com uma ``necessidade'' e
``exigência'' que o inscrevem na função fálica dos primeiros
ensinamentos de Lacan), o que move nossas vidas não é a repetição da
negatividade, mas uma tendência afirmativa a produzir diferença,
ancorada nesse ``excedente'', que é um excesso da ordem da experiência.
Seguindo essa linha de pensamento, \emph{conatus} se parece mais com os
textos iniciais de Freud, que atribuem valor a tal energia libidinal que
é ``convertida'' em vários sintomas ou em angústia. É o fluxo energético
presente em suas primeiras formulações sobre angústia que pode ser
vinculado ao ``caos'' de Guattari, mesmo quando, mais tarde, foi
articulado por Freud como o encontro das pulsões de vida e morte.

Para mapear esse entendimento do excedente com o
``conhecimento"-do"-corpo'' que está aquém ou além do limite da linguagem,
como o que está em jogo nessas formulações sobre angústia, estou
tentando enfatizar como politicamente mais potente e interessante, como
um movimento para além da marca patriarcal da cultura moderna, devemos
revisitar a questão da ``diferença'' e do ``antagonismo''. O excedente,
nessa compreensão espinozista, pressupõe diferença. Assim, ao invés de
pensar em uma ontologia do sujeito em que a afirmação não possui
``antagonismo'',\footnote{Conforme Žižek o expressaria em uma crítica corajosa ao
materialismo pós"-deleuziano, cf. \textsc{žižek}, 2010.} a própria produção
constante de excedente é antagônica, e é essa complicação de uma
concepção conativa distintiva do sujeito, da natureza e do corpo que faz
a vida ir ``adiante''. Esse modo de continuidade é também
necessariamente singular e criativo e não irá se repetir em
negatividade, mas transformar"-se em ruptura. A ruptura caótica que
Guattari atribui ao que está para além da linguagem pode ser relacionada
com a experiência da angústia, ao mesmo tempo em que também informa o
debate contemporâneo acerca de uma ``preservação'' da diferença sexual
como um antídoto para o capitalismo neoliberal. Pensar em termos
energéticos, ou em ``acumulação'', ``descarga'', ``troca'' e
``conversão'' de energia libidinosa, como Freud o fez em seus textos
iniciais sobre angústia, permite"-nos capturar uma singularidade
distintiva de um excedente evidente na angústia e que não está atrelado
à diferença sexual e ao Simbolismo.

Em termos muito simples, a cartografia dessa lógica pode ser mapeada da
seguinte forma: apenas ao ``seguir'' existindo e vivendo pode a
diferença entre os diferentes atributos da substância ser acumulado como
um excedente. O excesso é produzido pela dissonância entre
``pensamentos'' e ``corpo''; ``ideias'' e ``matéria''; ``representação''
e o que existe para além dela, ou um ``conhecimento"-do"-corpo''. O
excedente é essa ``energia libidinosa'' caótica que Freud notou como
tentando achar uma válvula de escape para poder afetar e ser afetada por
outros corpos, para se mover; e, em seus meandros, é experimentada como
angústia.

Dessa forma, não é simplesmente ou somente a diferença sexual como
``estruturante'' do sujeito que pode garantir a singularidade ---
conforme o caminho que Zupančič defende em \emph{What is Sex?} (2017)
para uma crítica a teorias de gênero e \emph{queer}, que ela enxerga
como presas ao Imaginário. Há algo da ordem do corpo, da forma como ele
é experimentado, que é excessivo. Esse excedente gerado na diferença
entre um ``conhecimento"-do"-corpo'' e a ``consciência'' como tal, que
revela singularidade e criatividade na subjetividade, é o que está
presente na angústia e, portanto, é frutífero para uma leitura feminista
de Freud. Angústia, a partir da perspectiva de leitura adotada neste
capítulo, traz à tona um corpo em fluxo, aberto a atualizações que não
estão atreladas ao pensamento ou à simbolização, sendo, portanto, um
sinal de um tipo particular de ``conhecimento"-do"-corpo''. Angústia,
nesses textos de Freud anteriores a 1920, não pode ser subsumida em
palavras, ou interpretação e significado. Angústia, nesse sentido, é
``sem significado''; porém é transformativo, ao insistir e empurrar a
energia libidinal que é a ``vida'' sobre a materialidade do corpo.

Considero essa leitura fértil a partir de uma perspectiva feminista, já
que ela me permite pensar a angústia na clínica psicanalítica
contemporânea de maneira diferente, como um afeto que está para além da
linguagem e também para além do que Lacan chamou de ``Nome"-do"-Pai'',
abrindo uma ecologia feminista dentro desse conceito de angústia e em
sua realidade clínica. Se para Lacan, no \emph{Seminário 10}, a angústia é uma
aparição do Real, e o Real é caótico, ou sem lei, há algo em relação à
ontologia desse mesmo ``caos'' que está politicamente em jogo. Até a
chegada dos enigmáticos ensinamentos posteriores de Lacan, em que um
Real que ``nada tem a ver'' com o Simbólico aparece de um modo um tanto
frágil, os sujeitos estão necessariamente presos ao significante e,
portanto, ao ``Nome"-do"-Pai''. Conforme dito por Miller,

\begin{quote}
Sem o Nome"-do"-Pai há apenas o caos. Caos significa fora da lei, um caos
no simbólico. Sem o Nome"-do"-Pai não há linguagem, há apenas
\emph{lalangue}. Sem o Nome"-do"-Pai há, propriamente, corpo algum, há
apenas o corpóreo, a carne, o organismo, a matéria, a imagem. Há eventos
do corpo, eventos que destroem o corpo. Sem o Nome"-do"-Pai, há um
sem"-o"-corpo'' (\textsc{miller}, 2003).
\end{quote}

Essa ruptura caótica, com sua presença avassaladora, ressoa o que é
descrito por ``ataques'' de angústia, ou a angústia em todo o seu volume
ensurdecedor. Ao mesmo tempo, ``lalangue'', esse contentamento poético
da ordem do corpo, esse modo singular, ímpar, inventivo de falar, também
é parte de tal ``caos''. Nesse contexto, a angústia marca um território
de tensão, esse limiar entre a relação com a Lei (e, por extensão, com o
arranjo patriarcal) e tudo que existe para além dele, o caos que está em
fluxo através do corpo e não pode ser capturado pela linguagem ou por
palavras.

Uma intervenção tão simples que se prende a tal ``caos'', a um corpo
conativo que tem uma existência política coletiva, é importante como
ecologia feminista e também como um modo de se ponderar sobre as
possibilidades de emancipação que estão para além da lei patriarcal. A
despretensiosa demarcação da angústia como experiência direta daquilo
que não é capturado pela linguagem, mas que está dentro dessa ontologia
afirmativa, aponta para uma possibilidade de emancipação que não depende
do Outro e, consequentemente, da Lei e das modernas epistemologias
coloniais patriarcais. Não importa o quanto os corpos estão
``disciplinados'', em um sentido foucaultiano, o caos irá escapar
através da angústia. E, diferentemente da lógica dos sintomas e de suas
interpretações clínicas, o ``excedente'' da ordem do
``conhecimento"-do"-corpo'' não deve receber um significado por meio da
mesma linguagem que o modifica Ao invés disso, o excedente distintivo e
afirmativo de tal ``conhecimento"-do"-corpo'' deve ser abordado
clinicamente de uma maneira criativa e transformadora --- não com
construções que permeiam significados, mas com conexões poéticas. Nesse
sentido, o que essa leitura espinozista dos textos iniciais de Freud
sobre a angústia pode indicar é que a angústia deve ser manejada
criativamente e coletivamente. Ela é, como vimos neste capítulo, um
afeto político.

\pagebreak

\section{bibliografia}

\begin{Parskip}
\textsc{bennett}, Jane. \emph{Vibrant
Matter: A Political Ecology of Things.} \versal{USA}: Duke University Press, 2010.

\textsc{bottici}, Chiara. ``Bodies in plural: Towards an anarcha"-feminist
manifesto''. In: \emph{Thesis Eleven}, 142(1), 2017, p. 91--111.

\textsc{braidotti}, Rosi. \emph{Per Una Politica Affermativa.} Milano: Mimesis
Edizioni, 2017.

\textsc{butler}, Judith. \emph{The Psychic Life of Power: Theories in
Subjection.} Stanford: Stanford University Press, 1997.

\textsc{cixous}, Helene. ``The Laugh of the Medusa''. In: \emph{Signs}, vol. 1, no.
4, Summer, 1976, p. 875-893.

\textsc{curley}, Edwin. \emph{A Spinoza Reader. The Ethics and Other works.
Benedict de Spinoza}. Ed. and Trans. Edwin Curley. Princeton, New Jersey:
Princeton University Press, 1994.

\textsc{deleuze}, Gilles. and \textsc{guattari}, Felix. \emph{A Thousand Plateaus --
Capitalism and Schizophrenia}. Minneapolis and London: University of
Minnesota Press, 1987.

\_\_\_\_\_\_\_\_. and \_\_\_\_\_\_\_\_. \emph{Anti"-Oedipus -- Capitalism and
Schizophrenia.} Minneapolis: University of Minnesota Press, 1983.

\textsc{ettinger}, Bracha. \emph{The Matrixial Borderspace}. \versal{USA}: University of
Minnesota Press, 2006.

\textsc{freud}, Sigmund. ``Inhibitions, Symptoms and Anxiety''. In: \emph{The Standard
Edition of the Complete Psychological Works of Sigmund Freud, Volume \versal{XX}
(1925- 1926): An Autobiographical Study, Inhibitions, Symptoms and
Anxiety, The Question of Lay Analysis and Other Works}, 1926, p. 75-176.

\_\_\_\_\_\_\_\_. ``Introductory Lectures on Psycho"-Analysis''. In: \emph{The
Standard Edition of the Complete Psychological Works of Sigmund Freud,
Volume \versal{XVI} (1916 1917): Introductory Lectures on Psycho"-Analysis (Part
\versal{III})}, 1917, p. 241-463.

\_\_\_\_\_\_\_\_.  ``Draft E. How Anxiety Originates'', June 6, 1894. In:
\emph{The Complete Letters of Sigmund Freud to Wilhelm Fliess,
1887-1904}, 1894a, p. 78-83.

\_\_\_\_\_\_\_\_.  ``On The Grounds for Detaching a Particular Syndrome
From Neurasthenia Under The Description `Anxiety Neurosis'". In: \emph{The
Standard Edition of the Complete Psychological Works of Sigmund Freud,
Volume \versal{III} (1893-1899): Early Psycho"-Analytic Publications}, 1894b, p. 85-115.

\_\_\_\_\_\_\_\_. ``Draft A1 from Extracts From the Fliess Papers''. In:
\emph{The Standard Edition of the Complete Psychological Works of
Sigmund Freud, Volume \versal{I} ( 1886-1899): Pre"-Psycho"-Analytic Publications
and Unpublished Drafts}, 1892a, p. 177-178.

\_\_\_\_\_\_\_\_. ``Draft B from Extracts From the Fliess Papers''. In:
\emph{The Standard Edition of the Complete Psychological Works of
Sigmund Freud, Volume \versal{I} ( 1886-1899): Pre"-Psycho"-Analytic Publications
and Unpublished Drafts}, 1892b, p. 179-184.

\textsc{freud}, Sigmund. \& \textsc{fliess}, Wilhelm. \emph{A correspondência completa de Sigmund
Freud e Wilhelm Fliess 1887-1904}. Rio de Janeiro: Imago, 1986.

\_\_\_\_\_\_\_\_. (1917) ``Conferências introdutórias''. In: \emph{Obras
completes brasileiras, vol. 13}. São Paulo: Companhia das Letras, 2014.

\textsc{grosz}, Elizabeth. \emph{The Incorporeal: Ontology, Ethics, and the Limits
of Materialism}. New York: Columbia University Press, 2017.

\_\_\_\_\_\_\_\_. \emph{Chaos, Territory, Art: Deleuze and the Framing of
the Earth}. New York: Columbia University Press, 2008.

\_\_\_\_\_\_\_\_. ``Deleuze, Bergson and the Concept of Life''. In: \emph{Revue
internationale de philosophie}, 241(3), 2017, p. 287-300.

\textsc{guattari}, Felix. \emph{Chaosophy: Texts and Interviews 1972-1977.}
\versal{USA}: Semiotext(e), 2009.

\_\_\_\_\_\_\_\_. \emph{Chaosmosis. An Ethico"-aesthetic paradigm}.
Bloomington and Indianapolis: Indiana University Press, 1995.

\_\_\_\_\_\_\_\_. \emph{Schizoanalytic Cartographies}. London and Oxford: Bloomsbury, 1989.

\textsc{guattari}, Felix. and \textsc{rolnik}, Suely. \emph{Molecular Revolution in
Brazil.} \versal{USA}: Semiotext(e), 2007.

\_\_\_\_\_\_\_\_. e \_\_\_\_\_\_\_\_. \emph{Cartografias do desejo}. Petrópolis:
Vozes, 1996.

\textsc{kordela}, Kiarina. \emph{Surplus: Spinoza, Lacan.} \versal{USA}: \versal{SUNY} Press, 2007.

\textsc{lacan}, J. \emph{Anxiety. The Seminar of Jacques Lacan Book \versal{X}}. Ed.
J"-A Miller. London: Polity Press, 2016.

\_\_\_\_\_\_\_\_. (1960) ``The Subversion of the Subject and the Dialectic of
Desire in the Freudian Unconscious''. In: \emph{Écrits: The First Complete
Edition in English}. Trans. Bruce Fink. New York: W. W. Norton and Company, 2005.

\textsc{miller}, Jacques-Alain. ``Lacan's Later Teaching''. In: \emph{Lacanian Ink}, vol.~21,
Spring, 2003. Online:
\textless{}\emph{https://bit.ly/2ksWNgP}\textgreater{}.

\textsc{mitchell}, Juliet. \emph{Mad Men And Medusas: Reclaiming Hysteria.} \versal{UK}:
Basic Books, 2000.

\_\_\_\_\_\_\_\_. \emph{Psychoanalysis and Feminism}. Oxford: Patheon, 1974.

\textsc{soler}, Colette. \emph{Lacanian Affects: The Function of Affect in
Lacan's Work.} London: Routledge, 2014.

\textsc{spinoza}, Baruch. \emph{Ética}. Belo Horizonte: Autêntica, 2009.

\textsc{watson}, Janell. \emph{Guattari's Diagrammatic Thought. Writing Between
Lacan and Deleuze.} London and Oxford: Bloomsbury, 2011.

\textsc{webster}, Jamieson. \emph{Conversion Disorder: Listening to the Body in
Psychoanalysis.} New York: Columbia University Press, 2018.

\textsc{žižek}, Slavoj. \emph{Interrogating the Real}. London: Continuum, 2010.

\_\_\_\_\_\_\_\_. \emph{Enjoy Your Symptom!}. London: Routledge, 1992.

\textsc{zupančič}, Alenka. \emph{What is Sex?}. \versal{USA: MIT} Press, 2017.
\end{Parskip}